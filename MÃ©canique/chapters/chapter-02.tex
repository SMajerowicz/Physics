\chapter{Lois de conservation}

\section{\'Energie}

Les \emph{int\'egrales premi\`eres} ou \emph{int\'egrales du mouvement}, sont les fonctions de $q_{i}$ et $\dot{q}_{i}$ qui conservent une valeur constante au cours du mouvement. Elles sont additives, c'est-\`a-dire qu'avant et après une int\'eraction, leur somme garde une valeur identique.

Commen\c{c}ons par l'int\'egrale premi\`ere qui d\'ecoule de l'\emph{uniformi\'e du temps}.

Pour un syst\`eme ferm\'e \`a $s$ degr\'es de libert\'e, nous avons :
\be
	\dfrac{\mathrm{d}L}{\mathrm{dt}} = \sum_{i=1}^{s}\dfrac{\partial L}{\partial q_{i}}\dfrac{\mathrm{d} q_{i}}{\mathrm{dt}} + \sum_{i=1}^{s}\dfrac{\partial L}{\partial \dot{q}_{i}}\dfrac{\mathrm{d}\dot{q}_{i}}{\mathrm{dt}} + \dfrac{\partial L}{\partial \mathrm{t}}
\ee
L'uniformit\'e du temps donne $\dfrac{\partial L}{\partial \mathrm{t}} = 0$ et en utilisant l'\'equation (\ref{EQ:2_6}), nous avons :
\bea
	\dfrac{\mathrm{d}L}{\mathrm{dt}} & = & \sum_{i=1}^{s}\dfrac{\mathrm{d}}{\mathrm{dt}}\left(\dfrac{\partial L}{\partial \dot{q}_{i}}\dot{q}_{i}\right) + \sum_{i=1}^{s}\dfrac{\partial L}{\partial \dot{q}_{i}}\ddot{q}_{i} \nonumber \\
	& = & \sum_{i=1}^{s}\dfrac{\mathrm{d}}{\mathrm{dt}}\left(\dfrac{\partial L}{\partial \dot{q}_{i}}\dot{q}_{i}\right) \nonumber \\
	0 & = & \dfrac{\mathrm{d}}{\mathrm{dt}}\left(\sum_{i=1}^{s}\dfrac{\partial L}{\partial \dot{q}_{i}}\dot{q}_{i} - L\right)
\eea
Par cons\'equent, la quantit\'e :
\be
	E = \sum_{i=1}^{s}\dfrac{\partial L}{\partial \dot{q}_{i}}\dot{q}_{i} - L \label{EQ:6_1}
\ee
est constante dans le temps pour un syst\`eme ferm\'e. $E$ est l'\emph{\'energie} du syst\`eme.