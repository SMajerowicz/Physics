\chapter{Lois de conservation}

\section{\'Energie}

Les \emph{int\'egrales premi\`eres} ou \emph{int\'egrales du mouvement}, sont les fonctions de $q_{i}$ et $\dot{q}_{i}$ qui conservent une valeur constante au cours du mouvement. Elles sont additives, c'est-\`a-dire qu'avant et après une int\'eraction, leur somme garde une valeur identique.

Commen\c{c}ons par l'int\'egrale premi\`ere qui d\'ecoule de l'\emph{uniformi\'e du temps}.

Pour un syst\`eme ferm\'e \`a $s$ degr\'es de libert\'e, nous avons :
\be
	\dfrac{\mathrm{d}L}{\mathrm{dt}} = \sum_{i=1}^{s}\dfrac{\partial L}{\partial q_{i}}\dfrac{\mathrm{d} q_{i}}{\mathrm{dt}} + \sum_{i=1}^{s}\dfrac{\partial L}{\partial \dot{q}_{i}}\dfrac{\mathrm{d}\dot{q}_{i}}{\mathrm{dt}} + \dfrac{\partial L}{\partial \mathrm{t}}
\ee
L'uniformit\'e du temps donne $\dfrac{\partial L}{\partial \mathrm{t}} = 0$ et en utilisant l'\'equation (\ref{EQ:2_6}), nous avons :
\bea
	\dfrac{\mathrm{d}L}{\mathrm{dt}} & = & \sum_{i=1}^{s}\dfrac{\mathrm{d}}{\mathrm{dt}}\left(\dfrac{\partial L}{\partial \dot{q}_{i}}\dot{q}_{i}\right) + \sum_{i=1}^{s}\dfrac{\partial L}{\partial \dot{q}_{i}}\ddot{q}_{i} \nonumber \\
	& = & \sum_{i=1}^{s}\dfrac{\mathrm{d}}{\mathrm{dt}}\left(\dfrac{\partial L}{\partial \dot{q}_{i}}\dot{q}_{i}\right) \nonumber \\
	0 & = & \dfrac{\mathrm{d}}{\mathrm{dt}}\left(\sum_{i=1}^{s}\dfrac{\partial L}{\partial \dot{q}_{i}}\dot{q}_{i} - L\right)
\eea
Par cons\'equent, la quantit\'e :
\be
	E = \sum_{i=1}^{s}\dfrac{\partial L}{\partial \dot{q}_{i}}\dot{q}_{i} - L \label{EQ:6_1}
\ee
est constante dans le temps pour un syst\`eme ferm\'e. $E$ est l'\emph{\'energie} du syst\`eme. Et puisque la fonction de Lagrange est additive, l'\'energie l'est aussi. Les syst\`emes m\'ecaniques dont l'\'energie se conserve sont appel\'es \emph{syst\`emes conservatifs}. Enfin, si le syst\`eme est plac\'e dans un champ ext\'erieur constant par rapport au temps alors la loi de conservation de l'\'energie reste valable car la fonction de Lagrange est implicitement ind\'ependant du temps.

\`A partir de l'\'equation (\ref{EQ:5_1}, nous pouvons \'ecrire :
\be
	\sum_{i=1}^{s}\dfrac{\partial L}{\partial\dot{q}_{i}}\dot{q}_{i} = \sum_{i=1}^{s}\dfrac{\partial (T-U)}{\partial\dot{q}_{i}}\dot{q}_{i} = \sum_{i=1}^{s}\dfrac{\partial T}{\partial\dot{q}_{i}}\dot{q}_{i}
\ee
car l'\'energie potentielle $U$ ne d\'epend que des coordonn\'ees (voire du temps).
Nous savons que l'\'energie cin\'etique $T = f(\dot{q}^{2})$. Or pour une fonction homog\`ene de degr\'e $\alpha$, telle $f(t.x) = t^{\alpha}f(x)$, alors le th\'eor\`eme d'Euler donne pour cette fonction :
\be
	\sum_{i} x_{i}\dfrac{\partial f(x)}{\partial x_{i}} = \alpha f(x)
\ee
Appliqu\'ee \`a l'\'energie cin\'etique, cela donne :
\be
	\sum_{i=1}^{s}\dfrac{\partial T}{\partial\dot{q}_{i}}\dot{q}_{i} = 2T = \sum_{i=1}^{s}\dfrac{\partial L}{\partial\dot{q}_{i}}\dot{q}_{i} \label{EQ:6_1_1}
\ee
ce qui permet de conclure \`a, en compl\'etant l'\'equation (\ref{EQ:6_1}) :
\be
	E = T(q,\dot{q}) + U(q) \label{EQ:6_2}
\ee
soit en coordonn\'ees cart\'esiennes :
\be
	E = \sum_{a}\dfrac{m_{a}\vec{v}_{a}^{\,2}}{2} + U(\begin{Bmatrix}\vec{r}_{i}\end{Bmatrix}_{1}^{s}) \label{EQ:6_3}
\ee

\section{Impulsion}

Apr\`es l'unformit\'e du temps, nous allons \'etudier les cons\'equences de l'homog\'en\'eit\'e de l'espace. Dans ce cadre, les propri\'et\'es m\'ecaniques d'un syst\`eme ferm\'e ne changent pas lors d'un d\'eplacement parall\`ele du syst\`eme dans son entier. Supposons ainsi que la fonction de Lagrange reste inchang\'ee pour un d\'eplacement infinit\'esimal tel que : $\vec{r}_{a} \rightarrow \vec{r}_{a} + \vec{\epsilon}$, i.e. $\delta L = 0$ :
\bea
	\delta L(\vec{r}_{a},\vec{v}_{a}) & = & \sum_{a}\dfrac{\partial L}{\partial\vec{r}_{a}}\delta\vec{r}_{a} + \sum_{a}\dfrac{\partial L}{\partial\vec{v}_{a}}\delta\vec{v}_{a} \nonumber \\
	& = & \sum_{a}\dfrac{\partial L}{\partial\vec{r}_{a}}\vec{\epsilon} \text{ puisque }\delta\vec{v}_{a} = \vec{0} \nonumber \\
\eea
donc $\delta L = 0$ est \'equivalent \`a :
\bea
	\forall \vec{\epsilon}\text{, }\sum_{a}\dfrac{\partial L}{\partial\vec{r}_{a}}\vec{\epsilon} & = & 0 \nonumber \\
	\forall \vec{\epsilon}\text{, }\vec{\epsilon}\cdot\sum_{a}\dfrac{\partial L}{\partial\vec{r}_{a}} & = & 0 \nonumber \\
	\Leftrightarrow \sum_{a}\dfrac{\partial L}{\partial\vec{r}_{a}} & = & 0 \label{EQ:7_1}
\eea
Or les \'equations de Lagrange (\ref{EQ:5_2}) donnent :
\bea
	\dfrac{\mathrm{d}}{\mathrm{dt}}\left(\dfrac{\partial L}{\partial \vec{v}_{a}}\right) & = & \dfrac{\partial L}{\partial \vec{r}_{a}} \nonumber \\
	\Leftrightarrow \sum_{a}\left(\dfrac{\mathrm{d}}{\mathrm{dt}}\left(\dfrac{\partial L}{\partial \vec{v}_{a}}\right)\right) & = & \sum_{a}\left(\dfrac{\partial L}{\partial \vec{r}_{a}}\right) \nonumber
\eea
En appliquant l'\'equation (\ref{EQ:7_1}), nous avons :
\be
	\sum_{a}\left(\dfrac{\mathrm{d}}{\mathrm{dt}}\left(\dfrac{\partial L}{\partial \vec{v}_{a}}\right)\right) = \dfrac{\mathrm{d}}{\mathrm{dt}}\sum_{a}\left(\dfrac{\partial L}{\partial \vec{v}_{a}}\right) = \vec{0}
\ee
Ainsi, dans un syst\`eme ferm\'e, la quantit\'e vectorielle :
\be
	\vec{P} = \sum_{a}\dfrac{\partial L}{\partial \vec{v}_{a}} \label{EQ:7_2}
\ee
est inchang\'ee pendant le mouvement. Le vecteur $\vec{P}$ est l'\emph{impulsion}\footnote{Anciennement appel\'ee \emph{quantit\'e de mouvement}} du syt\`eme. En utilisant l'\'equation (\ref{EQ:5_1}), nous obtenons :
\bea
	L & = & \sum_{a}\dfrac{m_{a}\vec{v}_{a}^{\,2}}{2} - U(\begin{Bmatrix}\vec{r}_{a}\end{Bmatrix}_{1}^{s}) \nonumber \\
	\Rightarrow \dfrac{\partial L}{\partial \vec{v}_{a}} & = & \dfrac{m_{a}}{2}\dfrac{2\partial\vec{v}_{a}}{\partial\vec{v}_{a}} - \dfrac{\partial U}{\partial\vec{v}_{a}} \nonumber \\
	& = & m_{a}\vec{v}_{a}
\eea
Ainsi :
\be
	\vec{P} = \sum_{a}m_{a}\vec{v}_{a} \label{EQ:7_3}
\ee
avec $\vec{P} = \sum_{a}\vec{p}_{a}$ pour un syst\`eme ferm\'e en l'absence de champ ext\'erieur.

Supposons l'existence d'un champ ext\'erieur telle que $U\rightarrow U(z)$ en coordonn\'ees cart\'esiennes, les \'equations de Newton (\ref{EQ:5_3}) donnent :
\bea
	m_{a}\dfrac{\mathrm{d}v_{x}}{\mathrm{d}x} & = & - \dfrac{\partial U}{\partial x} = 0 \nonumber \\
	m_{a}\dfrac{\mathrm{d}v_{y}}{\mathrm{d}y} & = & - \dfrac{\partial U}{\partial y} = 0 \nonumber \\
	m_{a}\dfrac{\mathrm{d}v_{z}}{\mathrm{d}z} & = & - \dfrac{\partial U}{\partial z} \nonumber
\eea
et donc, dans ce cas, $P_{x}$ et $P_{y}$ se conservent donc.

Une cons\'equence de l'\'equation (\ref{EQ:7_1}) en accord avec celle de (\ref{EQ:5_4}) pour la force $\vec{F}_{a}$ agissant sur le point mat\'eriel $a$ est que :
\be
	\sum_{a}\vec{F}_{a} = \vec{0} \label{EQ:7_4}
\ee
i.e. que la somme de toutes les forces s'exer\c{c}ant sur toutes les particules d'un syst\`eme ferm\'e est absolument nulle. Dans le cas particulier de deux particules, il y a $\vec{F}_{1} + \vec{F}_{2} = \vec{0}$. Les deux forces sont \'egales mais oppos\'ees, il s'agit de la loi de l'\'egalit\'e entre l'action et la r\'eaction.

Dans le cadre du mouvement d\'ecrit par les coordonn\'ees g\'en\'eralis\'ees, nous avons :
\begin{itemize}
	\item les \emph{impulsions g\'en\'eralis\'ees} :
	\be
		p_{i} = \dfrac{\partial L}{\partial\dot{q}_{i}} \label{EQ:7_5}
	\ee
	\item les \emph{forces g\'en\'eralis\'ees} :
	\be
		F_{i} = \dfrac{\partial L}{\partial q_{i}} \label{EQ:7_6}
	\ee
	\item les \'equations de Lagrange :
	\be
		\dot p_{i} = F_{i} \label{EQ:7_7}
	\ee
\end{itemize}

\section{Centre d'inertie}

Supposons deux r\'ef\'erentiels $(K)$ et $(K')$ tels que le second se meut à la vitesse $\vec{V}$ par rapport au premier. Nous avons donc $\vec{v}_{a} = \vec{v'}_{a} + \vec{V}$, soit, pour l'impulsion :
\bea
	\vec{P} = \sum_{a}m_{a}\vec{v}_{a} & = & \sum_{a}m_{a}\vec{v'}_{a} + \sum_{a}m_{a}\vec{V} \nonumber \\
	\Leftrightarrow \vec{P} & = & \vec{P'} + \vec{V}\sum_{a}m_{a} \label{EQ:8_1}
\eea
Il existe toujours un r\'ef\'erentiel $(K')$ tel que le syst\`eme est consid\'er\'e comme au repos, soit $\vec{P'} = \vec{0}$. La vitesse de ce r\'ef\'erentiel par rapport \`a $(K)$ peut s'\'ecrire :
\be
	\vec{V} = \dfrac{\vec{P}}{\sum_{a}m_{a}} = \dfrac{\sum_{a}m_{a}\vec{v}_{a}}{\sum_{a}m_{a}} \label{EQ:8_2}
\ee
et elle d\'efinit la vitesse dans son ensemble du syst\`eme m\'ecanique ayant une impulsion $\vec{P}$ non nulle. Et une analogie avec le mouvement d'un point mat\'eriel est \'evidente en posant sa masse comme $\mu = \sum_{a}m_{a}$. La masse est donc \emph{additive}.
La relation (\ref{EQ:8_2}) peut aussi s'\'ecrire comme :
\bea
	\vec{V} = \dfrac{\vec{R}}{\mathrm{dt}} & = & \dfrac{\sum_{a}m_{a}\dfrac{\vec{r}_{a}}{\mathrm{dt}}}{\sum_{a}m_{a}} = \dfrac{\mathrm{d}}{\mathrm{dt}}\left(\dfrac{\sum_{a}m_{a}\vec{r}_{a}}{\sum_{a}m_{a}}\right) \nonumber \\
	\Leftrightarrow \vec{R} & = & \dfrac{\sum_{a}m_{a}\vec{r}_{a}}{\sum_{a}m_{a}} \label{EQ:8_3}
\eea
Ainsi la vitesse d'un syst\`eme dans son ensemble est la vitesse de d\'eplacement du point de l'espace dont le rayon vecteur est d\'efini par l'\'equation (\ref{EQ:8_3}). Ce point est appel\'e \emph{centre d'inertie} du syst\`eme m\'ecanique.

Il s'agit d'une g\'en\'eralisation de la loi d'inertie \'enonc\'ee en (\ref{EQ:3_2}) pour un syst\`eme ferm\'e dont le centre d'inertie co\"incide avec le point mat\'eriel. En conclusion, pour \'etudier les propri\'et\'es m\'ecaniques d'une syst\`eme ferm\'e, il est int\'eressant de choisir le r\'ef\'erentiel dans lequel son centre d'inertie est au repos et le mouvement rectiligne et uniforme du syst\`eme peut alors \^etre occult\'e.
Soit $E_{\mathrm{int}}$ l'\'energie interne d'un syst\`eme ferm\'e au repos, elle comprend l'\'energie cin\'etiques du mouvement relatif des particules dans le syst\`eme ainsi que l'\'energie potentielle de leur int\'eraction mutuelle. L'\'energie totale d'un syst\`eme m\^u dans son ensemble par une vitesse $\vec{V}$ peut s'\'ecrire :
\be
	E = \dfrac{\mu \vec{V}^{\,2}}{2} + E_{\mathrm{int}} \label{EQ:8_4}
\ee
En effet, dans le r\'ef\'erentiel $(K)$ :
\bea
	E & = & \dfrac{1}{2}\sum_{a}m_{a}\vec{v}_{a}^{\,2} + U = \dfrac{1}{2}\sum_{a}m_{a}(\vec{v'}_{a} + \vec{V})^{\,2} + U \nonumber \\
	& = & \dfrac{1}{2}\sum_{a}m_{a}\vec{v'}_{a}^{\,2} + \dfrac{1}{2}\sum_{a}m_{a}\vec{V}^{\,2} + \sum_{a}m_{a}\vec{v'}_{a}\vec{V} + U \nonumber \\
	& = & \dfrac{\mu}{2}\vec{V} + \vec{V}\sum_{a}m_{a}\vec{v'}_{a} + \dfrac{1}{2}\sum_{a}m_{a}\vec{v'}_{a}^{\,2} + U \\
\eea
Puisque :
\bea
	E' & = & \dfrac{1}{2}\sum_{a}m_{a}\vec{v'}_{a}^{\,2} + U \nonumber \\
	\vec{P'} & = & \sum_{a}m_{a}\vec{v'}_{a} \nonumber
\eea
alors nous arrivons \`a :
\be
	E = E' + \vec{V}\cdot\vec{P'} + \dfrac{\mu}{2}\vec{V} \label{EQ:8_5}
\ee
qui est la loi de transformation de l'\'energie lors d'un changement de r\'ef\'erentiel se mouvant l'un par rapport \`a l'autre comme d\'efini. Aussi, dans le cas d'un syst\`eme m\'ecanique au repos dans le r\'ef\'erentiel $(K')$, alors $\vec{P'} = \vec{0}$  et $E'$ est l'\'energie interne de ce syst\`eme. Nous retombons alors sur l'\'equation (\ref{EQ:8_4}).

\section{Moment cinétique}

Apr\`es l'uniformit\'e du temps, l'homog\'en\'eit\'e de l'espace, nous allons \'etudier les cons\'equences de l'isotropie de l'espace. Celle-ci implique que les propri\'et\'es m\'ecaniques d'un syst\`eme ferm\'e ne changent pas lors d'une rotation. Nous consid\'erons donc que la fonction de Lagrange reste inchang\'ee lors d'une rotation infinit\'esimale.

\begin{figure}[htb!]
	\begin{center}
		\begin{picture}(150,200)(0,0)
			%axis
			\linethickness{0.05mm}
			\put(25,0){\vector(0,1){200}}\put(30,180){$\delta\vec{\varphi}$}
			\multiput(25,125)(5,0){20}{\line(1,0){3}}
			\multiput(25,125)(8,2){10}{\line(4,1){4}}
			%arms
			\linethickness{0.5mm}
			\put(25,25){\vector(1,1){100}}\put(75,50){$\vec{r}$}
			\put(12,20){$O$}
			\put(25,25){\vector(2,3){80}}
			\put(125,125){\vector(-1,1){20}}\put(120,135){$\delta\vec{r}$}
			%angles
			\linethickness{0.05mm}
			\qbezier(50,50),(50,55),(45,55)
			\put(52,57){$\theta$}
			\qbezier(50,125),(50,130),(45,130)
			\put(50,115){$\delta\varphi$}
		\end{picture}
		\caption{Rotation infinit\'esimale}\label{FIG:CHAP9_1}
	\end{center}
\end{figure}

Sur la figure (\ref{FIG:CHAP9_1}), le syst\`eme subit une rotation $\delta\vec{\varphi}$ qui donne la direction, l'axe de la rotation, et l'angle $\delta\varphi$ de rotation. Nous avons $\lvert\delta\vec{r}\rvert = r\sin\theta\sin\delta\varphi$. Or $\sin\delta\varphi \approx \delta\varphi$ puisque la rotation est infinit\'esimale. Donc $\lvert\delta\vec{r}\rvert = r\sin\theta\delta\varphi$. De plus, $\delta\vec{r}$ est perpendiculaire au plan défini par $(\delta\vec{\varphi},\vec{r})$, on peut en conclure :
\be
	\delta\vec{r} = \delta\vec{\varphi}\wedge\vec{r} \label{EQ:9_1}
\ee
Par d\'efinition :
\bea
	\dfrac{\mathrm{d}\delta\vec{r}}{\mathrm{dt}} & = & \delta\left(\dfrac{\mathrm{d}\vec{r}}{\mathrm{dt}}\right) \nonumber \\
	\Leftrightarrow \dfrac{\mathrm{d}}{\mathrm{dt}}(\delta\vec{\varphi}\wedge\vec{r}) & = & \delta\vec{v} \nonumber \\
	\Leftrightarrow \delta\vec{v} & = & \dfrac{\mathrm{d}\delta\vec{\varphi}}{\mathrm{dt}}\wedge\vec{r} + \vec{\varphi}\wedge\dfrac{\mathrm{d}\vec{r}}{\mathrm{dt}} \nonumber \\
	\Leftrightarrow \delta\vec{v} & = & \delta\vec{\varphi}\wedge\vec{v} \label{EQ:9_2}
\eea
car la rotation est ici ind\'ependante du temps.

La fonction de Lagrange est une fonction des coordonn\'ees et des vitesses des points mat\'eriels du syst\`eme aussi :
\be
	L = \sum_{a=1}^{s}L_{a}(\vec{r}_{a},\vec{v}_{a})
\ee
et sa variation s'\'ecrit :
\be
	\delta L = \sum_{a=1}^{s}\left(\dfrac{\partial L}{\partial\vec{r}_{a}}\delta\vec{r}_{a} + \dfrac{\partial L}{\partial\vec{v}_{a}}\delta\vec{v}_{a}\right)
\ee
et nous pouvons \'ecrire que cette variation est nulle car nous avons suppos\'e plus haut que la fonction de Lagrange est invariante dans le cadre d'une rotation. Les \'equations de Lagrange (\ref{EQ:2_6}) associ\'ees à l'\'egalit\'e (\ref{EQ:7_2}) implique :
\bea
	\delta L & = & \sum_{a}(\vec{\dot{p}}_{a}\cdot\delta\vec{r}_{a} + \vec{p}_{a}\cdot\delta\vec{v}_{a}) \nonumber \\
	& = & \sum_{a}\left(\vec{\dot{p}}_{a}\cdot(\delta\vec{\varphi}\wedge\vec{r}_{a}) + \vec{p}_{a}\cdot(\delta\vec{\varphi}\wedge\vec{v}_{a})\right)
\eea
en y incluant les \'equations (\ref{EQ:9_1}) et (\ref{EQ:9_2}). Cela donne gr\^ace \`a l'invariance par permutation circulaire du produit mixte\footnote{$(\vec{u}\wedge\vec{v})\cdot\vec{w} = (\vec{v}\wedge\vec{w})\cdot\vec{u} = (\vec{w}\wedge\vec{u})\cdot\vec{v} $} :
\be
	\delta L = \delta\vec{\varphi}\cdot\sum_{a}(\vec{r}_{a}\wedge\vec{\dot{p}}_{a} + \vec{v}_{a}\wedge\vec{p}_{a})
\ee
or :
\bea
	\dfrac{\mathrm{d}\vec{r}_{a}\wedge\vec{p}_{a}}{\mathrm{dt}} & = & \dfrac{\mathrm{d}\vec{r}_{a}}{\mathrm{dt}}\wedge\vec{p}_{a} + \vec{r}_{a}\wedge\dfrac{\mathrm{d}\vec{p}_{a}}{\mathrm{dt}} \nonumber \\
	& = & \vec{v}_{a}\wedge\vec{p}_{a} + \vec{r}_{a}\wedge\vec{\dot{p}}_{a}
\eea
Donc par l'invariance de la fonction de Lagrange du syst\`eme et par le fait nous raisonnons avec $\delta\vec{\varphi}$ arbitrairement choisi, nous arrivons \`a :
\be
	\dfrac{\mathrm{d}}{\mathrm{dt}}\left(\sum_{a}\vec{r}_{a}\wedge\vec{p}_{a}\right) = 0
\ee
permettant de conclure que lors du mouvement d'un sys\`eme ferm\'e, la quantit\'e d\'efinie comme le \emph{moment cin\'etique} :
\be
	\vec{M} = \sum_{a}\vec{r}_{a}\wedge\vec{p}_{a} \label{EQ:9_3}
\ee
se conserve et poss\`ede le comportement d'additivit\'e.

Au final, nous avons donc sept int\'egrales premi\`eres additives : l'\'energie et les trois composantes des vecteurs quantit\'e et moment cin\'etique.

Comme les rayons vecteurs des points mat\'eriels interviennent dans la d\'efinition du moment cin\'etique, sa valeur d\'epend du choix de l'origine des coordonn\'ees. Prenons l'exemple de deux r\'f\'erentiels tels $\vec{r}_{a} = \vec{r'}_{a} + \vec{a}$, soit $\vec{v}_{a} = \vec{v'}_{a} \Leftrightarrow \vec{p}_{a} = \vec{p'}_{a}$, alors le moment cin\'etique s'\'ecrit :
\bea
	\vec{M} & = & \sum_{a}\vec{r}_{a}\wedge\vec{p}_{a} \nonumber \\
	& = & \sum_{a}\vec{r'}_{a}\wedge\vec{p}_{a} + \sum_{a}\vec{a}\wedge\vec{p}_{a} \nonumber \\
	& = & \sum_{a}\vec{r'}_{a}\wedge\vec{p'}_{a} + \vec{a}\wedge\sum_{a}\vec{p}_{a} \nonumber \\
	\vec{M} & = & \vec{M'} + \vec{a}\wedge\vec{P} \label{EQ:9_4}
\eea
avec $\vec{M'} = \sum_{a}\vec{r'}_{a}\wedge\vec{p'}_{a}$. Par voie de cons\'equence, le moment cin\'etique ne d\'epend du choix de l'origine des coordonn\'ees si et seulement si le syst\`eme est au repose, i.e. $\vec{P} = \vec{0}$.

Consid\'erons d\'esormais deux r\'f\'erentiels $(K)$ et $(K')$ tels $\vec{v}_{a} = \vec{v'}_{a} + \vec{V}$ et \`a l'instant donn\'e, $\vec{r}_{a} = \vec{r'}_{a}$ alors :
\bea
	\vec{M} & = & \sum_{a}m_{a}\vec{r}_{a}\wedge\vec{v}_{a} \nonumber \\
	& = & \sum_{a}m_{a}\vec{r}_{a}\wedge\vec{v'}_{a} + \sum_{a}m_{a}\vec{r}_{a}\wedge\vec{V} \nonumber \\
	& = & \sum_{a}m_{a}\vec{r'}_{a}\wedge\vec{v'}_{a} + \sum_{a}m_{a}\vec{r}_{a}\wedge\vec{V} \nonumber \\
	\vec{M} & = & \vec{M'} + \mu\vec{R}\wedge\vec{V} \label{EQ:9_5}
\eea
avec $\mu$ et $\vec{R}$ d\'efinis par la relation (\ref{EQ:8_3}). \`A l'instar des \'equations (\ref{EQ:8_1}) pour l'\'energie et (\ref{EQ:8_5}) pour l'impulsion, l'\'equation (\ref{EQ:9_5}) est la loi de transformation du moment cin\'etique lors du passage d'un r\'ef\'erentiel \`a un autre. Si dans le r\'ef\'erentiel $(K')$, le syst\`eme est au repos alors la relation ((\ref{EQ:8_2}) qui d\'efinit $\vec{V}$ comme la vitesse du centre d'inertie par rapport à $(K)$ permet d'\'ecrire $\mu\vec{V}$ comme l'impulsion totale et :
\be
	\vec{M} = \vec{M'} + \vec{R}\wedge\vec{P} \label{EQ:9_6}
\ee
où $\vec{M'}$ est le moment cin\'etique propre, soit le moment dans le r\'ef\'erentiel dans lequel le syst\`eme est au repos et $\vec{R}\wedge\vec{P}$ est le moment cin\'etique li\'e au mouvement du syst\`eme dans son ensemble.
Quelques cas particuliers où un champ ext\'erieur est pr\'esent :
\begin{itemize}
	\item le champ est sym\'etrique par rapport \`a un axe. $\delta L = 0$ reste vrai si la rotation du syst\`eme est d\'efinie autour de l'axe de sym\'etrie du champ ext\'erieur. En cons\'equence, la projection du moment cin\'etique sur cet axe se conserve ;
	\item le champ est \`a sym\'etrie centrale. L'invariance de la fonction de Lagrange est vraie dans le cas o\`u la rotation s'effectue autour d'un axe quelconque passant par le centre de sym\'etrie du champ et donc la projection du moment cin\'etique sur cet axe dont l'origine est le centre de sym\'etrie du champ se conserve ;
	\item le champ est uniforme le long d'un axe, disons $z$. En coordonn\'ees cylindriques, nous avons :
	\be
		\vec{r}_{a} = \begin{pmatrix} r_{a}\cos\varphi_{a} \\ r_{a}\sin\varphi_{a} \\ z_{a} \end{pmatrix}
	\ee
	et donc\footnote{$\vec{u}\wedge\vec{v} = (u_{y}v_{z}-u_{z}v_{y}\,;\,u_{z}v_{x}-u_{x}v_{z}\,;\,u_{y}v_{z}-u_{z}v_{y})$} :
	\bea
		M_{z} & = & \sum_{a}{M_{a}}_{z} \nonumber \\
		& = & \sum_{a}\left(r_{a}\cos\varphi_{a}m_{a}r_{a}\cos\varphi_{a}\dot{\varphi_{a}} + r_{a}\sin\varphi_{a}m_{a}r_{a}\sin\varphi_{a}\right) \nonumber \\
		& = & \sum_{a}\left(r_{a}^{2}m_{a}\dot{\varphi_{a}}(\cos^{2}\varphi_{a} + \sin^{2}\varphi_{a})\right) \nonumber \\
		& = & \sum_{a}r_{a}^{2}m_{a}\dot{\varphi_{a}} \label{EQ:9_8}
	\eea
	La relation (\ref{EQ:4_5}) donne en coordonn\'ees cylindriques la fonction de Lagrange :
	\be
		L = \sum_{a}\dfrac{m_{a}}{2}(\dot{r_{a}}^{2} + r_{a}^{2}\dot{\varphi_{a}}^{2} + \dot{z_{a}}^{2}) - U
	\ee
	donc :
	\be
		\sum_{a}\dfrac{\partial L}{\partial \dot{\varphi_{a}}} = \sum_{a}\dfrac{m_{a}}{2}2r_{a}^{2}\dot{\varphi_{a}}
	\ee
	car $\dfrac{\partial U}{\partial \dot{\varphi_{a}}} = 0$, ce qui permet de conclure \`a :
	\be
		M_{z} = \sum_{a}\dfrac{\partial L}{\partial \dot{\varphi_{a}}} \label{EQ:9_7}
	\ee
\end{itemize}

\subsection{Similitude m\'ecanique}

Nous avons d\'ej\`a vu que les \'equations de Lagrange (\ref{EQ:2_6}) sont toujours valables lorsque la fonction de Lagrange est multipli\'ee par une constante. Consid\'erons le cas o\`u l'\'energie potentielle $U$ d'un syst\`eme soit une fonction homog\`ene de degr\'e $k$, i.e. telle que :
\be
	U\left(\alpha\begin{Bmatrix}\vec{r}_{i}\end{Bmatrix}_{1}^{n}\right) = \alpha^{k}U\left(\begin{Bmatrix}\vec{r}_{i}\end{Bmatrix}_{1}^{n}\right) \label{EQ:10_1}
\ee
Effectuons la transformation suivante, si $\vec{r}_{a} \mapsto \vec{r'}_{a} = \alpha\vec{r}_{a}$, alors $\mathrm{t} \mapsto \mathrm{t}' = \beta\mathrm{t}$. En cons\'equence :
\bea
	\forall a \in {1,\ldots\,n}\text{, }\vec{v}_{a} = \dfrac{\mathrm{d}\vec{r}_{a}}{\mathrm{dt}} & \mapsto & \dfrac{\mathrm{d}\vec{r'}_{a}}{\mathrm{dt}'} = \dfrac{\alpha\mathrm{d}\vec{r}_{a}}{\beta\mathrm{dt}} = \dfrac{\alpha}{\beta}\vec{v}_{a} \nonumber \\
	T_{a} = \dfrac{m_{a}}{2}v_{a}^{2} & \mapsto & \dfrac{m_{a}}{2}\left(\dfrac{\alpha}{\beta}\right)^{2}v_{a}^{2} = \left(\dfrac{\alpha}{\beta}\right)^{2}T_{a}^{2} \nonumber \\
	U & \mapsto & \alpha^{k}U \nonumber
\eea
En posant la condition suivante :
\be
	\left(\dfrac{\alpha}{\beta}\right)^{2} = \alpha^{k} \Leftrightarrow \beta = \alpha^{1 - \dfrac{k}{2}}
\ee
la fonction de Lagrange se transforme ainsi :
\bea
	L = \sum_{a}T_{a} - U & \mapsto & \left(\dfrac{\alpha}{\beta}\right)^{2}\sum_{a}T_{a} - \alpha^{k}U \nonumber \\
	& = & \dfrac{\alpha^{2}}{\alpha^{2-k}}\sum_{a}T_{a} - \alpha^{k}U \nonumber \\
	& = & \alpha^{k}\sum_{a}T_{a} - \alpha^{k}U \nonumber \\
	L & \mapsto & \alpha^{k}L
\eea
et comme $\alpha^{k}$ est une constante alors les \'equations de Lagrange (\ref{EQ:2_6}) restent inchang\'ees et pleinement valables. En conclusion, si l'\'energie potentielle d'un syst\`eme est une fonction homog\`ene de degr\'e $k$ des coordonn\'ees alors les \'equations du mouvement admettent des trajectoires g\'eom\'etriques semblables et nous pouvons en d\'eduire certaines caract\'eristiques de proportionnalit\'e.

Puisque la transformation peut aussi s'\'ecrire : $l' = \alpha l$ et $\mathrm{t}' = \beta \mathrm{t}$ alors, nous avons de facto en appliquant la condition ci-dessus :
\be
	\dfrac{\mathrm{t}'}{\mathrm{t}} = \left(\dfrac{l'}{l}\right)^{1-\frac{k}{2}} \label{EQ:10_2}
\ee
et plusieurs relations peuvent en \^etre d\'eduites :
\bea
	\dfrac{l'}{l} & = & \dfrac{l'}{\mathrm{t}'}\dfrac{\mathrm{t}}{l} = \dfrac{l'}{l}\dfrac{\mathrm{t}}{\mathrm{t}'} = \dfrac{l'}{l}\left(\dfrac{l'}{l}\right)^{\frac{k}{2}-1} = \left(\dfrac{l'}{l}\right)^{\frac{k}{2}} \\
	\dfrac{E'}{E} & = & \left(\dfrac{v'}{v}\right)^{2} = \left(\dfrac{l'}{l}\right)^{k} \\
	\dfrac{M'}{M} & = & \dfrac{l'v'}{lv} = \dfrac{l'}{l}\left(\dfrac{l'}{l}\right)^{\frac{k}{2}} = \left(\dfrac{l'}{l}\right)^{1+\frac{k}{2}} \label{EQ:10_3}
\eea
Regardons quelques cas particuliers :
\begin{itemize}
	\item celui des petites oscillations pour lesquelles $k=2$ alors l'\'equation (\ref{EQ:10_2}) dont $\mathrm{t}' = \mathrm{t}$ impliquant que la période d'oscillation ne d\'epend pas de l'amplitude du mouvement ;
	\item celui d'un champ de force homog\`ene, l'\'equation (\ref{EQ:5_8}) a pour conlusion que l'\'energie potentielle est fonction lin\'eaire des coordonn\'ees, soit $k=1$ et l'\'equation (\ref{EQ:10_2}) donne :
		\be
			\dfrac{\mathrm{t}'}{\mathrm{t}} = \left(\dfrac{l'}{l}\right)^{\frac{1}{2}}
		\ee
	\item celui de l'attraction newtonienne, gravitationnelle de deux masses, ou coulombienne entre deux charges, l'\'energie potentielle est une fonction inverse de la distance entre les deux points mat\'eriels, i.e. $k=-1$. Nous obtenons donc :
		\bea
			\dfrac{\mathrm{t}'}{\mathrm{t}} & = & \left(\dfrac{l'}{l}\right)^{\frac{3}{2}} \nonumber \\
			\Leftrightarrow \left(\dfrac{\mathrm{t}'}{\mathrm{t}}\right)^{2} = \left(\dfrac{l'}{l}\right)^{3}
		\eea
		C'est exactement la \emph{troisi\`eme loi de Kepler}.
\end{itemize}

D\'esormais, nous allons supposer que le mouvement du syst\`eme s'effectue dans une r\'egion limit\'ee de l'espace pour en d\'eduire une relation entre la moyenne des \'energies cin\'etique et potentielle, connue comme le \emph{th\'eor\`eme du Viriel}. Puisque l'\'energie cin\'etique $T$ est une fonction quadratique des vitesses, le th\'eor\`eme d'Euler donne, voir l'\'equation (\ref{EQ:6_1_1}) :
\be
	\sum_{a}\dfrac{\partial T}{\partial\vec{v}_{a}} = 2T
\ee
or en utilisant l'\'equation (\ref{EQ:7_2}) :
\be
	\dfrac{\partial T}{\partial\vec{v}_{a}} = \vec{p}_{a}
\ee
et le fait que :
\be
	\dfrac{\mathrm{d}}{\mathrm{dt}}(\sum_{a}\vec{p}_{a}\vec{r}_{a}) = \sum_{a}\vec{\dot{p}}_{a}\vec{r}_{a} + \sum_{a}\vec{p}_{a}\vec{v}_{a}
\ee
nous pouvons \'ecrire :
\be
	2T = \sum_{a}\vec{p}_{a}\vec{v}_{a} = \dfrac{\mathrm{d}}{\mathrm{dt}}(\sum_{a}\vec{p}_{a}\vec{r}_{a}) - \sum_{a}\vec{\dot{p}}_{a}\vec{r}_{a} \label{EQ:10_4}
\ee

La d\'efinition de la moyenne d'une fonction quelconque du temps $f(\mathrm{t})$ est :
\be
	\overline{f} = \lim_{\tau\to\infty}\dfrac{1}{\tau}\int_{0}^{\tau} f(\mathrm{t})\mathrm{dt}
\ee
Dans le cas où $f(\mathrm{t})=\dfrac{\mathrm{d}F(\mathrm{t})}{\mathrm{dt}}$ avec $F(\mathrm{t})$ une fonction born\'ee, i.e. sans valeurs infinies alors :
\be
	\overline{f} = \lim_{\tau\to\infty}\dfrac{1}{\tau}\int_{0}^{\tau} \dfrac{\mathrm{d}F(\mathrm{t})}{\mathrm{dt}}\mathrm{dt} = \lim_{\tau\to\infty}\dfrac{1}{\tau}\int_{0}^{\tau} \mathrm{d}F(\mathrm{t}) = \lim_{\tau\to\infty}\dfrac{F(\tau) - F(0)}{\tau} = 0
\ee
car $F(\tau) - F(0)$ est une quantit\'e finie.

Prenons l'exemple d'un syst\`eme m\'ecanique en mouvement dans une r\'egion finie de l'espace avec des vitesses finies. Par cons\'equent, la quantit\'e $\sum_{a}\vec{p}_{a}\vec{r}_{a}$ est born\'ee et dans le calcul de la valeur moyenne de l'\'energie cin\'etique, le premier terme de l'\'equation (\ref{EQ:10_4}) s'annule.

Les \'equations de Newton (\ref{EQ:5_4}) et (\ref{EQ:7_7}) donne :
\be
	\vec{F}_{a} = -\dfrac{\partial U}{\partial\vec{r}_{a}} = \vec{\dot{p}}_{a}
\ee
ce qui donne finalement :
\be
	2\overline{T} = \sum_{a}\vec{r}_{a}\dfrac{\partial \overline{U}}{\partial\vec{r}_{a}} \label{EQ:10_5}
\ee
Le th\'eor\`eme d'Euler appliqu\'e \`a l'\'energie potentielle, fonction homogn\`ene de degr\'e $k$, donne :
\bea
	\sum_{a}\vec{r}_{a}\dfrac{\partial U}{\partial\vec{r}_{a}} & = & kU \nonumber \\
	\sum_{a}\vec{r}_{a}\dfrac{\partial \overline{U}}{\partial\vec{r}_{a}} & = & k\overline{U}
\eea
et donc \emph{le th\'eor\`eme du Viriel} :
\be
	2\overline{T} = k\overline{U} \label{EQ:10_6}
\ee
Parce que la loi de conservation de l'\'energie s'applique, nous avons $\overline{E} = E$. Or :
\bea
	E & = & T + U \nonumber \\
	\overline{E} & = & \lim_{\tau\to\infty}\dfrac{1}{\tau}\int_{0}^{\tau} T+U\mathrm{dt} \nonumber \\
	\overline{E} & = & \lim_{\tau\to\infty}\dfrac{1}{\tau}\int_{0}^{\tau} T\mathrm{dt} + \lim_{\tau\to\infty}\dfrac{1}{\tau}\int_{0}^{\tau} U\mathrm{dt}\nonumber \\
	\overline{E} & = & \overline{T} + \overline{U}
\eea
et en appliquant le th\'eor\`eme du Viriel (\ref{EQ:10_6}), on obtient :
\be
	E = (\frac{k}{2} +1)\overline{U} = (1 + \frac{2}{k})\overline{T}
\ee
soit finalement :
\be
	\begin{cases}
		\overline{U} = \dfrac{2}{2+k}E \\
		\overline{T} = \dfrac{k}{2+k}E \label{EQ:10_7}
	\end{cases}
\ee

De la m\^eme mani\`ere que pr\'ec\'edemment, quelques cas particuliers :
\begin{itemize}
	\item pour les petites oscillations, $k=2$ donne $\overline{T} = \overline{U} = \frac{1}{2}E$
	\item pour l'int\'eraction newtonienne ou coulombienne, $k=-1$, il y a $2\overline{T} = -\overline{U}$, impliquant $\overline{T} = -E$ et $\overline{U} = E$. En conclusion, le mouvement s'effectue dans une r\'egion finie de l'espace si et seulement si l'\'energie totale du syst\`eme est n\'egative.
\end{itemize}