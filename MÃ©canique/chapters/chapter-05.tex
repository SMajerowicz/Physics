\chapter{Petites oscillations}

\section{Oscillations lin\'eaires libres}

Les \emph{petites oscillations} sont celles faites par un syst\`eme au voisinage de sa position d'\'equilibre stable et nous étudions ici le cas le plus simple, celui où il n'y a qu'un seul degr\'e de libert\'e. Un \'equilibre stable intervient quand l'\'energie potentielle $U(q)$ est minimale. D\'efinissons que $U$ soit minimale en $q = q_{0}$. L'\'equation (\ref{EQ:5_4}) montre qu'un \'ecart par rapport \`a la position $q_{0}$ engendre une force \'egale \`a $-\frac{\mathrm{d}U}{\mathrm{d}q}$ qui ram\`ene le syst\`eme \`a sa position d'\'equilibre stable. R\'ealisons un d\'eveloppement de Taylor de $U(q)$ jusqu'au second ordre :
\be
	U(q) = U(q_{0}) + U'(q_{0})(q - q_{0}) + \dfrac{U''(q_{0})}{2}(q - q_{0})^{2} \Rightarrow U(q) - U(q_{0}) = \dfrac{k}{2}(q - q_{0})^{2}
\ee
car $U'(q_{0}) = 0$ puisque $q_{0}$ est la position d'\'equilibre et nous avons pos\'e $k = U''(q_{0})$. En d\'efinissant :
\be
	x = q - q_{0} \label{EQ:21_1}
\ee
qui repr\'esente l'\'ecart par rapport \`a la position d'\'equilibre, nous avons, en posant $U(q_{0}) = 0$ :
\be
	U(x) = \dfrac{kx^{2}}{2} \label{EQ:21_2}
\ee
Au regard des relations (\ref{EQ:4_1}) et (\ref{EQ:5_5}), l'\'energie cin\'etique pour une particule et dans le cas d'un unique degr\'e de libert\'e, s'\'ecrit $T = \frac{1}{2}a(q)\dot{q}^{2}$. Or dans le cadre des petites oscillations, $x \approx q$. De la m\^eme mani\`ere, $a(q) \approx q(a_{0})$, quantit\'e qui peut s'identifier \`a la masse de la particule si $x$ est une coordonn\'ee cart\'esienne. L'\'energie cin\'etique peut donc se formuler comme $T = \frac{1}{2}m\dot{x}^{2}$. Par cons\'equent, la fonction de Lagrange pour un syst\`eme r\'ealisant de petites oscillations, d\'enomm\'e aussi \emph{oscillateur lin\'eaire} est :
\be
	L = \dfrac{m\dot{x}^{2}}{2} - \dfrac{kx^{2}}{2} \label{EQ:21_3}
\ee
Dans notre cas \'epur\'e, l'\'equation du mouvement s'obtient \`a partir de la relation (\ref{EQ:5_2}) :
\be
	\dfrac{\mathrm{d}}{\mathrm{d}t}\left(\dfrac{\partial L}{\partial \dot{x}}\right) = \dfrac{\partial L}{\partial x} \Leftrightarrow \dfrac{\mathrm{d}m\dot{x}}{\mathrm{d}t} = -kx \Leftrightarrow m\ddot{x} + kx = 0 \label{EQ:21_4}
\ee
\'Equation qui peut \'egalement s'\'ecrire :
\be
	\ddot{x} + \omega^{2}x = 0 \label{EQ:21_5}
\ee
en posant la quantit\'e :
\be
	\omega = \sqrt{\dfrac{k}{m}} \label{EQ:21_6}
\ee
qui est appel\'ee \emph{fr\'equence angulaire}. La quantit\'e $\omega$ ne d\'epend pas des conditions initiales du mouvement mais uniquement des propri\'et\'es m\'ecaniques du syst\`eme, c'est une constante fondamentale des oscillations. Ceci n'est valable que dans le cas qui nous concerne, celui des petites oscillations, ou encore quand $U \propto x^{2}$. Pour les cas g\'en\'eraux, nous pouvons nous reporter \`a l'\'equation (\ref{EQ:11_EX2A_1}) qui d\'efinit la p\'eriode d'oscillations $T$ pour $U = A\lvert x^{n} \rvert$, sachant que $\omega = 2\pi/T$.

L'\'equation (\ref{EQ:21_5} admet deux solutions ind\'ependantes en $\cos(\omega t)$ et en $\sin(\omega t)$. Donc la solution g\'en\'erale s'\'ecrit :
\be
	x = c_{1}\cos(\omega t) + c_{2}\sin(\omega t) \label{EQ:21_7}
\ee
ou encore :
\be
	x = a\cos(\omega t + \alpha) \label{EQ:21_8}
\ee
o\`u $a$ est l'\emph{amplitude} et $\alpha$ la valeur initiale de la \emph{phase}, qui d\'epend de l'origine prise pour $t$.Comme $\cos(\omega t + \alpha) = \cos(\omega t)\cos\alpha - \sin(\omega t)\sin\alpha$, alors nous pouvons en d\'eduire, \`a partir de la relation (\ref{EQ:21_7}), $c_{1} = a\cos\alpha$ et $c_{2} = -a\sin\alpha$. De plus :
\be
	\begin{cases}
		c_{1}^{2} + c_{2}^{2} = a^{2}(\cos^{2}\alpha + \sin^{2}\alpha) \Rightarrow a = \sqrt{c_{1}^{2} + c_{2}^{2}} \\
		\dfrac{\sin\alpha}{\cos\alpha} = \tan\alpha = -\dfrac{c_{2}}{c_{1}} \label{EQ:21_9}
	\end{cases}
\ee
L'\'energie totale du syst\`eme vaut, en y ajoutant la formule (\ref{EQ:21_8}) :
\bea
	E & = & T + U = \dfrac{m\dot{x}^{2}}{2} + \dfrac{kx^{2}}{2} = \dfrac{m}{2}(\dot{x}^{2} + \omega^{2}x^{2}) \nonumber \\
	& = & \dfrac{m}{2}(a^{2}\omega^{2}\sin^{2}(\omega t + \alpha) + a^{2}\omega^{2}\cos^{2}(\omega t + \alpha)) = \dfrac{m\omega^{2}a^{2}}{2} \label{EQ:21_10}
\eea
Il est souvent ais\'e de passer dans le domaine complexe pour simplifier les op\'erations math\'ematiques et de consid\'erer la solution (\ref{EQ:21_8}) comme :
\be
	x = \Re{\{Ae^{i\omega t}\}} \label{EQ:21_11}
\ee
avec l'\emph{amplitude complexe} s'exprimant comme :
\be
	A = ae^{i\alpha} \label{EQ:21_12}
\ee
dont le module est l'amplitude ordinaire et l'argument la phase initiale.

\section{Oscillations forc\'ees}

Les \emph{oscillations forc\'ees} sont les oscillations d'un syst\`eme soumis \`a un champ ext\'erieur variable. En restant dans l'hypoth\`ese des petites oscillations, cela implique que le champ ext\'erieur est suffisamment faible pour provoquer des d\'eplacements faibles \'egalement. De plus, ici, nous restons dans le cadre d'un unique degr\'e de libert\'e.

\subsection{Cas g\'en\'eral}

L'\'energie potentielle totale du syst\`eme s'\'ecrit avec deux termes, \`a savoir l'\'energie potentielle propre en $\frac{1}{2}kx^{2}$ et celle due \`a l'action ext\'erieur $U_{e}(x,t)$. Au premire ordre, cette derni\`ere s'\'ecrit :
\be
	U_{e}(x,t) = U_{e}(0,t) + x\left(\dfrac{\partial U_{e}}{\partial x}\right)(0,t)
\ee
o\`u $U_{e}(0,t)$ est une fonction ne d\'ependant que du temps et de facto, elle est le r\'esultat d'une d\'eriv\'ee totale par rapport au temps. Dans cette partie est n\'eglig\'ee dans l'\'equation de Lagrange, voir (\ref{EQ:2_8}). De plus, la formule (\ref{EQ:5_8}) permet de d\'eduire que $-\frac{\partial U_{e}}{\partial x}$ est la force ext\'erieure qui s'exerce sur le syst\`eme. Elle est fonction du temps et nous la d\'esignons $F(t)$. Par cons\'equent, $U_{e}(x,t) = -xF(t)$ et la fonction de Lagrange peut se formuler ainsi :
\be
	L = \dfrac{m}{2}\dot{x}^{2} - \left(\frac{k}{2}x^{2} - xF(t)\right) \label{EQ:22.1}
\ee
L'\'equation du mouvement provenant de la formule (\ref{EQ:5_2}) permet d'en d\'eduite :
\bea
	\dfrac{\mathrm{d}}{\mathrm{d}t}\left(\dfrac{\partial L}{\partial \dot{x}}\right) & = & \dfrac{\partial L}{\partial x} \Leftrightarrow \dfrac{\mathrm{d}m\dot{x}}{\mathrm{d}t} = -kx + F(t) + x\dfrac{\partial F(t)}{\partial x} \nonumber \\
	& \Leftrightarrow & m\ddot{x} + kx = F(t) \Leftrightarrow \ddot{x} + \omega^{2}x = \dfrac{F(t)}{m} \label{EQ:22_2}
\eea
avec $\omega = \sqrt{\frac{k}{m}}$ la fr\'equence des oscillations. L'\'equation (\ref{EQ:22_2}) est une \'equation diff\'erentielle du second ordre avec des c{\oe}fficients constants et un second membre. La solution g\'en\'erale est de la forme $x = x_{0} + x_{1}$ telle que :
\begin{itemize}
	\item $x_{0}$ est la solution g\'en\'erale de l'\'equation sans second membre qui repr\'esente les oscillations libres d\'etermin\'ee aux \'equations (\ref{EQ:21_7}) et (\ref{EQ:21_8})
	\item $x_{1}$ est une int\'egrale particuli\`ere de l'\'equation avec le second membre
\end{itemize}

\subsection{Cas particulier d'une force ext\'erieure p\'eriodique}

Consid\'erons que la force ext\'erieure s'exprime telle que :
\be
	F(t) = f\cos(\gamma t + \beta) \label{EQ:22_3}
\ee
Pour cette hypoth\`ese, l'int\'egrale particuli\`ere de l'\'equation (\ref{EQ:22_2}) peut \^etre $x_{1} = b\cos(\gamma t + \beta)$, ce qui nous donne :
\bea
	& & -b\gamma^{2}\cos(\gamma t + \beta) + \omega^{2}b\cos(\gamma t + \beta) = \dfrac{f}{m}\cos(\gamma t + \beta) \Leftrightarrow b(\omega^{2} - \gamma^{2}) = \dfrac{f}{m} \nonumber \\
	& \Leftrightarrow & b = \dfrac{f}{m(\omega^{2} - \gamma^{2})}
\eea
Le mouvement se compose donc en ajoutant la solution g\'en\'erale (\ref{EQ:21_8}) :
\be
	x = a\cos(\omega t + \alpha) + \frac{f}{m(\omega^{2} - \gamma^{2})}\cos(\gamma t + \beta) \label{EQ:22_4}
\ee
o\`u les quantit\'es $a$ et $\alpha$ sont d\'eduites des conditions initiales. Le mouvement est ainsi compos\'e de la somme de deux oscillations, la premi\`ere avec la fr\'equence propre du syst\`eme et la seconde avec la fr\'equence de la force ext\'erieure. La formule (\ref{EQ:22_4}) est ind\'etermin\'ee dans le cas o\`u $\omega = \gamma$, il y a alors \emph{r\'esonnance}.

\subsection{Solution r\'eelle lors de la r\'esonnance avec une force ext\'erieure}

Partons de la formule (\ref{EQ:22_4}) pour la reformuler ainsi :
\be
	x = a\cos(\omega t + \alpha) + \frac{f}{m(\omega^{2} - \gamma^{2})}(\cos(\gamma t + \beta) - \cos(\omega t + \beta))
\ee
Ainsi lorsque $\gamma \rightarrow \omega$, alors nous avons une ind\'etermination de type $0/0$ pour le second terme. Appliquons alors la r\`egle de L'Hospital\footnote{Si $f$ et $g$ sont deux fonctions d\'efinies sur $[a;b[$, d\'erivables en $a$ et telles que $f(a) = g(a) = 0$ et $g'(a) \neq 0$ alors $\lim_{x \rightarrow a^{+}}\dfrac{f(x)}{g(x)} = \dfrac{f'(a)}{g'(a)}$} avec :
\be
	\begin{cases}
		f : \omega \mapsto \cos(\gamma t + \beta) - \cos(\omega t + \beta) \\
		g : \omega \mapsto w^{2} - \gamma^{2}
	\end{cases}
\ee
Cela nous donne :
\be
	\lim_{\gamma \rightarrow \omega}\dfrac{f(\gamma)}{g(\gamma)} = \dfrac{f'(\gamma = \omega)}{g'(\gamma = \omega)} = \dfrac{-t\sin(\omega t + \beta)}{-2\omega}
\ee
et nous permet d'\'ecrire :
\be
	x = a\cos(\omega t + \alpha) + \frac{f}{2m\omega}t\sin(\omega t + \beta) \label{EQ:22_5}
\ee
Dans le cas de la r\'esonnance, l'amplitude augmente lin\'eairement avec le temps tant que nous restons dans le cadre des petites oscillations.

\subsection{Solution complexe lors de la r\'esonnance avec une force ext\'erieure}

Au voisinage de la r\'esonnance, d\'efinissons la fr\'equence $\gamma = \omega + \epsilon$ avec $\epsilon$ petit devant $\omega$. Dans le domaine complexe, il est possible de g\'en\'eraliser l'expression (\ref{EQ:22_4}) telle que :
\be
	x = Ae^{i\omega t} + Be^{i(\omega + \epsilon)t} = \left(A + Be^{i\epsilon t}\right)e^{i\omega t} \label{EQ:22_6}
\ee
Comme $\epsilon \ll \omega$, l'expression $A + Be^{i\epsilon t}$ varie peu lors d'une p\'eriode $2pi/\omega$. Au voisinage de la r\'esonnance, le mouvement se constitue de petites oscillations \`a amplitude variable. L'amplitude reste une grandeure r\'eelle aussi nous d\'efinissons $C = \lvert A + Be^{i\epsilon t} \lvert$. La relation (\ref{EQ:21_12}) permet d'\'ecrire $A = ae^{i\alpha}$ et $B = be^{i\beta}$, et ainsi l'amplitude $C$ devient :
\bea
	C & = & \lvert ae^{i\alpha} + be^{i(\epsilon t + \beta)} \rvert \nonumber \\
	\Leftrightarrow C^{2} & = & \lvert ae^{i\alpha} \rvert^{2} + \lvert be^{i(\epsilon t + \beta)} \rvert^{2} + 2\lvert ae^{i\alpha} \rvert \cdot \lvert be^{i(\epsilon t + \beta)} \rvert \cos(\epsilon t + \beta - \alpha) \nonumber \\
	C^{2} & = & a^{2} + b^{2} + 2ab\cos(\epsilon t + \beta - \alpha) \label{EQ:22_7}
\eea
Les valeurs extremum de $C$ sont :
\begin{itemize}
	\item $C_{max}^{2} = a^{2} + b^{2} + 2ab \Rightarrow C_{max} = a + b$
	\item $C_{min}^{2} = a^{2} + b^{2} - 2ab \Rightarrow C_{min} = \lvert a - b \rvert$
\end{itemize}
Par cons\'equent, l'amplitude $C$ oscille p\'eriodiquement entre les deux valeurs pr\'ec\'edentes avec une fr\'equence $\epsilon$, ph\'enom\`ene appel\'e \emph{battements}.

Ensuite, nous pouvons remarquer que l'\'equation (\ref{EQ:22_2}) peut s\'ecrire :
\bea
	\ddot{x} + i\omega\dot{x} - i\omega\dot{x} - i\omega\cdot i\omega x & = & \dfrac{F(t)}{m} \Leftrightarrow \dfrac{\mathrm{d}}{\mathrm{d}t}(\dot{x} + i\omega x) - i\omega(\dot{x} + i\omega x) = \dfrac{F(t)}{m} \nonumber \\
	\Rightarrow \dfrac{\mathrm{d}\xi}{\mathrm{d}t} - i\omega\xi = \dfrac{F(t)}{m} \label{EQ:22_8}
\eea
en d\'efinissant la quantit\'e $\xi$ telle que :
\be
	\xi = \dot{x} + i\omega x \label{EQ:22_9}
\ee
L'\'equation (\ref{EQ:22_8}) n'est plus du second ordre.