\chapter{Probl\`emes sur les lois de conservation}

\section{Changement de direction d'une particule}

\begin{figure}[htb!]
	\begin{center}
		\begin{picture}(150,100)(0,0)
			%axis
			\linethickness{0.05mm}
			\multiput(75,0)(0,10){10}{\line(0,1){8}}
			\multiput(20,50)(7,0){6}{\line(1,0){5}}
			\multiput(100,50)(7,0){6}{\line(1,0){5}}
			%particles
			\put(20,50){\color{black}\circle*{5}}\put(6,47){$m$}
			\put(100,50){\color{black}\circle*{5}}
			%velocities
			\put(20,50){\color{black}\vector(3,1){40}}\put(35,65){$\vec{v}_{1}$}
			\put(100,50){\color{black}\vector(2,1){40}}\put(110,65){$\vec{v}_{2}$}
			%angles
			\linethickness{0.05mm}
			\qbezier(40,50),(40,52),(38,55)
			\put(45,52){$\theta_{1}$}
			\qbezier(115,50),(115,52),(113,55)
			\put(120,52){$\theta_{2}$}
			%potential energies
			\put(25,10){$U_{1}(t_{1})$}
			\put(110,10){$U_{2}(t_{2})$}
		\end{picture}
		\caption{Trajectoire de la particule}\label{FIG:2_1}
	\end{center}
\end{figure}