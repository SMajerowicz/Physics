\chapter{Probl\`emes sur les lois de conservation}

\section{Changement de direction d'une particule}

\begin{figure}[htb!]
	\begin{center}
		\begin{picture}(150,100)(0,0)
			%axis
			\linethickness{0.05mm}
			\multiput(75,0)(0,10){10}{\line(0,1){8}}
			\multiput(20,50)(7,0){6}{\line(1,0){5}}
			\multiput(100,50)(7,0){6}{\line(1,0){5}}
			%particles
			\put(20,50){\color{black}\circle*{5}}\put(6,47){$m$}
			\put(100,50){\color{black}\circle*{5}}
			%velocities
			\put(20,50){\color{black}\vector(3,1){40}}\put(35,65){$\vec{v}_{1}$}
			\put(100,50){\color{black}\vector(2,1){40}}\put(110,65){$\vec{v}_{2}$}
			%angles
			\linethickness{0.05mm}
			\qbezier(40,50),(40,52),(38,55)
			\put(45,52){$\theta_{1}$}
			\qbezier(115,50),(115,52),(113,55)
			\put(120,52){$\theta_{2}$}
			%potential energies
			\put(25,10){$U_{1}(t_{1})$}
			\put(110,10){$U_{2}(t_{2})$}
		\end{picture}
		\caption{Trajectoire de la particule}\label{FIG:2_1}
	\end{center}
\end{figure}

Par application de la conservation de l'\'energie, nous pouvons \'ecrire :
\bea
	E_{1} & = & E_{2} \nonumber \\
	\dfrac{m\vec{v_{1}}^{\,2}}{2} + U_{1} & = & \dfrac{m\vec{v_{2}}^{\,2}}{2} + U_{2}
\eea
Alors que l'application de la conservation de mouvement suivant l'axe vertical, puisque l'\'energie potentielle est constante suivant cette axe, permet d'\'ecrire :
\bea
	mv_{1}\sin(\theta_{1}) & = & mv_{2}\sin(\theta_{2}) \nonumber \\
	\Leftrightarrow v_{2} & = & v_{1}\dfrac{\sin(\theta_{1})}{\sin(\theta_{2})}
\eea
En reprenant la conservation de l'\'energie, nous avons ainsi :
\bea
	\dfrac{m\vec{v_{1}}^{\,2}}{2} + U_{1} & = & \dfrac{m\vec{v_{2}}^{\,2}}{2} + U_{2} \nonumber \\
	\Leftrightarrow \dfrac{mv_{1}^{2}}{2} + U_{1} & = & \dfrac{mv_{2}^{2}}{2} + U_{2} \nonumber \\
	\Leftrightarrow \dfrac{mv_{1}^{2}}{2} + U_{1} & = & \dfrac{mv_{1}^{2}\sin^{2}(\theta_{1})}{2\sin^{2}(\theta_{2})} + U_{2} \nonumber \\
	\Leftrightarrow \dfrac{mv_{1}^{2}}{2}\left(\dfrac{\sin^{2}(\theta_{1})}{\sin^{2}(\theta_{2})} - 1\right) & = & U_{1} - U_{2} \nonumber \\
	\Leftrightarrow \dfrac{\sin(\theta_{1})}{\sin(\theta_{2})} & = & \sqrt{1+2\dfrac{U_{1} - U_{2}}{mv_{1}^{2}}}
\eea

\section{Transformation de l'action d'un r\'ef\'erentiel galil\'een \`a un autre}

Le passage d'un r\'ef\'erentiel galil\'een \`a un autre s'op\`ere tel que, voir \'equation (\ref{EQ:3_3}):
\bea
	\vec{r} = \vec{r'} + \vec{V}\mathrm{t} \nonumber \\
	\Leftrightarrow \vec{v} = \vec{v'} + \vec{V} \nonumber
\eea
La fonction de Lagrange devient :
\bea
	L & = & T - U \nonumber \\
	& = & \sum_{a=1}^{s}\dfrac{m_{a}\vec{v}_{a}^{\,2}}{2} - U(\begin{Bmatrix}\vec{r}_{a}\end{Bmatrix}^{s}_{1}) \nonumber \\
	& = & \sum_{a}\dfrac{m_{a}\vec{v'}_{a}^{\,2}}{2} + \sum_{a}\dfrac{m_{a}\vec{V}^{\,2}}{2} + \sum_{a}m_{a}\vec{v'}_{a}\cdot\vec{V} - U \nonumber \\
	& = & \sum_{a}\dfrac{m_{a}\vec{v'}_{a}^{\,2}}{2} - U + \vec{V}\cdot\sum_{a}m_{a}\vec{v'}_{a} + \vec{V}^{\,2}\sum_{a}\dfrac{m_{a}}{2}
\eea
ce qui peut aussi s'\'ecrire :
\be
	L = L' + \vec{V}\cdot\vec{P'} + \dfrac{1}{2}\mu\vec{V}^{\,2}
\ee
Ce qui implique pour l'action du syst\`eme : 
\be
	S = \int L\mathrm{dt} = \int L'\mathrm{dt} + \vec{V}\cdot\int\vec{P'}\mathrm{dt} + \dfrac{1}{2}\mu\vec{V}^{\,2}\int \mathrm{dt}
\ee
or l'\'equation (\ref{EQ:8_3}) donne pour la d\'efinition du centre d'inertie :
\be
	\vec{R} = \dfrac{\sum_{a}m_{a}\vec{r}_{a}}{\sum_{a}m_{a}}
\ee
En d\'erivant par le temps l'expression pr\'ec\'edente :
\be
	\mu\dfrac{\vec{R}}{\mathrm{dt}} = \sum_{a}m_{a}\dfrac{\mathrm{d}\vec{r}_{a}}{\mathrm{dt}} = \sum_{a}m_{a}\vec{v}_{a} = \vec{P}
\ee
Donc l'action devient :
\bea
	S & = & S' + \vec{V}\cdot\int\mu\dfrac{\vec{R'}}{\mathrm{dt}}\mathrm{dt} + \dfrac{1}{2}\mu\vec{V}^{\,2}\mathrm{t} \nonumber \\
	& = & S' + \mu\vec{V}\cdot\vec{R'} + \dfrac{1}{2}\mu\vec{V}^{\,2}\mathrm{t}
\eea