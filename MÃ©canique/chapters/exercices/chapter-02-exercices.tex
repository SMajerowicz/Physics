\chapter{Probl\`emes sur les lois de conservation}

\section{Changement de direction d'une particule}

\begin{figure}[htb!]
	\begin{center}
		\begin{picture}(150,100)(0,0)
			%axis
			\linethickness{0.05mm}
			\multiput(75,0)(0,10){10}{\line(0,1){8}}
			\multiput(20,50)(7,0){6}{\line(1,0){5}}
			\multiput(100,50)(7,0){6}{\line(1,0){5}}
			%particles
			\put(20,50){\color{black}\circle*{5}}\put(6,47){$m$}
			\put(100,50){\color{black}\circle*{5}}
			%velocities
			\put(20,50){\color{black}\vector(3,1){40}}\put(35,65){$\vec{v}_{1}$}
			\put(100,50){\color{black}\vector(2,1){40}}\put(110,65){$\vec{v}_{2}$}
			%angles
			\linethickness{0.05mm}
			\qbezier(40,50),(40,52),(38,55)
			\put(45,52){$\theta_{1}$}
			\qbezier(115,50),(115,52),(113,55)
			\put(120,52){$\theta_{2}$}
			%potential energies
			\put(25,10){$U_{1}(t_{1})$}
			\put(110,10){$U_{2}(t_{2})$}
		\end{picture}
		\caption{Trajectoire de la particule}\label{FIG:2_1}
	\end{center}
\end{figure}

Par application de la conservation de l'\'energie, nous pouvons \'ecrire :
\bea
	E_{1} & = & E_{2} \nonumber \\
	\dfrac{m\vec{v_{1}}^{\,2}}{2} + U_{1} & = & \dfrac{m\vec{v_{2}}^{\,2}}{2} + U_{2}
\eea
Alors que l'application de la conservation de mouvement suivant l'axe vertical, puisque l'\'energie potentielle est constante suivant cette axe, permet d'\'ecrire :
\bea
	mv_{1}\sin(\theta_{1}) & = & mv_{2}\sin(\theta_{2}) \nonumber \\
	\Leftrightarrow v_{2} & = & v_{1}\dfrac{\sin(\theta_{1})}{\sin(\theta_{2})}
\eea
En reprenant la conservation de l'\'energie, nous avons ainsi :
\bea
	\dfrac{m\vec{v_{1}}^{\,2}}{2} + U_{1} & = & \dfrac{m\vec{v_{2}}^{\,2}}{2} + U_{2} \nonumber \\
	\Leftrightarrow \dfrac{mv_{1}^{2}}{2} + U_{1} & = & \dfrac{mv_{2}^{2}}{2} + U_{2} \nonumber \\
	\Leftrightarrow \dfrac{mv_{1}^{2}}{2} + U_{1} & = & \dfrac{mv_{1}^{2}\sin^{2}(\theta_{1})}{2\sin^{2}(\theta_{2})} + U_{2} \nonumber \\
	\Leftrightarrow \dfrac{mv_{1}^{2}}{2}\left(\dfrac{\sin^{2}(\theta_{1})}{\sin^{2}(\theta_{2})} - 1\right) & = & U_{1} - U_{2} \nonumber \\
	\Leftrightarrow \dfrac{\sin(\theta_{1})}{\sin(\theta_{2})} & = & \sqrt{1+2\dfrac{U_{1} - U_{2}}{mv_{1}^{2}}}
\eea