\chapter{Probl\`emes d'\'equations du mouvement}

L'objectif est de trouver la fonction de Lagrange dans les cas suivants plac\'es dans un champ de pesanteur constant $g$.

\section{Probl\`eme du pendule double oscillant}

\begin{figure}[htb!]
	\begin{center}
		\begin{picture}(150,200)(0,0)
			%axis
			\linethickness{0.05mm}
			\multiput(0,200)(10,0){15}{\line(1,0){8}}\put(150,195){$x$}
			\multiput(0,0)(0,10){20}{\line(0,1){8}}\put(-5,-10){$y$}
			\multiput(75,90)(0,-10){7}{\line(0,-1){8}}
			%socle
			\put(0,200){\color{black}\circle*{5}}
			%arms
			\linethickness{0.5mm}
			\put(0,200){\line(10,-15){75}}
			\put(40,150){$l_{1}$}
			\put(75,90){\color{black}\circle*{10}}\put(80,95){$m_{1}$}
			\put(75,90){\line(15,-10){50}}
			\put(105,75){$l_{2}$}
			\put(125,55){\color{black}\circle*{10}}\put(130,60){$m_{2}$}
			%angles
			\linethickness{0.05mm}
			\qbezier(0,180),(5,182),(10,185)
			\put(3,170){$\varphi_{1}$}
			\qbezier(75,70),(82,70),(93,80)
			\put(83,60){$\varphi_{2}$}
		\end{picture}
		\caption{Pendule double oscillant}\label{FIG:1_1}
	\end{center}
\end{figure}

La fonction de Lagrange du syst\`eme $L = L_{1} + L_{2}$. Calculons d'abord $L_{1} = T_{1} - U_{1}$ avec :
\be
	\begin{cases}
		T_{1} = \dfrac{m_{1}}{2}\left(l_{1}\dot{\varphi_{1}}\right)^{2} \\
		U_{1} = -m_{1}gl_{1}\cos(\varphi_{1})
	\end{cases}
\ee
Pour $L_{2}$, partons des coordonn\'ees cart\'esiennes de $m_{2}$, soit :
\be
	\begin{cases}
		x_{2} = l_{1}\sin(\varphi_{1}) + l_{2}\sin(\varphi_{2}) \\
		y_{2} = l_{1}\cos(\varphi_{1}) + l_{2}\cos(\varphi_{2}) \\
	\end{cases}
\ee
or il est clair que :
\be
\dfrac{\mathrm{d}f(u(\mathrm{t}))}{\mathrm{dt}} = \dfrac{\mathrm{d}f(u(\mathrm{t}))}{\mathrm{d}u(\mathrm{t})}\dfrac{\mathrm{d}u(\mathrm{t})}{\mathrm{dt}}
\ee
Cela implique donc :
\be
	\begin{cases}
		\dot{x_{2}} = \dfrac{\mathrm{d}x}{\mathrm{dt}} = l_{1}\cos(\varphi_{1})\dot{\varphi_{1}} + l_{2}\cos(\varphi_{2})\dot{\varphi_{2}} \\[0.25cm]
		\dot{y_{2}} = \dfrac{\mathrm{d}y}{\mathrm{dt}} = -l_{2}\sin(\varphi_{1})\dot{\varphi_{1}} - l_{2}\sin(\varphi_{2})\dot{\varphi_{2}} \\
	\end{cases}
\ee
L'\'energie cin\'etique $T_{2}$ s'\'ecrit alors :
\bea
	T_{2} & = & \dfrac{m_{2}}{2}(\dot{x_{2}}^{2} + \dot{y_{2}}^{2}) \nonumber \\
	& = & \dfrac{m_{2}}{2}(l_{1}^{2}\cos^{2}(\varphi_{1})\dot{\varphi_{1}}^{2} + 2l_{1}l_{2}\cos(\varphi_{1})\cos(\varphi_{2})\dot{\varphi_{1}}\dot{\varphi_{2}} + l_{2}^{2}\cos^{2}(\varphi_{2})\dot{\varphi_{2}}^{2} \nonumber \\
	& + & l_{1}^{2}\sin^{2}(\varphi_{1})\dot{\varphi_{1}}^{2} + 2l_{1}l_{2}\sin(\varphi_{1})\sin(\varphi_{2})\dot{\varphi_{1}}\dot{\varphi_{2}} + l_{2}^{2}\sin^{2}(\varphi_{2})\dot{\varphi_{2}}^{2}) \nonumber
\eea
or, nous savons que :
\be
	\begin{cases}
		\cos^2(\alpha) + \sin^{2}(\alpha) = 1 \\
		\cos(\alpha - \beta) = \cos(\alpha)\cos(\beta) + \sin(\alpha)\sin(\beta)
	\end{cases}
\ee
ce qui conclut pour l'\'energie cin\'etique \`a :
\be
	T_{2} = \dfrac{m_{2}}{2}(l_{1}^{2}\dot{\varphi_{1}}^{2} + l_{2}^{2}\dot{\varphi_{2}}^{2} + 2l_{1}l_{2}\cos(\varphi_{1} - \varphi_{2})\dot{\varphi_{1}}\dot{\varphi_{2}})
\ee
De mani\`ere \'evidente, l'\'energie potentielle est :
\be
	U_{2} = -m_{2}g(l_{1}\cos(\varphi_{1}) + l_{2}\cos(\varphi_{2}))
\ee
En rassemblant, la fonction de Lagrange du syst\`eme m\'ecanique total :
\bea
	L & = & T_{1} + T_{2} - U_{1} - U{_2} \nonumber \\
	& = & \dfrac{m_{1}}{2}\left(l_{1}\dot{\varphi_{1}}\right)^{2} + \dfrac{m_{2}}{2}(l_{1}^{2}\dot{\varphi_{1}}^{2} + l_{2}^{2}\dot{\varphi_{2}}^{2} + 2l_{1}l_{2}\cos(\varphi_{1} - \varphi_{2})\dot{\varphi_{1}}\dot{\varphi_{2}}) \nonumber \\
	& + & m_{1}gl_{1}\cos(\varphi_{1}) + m_{2}g(l_{1}\cos(\varphi_{1}) + l_{2}\cos(\varphi_{2})) \nonumber \\
	& = & \dfrac{m_{1}+m_{2}}{2}l_{1}^{2}\dot{\varphi_{1}}^{2} + \dfrac{m_{2}}{2}l_{2}^{2}\dot{\varphi_{2}}^{2} + m_{2}l_{1}l_{2}\cos(\varphi_{1} - \varphi_{2})\dot{\varphi_{1}}\dot{\varphi_{2}} \nonumber \\
	& + & (m_{1}+m_{2})gl_{1}\cos(\varphi_{1}) + m_{2}gl_{2}\cos(\varphi_{2})
\eea

\section{Probl\`eme du pendule plan}

\begin{figure}[htb!]
	\begin{center}
		\begin{picture}(150,150)(0,0)
			%axis
			\linethickness{0.05mm}
			\multiput(0,150)(10,0){15}{\line(1,0){8}}\put(150,145){$x$}
			\multiput(0,0)(0,10){15}{\line(0,1){8}}\put(-5,-10){$y$}
			\multiput(50,150)(0,-10){10}{\line(0,-1){8}}
			%arms
			\linethickness{0.5mm}
			\put(50,150){\color{black}\circle*{10}}\put(60,140){$m_{1}$}
			\put(50,150){\line(10,-15){75}}
			\put(90,100){$l$}
			\put(125,40){\color{black}\circle*{10}}\put(130,45){$m_{2}$}
			%angles
			\linethickness{0.05mm}
			\qbezier(50,130),(55,132),(60,135)
			\put(53,120){$\varphi$}
		\end{picture}
		\caption{Pendule plan}\label{FIG:1_2}
	\end{center}
\end{figure}

En suivant la m\^eme m\'ethodologie que dans l'exercice pr\'ec\'edent, nous avons avec $x$ l'abscisse de $m_{1}$ :
\be
	\begin{cases}
		T_{1} = \dfrac{m_{1}}{2}\dot{x}^{2} \\
		U_{1} = 0
	\end{cases}
\ee

\section{Probl\`eme du pendule plan sur un cercle}

\subsection{Cas fr\'equence constante}

\subsection{Cas oscillations horizontales}

\subsection{Cas oscillations verticales}

\section{Probl\`eme du pendule rotatif à trois masses}