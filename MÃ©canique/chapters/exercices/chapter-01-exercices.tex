\chapter{Probl\`emes d'\'equations du mouvement}

L'objectif est de trouver la fonction de Lagrange dans les cas suivants plac\'es dans un champ de pesanteur constant $g$.

\section{Probl\`eme 1}

\begin{figure}[htb!]
	\begin{center}
		\begin{picture}(150,200)(0,0)
			%axis
			\linethickness{0.05mm}
			\multiput(0,200)(10,0){15}{\line(1,0){8}}\put(150,195){$x$}
			\multiput(0,0)(0,10){20}{\line(0,1){8}}\put(-5,-10){$y$}
			\multiput(75,90)(0,10){7}{\line(0,-1){8}}
			%socle
			\put(0,200){\color{black}\circle*{5}}
			%arms
			\linethickness{0.5mm}
			\put(0,200){\line(10,-15){75}}\put(75,90){\color{black}\circle*{10}}\put(80,95){$m_{1}$}
			\put(75,90){\line(15,-10){50}}\put(125,55){\color{black}\circle*{10}}\put(130,60){$m_{2}$}
		\end{picture}
		\caption{Pendule double oscillant}\label{FIG:1_1}
	\end{center}
\end{figure}
