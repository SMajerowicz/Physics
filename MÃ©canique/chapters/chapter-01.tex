\chapter{\'Equations du mouvement}
\section{Coordonn\'ees g\'en\'eralis\'ees}

Pour un point mat\'eriel, nous avons en coordonn\'ees cart\'esiennes :
\begin{itemize}
\item le rayon vecteur tel que :
	\be
		\vec{r} = \begin{pmatrix} x \\ y \\ z \end{pmatrix}
	\ee
\item la vitesse telle que :
	\be
		\vec{v} = \dfrac{{\rm d}\vec{r}}{{\rm dt}}
	\ee
\item l'acc\'el\'eration telle que :
	\be
		\vec{a} = \dfrac{{\rm d}\vec{v}}{{\rm d}t} = \frac{{\rm d}^{2}\vec{r}}{{\rm dt^{2}}}
	\ee
\end{itemize}

Les coordonn\'ees cart\'esiennes ne sont pas toujours les plus adapt\'ees. Un autre syt\`eme de coordonn\'ees peut \^etre plus commode \`a utiliser. Il convient de choisir alors $s$ grandeurs quelconques $\begin{Bmatrix}q_{i}\end{Bmatrix}^{s}_{1}$ pour d\'efinir la position d'un syst\`eme ($s$ degr\'es de libert\'es), ce sont ses \emph{coordonn\'ees g\'en\'eralis\'ees} et les d\'riv\'ees $\begin{Bmatrix}\dot{q}_{i}\end{Bmatrix}^{s}_{1}$, ses \emph{vitesses g\'en\'eralis\'ees}.

Les relations qui lient les acc\'el\'erations aux coordonn\'ees et aux vitesses sont appel\'ees les \emph{\'equations du mouvement}.

\section{Le principe de moindre action}

La formule la plus g\'en\'erale de la loi du mouvement des syst\`emes mécaniques est celle du \emph{principe de moindre action} (ou principe de Hamilton). Il introduit la \emph{fonction de Lagrange} d\'efinie telle que :
\be
	L(\begin{Bmatrix}q_{i}\end{Bmatrix}^{s}_{1},\begin{Bmatrix}\dot{q}_{i}\end{Bmatrix}^{s}_{1}, {\rm t}) = L(q,\dot{q},{\rm t})
\ee
Entre les instants ${\rm t}_{1}$ et ${\rm t}_{2}$, le système se meut de telle mani\`ere que l'\emph{action} :
\be
	S = \int_{{\rm t}_{1}}^{{\rm t}_{2}} L(q,\dot{q},{\rm t}) d{\rm t} \label{EQ:2_1}
\ee
ait la plus petite valeur possible.

Partons d'un seul degr\'e de libert\'e et d\'efinissons $q=q({\rm t})$ telle que $S$ soit minimale. Cela signifie que $S$ a une valeur plus grande si :
\be
	q({\rm t}) \rightarrow q({\rm t})+\delta q({\rm t}) \label{EQ:2_2}
\ee
avec $\delta q({\rm t})$ est la variation de $q({\rm t})$. Or en ${\rm t}_{1}$ et ${\rm t}_{2}$, toutes les fonctions $q({\rm t})$ doivent avoir des valeurs identiques (les trajectoires diff\`erent mais pas les conditions initiales, ni finales). Donc $\forall q({\rm t})$, nous avons :
\be
	\delta q({\rm t}_{1})=\delta q({\rm t}_{2})=0 \label{EQ:2_3}
\ee
De plus $\dot{q}=\dfrac{{\rm d}q}{\rm dt}$ dont $\dfrac{{\rm d}(q+\delta q)}{\rm dt}=\dot{q}+\delta\dot{q}$.
\be
	S(q+\delta q, \dot{q}+\delta \dot{q}, {\rm t}) - S(q,\dot{q},{\rm t}) = \delta S = \int_{{\rm t}_{1}}^{{\rm t}_{2}} L(q+\delta q, \dot{q}+\delta \dot{q}, {\rm t}) d{\rm t} - \int_{{\rm t}_{1}}^{{\rm t}_{2}} L(q,\dot{q},{\rm t}) d{\rm t}
\ee
Or par d\'efinition, nous avons :
\be
	\delta L(q,\dot{q},{\rm t}) = \dfrac{\partial L}{\partial q}\delta q + \dfrac{\partial L}{\partial \dot{q}}\delta \dot{q} + \dfrac{\partial L}{\partial {\rm t}}\delta {\rm t}
\ee
Mais puisque $\delta {\rm t}=0$, cela donne, au premier ordre (développement en s\'erie de Taylor) :
\be
	L(q+\delta q, \dot{q}+\delta \dot{q}, {\rm t}) \approx L(q,\dot{q},{\rm t}) + \dfrac{\partial L}{\partial q}\delta q + \dfrac{\partial L}{\partial \dot{q}}\delta \dot{q}
\ee
ou encore, le principe de moindre action peut s'\'ecrire :
\be
	\delta S = \delta \int_{{\rm t}_{1}}^{{\rm t}_{2}} L(q,\dot{q},{\rm t}) d{\rm t} = 0 \label{EQ:2_4}
\ee
et, de facto :
\be
	\int_{{\rm t}_{1}}^{{\rm t}_{2}} \left(\dfrac{\partial L}{\partial q}\delta q + \dfrac{\partial L}{\partial \dot{q}}\delta \dot{q}\right) {\rm dt} = 0
\ee
Ensuite, remarquons que :
\be
	\delta \dot{q} = \delta\left(\dfrac{{\rm d}q}{{\rm dt}}\right) = \dfrac{{\rm d}\delta q}{{\rm dt}}
\ee
L'\'equation \ref{EQ:2_4} devient alors :
\be
	\int_{{\rm t}_{1}}^{{\rm t}_{2}} \dfrac{\partial L}{\partial q}\delta q + \dfrac{\partial L}{\partial \dot{q}}\dfrac{{\rm d}\delta q}{{\rm dt}} \delta {\rm t} = 0
\ee
En se rappelant l'intégration par parties de Brook Taylor qui dit que :
\be
	\int_{a}^{b} u(x)v'(x){\rm dx} = \left[u(x)v(x)\right]_{a}^{b} - \int_{a}^{b} u'(x)v(x){\rm dx}
\ee
alors :
\be
	\int_{{\rm t}_{1}}^{{\rm t}_{2}} \dfrac{\partial L}{\partial \dot{q}}\dfrac{{\rm d}\delta q}{{\rm dt}} \delta {\rm t} = \left[\dfrac{\partial L}{\partial \dot{q}}\delta q\right]_{{\rm t}_{1}}^{{\rm t}_{2}} - \int_{{\rm t}_{1}}^{{\rm t}_{2}} \dfrac{{\rm d}}{{\rm dt}}\left(\dfrac{\partial L}{\partial \dot{q}}\right) \delta q{\rm dt}
\ee
L'\'equation \ref{EQ:2_4} s'\'ecrit donc :
\bea
	\delta S & = & \int_{{\rm t}_{1}}^{{\rm t}_{2}} \dfrac{\partial L}{\partial q}\delta q {\rm dt} + \left[\dfrac{\partial L}{\partial \dot{q}}\delta q\right]_{{\rm t}_{1}}^{{\rm t}_{2}} - \int_{{\rm t}_{1}}^{{\rm t}_{2}} \dfrac{{\rm d}}{{\rm dt}}\left(\dfrac{\partial L}{\partial \dot{q}}\right) \delta q{\rm dt} \nonumber \\
	& = & \left[\dfrac{\partial L}{\partial \dot{q}}\delta q\right]_{{\rm t}_{1}}^{{\rm t}_{2}} - \int_{{\rm t}_{1}}^{{\rm t}_{2}} \left(\dfrac{\partial L}{\partial q}-\dfrac{{\rm d}}{{\rm dt}}\left(\dfrac{\partial L}{\partial \dot{q}}\right)\right) \delta q{\rm dt} \label{EQ:2_5}
\eea
or l'application de \ref{EQ:2_3} dans \ref{EQ:2_5} implique directement :
\be
	\delta S = \int_{{\rm t}_{1}}^{{\rm t}_{2}} \left(\dfrac{\partial L}{\partial q}-\dfrac{{\rm d}}{{\rm dt}}\left(\dfrac{\partial L}{\partial \dot{q}}\right)\right) \delta q{\rm dt}
\ee
Le principe de moindre action donne $\forall q$, $\delta S = 0$ et l'expression ci-dessus doit \^etre valide quelque soit la valeur de $\delta q$. Cela a pour cons\'equence :
\be
	\dfrac{\partial L}{\partial q}-\dfrac{{\rm d}}{{\rm dt}}\left(\dfrac{\partial L}{\partial \dot{q}}\right) = 0
\ee
ce qui donne les \emph{\'equations de Lagrange} lorsqu'il y a $s$ degr\'es de libert\'e :
\be
	\forall i \in \left(1, 2, \ldots, s\right), \dfrac{\partial L}{\partial q_{i}}-\dfrac{{\rm d}}{{\rm dt}}\left(\dfrac{\partial L}{\partial \dot{q_{i}}}\right) = 0\label{EQ:2_6}
\ee
c'est-\`a-dire $s$ \'equations diff\'erentielles du second ordre à $s$ inconnues, $q_{i}({\rm t})$.

Si (A) et (B) sont deux syst\`emes ferm\'es suffisamment \'eloign\'es pour n\'egliger leur interaction mutuelle alors :
\be
	\lim L = L_{A} + L_{B} \label{EQ:2_7}
\ee
Il s'agit de l'additivit\'e de la fonction de Lagrange. De la m\^eme mani\`ere, la multiplication par une constante de la fonction de Lagrange d'un syst\`eme ferm\'e n'influe pas les \'equations du mouvement.

Une derni\`ere remarque en consid\'erant la fonction de Lagrange $L'$ telle que :
\be
	L'(q,\dot{q},t)=L(q,\dot{q},t) + \frac{{\rm d}f(q,{\rm t})}{{\rm dt}}
\ee
alors les int\'egrales \ref{EQ:2_1} donnent :
\bea
	S' & = & \int_{{\rm t}_{1}}^{{\rm t}_{2}} L'(q,\dot{q},{\rm t}) d{\rm t} = \int_{{\rm t}_{1}}^{{\rm t}_{2}} L(q,\dot{q},{\rm t}) d{\rm t} + \int_{{\rm t}_{1}}^{{\rm t}_{2}} \dfrac{{\rm d}f(q,{\rm t})}{{\rm dt}} d{\rm t} \nonumber \\
	& = & S + \int_{{\rm t}_{1}}^{{\rm t}_{2}} {\rm d}f(q,{\rm t}) = S + f(q({\rm t}_{2})) - f(q({\rm t}_{1}))
\eea
et impliquent que le principe de moindre action $\delta S = 0$ co\"incide avec $\delta S' = 0$. Ainsi, la fonction de Lagrange n'est d\'etermin\'ee qu'\`a la d\'eriv\'ee totale d'une fonction quelconque des coordonn\'ees et du temps.

\section{Le principe de relativit\'e de Galil\'ee}

Il est n\'ecessaire de choisir un syst\`eme de r\'ef\'erence pour \'etudier les ph\'enom\`enes m\'ecaniques et il est pr\'ef\'erable de le choisir afin que les lois de la m\'ecanique y soit les plus simples possibles. Et il est toujours possible de trouver un r\'ef\'erentiel tel que l'espace est homog\`ene et isotrope et le temps uniforme. En d'autres termes, la fonction de Lagrange d'un point mat\'eriel se mouvant librement dans un syst\`eme galil\'een en coordonn\'ees cart\'esiennes :
\begin{itemize}
	\item homog\'en\'it\'e de l'espace : $L(\vec{r},\vec{v},{\rm t}) \Rightarrow L(\vec{v},{\rm t})$
	\item uniformit\'e du temps : $L(\vec{v},{\rm t}) \Rightarrow L(\vec{v})$
	\item isotropie de l'espace : $L(\vec{v}) \Rightarrow L(\lVert\vec{v}\rVert)$
\end{itemize}
En r\'esum\'e, nous avons :
\be
	L = L(v^{2}) \label{EQ:3_1}
\ee
ce qui implique que $\dfrac{\partial L}{\partial \vec{r}} = 0$. Et par application des \'equations de Lagrange (\ref{EQ:2_6}), nous avons :
\be
	\dfrac{{\rm d}}{{\rm dt}}\left(\dfrac{\partial L}{\partial \vec{v}}\right) = 0
\ee
et comme la fonction de Lagrange n'est fonction que de la vitesse, ou encore $\dfrac{\partial L({\vec{v}}^{2})}{\partial \vec{v}} \propto \vec{v}$ (voir \ref{EQ:3_1}), dans ce cas alors :
\be
	\vec{v} = \vec{Cte} \label{EQ:3_2}
\ee
Dans un r\'ef\'erentiel galil\'een, tout mouvement libre s'effectue donc \`a une vitesse constante en grandeur et en direction. C'est la \emph{loi de l'inertie}. Supposons deux r\'ef\'erentiels galil\'eens ($K$) et ($K'$) tels que $\vec{KK'}=\vec{V}{\rm t}$. Alors $\vec{r}=\vec{KK'}+\vec{r'}$, soit :
\be
	\vec{r} = \vec{r'} + \vec{V}{\rm t} \label{EQ:3_3}
\ee
En M\'ecanique classique, le temps est absolu, aussi :
\be
	{\rm t} = {\rm t'} \label{EQ:3_4}
\ee
Les formules (\ref{EQ:3_3}) et (\ref{EQ:3_4}) sont les \emph{transformations de Galil\'ee}.

\section{Fonction de Lagrange d'un point mat\'eriel libre}