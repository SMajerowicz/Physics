\chapter{Int\'egration des \'equations du mouvement}

\section{Mouvement lin\'eaire}

L'\'equation (\ref{EQ:5_5}) appliqu\'ee \`a un point mat\'eriel donne en coordonn\'ees g\'en\'eralis\'ees :
\be
	L = \frac{1}{2}a(q)\dot{q}^{2} - U(q) \label{EQ:11_1}
\ee
qui s'\'ecrit en coordonn\'ees cart\'esiennes :
\be
	L = \frac{1}{2}m\dot{x}^{2} - U(x) \label{EQ:11_2}
\ee
En \'ecrivant l'\'energie totale :
\bea
	E = T + U & = & \frac{1}{2}m\dot{x}^{2} + U(x) \nonumber \\
	\dot{x}^{2} & = & \frac{2}{m}(E-U(x)) \nonumber \\
	\dfrac{\mathrm{d}x}{\mathrm{dt}} & = & \sqrt{\frac{2}{m}(E-U(x))} \nonumber \\
	t & = & \sqrt{\frac{m}{2}}\int{\dfrac{\mathrm{d}x}{\sqrt{E-U(x)}}} + cste \label{EQ:11_3}
\eea
o\`u $E$ et $cste$ sont des constantes du mouvement, la premi\`ere par la loi de conservation de l'\'energie et la seconde par int\'egration.

Lors d'un mouvement, l'\'energie totale est toujours sup\'erieure \`a l'\'energie potentielle car l'\'energie cint\'etique ne peut \^etre n\'egative. Aussi le mouvement ne peut \^etre possible que si $U(x)<E$.

\begin{figure}[htb!]
	\begin{center}
		\begin{picture}(200,150)(0,0)
			%axis
			\linethickness{0.05mm}
			\put(0,0){\line(1,0){200}}\put(202,-2){$x$}
			\put(0,0){\line(0,1){140}}\put(-2,142){$U$}
			%curve
			\linethickness{0.5mm}
			\qbezier(-5,130),(15,130),(25,110)
			\qbezier(25,110),(35,90),(55,90)
			\qbezier(55,90),(80,90),(90,100)
			\qbezier(90,100),(100,110),(125,110)
			\qbezier(125,110),(150,110),(190,20)
			%limit
			\linethickness{0.05mm}
			\multiput(-5,100)(10,0){20}{\line(1,0){8}}\put(195,98){$U=E$}
			\multiput(32,0)(0,5){20}{\line(0,1){3}}\put(34,102){$A$}\put(28,-7){$x_{1}$}
			\multiput(90,0)(0,5){20}{\line(0,1){3}}\put(82,102){$B$}\put(86,-7){$x_{2}$}
			\put(142,102){$C$}
		\end{picture}
		\caption{Exemple d'une fonction d\'efinissant l'\'energie potentielle}\label{FIG:3_1}
	\end{center}
\end{figure}

Sur la figure (\ref{FIG:3_1}), la condition $U(x)<E$ implique que le mouvement n'est possible que sur les intervalles tels que $x\in\left[A\,;B\right]$ et $x\in\left[C\,;+\infty\right[$. Dans le premier cas, le mouvement est dit fini et oscillatoire. Dans le second cas, le mouvement est dit inifini. Les points tels que :
\be
	U(x) = E \label{EQ:11_4}
\ee
sont les \emph{points d'arr\^et}. En effet, dans ce cas, l'\'energie cin\'etique du point mat\'eriel est nulle car sa vitesse aussi de facto.

Dans la relation (\ref{EQ:5_1}), nous avons vu que le temps n'est pas qu'uniforme, il est aussi isotrope. Ainsi la transformation $\mathrm{t}\mapsto -\mathrm{t}$ laisse inchang\'ee la fonction de Lagrange et les \'equations du mouvement. Ainsi $\Delta\mathrm{t}(x_{1}\rightarrow x_{2}) = \Delta\mathrm{t}(x_{2}\rightarrow x_{1})$ et la période d'oscillation $\mathrm{T}(E)=\Delta\mathrm{t}(x_{1}\rightarrow x_{2}\rightarrow x_{1})$ vaut $2\Delta\mathrm{t}(x_{1}\rightarrow x_{2})$. L'\'equation (\ref{EQ:11_3}) s'\'ecrit alors :
\bea
	\mathrm{T}(E) & = & 2\sqrt{\frac{m}{2}}\int_{x_{1}(E)}^{x_{2}(E)}{\dfrac{\mathrm{d}x}{\sqrt{E-U(x)}}} \nonumber \\
	\mathrm{T}(E) & = & \sqrt{2m}\int_{x_{1}(E)}^{x_{2}(E)}{\dfrac{\mathrm{d}x}{\sqrt{E-U(x)}}}\label{EQ:11_5}
\eea
qui d\'etermine la p\'eriode d'oscillation de la particule mat\'erielle en fonction de son \'energie totale.

\section{D\'efinition de l'\'energie potentielle en fonction de la p\'eriode des oscillations}

\begin{figure}[htb!]
	\begin{center}
		\begin{picture}(200,150)(0,0)
			%axis
			\linethickness{0.05mm}
			\put(0,0){\line(1,0){200}}\put(202,-2){$x$}
			\put(100,0){\line(0,1){140}}\put(98,142){$U$}
			%curve
			\linethickness{0.5mm}
			\qbezier(35,120),(40,0),(100,0)
			\qbezier(100,0),(175,0),(180,120)
			\put(47,50){$x_{1}(U)$}
			\put(132,50){$x_{2}(U)$}
			%limit
			\linethickness{0.05mm}
			\multiput(-5,80)(10,0){20}{\line(1,0){8}}\put(195,78){$U=E$}
			\multiput(40,0)(0,5){16}{\line(0,1){3}}\put(35,-7){$x_{1}$}
			\multiput(175,0)(0,5){16}{\line(0,1){3}}\put(170,-7){$x_{2}$}
		\end{picture}
		\caption{Fonction d\'efinissant l'\'energie potentielle ayant qu'un unique minimum}\label{FIG:3_2}
	\end{center}
\end{figure}

L'objectif de ce paragraphe est de trouver la fonction d\'efinissant l'\'energie potentielle d'un champ $U$ connaissant la p\'eriode d'oscillations $\mathrm{T}(E)$ du mouvement animant une particule. Pour cela, nous supposons que la fonction recherch\'ee $U$ n'a qu'un unique minimum dans la r\'egion du mouvement et que, comme sur la figure (\ref{FIG:3_2}), $U(0) = 0$.

Transformons l'int\'egrale (\ref{EQ:11_5}) avec $x$ fonction de $U$ telle que pour une valeur de l'\'energie potentielle $U$ il existe deux valeurs de $x$ diff\'erentes. Elle devient alors la somme de deux int\'egrales sur deux r\'egions diff\'erentes :
\bea
	\mathrm{T}(E) & = & \sqrt{2m}\int_{x_{1}(E)}^{x_{2}(E)}\dfrac{\mathrm{d}x}{\sqrt{E - U(x)}} \nonumber \\
	& = & -\sqrt{2m}\int_{0}^{E}\dfrac{\mathrm{d}x_{1}}{\mathrm{d}U}\dfrac{\mathrm{d}U}{\sqrt{E - U}} + \sqrt{2m}\int_{0}^{E}\dfrac{\mathrm{d}x_{2}}{\mathrm{d}U}\dfrac{\mathrm{d}U}{\sqrt{E - U}} \nonumber \\
	& = & \sqrt{2m}\int_{0}^{E}\left[\dfrac{\mathrm{d}x_{2}}{\mathrm{d}U} - \dfrac{\mathrm{d}x_{1}}{\mathrm{d}U}\right]\dfrac{\mathrm{d}U}{\sqrt{E - U}}
\eea
Elle peut \^etre \'etendue \`a :
\bea
	\int_{0}^{\alpha}\dfrac{\mathrm{T}(E)\mathrm{d}E}{\sqrt{\alpha - E}} & = & \sqrt{2m}\int_{0}^{\alpha}\int_{0}^{E}\left[\dfrac{\mathrm{d}x_{2}}{\mathrm{d}U} - \dfrac{\mathrm{d}x_{1}}{\mathrm{d}U}\right]\dfrac{\mathrm{d}U\mathrm{d}E}{\sqrt{(\alpha - E)(E - U)}} \nonumber \\
	& =& \sqrt{2m}\int_{0}^{\alpha}\left[\dfrac{\mathrm{d}x_{2}}{\mathrm{d}U} - \dfrac{\mathrm{d}x_{1}}{\mathrm{d}U}\right]\mathrm{d}U\int_{U}^{\alpha}\dfrac{\mathrm{d}E}{\sqrt{(\alpha - E)(E - U)}}
\eea
La seconde implication dans sa deuxi\`eme int\'egrale est en $\int_{U}^{\alpha}$ car en dehors de cet intervalle, $\int_{0}^{U}$ et $\int_{\alpha}^{E}$ sont ind\'efinie car $(E-U)$ et $(\alpha-E)$ sont respectivement n\'egatives.

Faisons d\'esormais le calcul de la seconde int\'egrale, soit :
\be
	\int_{U}^{\alpha}\dfrac{\mathrm{d}E}{\sqrt{(\alpha - E)(E - U)}}
\ee
en posant :
\be
	y = \dfrac{2E - U - \alpha}{\alpha - U}
\ee
qui donne :
\be
	\begin{cases}
		E = \dfrac{(\alpha - U)y + U + \alpha}{2} \\
		\mathrm{d}y = \dfrac{2\mathrm{d}E}{\alpha - U} \Leftrightarrow \mathrm{d}E = \dfrac{\alpha - U}{2}\mathrm{d}y
	\end{cases}
\ee
et :
\be
	\begin{cases}
		E = U \Rightarrow y = \dfrac{U - \alpha}{\alpha - U} = -1 \\
		E = \alpha \Rightarrow y = \dfrac{-U + \alpha}{\alpha - U} = 1
	\end{cases}
\ee