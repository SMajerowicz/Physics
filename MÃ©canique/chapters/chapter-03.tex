\chapter{Int\'egration des \'equations du mouvement}

\section{Mouvement lin\'eaire}

L'\'equation (\ref{EQ:5_5}) appliqu\'ee \`a un point mat\'eriel donne en coordonn\'ees g\'en\'eralis\'ees :
\be
	L = \frac{1}{2}a(q)\dot{q}^{2} - U(q) \label{EQ:11_1}
\ee
qui s'\'ecrit en coordonn\'ees cart\'esiennes :
\be
	L = \frac{1}{2}m\dot{x}^{2} - U(x) \label{EQ:11_2}
\ee
En \'ecrivant l'\'energie totale :
\bea
	E = T + U & = & \frac{1}{2}m\dot{x}^{2} + U(x) \nonumber \\
	\dot{x}^{2} & = & \frac{2}{m}(E-U(x)) \nonumber \\
	\dfrac{\mathrm{d}x}{\mathrm{dt}} & = & \sqrt{\frac{2}{m}(E-U(x))} \nonumber \\
	t & = & \sqrt{\frac{m}{2}}\int{\dfrac{\mathrm{d}x}{\sqrt{E-U(x)}}} + cste \label{EQ:11_3}
\eea
o\`u $E$ et $cste$ sont des constantes du mouvement, la premi\`ere par la loi de conservation de l'\'energie et la seconde par int\'egration.

Lors d'un mouvement, l'\'energie totale est toujours sup\'erieure \`a l'\'energie potentielle car l'\'energie cint\'etique ne peut \^etre n\'egative. Aussi le mouvement ne peut \^etre possible que si $U(x)<E$.

\begin{figure}[htb!]
	\begin{center}
		\begin{picture}(200,150)(0,0)
			%axis
			\linethickness{0.05mm}
			\put(0,0){\line(1,0){200}}\put(202,-2){$x$}
			\put(0,0){\line(0,1){140}}\put(-2,142){$U$}
			%curve
			\linethickness{0.5mm}
			\qbezier(-5,130),(15,130),(25,110)
			\qbezier(25,110),(35,90),(55,90)
			\qbezier(55,90),(80,90),(90,100)
			\qbezier(90,100),(100,110),(125,110)
			\qbezier(125,110),(150,110),(190,20)
			%limit
			\linethickness{0.05mm}
			\multiput(-5,100)(10,0){20}{\line(1,0){8}}\put(195,98){$U=E$}
			\multiput(32,0)(0,5){20}{\line(0,1){3}}\put(34,102){$A$}\put(28,-7){$x_{1}$}
			\multiput(90,0)(0,5){20}{\line(0,1){3}}\put(82,102){$B$}\put(86,-7){$x_{2}$}
			\put(142,102){$C$}
		\end{picture}
		\caption{Exemple d'une fonction d\'efinissant l'\'energie potentielle}\label{FIG:3_1}
	\end{center}
\end{figure}

Sur la figure (\ref{FIG:3_1}), la condition $U(x)<E$ implique que le mouvement n'est possible que sur les intervalles tels que $x\in\left[A\,;B\right]$ et $x\in\left[C\,;+\infty\right[$. Dans le premier cas, le mouvement est dit fini et oscillatoire. Dans le second cas, le mouvement est dit inifini. Les points tels que :
\be
	U(x) = E \label{EQ:11_4}
\ee
sont les \emph{points d'arr\^et}. En effet, dans ce cas, l'\'energie cin\'etique du point mat\'eriel est nulle car sa vitesse aussi de facto.

Dans la relation (\ref{EQ:5_1}), nous avons vu que le temps n'est pas qu'uniforme, il est aussi isotrope. Ainsi la transformation $\mathrm{t}\mapsto -\mathrm{t}$ laisse inchang\'ee la fonction de Lagrange et les \'equations du mouvement. Ainsi $\Delta\mathrm{t}(x_{1}\rightarrow x_{2}) = \Delta\mathrm{t}(x_{2}\rightarrow x_{1})$ et la période d'oscillation $\mathrm{T}(E)=\Delta\mathrm{t}(x_{1}\rightarrow x_{2}\rightarrow x_{1})$ vaut $2\Delta\mathrm{t}(x_{1}\rightarrow x_{2})$. L'\'equation (\ref{EQ:11_3}) s'\'ecrit alors :
\bea
	\mathrm{T}(E) & = & 2\sqrt{\frac{m}{2}}\int_{x_{1}(E)}^{x_{2}(E)}{\dfrac{\mathrm{d}x}{\sqrt{E-U(x)}}} \nonumber \\
	\mathrm{T}(E) & = & \sqrt{2m}\int_{x_{1}(E)}^{x_{2}(E)}{\dfrac{\mathrm{d}x}{\sqrt{E-U(x)}}}\label{EQ:11_5}
\eea
qui d\'etermine la p\'eriode d'oscillation de la particule mat\'erielle en fonction de son \'energie totale.

\section{D\'efinition de l'\'energie potentielle en fonction de la p\'eriode des oscillations}

\begin{figure}[htb!]
	\begin{center}
		\begin{picture}(200,150)(0,0)
			%axis
			\linethickness{0.05mm}
			\put(0,0){\line(1,0){200}}\put(202,-2){$x$}
			\put(100,0){\line(0,1){140}}\put(98,142){$U$}
			%curve
			\linethickness{0.5mm}
			\qbezier(35,120),(40,0),(100,0)
			\qbezier(100,0),(175,0),(180,120)
			\put(47,50){$x_{1}(U)$}
			\put(132,50){$x_{2}(U)$}
			%limit
			\linethickness{0.05mm}
			\multiput(-5,80)(10,0){20}{\line(1,0){8}}\put(195,78){$U=E$}
			\multiput(40,0)(0,5){16}{\line(0,1){3}}\put(35,-7){$x_{1}$}
			\multiput(175,0)(0,5){16}{\line(0,1){3}}\put(170,-7){$x_{2}$}
		\end{picture}
		\caption{Fonction d\'efinissant l'\'energie potentielle ayant qu'un unique minimum}\label{FIG:3_2}
	\end{center}
\end{figure}

L'objectif de ce paragraphe est de trouver la fonction d\'efinissant l'\'energie potentielle d'un champ $U$ connaissant la p\'eriode d'oscillations $\mathrm{T}(E)$ du mouvement animant une particule. Pour cela, nous supposons que la fonction recherch\'ee $U$ n'a qu'un unique minimum dans la r\'egion du mouvement et que, comme sur la figure (\ref{FIG:3_2}), $U(0) = 0$.

Transformons l'int\'egrale (\ref{EQ:11_5}) avec $x$ fonction de $U$ telle que pour une valeur de l'\'energie potentielle $U$ il existe deux valeurs de $x$ diff\'erentes. Elle devient alors la somme de deux int\'egrales sur deux r\'egions diff\'erentes :
\bea
	\mathrm{T}(E) & = & \sqrt{2m}\int_{x_{1}(E)}^{x_{2}(E)}\dfrac{\mathrm{d}x}{\sqrt{E - U(x)}} \nonumber \\
	& = & -\sqrt{2m}\int_{0}^{E}\dfrac{\mathrm{d}x_{1}}{\mathrm{d}U}\dfrac{\mathrm{d}U}{\sqrt{E - U}} + \sqrt{2m}\int_{0}^{E}\dfrac{\mathrm{d}x_{2}}{\mathrm{d}U}\dfrac{\mathrm{d}U}{\sqrt{E - U}} \nonumber \\
	& = & \sqrt{2m}\int_{0}^{E}\left[\dfrac{\mathrm{d}x_{2}}{\mathrm{d}U} - \dfrac{\mathrm{d}x_{1}}{\mathrm{d}U}\right]\dfrac{\mathrm{d}U}{\sqrt{E - U}}
\eea
Elle peut \^etre \'etendue \`a :
\bea
	\int_{0}^{\alpha}\dfrac{\mathrm{T}(E)\mathrm{d}E}{\sqrt{\alpha - E}} & = & \sqrt{2m}\int_{0}^{\alpha}\int_{0}^{E}\left[\dfrac{\mathrm{d}x_{2}}{\mathrm{d}U} - \dfrac{\mathrm{d}x_{1}}{\mathrm{d}U}\right]\dfrac{\mathrm{d}U\mathrm{d}E}{\sqrt{(\alpha - E)(E - U)}} \nonumber \\
	& =& \sqrt{2m}\int_{0}^{\alpha}\left[\dfrac{\mathrm{d}x_{2}}{\mathrm{d}U} - \dfrac{\mathrm{d}x_{1}}{\mathrm{d}U}\right]\mathrm{d}U\int_{U}^{\alpha}\dfrac{\mathrm{d}E}{\sqrt{(\alpha - E)(E - U)}}
\eea
La seconde implication dans sa deuxi\`eme int\'egrale est en $\int_{U}^{\alpha}$ car en dehors de cet intervalle, $\int_{0}^{U}$ et $\int_{\alpha}^{E}$ sont ind\'efinie car $(E-U)$ et $(\alpha-E)$ sont respectivement n\'egatives.

Faisons d\'esormais le calcul de la seconde int\'egrale, soit :
\be
	\int_{U}^{\alpha}\dfrac{\mathrm{d}E}{\sqrt{(\alpha - E)(E - U)}}
\ee
en posant :
\be
	y = \dfrac{2E - U - \alpha}{\alpha - U}
\ee
qui donne :
\be
	\begin{cases}
		E = \dfrac{(\alpha - U)y + U + \alpha}{2} \\
		\mathrm{d}y = \dfrac{2\mathrm{d}E}{\alpha - U} \Leftrightarrow \mathrm{d}E = \dfrac{\alpha - U}{2}\mathrm{d}y
	\end{cases}
\ee
et :
\be
	\begin{cases}
		E = U \Rightarrow y = \dfrac{U - \alpha}{\alpha - U} = -1 \\
		E = \alpha \Rightarrow y = \dfrac{-U + \alpha}{\alpha - U} = 1
	\end{cases}
\ee
Ainsi :
\bea
	\int_{U}^{\alpha}\dfrac{\mathrm{d}E}{\sqrt{(\alpha - E)(E - U)}} & = & \int_{-1}^{1}\dfrac{(\alpha - U)\mathrm{d}y}{2\sqrt{\frac{(2\alpha - (\alpha - U)y + U + \alpha)}{2}\frac{((\alpha - U)y + U + \alpha - 2U)}{2}}} \nonumber \\
	& = & \int_{-1}^{1}\dfrac{(\alpha - U)\mathrm{d}y}{2\sqrt{\frac{(\alpha - U - (\alpha - U)y)}{2}\frac{(\alpha - U + (\alpha - U)y)}{2}}} \nonumber \\
	& = & \int_{-1}^{1}\dfrac{(\alpha - U)\mathrm{d}y}{2\frac{\alpha - U}{2}\sqrt{(1 - y)(1 + y)}} \nonumber \\
	& = & \int_{-1}^{1}\dfrac{\mathrm{d}y}{\sqrt{1 - y^{2}}}
\eea
En posant $u = y^{2}$ soit $y = u^{1/2}$ et $\mathrm{d}u = 2y\mathrm{d}y = 2u^{1/2}\mathrm{d}y \Leftrightarrow \mathrm{d}y = \frac{1}{2}u^{-1/2}\mathrm{d}u$, l'int\'egrale pr\'ec\'edente s'\'ecrit :
\bea
	\int_{U}^{\alpha}\dfrac{\mathrm{d}E}{\sqrt{(\alpha - E)(E - U)}} & = & \int_{-1}^{1}\dfrac{\mathrm{d}y}{\sqrt{1 - y^{2}}} \nonumber \\
	& = & 2 \int_{0}^{1}\dfrac{1}{2}\dfrac{u^{-1/2}\mathrm{d}u}{(1-u)^{1/2}} \nonumber \\
	& = & \int_{0}^{1}u^{-1/2}(1-u)^{-1/2}\mathrm{d}u \nonumber \\
	& = & \mathrm{B}\left(\frac{1}{2},\frac{1}{2}\right) = \dfrac{\Gamma(\frac{1}{2})\Gamma(\frac{1}{2})}{\Gamma(1)} = \pi
\eea
De fait :
\bea
	\int_{0}^{\alpha}\dfrac{\mathrm{T}(E)\mathrm{d}E}{\sqrt{\alpha - E}} & = & \sqrt{2m}\pi\int_{0}^{\alpha}\left[\dfrac{\mathrm{d}x_{2}}{\mathrm{d}U} - \dfrac{\mathrm{d}x_{1}}{\mathrm{d}U}\right]\mathrm{d}U \nonumber \\
	& = & \sqrt{2m}\pi\left[x_{2}(\alpha) - x_{2}(0) - x_{1}(\alpha) + x_{1}(0)\right]
\eea
Or par hypoth\`ese, $x_{2}(0) = x_{1}(0) = 0$, donc :
\be
	\int_{0}^{\alpha}\dfrac{\mathrm{T}(E)\mathrm{d}E}{\sqrt{\alpha - E}} = \sqrt{2m}\pi\left[x_{2}(\alpha) - x_{1}(\alpha)\right]
\ee
Dans le cas o\`u nous posons $\alpha = U$, nous pouvons en d\'eduire :
\be
	x_{2}(U) - x_{1}(U) = \dfrac{1}{\sqrt{2m}\pi}\int_{0}^{U}\dfrac{\mathrm{T}(E)\mathrm{d}E}{\sqrt{U - E}} \label{EQ:12_1}
\ee
De plus, si le champ d'\'energie potentielle est paire tel que $U(x) = U(-x)$ ou encore $x_{1}(U) = -x_{2}(U) = x(U)$, alors l'\'equation (\ref{EQ:12_1}) devient, puisque $x_{2}(U) - x_{1}(U) = 2x(U)$ :
\be
	x(U) = \dfrac{1}{2\pi\sqrt{2m}}\int_{0}^{U}\dfrac{\mathrm{T}(E)\mathrm{d}E}{\sqrt{U - E}} \label{EQ:12_2}
\ee

\section{Masse r\'eduite}

L'important probl\`eme du mouvement s'un syst\`eme ferm\'e \`a deux particules r\'eagissant l'une sur l'autre, ou encore \emph{probl\`eme des deux corps}, admet une solution g\'en\'erale compl\`ete. Pour le r\'esoudre, nous allons le simplifier en d\'ecomposant le mouvement du syst\`eme en celui du centre d'inertie et des deux points mat\'eriels par rapport \`a ce dernier. L'\'energie potentielle d'int\'eraction ne d\'epend que de la distance entre les deux particules, aussi la fonction de Lagrange s'\'ecrit :
\be
	L = \dfrac{m_{1}\dot{r}_{1}^{2}}{2} + \dfrac{m_{2}\dot{r}_{2}^{2}}{2} - U(\lvert \vec{r}_{1} - \vec{r}_{2} \rvert) \label{EQ:13_1}
\ee
D\'efinissons $\vec{r} = \vec{r}_{1} - \vec{r}_{2}$ et pla\c{c}ons l'origine des coordonn\'ees au centre d'inertie. Alors l'\'equation (\ref{EQ:8_3}) donne $\vec{R} = \vec{0}$ ou encore : $\sum m_{a}\vec{r}_{a} = \vec{0}$, soit :
\be
	m_{1}\vec{r}_{1} + m_{2}\vec{r}_{2} = \vec{0}
\ee
Nous pouvons en conclure que :
\be
	\vec{r} = \begin{cases}
		\vec{r}_{1} + \dfrac{m_{1}}{m_{2}}\vec{r}_{1} = \dfrac{m_{1} + m_{2}}{m_{2}}\vec{r}_{1} \\
		- \dfrac{m_{2}}{m_{1}}\vec{r}_{2} - \vec{r}_{2} = -\dfrac{m_{1} + m_{2}}{m_{1}}\vec{r}_{2}
	\end{cases}
\ee
soit :
\be
	\begin{cases}
		\vec{r}_{1} = \dfrac{m_{2}}{m_{1} + m_{2}}\vec{r} \\
		\vec{r}_{2} = \dfrac{-m_{1}}{m_{1} + m_{2}}\vec{r}
	\end{cases}\label{EQ:13_2}
\ee
Puisque :
\be
	\begin{cases}
		\vec{\dot{r}}_{1} = \dfrac{m_{2}}{m_{1} + m_{2}}\vec{\dot{r}} \\
		\vec{\dot{r}}_{2} = \dfrac{-m_{1}}{m_{1} + m_{2}}\vec{\dot{r}}
	\end{cases}
\ee
la fonction de Lagrange se d\'eveloppe ainsi :
\bea
	L & = & \dfrac{m_{1}}{2}\left(\dfrac{m_{2}}{m_{1} + m_{2}}\right)^{2}\vec{\dot{r}}^{\,2} + \dfrac{m_{2}}{2}\left(\dfrac{-m_{1}}{m_{1} + m_{2}}\right)^{2}\vec{\dot{r}}^{\,2} - U(\vec{r}) \nonumber \\
	& = & \dfrac{m_{1}m_{2}^{2} + m_{1}^{2}m_{2}}{2(m_{1} + m_{2})^{2}}\vec{\dot{r}}^{\,2} - U(\vec{r}) \nonumber \\
	& = & \dfrac{m_{1}m_{2}(m_{2} + m_{1})}{2(m_{1} + m_{2})^{2}}\vec{\dot{r}}^{\,2} - U(\vec{r})
\eea
et on peut conclure en posant \emph{la masse r\'eduite} :
\be
	m = \dfrac{m_{1}m_{2}}{(m_{1} + m_{2})} \label{EQ:13_4}
\ee
\`a :
\be
	L = \dfrac{m}{2}\vec{\dot{r}}^{\,2} - U(\vec{r}) \label{EQ:13_3}
\ee
qui est la fonction de Lagrange d'un point mat\'eriel de masse $m$ se d\'epla\c{c}ant dans le champ ext\'erieur $U(\vec{r})$ qui est sym\'etrique par rapport au point immobile des coordonn\'ees, soit le centre d'inertie. Ainsi le calcul de $\vec{r} = \vec{r}(\mathrm{t})$ permet gr\^ace aux \'equations (\ref{EQ:13_2}) d'en d\'eduire les trajectoires $\vec{r}_{1}(\mathrm{t})$ et $\vec{r}_{2}(\mathrm{t})$ des particules de masse respective $m_{1}$ et $m_{2}$.