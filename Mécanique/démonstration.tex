\documentclass[12pt,a4paper]{report}

\usepackage[utf8]{inputenc}
\usepackage[french]{babel}
\usepackage[autolanguage]{numprint}
\usepackage[T1]{fontenc}
\usepackage{amsfonts,amsmath,amssymb}
\usepackage{bigints}

\allowdisplaybreaks

\DeclareMathOperator{\arcsinh}{arcsinh}
\DeclareMathOperator{\arccosh}{arccosh}
\DeclareMathOperator{\arctanh}{arctanh}
\DeclareMathOperator{\cotan}{cotan}

\usepackage[pdftex]{pict2e}
\usepackage[dvipsnames]{xcolor}
\usepackage{graphicx}

\graphicspath{
    {chapters/graphics/}
	{chapters/exercices/graphics/}
}

\setcounter{tocdepth}{3}

\setlength{\paperwidth}{21cm}
\setlength{\paperheight}{29.7cm}
\setlength{\evensidemargin}{-1.0cm}
\setlength{\oddsidemargin}{-1.5cm}
\setlength{\topmargin}{-2cm}
\setlength{\headsep}{0.15cm}
\setlength{\headheight}{0.7cm}
\setlength{\textheight}{26cm}
\setlength{\textwidth}{19cm}

\newcommand{\be}{\begin{equation}}
\newcommand{\ee}{\end{equation}}
\newcommand{\benn}{\begin{equation*}}
\newcommand{\eenn}{\end{equation*}}
\newcommand{\bea}{\begin{eqnarray}}
\newcommand{\eea}{\end{eqnarray}}

\newcommand{\trint}{\mathop{\int\!\!\!\int\!\!\!\int}\limits}
\newcommand{\dint}{\mathop{\int\!\!\!\int}\limits}

\title{\'Etapes de calcul et d\'emonstrations du tome 1 <<~M\'ecanique~>>  de Physique Th\'eorique de L. Landau \&  E. Lifchitz}
\author{S\'ebastien Majerowicz}
\date{2023 -- 2025}

\begin{document}

\maketitle

\begin{abstract}
Ce document est une aide pour la compr\'ehension du premier livre des c\'el\`ebres cours de Landau \& Lifschitz au travers de la d\'emonstration des formules et de la r\'esolution des exercices propos\'es.
\end{abstract}

\tableofcontents
\listoffigures

\clearpage
\pagenumbering{arabic}
\setcounter{page}{1}

\chapter{\'Equations du mouvement}
\section{Coordonn\'ees g\'en\'eralis\'ees}

Pour un point mat\'eriel, nous avons en coordonn\'ees cart\'esiennes :
\begin{itemize}
\item le rayon vecteur tel que :
	\be
		\vec{r} = \begin{pmatrix} x \\ y \\ z \end{pmatrix}
	\ee
\item la vitesse telle que :
	\be
		\vec{v} = \dfrac{{\rm d}\vec{r}}{{\rm dt}}
	\ee
\item l'acc\'el\'eration telle que :
	\be
		\vec{a} = \dfrac{{\rm d}\vec{v}}{{\rm d}t} = \frac{{\rm d}^{2}\vec{r}}{{\rm dt^{2}}}
	\ee
\end{itemize}

Les coordonn\'ees cart\'esiennes ne sont pas toujours les plus adapt\'ees. Un autre syt\`eme de coordonn\'ees peut \^etre plus commode \`a utiliser. Il convient de choisir alors $s$ grandeurs quelconques $\begin{Bmatrix}q_{i}\end{Bmatrix}^{s}_{1}$ pour d\'efinir la position d'un syst\`eme ($s$ degr\'es de libert\'es), ce sont ses \emph{coordonn\'ees g\'en\'eralis\'ees} et les d\'eriv\'ees $\begin{Bmatrix}\dot{q}_{i}\end{Bmatrix}^{s}_{1}$, ses \emph{vitesses g\'en\'eralis\'ees}.

Les relations qui lient les acc\'el\'erations aux coordonn\'ees et aux vitesses sont appel\'ees les \emph{\'equations du mouvement}.

\section{Le principe de moindre action}

La formule la plus g\'en\'erale de la loi du mouvement des syst\`emes mécaniques est celle du \emph{principe de moindre action} (ou principe de Hamilton). Il introduit la \emph{fonction de Lagrange} d\'efinie telle que :
\be
	L(\begin{Bmatrix}q_{i}\end{Bmatrix}^{s}_{1},\begin{Bmatrix}\dot{q}_{i}\end{Bmatrix}^{s}_{1}, {\rm t}) = L(q,\dot{q},{\rm t})
\ee
Entre les instants ${\rm t}_{1}$ et ${\rm t}_{2}$, le système se meut de telle mani\`ere que l'\emph{action} :
\be
	S = \int_{{\rm t}_{1}}^{{\rm t}_{2}} L(q,\dot{q},{\rm t}) d{\rm t} \label{EQ:2_1}
\ee
ait la plus petite valeur possible.

Partons d'un seul degr\'e de libert\'e et d\'efinissons $q=q({\rm t})$ telle que $S$ soit minimale. Cela signifie que $S$ a une valeur plus grande si :
\be
	q({\rm t}) \rightarrow q({\rm t})+\delta q({\rm t}) \label{EQ:2_2}
\ee
avec $\delta q({\rm t})$ est la variation de $q({\rm t})$. Or en ${\rm t}_{1}$ et ${\rm t}_{2}$, toutes les fonctions $q({\rm t})$ doivent avoir des valeurs identiques (les trajectoires diff\`erent mais pas les conditions initiales, ni finales). Donc $\forall q({\rm t})$, nous avons :
\be
	\delta q({\rm t}_{1})=\delta q({\rm t}_{2})=0 \label{EQ:2_3}
\ee
De plus $\dot{q}=\dfrac{{\rm d}q}{\rm dt}$ dont $\dfrac{{\rm d}(q+\delta q)}{\rm dt}=\dot{q}+\delta\dot{q}$.
\be
	S(q+\delta q, \dot{q}+\delta \dot{q}, {\rm t}) - S(q,\dot{q},{\rm t}) = \delta S = \int_{{\rm t}_{1}}^{{\rm t}_{2}} L(q+\delta q, \dot{q}+\delta \dot{q}, {\rm t}) d{\rm t} - \int_{{\rm t}_{1}}^{{\rm t}_{2}} L(q,\dot{q},{\rm t}) d{\rm t}
\ee
Or par d\'efinition, nous avons :
\be
	\delta L(q,\dot{q},{\rm t}) = \dfrac{\partial L}{\partial q}\delta q + \dfrac{\partial L}{\partial \dot{q}}\delta \dot{q} + \dfrac{\partial L}{\partial {\rm t}}\delta {\rm t}
\ee
Mais puisque $\delta {\rm t}=0$, cela donne, au premier ordre (développement en s\'erie de Taylor) :
\be
	L(q+\delta q, \dot{q}+\delta \dot{q}, {\rm t}) \approx L(q,\dot{q},{\rm t}) + \dfrac{\partial L}{\partial q}\delta q + \dfrac{\partial L}{\partial \dot{q}}\delta \dot{q}
\ee
ou encore, le principe de moindre action peut s'\'ecrire :
\be
	\delta S = \delta \int_{{\rm t}_{1}}^{{\rm t}_{2}} L(q,\dot{q},{\rm t}) d{\rm t} = 0 \label{EQ:2_4}
\ee
et, de facto :
\be
	\int_{{\rm t}_{1}}^{{\rm t}_{2}} \left(\dfrac{\partial L}{\partial q}\delta q + \dfrac{\partial L}{\partial \dot{q}}\delta \dot{q}\right) {\rm dt} = 0
\ee
Ensuite, remarquons que :
\be
	\delta \dot{q} = \delta\left(\dfrac{{\rm d}q}{{\rm dt}}\right) = \dfrac{{\rm d}\delta q}{{\rm dt}}
\ee
L'\'equation (\ref{EQ:2_4}) devient alors :
\be
	\int_{{\rm t}_{1}}^{{\rm t}_{2}} \dfrac{\partial L}{\partial q}\delta q + \dfrac{\partial L}{\partial \dot{q}}\dfrac{{\rm d}\delta q}{{\rm dt}} \delta {\rm t} = 0
\ee
En se rappelant l'intégration par parties de Brook Taylor qui dit que :
\be
	\int_{a}^{b} u(x)v'(x){\rm dx} = \left[u(x)v(x)\right]_{a}^{b} - \int_{a}^{b} u'(x)v(x){\rm dx}
\ee
alors :
\be
	\int_{{\rm t}_{1}}^{{\rm t}_{2}} \dfrac{\partial L}{\partial \dot{q}}\dfrac{{\rm d}\delta q}{{\rm dt}} \delta {\rm t} = \left[\dfrac{\partial L}{\partial \dot{q}}\delta q\right]_{{\rm t}_{1}}^{{\rm t}_{2}} - \int_{{\rm t}_{1}}^{{\rm t}_{2}} \dfrac{{\rm d}}{{\rm dt}}\left(\dfrac{\partial L}{\partial \dot{q}}\right) \delta q{\rm dt}
\ee
L'\'equation (\ref{EQ:2_4}) s'\'ecrit donc :
\bea
	\delta S & = & \int_{{\rm t}_{1}}^{{\rm t}_{2}} \dfrac{\partial L}{\partial q}\delta q {\rm dt} + \left[\dfrac{\partial L}{\partial \dot{q}}\delta q\right]_{{\rm t}_{1}}^{{\rm t}_{2}} - \int_{{\rm t}_{1}}^{{\rm t}_{2}} \dfrac{{\rm d}}{{\rm dt}}\left(\dfrac{\partial L}{\partial \dot{q}}\right) \delta q{\rm dt} \nonumber \\
	& = & \left[\dfrac{\partial L}{\partial \dot{q}}\delta q\right]_{{\rm t}_{1}}^{{\rm t}_{2}} - \int_{{\rm t}_{1}}^{{\rm t}_{2}} \left(\dfrac{\partial L}{\partial q}-\dfrac{{\rm d}}{{\rm dt}}\left(\dfrac{\partial L}{\partial \dot{q}}\right)\right) \delta q{\rm dt} \label{EQ:2_5}
\eea
or l'application de (\ref{EQ:2_3}) dans (\ref{EQ:2_5}) implique directement :
\be
	\delta S = \int_{{\rm t}_{1}}^{{\rm t}_{2}} \left(\dfrac{\partial L}{\partial q}-\dfrac{{\rm d}}{{\rm dt}}\left(\dfrac{\partial L}{\partial \dot{q}}\right)\right) \delta q{\rm dt}
\ee
Le principe de moindre action donne $\forall q$, $\delta S = 0$ et l'expression ci-dessus doit \^etre valide quelque soit la valeur de $\delta q$. Cela a pour cons\'equence :
\be
	\dfrac{\partial L}{\partial q}-\dfrac{{\rm d}}{{\rm dt}}\left(\dfrac{\partial L}{\partial \dot{q}}\right) = 0
\ee
ce qui donne les \emph{\'equations de Lagrange} lorsqu'il y a $s$ degr\'es de libert\'e :
\be
	\forall i \in \left(1, 2, \ldots, s\right), \dfrac{\partial L}{\partial q_{i}}-\dfrac{{\rm d}}{{\rm dt}}\left(\dfrac{\partial L}{\partial \dot{q_{i}}}\right) = 0\label{EQ:2_6}
\ee
c'est-\`a-dire $s$ \'equations diff\'erentielles du second ordre à $s$ inconnues, $q_{i}({\rm t})$.

Si (A) et (B) sont deux syst\`emes ferm\'es suffisamment \'eloign\'es pour n\'egliger leur interaction mutuelle alors :
\be
	\lim L = L_{A} + L_{B} \label{EQ:2_7}
\ee
Il s'agit de l'additivit\'e de la fonction de Lagrange. De la m\^eme mani\`ere, la multiplication par une constante de la fonction de Lagrange d'un syst\`eme ferm\'e n'influe pas les \'equations du mouvement.

Une derni\`ere remarque en consid\'erant la fonction de Lagrange $L'$ telle que :
\be
	L'(q,\dot{q},t)=L(q,\dot{q},t) + \frac{{\rm d}f(q,{\rm t})}{{\rm dt}} \label{EQ:2_8}
\ee
alors les int\'egrales (\ref{EQ:2_1}) donnent :
\bea
	S' & = & \int_{{\rm t}_{1}}^{{\rm t}_{2}} L'(q,\dot{q},{\rm t}) d{\rm t} = \int_{{\rm t}_{1}}^{{\rm t}_{2}} L(q,\dot{q},{\rm t}) d{\rm t} + \int_{{\rm t}_{1}}^{{\rm t}_{2}} \dfrac{{\rm d}f(q,{\rm t})}{{\rm dt}} d{\rm t} \nonumber \\
	& = & S + \int_{{\rm t}_{1}}^{{\rm t}_{2}} {\rm d}f(q,{\rm t}) = S + f(q({\rm t}_{2})) - f(q({\rm t}_{1}))
\eea
et impliquent que le principe de moindre action $\delta S = 0$ co\"incide avec $\delta S' = 0$. Ainsi, la fonction de Lagrange n'est d\'etermin\'ee qu'\`a la d\'eriv\'ee totale d'une fonction quelconque des coordonn\'ees et du temps.

\section{Le principe de relativit\'e de Galil\'ee}

Il est n\'ecessaire de choisir un syst\`eme de r\'ef\'erence pour \'etudier les ph\'enom\`enes m\'ecaniques et il est pr\'ef\'erable de le choisir afin que les lois de la m\'ecanique y soit les plus simples possibles. Et il est toujours possible de trouver un r\'ef\'erentiel tel que l'espace est homog\`ene et isotrope et le temps uniforme. En d'autres termes, la fonction de Lagrange d'un point mat\'eriel se mouvant librement dans un syst\`eme galil\'een en coordonn\'ees cart\'esiennes :
\begin{itemize}
	\item homog\'en\'it\'e de l'espace : $L(\vec{r},\vec{v},{\rm t}) \Rightarrow L(\vec{v},{\rm t})$
	\item uniformit\'e du temps : $L(\vec{v},{\rm t}) \Rightarrow L(\vec{v})$
	\item isotropie de l'espace : $L(\vec{v}) \Rightarrow L(\lVert\vec{v}\rVert)$
\end{itemize}
En r\'esum\'e, nous avons :
\be
	L = L(\vec{v}^{\,2}) \label{EQ:3_1}
\ee
ce qui implique que $\dfrac{\partial L}{\partial \vec{r}} = 0$. Et par application des \'equations de Lagrange (\ref{EQ:2_6}), nous avons :
\be
	\dfrac{{\rm d}}{{\rm dt}}\left(\dfrac{\partial L}{\partial \vec{v}}\right) = 0
\ee
et comme la fonction de Lagrange n'est fonction que de la vitesse, ou encore $\dfrac{\partial L({\vec{v}}^{\,2})}{\partial \vec{v}} \propto \vec{v}$ (voir \ref{EQ:3_1}), dans ce cas alors :
\be
	\vec{v} = \vec{Cte} \label{EQ:3_2}
\ee
Dans un r\'ef\'erentiel galil\'een, tout mouvement libre s'effectue donc \`a une vitesse constante en grandeur et en direction. C'est la \emph{loi de l'inertie}. Supposons deux r\'ef\'erentiels galil\'eens ($K$) et ($K'$) tels que $\vec{KK'}=\vec{V}{\rm t}$. Alors $\vec{r}=\vec{KK'}+\vec{r'}$, soit :
\be
	\vec{r} = \vec{r'} + \vec{V}{\rm t} \label{EQ:3_3}
\ee
En M\'ecanique classique, le temps est absolu, aussi :
\be
	{\rm t} = {\rm t'} \label{EQ:3_4}
\ee
Les formules (\ref{EQ:3_3}) et (\ref{EQ:3_4}) sont les \emph{transformations de Galil\'ee}.

\section{Fonction de Lagrange d'un point mat\'eriel libre}

Consid\'erons le mouvement libre d'un point mat\'eriel dans un syst\`eme gali\'een. Cela implique (\ref{EQ:3_1}), soit $L = L(v^{2})$. Avec un r\'ef\'erentiel $(K)$ qui se d\'eplace par rapport \`a un autre r\'ef\'erentiel (K') telle que $\vec{v'}=\vec{v} + \vec{\epsilon}$, nous avons :
\bea
L' = L(\vec{v'}^{\,2}) & = & L(\vec{v}^{\,2} + 2\vec{v}.\vec{\epsilon} + \vec{\epsilon}^{\,2}) \nonumber \\
 & = & L(\vec{v}^{\,2}) + \dfrac{\partial L}{\partial \vec{v}^{\,2}}2\vec{v}.\vec{\epsilon}
\eea
en supprimant les ordres sup\'erieurs à 2 en $\vec{\epsilon}$ dans le d\'eveloppement en s\'erie de Taylor de :
\be
	L(\vec{v}^{\,2} + 2\vec{v}.\vec{\epsilon} + \vec{\epsilon}^{\,2}) \approx L(\vec{v}^{\,2} + \dfrac{\partial L}{\partial \vec{v}^{\,2}}.(2\vec{v}.\vec{\epsilon}+\vec{\epsilon}^{\,2}) + \dfrac{\partial L}{2\partial \vec{v}^{\,4}}.(2\vec{v}.\vec{\epsilon}+\vec{\epsilon}^{\,2})^{2} + \ldots
\ee
En reprenant la fonction de Lagrange qui peut se d\'efinir comme (\ref{EQ:2_8}), nous pouvons pos\'e :
\bea
	\dfrac{\partial L}{\partial \vec{v}^{\,2}}2\vec{v}.\vec{\epsilon} & = & \dfrac{{\rm d}f(\vec{r},t)}{{\rm dt}} \nonumber \\
	\Leftrightarrow 2\dfrac{\partial L}{\partial \vec{v}^{\,2}}\dfrac{{\rm d}\vec{r}}{{\rm dt}}.\vec{\epsilon} & = & \dfrac{{\rm d}f(\vec{r},t)}{{\rm dt}} \nonumber \\
	\Leftrightarrow 2\dfrac{\partial L}{\partial \vec{v}^{\,2}}\dfrac{{\rm d}(\vec{r}.\vec{\epsilon})}{{\rm dt}} & = & \dfrac{{\rm d}f(\vec{r},t)}{{\rm dt}}
\eea
n'est vraie que si et seulement si $\dfrac{\partial L}{\partial \vec{v}^{\,2}} = \alpha$, avec $\alpha$ une constante. Or nous avons \ref{EQ:3_1}, qui implique :
\be
	L = L(\vec{v}^{\,2}) = \alpha\vec{v}^{\,2}
\ee
Dans le cadre où le r\'ef\'erentiel $(K)$ se meut à une vitesse $\vec{V}$ par rapport à $(K')$, la fonction de Lagrange devient :
\bea
L' = \alpha\vec{v'}^{\,2} & = & \alpha.(\vec{v} + \vec{V})^{2} \nonumber \\
& = & \alpha\vec{v}^{\,2} + 2\alpha\vec{v}.\vec{V} + \alpha\vec{V}^{\,2} \nonumber \\
& = & L + 2\alpha\dfrac{{\rm d}\vec{r}}{{\rm dt}}.\vec{V} + \alpha\dfrac{{\rm dt}}{{\rm dt}}\vec{V}^{\,2} \nonumber \\
& = & L + \dfrac{{\rm d}}{{\rm dt}}\left(2\alpha\vec{r}.\vec{V} + \alpha\vec{V}^{\,2}{\rm t}\right)
\eea
dont le second terme est une d\'eriv\'ee totale par rapport au temps d'une fonction qui ne d\'epend que de la position et du temps. Ce second terme peut donc être omis, voir (\ref{EQ:2_8}) et la fonction de Lagrange devient, en posant $\alpha=m/2$ :
\be
	L = \dfrac{m\vec{v}^{\,2}}{2} \label{EQ:4_1}
\ee
où $m$ est appel\'ee $masse$ du point mat\'eriel. Par additivité de la fonction de Lagrange dans le cadre de particules sans interactions :
\be
	L = \sum_{a}\dfrac{m_{a}\vec{v_{a}}^{\,2}}{2} \label{EQ:4_2}
\ee
Un point int\'eressant est que la masse ne peut \^etre n\'egative. En effet, dans un tel cas, l'action :
\be
	S = \int_{{\rm t_{1}}}^{{\rm t_{2}}}\dfrac{m\vec{v}^{\,2}}{2}{\rm dt}
\ee
ne peut alors pas avoir de minimum et le principe de moindre action ne peut pas \^etre rempli !

De mani\`ere \'evidente, le carr\'e de la vitesse s'\'ecrit en fonction de la longueur $l$ d'un arc tel que :
\be
	\vec{v}^{\,2} = \left(\dfrac{{\rm d}l}{{\rm dt}}\right)^{2} = \dfrac{{{\rm d}l}^{2}}{{\rm dt}^{2}} \label{EQ:4_3}
\ee
Cela permet d'\'ecrire la fonction de Lagrange :
\begin{itemize}
	\item en coordonn\'ees cart\'esiennes :
		\bea
			{{\rm d}l}^{2} & = & {{\rm d}x}^{2} + {{\rm d}y}^{2} + {{\rm d}z}^{2} \nonumber \\
			\Rightarrow L & = & \dfrac{m}{2}\left(\dot{x}^{2} + \dot{y}^{2} + \dot{z}^{2}\right) \label{EQ:4_4}
		\eea
	\item en coordonn\'ees cylindriques :
		\bea
			{{\rm d}l}^{2} & = & {{\rm d}r}^{2} + r^{2}{{\rm d}\phi}^{2} + {{\rm d}z}^{2} \nonumber \\
			\Rightarrow L & = & \dfrac{m}{2}\left(\dot{r}^{2} + r^{2}\dot{\phi}^{2} + \dot{z}^{2}\right) \label{EQ:4_5}
		\eea
	\item en coordonn\'ees sph\'eriques :
		\bea
			{{\rm d}l}^{2} & = & {{\rm d}r}^{2} + r^{2}{{\rm d}\theta}^{2} + r^{2}\sin^{2}(\theta){{\rm d}\varphi}^{2} \nonumber \\
			\Rightarrow L & = & \dfrac{m}{2}\left(\dot{r}^{2} + r^{2}\dot{\theta}^{2} + r^{2}\sin^{2}(\theta)\dot{\varphi}^{2}\right) \label{EQ:4_6}
		\eea
\end{itemize}

\section{Fonction de Lagrange d'un syst\`eme de points mat\'eriels}

Dans ce cas, un syst\`eme est ferm\'e s'il est compos\'e de points mat\'erielsr\'eagissant les uns avec les autres mais isol\'es de tout corps ext\'erieur. L'int\'eraction entre les points mat\'eriels peut \^etre d\'ecrite en ajoutant à la fonction de Lagrange une fonction d\'ependante des seules coordonn\'ees des particules. Ainsi, en reprenant (\ref{EQ:4_2}) :
\be
	L = \sum_{a}\dfrac{m_{a}\vec{v_{a}}^{\,2}}{2} - U\left(\begin{Bmatrix}\vec{r_{i}}\end{Bmatrix}_{1}^{s}\right) = T - U \label{EQ:5_1}
\ee
avec $T$ l'\'energie cin\'etique et $U$ l'\'energie potentielle du syst\`eme.

En m\'ecanique classique, les int\'eractions entre particules sont instantan\'ees (uniformit\'e du temps et relativit\'e de Galil\'ee). De la m\^eme mani\`ere, le changement de $t$ en $-t$ ne change ne la fonction de Lagrange, ni les \'equations du mouvement, c'est la r\'eversibili\'e de la m\'ecanique classique.

En passant l'\'equation (\ref{EQ:2_6}) des coordonn\'ees g\'en\'eralis\'ees en coordonn\'ees cart\'esiennes, nous avons pour la particule $a$ :
\be
	\dfrac{{\rm d}}{{\rm dt}}\left(\dfrac{\partial L}{\partial \vec{v}_{a}}\right) = \dfrac{\partial L}{\partial \vec{r}_{a}} \label{EQ:5_2}
\ee
En introduisant la fonction de Lagrange de (\ref{EQ:5_1}) :
\be
	\dfrac{\partial L}{\partial\vec{v}_{a}} = \dfrac{m_{a}}{2}\dfrac{\partial\vec{v}_{a}^{\,2}}{\partial\vec{v}_{a}} - \dfrac{\partial U}{\partial\vec{v}_{a}} = m_{a}\vec{v}_{a}
\ee
car l'\'energie potentielle ne d\'epend que des coordonn\'ees donc $\dfrac{\partial U}{\partial\vec{v}_{a}} = 0$. Et :
\bea
	\dfrac{\partial L}{\partial\vec{r}_{a}} & = & \dfrac{m_{a}}{2}\dfrac{\partial\vec{v}_{a}^{\,2}}{\partial\vec{r}_{a}} - \dfrac{\partial U}{\partial\vec{r}_{a}} \nonumber \\
	& = & \dfrac{m_{a}}{2}\dfrac{\partial}{\partial\vec{r}_{a}}\left(\dfrac{{\rm d}\vec{r}_{a}}{{\rm dt}}\right)^{2} - \dfrac{\partial U}{\partial\vec{r}_{a}} \nonumber \\
	& = & \dfrac{m_{a}}{2}\left[\dfrac{{\rm d}}{{\rm dt}}\left(\dfrac{\partial\vec{r}_{a}}{\partial\vec{r}_{a}}\right)\right]^{2} - \dfrac{\partial U}{\partial\vec{r}_{a}} \nonumber \\
	& = & - \dfrac{\partial U}{\partial\vec{r}_{a}}
\eea
car $\dfrac{\partial\vec{r}_{a}}{\partial\vec{r}_{a}} = 1$ ! Nous obtenons ainsi :
\be
	m_{a}\dfrac{{\rm d}\vec{r}_{a}}{{\rm dt}} = - \dfrac{\partial U}{\partial\vec{r}_{a}} \label{EQ:5_3}
\ee
Il s'agit des \emph{\'equations de Newton} et le vecteur :
\be
	\vec{F}_{a} = - \dfrac{\partial U}{\partial\vec{r}_{a}} \label{EQ:5_4}
\ee
est appel\'e la \emph{force} agissant sur le point mat\'eriel $a$. L'\'energie potentielle $U$ ne d\'ependant que des coordonn\'ees alors les acc\'el\'erations, voir (\ref{EQ:5_3}), ne d\'ependent \'egalement que des coordonn\'ees.
Si l'\'energie potentielle est d\'efinie \`a une constante pr\`es telle que $U'(\vec{r}_{a}) = U(\vec{r}_{a}) + \alpha$ alors l'\'equation (\ref{EQ:5_3}) ne change pas car $\dfrac{\partial U'}{\partial\vec{r}_{a}} = \dfrac{\partial U}{\partial\vec{r}_{a}}$. Pr\'ef\'erentiellement, $\alpha$ est choisie de telle mani\`ere \`a ce que l'\'energie potentielle tende vers z\'ero quand la distance entre les points mat\'eriels augmentent.

Pour passer des coordonn\'ees cart\'esiennes au coordonn\'ees g\'en\'eralis\'ees, il faut d\'efinir une transformation ind\'ependante du temps telle que :
\bea
	x_{a} & = & f_{a}\left(\begin{Bmatrix}\dot{q}_{i}\end{Bmatrix}_{1}^{s}\right) \\
	\Rightarrow \dot{x}_{a} = \dfrac{{\rm d}x_{a}}{{\rm dt}} & = & \sum_{k=1}^{s}\dfrac{\partial f_{a}(q_{k})}{\partial q_{k}}\dfrac{{\rm d}q_{k}}{{\rm dt}}
\eea
en employant la diff\'erentiel totale.

En reportant cela dans l'\'equation (\ref{EQ:5_1}), nous avons :
\bea
	L & = & \dfrac{1}{2}\sum_{a}m_{a}(\dot{x}^{2} + \dot{y}^{2} + \dot{z}^{2}) - U \nonumber \\
	& = & \dfrac{1}{2}\sum_{a}m_{a}\left[\left(\sum_{k=1}^{s}\dfrac{\partial f_{a}(q_{k})}{\partial q_{k}}\dot{q}_{k}\right)^{2} + \left(\sum_{k=1}^{s}\dfrac{\partial g_{a}(q_{k})}{\partial q_{k}}\dot{q}_{k}\right)^{2} + \left(\sum_{k=1}^{s}\dfrac{\partial h_{a}(q_{k})}{\partial q_{k}}\dot{q}_{k}\right)^{2}\right] - U(q) \nonumber \\
	& = & \dfrac{1}{2}\sum_{a}m_{a}\left[\sum_{i,j=1}^{s}\dfrac{\partial f_{a}(q_{i})}{\partial q_{i}}\dfrac{\partial f_{a}(q_{j})}{\partial q_{j}}\dot{q}_{i}\dot{q}_{j} + \sum_{i,j=1}^{s}\dfrac{\partial g_{a}(q_{i})}{\partial q_{i}}\dfrac{\partial g_{a}(q_{j})}{\partial q_{j}}\dot{q}_{i}\dot{q}_{j} + \sum_{i,j=1}^{s}\dfrac{\partial h_{a}(q_{i})}{\partial q_{i}}\dfrac{\partial h_{a}(q_{j})}{\partial q_{j}}\dot{q}_{i}\dot{q}_{j}\right] - U(q) \nonumber
\eea
Nous avons donc la fonction de Lagrange sous la forme :
\be
	L = \dfrac{1}{2}\sum_{a}m_{a}\left[\sum_{i,j=1}^{s}a_{ij}(q)\dot{q}_{i}\dot{q}_{j}\right] - U(q) \label{EQ:5_5}
\ee
avec :
\be
	a_{ij}(q) = \dfrac{\partial f_{a}(q_{i})}{\partial q_{i}}\dfrac{\partial f_{a}(q_{j})}{\partial q_{j}} + \dfrac{\partial g_{a}(q_{i})}{\partial q_{i}}\dfrac{\partial g_{a}(q_{j})}{\partial q_{j}} + \dfrac{\partial h_{a}(q_{i})}{\partial q_{i}}\dfrac{\partial h_{a}(q_{j})}{\partial q_{j}}
\ee
o\`u les c{\oe}fficients $a_{ij}$ ne d\'ependent que des coordonn\'ees. Aussi dans un syst\`eme de coordonn\'ees g\'en\'eralis\'ees, l'\'energie cin\'etique est fonction quadratique des vitesses mais peut aussi d\'ependre des coordon\'ees.

Consid\'erons le syst\`eme $A$ non ferm\'e et interagissant avec le syst\`eme $B$ anim\'e d'un mouvement potentiellement fonction du temps. $A$ se meut dans le champ cr\'e\'e par $B$ et le syst\`eme $A+B$ est ferm\'e. La fonction de Lagrange s'\'ecrit alors :
\be
	L_{A+B} = T_{A}(q_{A},\dot{q}_{A}) + T_{B}(q_{B},\dot{q}_{B}) - U(q_{A},q_{B})
\ee
L'\'energie cin\'etique $T_{B}$ n'est fonction que du temps et est donc le r\'esultat de la d\'eriv\'ee totale par rapport au temps d'une fonction, voir (\ref{EQ:2_8}). Par cons\'equent, elle peut \^etre soustrait de la fonction de Lagrange totale. Nous avons donc :
\be
	L_{A+B} = T_{A}(q_{A},\dot{q}_{A}) - U(q_{A},q_{B}({\rm t}))
\ee
avec $U$ qui d\'epend explicitement des coordonn\'ees et du temps.

Dans le cas particulier d'une unique particule dans un champ ext\'erieur, la fonction de Lagrange est :
\be
	L = \dfrac{mv^{2}}{2} - U(\vec{r},{\rm t}) \label{EQ:5_6}
\ee
et l'\'equation de Newton (\ref{EQ:5_3}) devient :
\be
	m\dfrac{{\rm d}\vec{v}}{{\rm dt}} = -\dfrac{\partial U(\vec{r},{\rm t})}{\partial \vec{r}} \label{EQ:5_7}
\ee
Dans le cadre d'un champ uniforme ($U$ ne d\'epend plus du temps), la force d\'efinie en (\ref{EQ:5_4}) devient :
\bea
	\vec{F} & = & -\dfrac{{\rm d} U(\vec{r})}{{\rm d}\vec{r}} \nonumber \\
	\vec{F}.d\vec{r} & = & - {\rm d} U(\vec{r}) \nonumber \\
	\vec{F}.\vec{r} & = & - U(\vec{r}) \label{EQ:5_8}
\eea
\chapter{Lois de conservation}

\section{\'Energie}

Les \emph{int\'egrales premi\`eres} ou \emph{int\'egrales du mouvement}, sont les fonctions de $q_{i}$ et $\dot{q}_{i}$ qui conservent une valeur constante au cours du mouvement. Elles sont additives, c'est-\`a-dire qu'avant et après une int\'eraction, leur somme garde une valeur identique.

Commen\c{c}ons par l'int\'egrale premi\`ere qui d\'ecoule de l'\emph{uniformi\'e du temps}.

Pour un syst\`eme ferm\'e \`a $s$ degr\'es de libert\'e, nous avons :
\be
	\dfrac{\mathrm{d}L}{\mathrm{dt}} = \sum_{i=1}^{s}\dfrac{\partial L}{\partial q_{i}}\dfrac{\mathrm{d} q_{i}}{\mathrm{dt}} + \sum_{i=1}^{s}\dfrac{\partial L}{\partial \dot{q}_{i}}\dfrac{\mathrm{d}\dot{q}_{i}}{\mathrm{dt}} + \dfrac{\partial L}{\partial \mathrm{t}}
\ee
L'uniformit\'e du temps donne $\dfrac{\partial L}{\partial \mathrm{t}} = 0$ et en utilisant l'\'equation (\ref{EQ:2_6}), nous avons :
\bea
	\dfrac{\mathrm{d}L}{\mathrm{dt}} & = & \sum_{i=1}^{s}\dfrac{\mathrm{d}}{\mathrm{dt}}\left(\dfrac{\partial L}{\partial \dot{q}_{i}}\dot{q}_{i}\right) + \sum_{i=1}^{s}\dfrac{\partial L}{\partial \dot{q}_{i}}\ddot{q}_{i} \nonumber \\
	& = & \sum_{i=1}^{s}\dfrac{\mathrm{d}}{\mathrm{dt}}\left(\dfrac{\partial L}{\partial \dot{q}_{i}}\dot{q}_{i}\right) \nonumber \\
	0 & = & \dfrac{\mathrm{d}}{\mathrm{dt}}\left(\sum_{i=1}^{s}\dfrac{\partial L}{\partial \dot{q}_{i}}\dot{q}_{i} - L\right)
\eea
Par cons\'equent, la quantit\'e :
\be
	E = \sum_{i=1}^{s}\dfrac{\partial L}{\partial \dot{q}_{i}}\dot{q}_{i} - L \label{EQ:6_1}
\ee
est constante dans le temps pour un syst\`eme ferm\'e. $E$ est l'\emph{\'energie} du syst\`eme. Et puisque la fonction de Lagrange est additive, l'\'energie l'est aussi. Les syst\`emes m\'ecaniques dont l'\'energie se conserve sont appel\'es \emph{syst\`emes conservatifs}. Enfin, si le syst\`eme est plac\'e dans un champ ext\'erieur constant par rapport au temps alors la loi de conservation de l'\'energie reste valable car la fonction de Lagrange est implicitement ind\'ependant du temps.

\`A partir de l'\'equation (\ref{EQ:5_1}, nous pouvons \'ecrire :
\be
	\sum_{i=1}^{s}\dfrac{\partial L}{\partial\dot{q}_{i}}\dot{q}_{i} = \sum_{i=1}^{s}\dfrac{\partial (T-U)}{\partial\dot{q}_{i}}\dot{q}_{i} = \sum_{i=1}^{s}\dfrac{\partial T}{\partial\dot{q}_{i}}\dot{q}_{i}
\ee
car l'\'energie potentielle $U$ ne d\'epend que des coordonn\'ees (voire du temps).
Nous savons que l'\'energie cin\'etique $T = f(\dot{q}^{2})$. Or pour une fonction homog\`ene de degr\'e $\alpha$, telle $f(t.x) = t^{\alpha}f(x)$, alors le th\'eor\`eme d'Euler donne pour cette fonction :
\be
	\sum_{i} x_{i}\dfrac{\partial f(x)}{\partial x_{i}} = \alpha f(x)
\ee
Appliqu\'ee \`a l'\'energie cin\'etique, cela donne :
\be
	\sum_{i=1}^{s}\dfrac{\partial T}{\partial\dot{q}_{i}}\dot{q}_{i} = 2T = \sum_{i=1}^{s}\dfrac{\partial L}{\partial\dot{q}_{i}}\dot{q}_{i}
\ee
ce qui permet de conclure \`a, en compl\'etant l'\'equation (\ref{EQ:6_1}) :
\be
	E = T(q,\dot{q}) + U(q) \label{EQ:6_2}
\ee
soit en coordonn\'ees cart\'esiennes :
\be
	E = \sum_{a}\dfrac{m_{a}\vec{v}_{a}^{\,2}}{2} + U(\begin{Bmatrix}\vec{r}_{i}\end{Bmatrix}_{1}^{s}) \label{EQ:6_3}
\ee

\section{Impulsion}
\chapter{Int\'egration des \'equations du mouvement}

\section{Mouvement lin\'eaire}

L'\'equation (\ref{EQ:5_5}) appliqu\'ee \`a un point mat\'eriel donne en coordonn\'ees g\'en\'eralis\'ees :
\be
	L = \frac{1}{2}a(q)\dot{q}^{2} - U(q) \label{EQ:11_1}
\ee
qui s'\'ecrit en coordonn\'ees cart\'esiennes :
\be
	L = \frac{1}{2}m\dot{x}^{2} - U(x) \label{EQ:11_2}
\ee
En \'ecrivant l'\'energie totale :
\bea
	E = T + U & = & \frac{1}{2}m\dot{x}^{2} + U(x) \nonumber \\
	\dot{x}^{2} & = & \frac{2}{m}(E-U(x)) \nonumber \\
	\dfrac{\mathrm{d}x}{\mathrm{dt}} & = & \sqrt{\frac{2}{m}(E-U(x))} \nonumber \\
	t & = & \sqrt{\frac{m}{2}}\bigintsss{\dfrac{\mathrm{d}x}{\sqrt{E-U(x)}}} + cste \label{EQ:11_3}
\eea
o\`u $E$ et $cste$ sont des constantes du mouvement, la premi\`ere par la loi de conservation de l'\'energie et la seconde par int\'egration.

Lors d'un mouvement, l'\'energie totale est toujours sup\'erieure \`a l'\'energie potentielle car l'\'energie cint\'etique ne peut \^etre n\'egative. Aussi le mouvement ne peut \^etre possible que si $U(x)<E$.

\begin{figure}[htb!]
	\begin{center}
		\begin{picture}(200,150)(0,0)
			%axis
			\linethickness{0.05mm}
			\put(0,0){\line(1,0){200}}\put(202,-2){$x$}
			\put(0,0){\line(0,1){140}}\put(-2,142){$U$}
			%curve
			\linethickness{0.5mm}
			\qbezier(-5,130),(15,130),(25,110)
			\qbezier(25,110),(35,90),(55,90)
			\qbezier(55,90),(80,90),(90,100)
			\qbezier(90,100),(100,110),(125,110)
			\qbezier(125,110),(150,110),(190,20)
			%limit
			\linethickness{0.05mm}
			\multiput(-5,100)(10,0){20}{\line(1,0){8}}\put(195,98){$U=E$}
			\multiput(32,0)(0,5){20}{\line(0,1){3}}\put(34,102){$A$}\put(28,-7){$x_{1}$}
			\multiput(90,0)(0,5){20}{\line(0,1){3}}\put(82,102){$B$}\put(86,-7){$x_{2}$}
			\put(142,102){$C$}
		\end{picture}
		\caption{Exemple d'une fonction d\'efinissant l'\'energie potentielle}\label{FIG:3_1}
	\end{center}
\end{figure}

Sur la figure (\ref{FIG:3_1}), la condition $U(x)<E$ implique que le mouvement n'est possible que sur les intervalles tels que $x\in\left[A\,;B\right]$ et $x\in\left[C\,;+\infty\right[$. Dans le premier cas, le mouvement est dit fini et oscillatoire. Dans le second cas, le mouvement est dit inifini. Les points tels que :
\be
	U(x) = E \label{EQ:11_4}
\ee
sont les \emph{points d'arr\^et}. En effet, dans ce cas, l'\'energie cin\'etique du point mat\'eriel est nulle car sa vitesse aussi de facto.

Dans la relation (\ref{EQ:5_1}), nous avons vu que le temps n'est pas qu'uniforme, il est aussi isotrope. Ainsi la transformation $\mathrm{t}\mapsto -\mathrm{t}$ laisse inchang\'ee la fonction de Lagrange et les \'equations du mouvement. Ainsi $\Delta\mathrm{t}(x_{1}\rightarrow x_{2}) = \Delta\mathrm{t}(x_{2}\rightarrow x_{1})$ et la période d'oscillation $\mathrm{T}(E)=\Delta\mathrm{t}(x_{1}\rightarrow x_{2}\rightarrow x_{1})$ vaut $2\Delta\mathrm{t}(x_{1}\rightarrow x_{2})$. L'\'equation (\ref{EQ:11_3}) s'\'ecrit alors :
\bea
	\mathrm{T}(E) & = & 2\sqrt{\frac{m}{2}}\bigintsss_{x_{1}(E)}^{x_{2}(E)}{\dfrac{\mathrm{d}x}{\sqrt{E-U(x)}}} \nonumber \\
	\mathrm{T}(E) & = & \sqrt{2m}\bigintsss_{x_{1}(E)}^{x_{2}(E)}{\dfrac{\mathrm{d}x}{\sqrt{E-U(x)}}}\label{EQ:11_5}
\eea
qui d\'etermine la p\'eriode d'oscillation de la particule mat\'erielle en fonction de son \'energie totale.

\section{D\'efinition de l'\'energie potentielle en fonction de la p\'eriode des oscillations}

\begin{figure}[htb!]
	\begin{center}
		\begin{picture}(200,150)(0,0)
			%axis
			\linethickness{0.05mm}
			\put(0,0){\line(1,0){200}}\put(202,-2){$x$}
			\put(100,0){\line(0,1){140}}\put(98,142){$U$}
			%curve
			\linethickness{0.5mm}
			\qbezier(35,120),(40,0),(100,0)
			\qbezier(100,0),(175,0),(180,120)
			\put(47,50){$x_{1}(U)$}
			\put(132,50){$x_{2}(U)$}
			%limit
			\linethickness{0.05mm}
			\multiput(-5,80)(10,0){20}{\line(1,0){8}}\put(195,78){$U=E$}
			\multiput(40,0)(0,5){16}{\line(0,1){3}}\put(35,-7){$x_{1}$}
			\multiput(175,0)(0,5){16}{\line(0,1){3}}\put(170,-7){$x_{2}$}
		\end{picture}
		\caption{Fonction d\'efinissant l'\'energie potentielle ayant qu'un unique minimum}\label{FIG:3_2}
	\end{center}
\end{figure}

L'objectif de ce paragraphe est de trouver la fonction d\'efinissant l'\'energie potentielle d'un champ $U$ connaissant la p\'eriode d'oscillations $\mathrm{T}(E)$ du mouvement animant une particule. Pour cela, nous supposons que la fonction recherch\'ee $U$ n'a qu'un unique minimum dans la r\'egion du mouvement et que, comme sur la figure (\ref{FIG:3_2}), $U(0) = 0$.

Transformons l'int\'egrale (\ref{EQ:11_5}) avec $x$ fonction de $U$ telle que pour une valeur de l'\'energie potentielle $U$ il existe deux valeurs de $x$ diff\'erentes. Elle devient alors la somme de deux int\'egrales sur deux r\'egions diff\'erentes :
\bea
	\mathrm{T}(E) & = & \sqrt{2m}\bigintsss_{x_{1}(E)}^{x_{2}(E)}\dfrac{\mathrm{d}x}{\sqrt{E - U(x)}} \nonumber \\
	& = & -\sqrt{2m}\bigintsss_{0}^{E}\dfrac{\mathrm{d}x_{1}}{\mathrm{d}U}\dfrac{\mathrm{d}U}{\sqrt{E - U}} + \sqrt{2m}\bigintsss_{0}^{E}\dfrac{\mathrm{d}x_{2}}{\mathrm{d}U}\dfrac{\mathrm{d}U}{\sqrt{E - U}} \nonumber \\
	& = & \sqrt{2m}\bigintsss_{0}^{E}\left[\dfrac{\mathrm{d}x_{2}}{\mathrm{d}U} - \dfrac{\mathrm{d}x_{1}}{\mathrm{d}U}\right]\dfrac{\mathrm{d}U}{\sqrt{E - U}} \nonumber
\eea
Elle peut \^etre \'etendue \`a :
\bea
	\bigintsss_{0}^{\alpha}\dfrac{\mathrm{T}(E)\mathrm{d}E}{\sqrt{\alpha - E}} & = & \sqrt{2m}\bigintsss_{0}^{\alpha}\bigintsss_{0}^{E}\left[\dfrac{\mathrm{d}x_{2}}{\mathrm{d}U} - \dfrac{\mathrm{d}x_{1}}{\mathrm{d}U}\right]\dfrac{\mathrm{d}U\mathrm{d}E}{\sqrt{(\alpha - E)(E - U)}} \nonumber \\
	& =& \sqrt{2m}\bigintsss_{0}^{\alpha}\left[\dfrac{\mathrm{d}x_{2}}{\mathrm{d}U} - \dfrac{\mathrm{d}x_{1}}{\mathrm{d}U}\right]\mathrm{d}U\bigintsss_{U}^{\alpha}\dfrac{\mathrm{d}E}{\sqrt{(\alpha - E)(E - U)}} \nonumber
\eea
La seconde implication dans sa deuxi\`eme int\'egrale est en $\int_{U}^{\alpha}$ car en dehors de cet intervalle, $\int_{0}^{U}$ et $\int_{\alpha}^{E}$ sont ind\'efinie car $(E-U)$ et $(\alpha-E)$ sont respectivement n\'egatives.

Faisons d\'esormais le calcul de la seconde int\'egrale, soit :
\benn
	\bigintsss_{U}^{\alpha}\dfrac{\mathrm{d}E}{\sqrt{(\alpha - E)(E - U)}}
\eenn
en posant :
\benn
	y = \dfrac{2E - U - \alpha}{\alpha - U}
\eenn
qui donne :
\be
	\begin{cases}
		E = \dfrac{(\alpha - U)y + U + \alpha}{2} \\
		\mathrm{d}y = \dfrac{2\mathrm{d}E}{\alpha - U} \Leftrightarrow \mathrm{d}E = \dfrac{\alpha - U}{2}\mathrm{d}y \nonumber
	\end{cases}
\ee
et :
\be
	\begin{cases}
		E = U \Rightarrow y = \dfrac{U - \alpha}{\alpha - U} = -1 \\
		E = \alpha \Rightarrow y = \dfrac{-U + \alpha}{\alpha - U} = 1 \nonumber
	\end{cases}
\ee
Ainsi :
\bea
	\bigintsss_{U}^{\alpha}\dfrac{\mathrm{d}E}{\sqrt{(\alpha - E)(E - U)}} & = & \bigintsss_{-1}^{1}\dfrac{(\alpha - U)\mathrm{d}y}{2\sqrt{\frac{(2\alpha - (\alpha - U)y + U + \alpha)}{2}\frac{((\alpha - U)y + U + \alpha - 2U)}{2}}} \nonumber \\
	& = & \bigintsss_{-1}^{1}\dfrac{(\alpha - U)\mathrm{d}y}{2\sqrt{\frac{(\alpha - U - (\alpha - U)y)}{2}\frac{(\alpha - U + (\alpha - U)y)}{2}}} \nonumber \\
	& = & \bigintsss_{-1}^{1}\dfrac{(\alpha - U)\mathrm{d}y}{2\frac{\alpha - U}{2}\sqrt{(1 - y)(1 + y)}} = \bigintsss_{-1}^{1}\dfrac{\mathrm{d}y}{\sqrt{1 - y^{2}}} \nonumber
\eea
En posant $u = y^{2}$ soit $y = u^{1/2}$ et $\mathrm{d}u = 2y\mathrm{d}y = 2u^{1/2}\mathrm{d}y \Leftrightarrow \mathrm{d}y = \frac{1}{2}u^{-1/2}\mathrm{d}u$, l'int\'egrale pr\'ec\'edente s'\'ecrit :
\bea
	\bigintsss_{U}^{\alpha}\dfrac{\mathrm{d}E}{\sqrt{(\alpha - E)(E - U)}} & = & \bigintsss_{-1}^{1}\dfrac{\mathrm{d}y}{\sqrt{1 - y^{2}}} = 2 \bigintsss_{0}^{1}\dfrac{1}{2}\dfrac{u^{-1/2}\mathrm{d}u}{(1-u)^{1/2}} \nonumber \\
	& = & \int_{0}^{1}u^{-1/2}(1-u)^{-1/2}\mathrm{d}u \nonumber \\
	& = & \mathrm{B}\left(\dfrac{1}{2},\dfrac{1}{2}\right) = \dfrac{\Gamma(\dfrac{1}{2})\Gamma(\dfrac{1}{2})}{\Gamma(1)} = \pi \nonumber
\eea
De fait :
\bea
	\bigintsss_{0}^{\alpha}\dfrac{\mathrm{T}(E)\mathrm{d}E}{\sqrt{\alpha - E}} & = & \sqrt{2m}\pi\bigintsss_{0}^{\alpha}\left[\dfrac{\mathrm{d}x_{2}}{\mathrm{d}U} - \dfrac{\mathrm{d}x_{1}}{\mathrm{d}U}\right]\mathrm{d}U \nonumber \\
	& = & \sqrt{2m}\pi\left[x_{2}(\alpha) - x_{2}(0) - x_{1}(\alpha) + x_{1}(0)\right] \nonumber
\eea
Or par hypoth\`ese, $x_{2}(0) = x_{1}(0) = 0$, donc :
\benn
	\bigintsss_{0}^{\alpha}\dfrac{\mathrm{T}(E)\mathrm{d}E}{\sqrt{\alpha - E}} = \sqrt{2m}\pi\left[x_{2}(\alpha) - x_{1}(\alpha)\right]
\eenn
Dans le cas o\`u nous posons $\alpha = U$, nous pouvons en d\'eduire :
\be
	x_{2}(U) - x_{1}(U) = \dfrac{1}{\sqrt{2m}\pi}\bigintsss_{0}^{U}\dfrac{\mathrm{T}(E)\mathrm{d}E}{\sqrt{U - E}} \label{EQ:12_1}
\ee
De plus, si le champ d'\'energie potentielle est paire tel que $U(x) = U(-x)$ ou encore $x_{1}(U) = -x_{2}(U) = x(U)$, alors l'\'equation (\ref{EQ:12_1}) devient, puisque $x_{2}(U) - x_{1}(U) = 2x(U)$ :
\be
	x(U) = \dfrac{1}{2\pi\sqrt{2m}}\bigintsss_{0}^{U}\dfrac{\mathrm{T}(E)\mathrm{d}E}{\sqrt{U - E}} \label{EQ:12_2}
\ee

\section{Masse r\'eduite}

L'important probl\`eme du mouvement s'un syst\`eme ferm\'e \`a deux particules r\'eagissant l'une sur l'autre, ou encore \emph{probl\`eme des deux corps}, admet une solution g\'en\'erale compl\`ete. Pour le r\'esoudre, nous allons le simplifier en d\'ecomposant le mouvement du syst\`eme en celui du centre d'inertie et des deux points mat\'eriels par rapport \`a ce dernier. L'\'energie potentielle d'int\'eraction ne d\'epend que de la distance entre les deux particules, aussi la fonction de Lagrange s'\'ecrit :
\be
	L = \dfrac{m_{1}\dot{r}_{1}^{2}}{2} + \dfrac{m_{2}\dot{r}_{2}^{2}}{2} - U(\lvert \vec{r}_{1} - \vec{r}_{2} \rvert) \label{EQ:13_1}
\ee
D\'efinissons $\vec{r} = \vec{r}_{1} - \vec{r}_{2}$ et pla\c{c}ons l'origine des coordonn\'ees au centre d'inertie. Alors l'\'equation (\ref{EQ:8_3}) donne $\vec{R} = \vec{0}$ ou encore : $\sum m_{a}\vec{r}_{a} = \vec{0}$, soit :
\benn
	m_{1}\vec{r}_{1} + m_{2}\vec{r}_{2} = \vec{0}
\eenn
Nous pouvons en conclure que :
\be
	\vec{r} = \begin{cases}
		\vec{r}_{1} + \dfrac{m_{1}}{m_{2}}\vec{r}_{1} = \dfrac{m_{1} + m_{2}}{m_{2}}\vec{r}_{1} \\
		- \dfrac{m_{2}}{m_{1}}\vec{r}_{2} - \vec{r}_{2} = -\dfrac{m_{1} + m_{2}}{m_{1}}\vec{r}_{2} \nonumber
	\end{cases}
\ee
soit :
\be
	\begin{cases}
		\vec{r}_{1} = \dfrac{m_{2}}{m_{1} + m_{2}}\vec{r} \\
		\vec{r}_{2} = \dfrac{-m_{1}}{m_{1} + m_{2}}\vec{r}
	\end{cases}\label{EQ:13_2}
\ee
Puisque :
\be
	\begin{cases}
		\vec{\dot{r}}_{1} = \dfrac{m_{2}}{m_{1} + m_{2}}\vec{\dot{r}} \\
		\vec{\dot{r}}_{2} = \dfrac{-m_{1}}{m_{1} + m_{2}}\vec{\dot{r}} \nonumber
	\end{cases}
\ee
la fonction de Lagrange se d\'eveloppe ainsi :
\bea
	L & = & \dfrac{m_{1}}{2}\left(\dfrac{m_{2}}{m_{1} + m_{2}}\right)^{2}\vec{\dot{r}}^{\,2} + \dfrac{m_{2}}{2}\left(\dfrac{-m_{1}}{m_{1} + m_{2}}\right)^{2}\vec{\dot{r}}^{\,2} - U(\vec{r}) \nonumber \\
	& = & \dfrac{m_{1}m_{2}^{2} + m_{1}^{2}m_{2}}{2(m_{1} + m_{2})^{2}}\vec{\dot{r}}^{\,2} - U(\vec{r}) = \dfrac{m_{1}m_{2}(m_{2} + m_{1})}{2(m_{1} + m_{2})^{2}}\vec{\dot{r}}^{\,2} - U(\vec{r}) \nonumber
\eea
et on peut conclure en posant \emph{la masse r\'eduite} :
\be
	m = \dfrac{m_{1}m_{2}}{(m_{1} + m_{2})} \label{EQ:13_4}
\ee
\`a :
\be
	L = \dfrac{m}{2}\vec{\dot{r}}^{\,2} - U(\vec{r}) \label{EQ:13_3}
\ee
qui est la fonction de Lagrange d'un point mat\'eriel de masse $m$ se d\'epla\c{c}ant dans le champ ext\'erieur $U(\vec{r})$ qui est sym\'etrique par rapport au point immobile des coordonn\'ees, soit le centre d'inertie. Ainsi le calcul de $\vec{r} = \vec{r}(\mathrm{t})$ permet gr\^ace aux \'equations (\ref{EQ:13_2}) d'en d\'eduire les trajectoires $\vec{r}_{1}(\mathrm{t})$ et $\vec{r}_{2}(\mathrm{t})$ des particules de masse respective $m_{1}$ et $m_{2}$.

\section{Mouvement dans un champ central}

Le fait de ramener le probl\`eme \`a deux corps \`a celui d'un seul nous conduit \`a devoir d\'eterminer le mouvement d'une particule dans un champ ext\'erieur o\`u son \'energie potentielle ne d\'epend que de la distance $r$ \`a un point immobile. Ce champ est appel\'e \emph{champ central}. De la relation (\ref{EQ:5_4}), la force est donn\'ee par :
\benn
	\vec{F} = -\dfrac{\partial U(r)}{\partial\vec{r}} = -\dfrac{\mathrm{d}U(r)}{\mathrm{d}r}\dfrac{\vec{r}}{r}
\eenn
et agissant sur la particule, elle ne d\'epend aussi que de $r$.

Par d\'efinition, \'equation (\ref{EQ:9_3}), le moment cin\'etique est :
\benn
	\vec{M} = \vec{r}\wedge\vec{p}
\eenn
et dans un champ central, il se conserve dans un r\'ef\'erentiel dont un axe passe par le centre et ce dernier est le centre du r\'ef\'erentiel. Dans ce cas de conservation et parce que $\vec{M}\perp\vec{p}$ alors le mouvement de la particule reste toujours dans le m\^eme plan perpendiculaire au moment cin\'etique. Ainsi, en coordonn\'ees polaires (cylindriques), nous avons de facto $\dot{z} = 0$ et la fonction de Lagrange est, voir l'\'equation (\ref{EQ:4_5}) :
\be
	L = \dfrac{m}{2}(\dot{r}^{2} + r^{2}\dot{\varphi}^{2}) - U(r) \label{EQ:14_1}
\ee
La coordonn\'ee $\varphi$ est dite \emph{cyclique} car elle n'appara\^it pas explicitement dans la d\'efinition de la fonction de Lagrange, alors que $r$ et $\dot{\varphi}$ en sont des coordonn\'ees \emph{principales}. Pour la coordonn\'ee $\varphi$, les \'equations de Lagrange (\ref{EQ:2_6}) sont :
\benn
	\dfrac{\mathrm{d}}{\mathrm{dt}}\left(\dfrac{\partial L}{\partial\dot{q}_{i}}\right) = \dfrac{\partial L}{\partial q_{i}} = 0
\eenn
Ainsi, l'impulsion g\'en\'eralis\'ee, \'equation (\ref{EQ:7_5}) :
\benn
	p_{i} = \dfrac{\partial L}{\partial\dot{q}_{i}}
\eenn
est constante dans le temps, i.e. $p_{i}$ est une int\'egrale premi\`ere, soit une int\'egrale du mouvement qui conserve sa valeur pendant le mouvement. L'\'equation (\ref{EQ:14_1}) permet d'\'ecrire\footnote{Il s'agit bien de $p_{\varphi}$ et non pas de $\dot{p}_{\varphi}$ commme inscrit dans le livre.} :
\benn
	\dfrac{\partial L}{\partial\dot{q}_{i}} = mr^{2}\dot{\varphi} = p_{\varphi}
\eenn
et la valeur de $p_{\varphi}$ se conserve dans le temps. En appliquant l'\'equation (\ref{EQ:9_7}) car nous avons choisi le r\'ef\'erentiel tel que la particule tourne dans un plan perpendiculaire \`a un axe passant par le centre du champ :
\bea
	M_{z} & = & \dfrac{\partial L}{\partial\dot{\varphi}_{i}} \nonumber \\
	\Leftrightarrow M & = & mr^{2}\dot{\varphi} \label{EQ:14_2}
\eea
Car les valeurs de $M$ et de $M_{z}$ se confondent par d\'efinition du choix fait pr\'ec\'edemment. La valeur de $M$ est donc conserv\'ee \'egalement. $\dot{\varphi}$ ne changeant pas de signe au cours du mouvement\footnote{En effet, $M$ est une valeur constante, commme $m$ alors que $r^{2}$ est toujours positive.}, $\varphi$ \'evolue de mani\`ere monotone.

\begin{figure}[htb!]
	\begin{center}
		\begin{picture}(200,150)(0,0)
			%curve
			\linethickness{0.5mm}
			\qbezier(120,150),(150,125),(170,75)
			\qbezier(170,75),(180,50),(190,5)
			%vectors
			\linethickness{0.05mm}
			\put(-8,47){$0$}
			\put(0,50){\vector(1,0){178}}
			\put(75,80){$\vec{r}$}
			\put(0,50){\vector(8,2){163}}
			%tangent
			\multiput(162,50)(0,5){8}{\line(0,1){3}}
			\put(135,65){$r\mathrm{d}\varphi$}
			%angle
			\qbezier(25,50),(25,55),(20,55)
			\put(27,52){$\mathrm{d}\varphi$}
		\end{picture}
		\caption{Aire balay\'ee}\label{FIG:3_7}
	\end{center}
\end{figure}

R\'efl\'echissons d\'esormais à l'interpr\'etation g\'eom\'etrique de la loi pr\'ec\'edente (\ref{EQ:14_2}) en remarquant que l'expression $\frac{1}{2}\vec{r}\cdot\vec{r}\mathrm{d}\varphi$ repr\'sente la surface form\'ee par les deux rayons vecteurs infiniment voisins et l'arc de la trajectoire de la figure (\ref{FIG:3_9}). En d\'esignant par $\mathrm{d}f$ cet \'el\'ement de surface, le moment cin\'etique de la particule devient :
\be
	M = 2m\dot{f} \label{EQ:14_3}
\ee
car en s'aidant de la relation (\ref{EQ:14_2}) :
\bea
	\mathrm{d}f & = & \dfrac{1}{2}\vec{r}\cdot\vec{r}\mathrm{d}\varphi = \dfrac{1}{2}r^{2}\mathrm{d}\varphi \nonumber \\
	\dot{f} = \dfrac{\mathrm{d}f}{\mathrm{dt}} & = & \dfrac{1}{2}r^{2}\dot{\varphi} \nonumber
\eea
La grandeur $\dot{f}$ est alors appel\'ee \emph{vitesse a\'erolaire}. Et comme le moment cin\'etique $M$ est constant pendant le mouvement alors la vitesse a\'erolaire est constante \'egalement, il s'agit de la \emph{seconde loi de Kepler}, \`a savoir que pendant des intervalles de temps \'egaux, le rayon vecteur d'un point mobile balaye des aires \'egales. Cette seconde loi est aussi appel\'ee \emph{int\'egrale des aires} qui est la loi de conservation du moment cin\'etique d'une particule en mouvement dans un champ central.

Pour obtenir la solution compl\`ete du mouvement d'une particule dans un champ central, utilisons les lois de conservation de l'\'energie (\'equation (\ref{EQ:6_1}) et du moment cin\'etique (\'equation (\ref{EQ:9_3}). D'apr\`es la relation (\ref{EQ:14_2}), $\dot{\varphi} = \frac{M}{mr^{2}}$, donc l'\'energie peut s'\'ecrire comme :
\be
	E = T + U = \dfrac{m}{2}(\dot{r}^{2} + r^{2}\dot{\varphi}^{2}) + U(r) = \dfrac{m}{2}\dot{r}^{2} + \dfrac{M^{2}}{2mr^{2}} + U(r) \label{EQ:14_4}
\ee
qui implique directement :
\be
	\dot{r} = \dfrac{\mathrm{d}r}{\mathrm{dt}} = \pm\sqrt{\frac{2}{m}(E - U(r)) - \frac{M^{2}}{m^{2}r^{2}}} \label{EQ:14_5}
\ee
En renversant, nous obtenons :
\benn
	\mathrm{dt} = \dfrac{\mathrm{d}r}{\sqrt{\frac{2}{m}(E - U(r)) - \frac{M^{2}}{m^{2}r^{2}}}}
\eenn
qui permet d'obtenir par intégration :
\be
	t = \bigintsss{\dfrac{\mathrm{d}r}{\sqrt{\frac{2}{m}(E - U(r)) - \frac{M^{2}}{m^{2}r^{2}}}}} + cste \label{EQ:14_6}
\ee
En reprenant l'\'equation (\ref{EQ:14_2}) sous la forme :
\bea
	M & = & mr^{2}\dot{\varphi} \nonumber \\
	\mathrm{d}\varphi & = & \dfrac{M}{mr^{2}}\mathrm{dt} \nonumber
\eea
et en l'injection dans la relation (\ref{EQ:14_5}), nous avons :
\bea
	\dfrac{\mathrm{d}r}{\mathrm{dt}} = \dfrac{M\mathrm{d}r}{mr^{2}\mathrm{d}\varphi} & = & \sqrt{\frac{2}{m}(E - U(r)) - \frac{M^{2}}{m^{2}r^{2}}} \nonumber \\
	\Leftrightarrow \mathrm{d}\varphi & = & \dfrac{M\mathrm{d}r}{mr^{2}\sqrt{\frac{2}{m}(E - U(r)) - \frac{M^{2}}{m^{2}r^{2}}}} \nonumber \\
	& = & \dfrac{\frac{M}{r^{2}}\mathrm{d}r}{\sqrt{2m(E - U(r)) - \frac{M^{2}}{r^{2}}}} \nonumber \\
	\Leftrightarrow \varphi & = & \bigintsss{\dfrac{\frac{M}{r^{2}}\mathrm{d}r}{\sqrt{2m(E - U(r)) - \frac{M^{2}}{r^{2}}}}} + cste \label{EQ:14_7}
\eea

Les formules (\ref{EQ:14_6}) et (\ref{EQ:14_7}) forment la solution g\'en\'erale. La premi\`ere donne implicitement la distance au centre $r$ en fonction du temps alors que la seconde \'etablit la relation entre $r$ et $\varphi$, soit l'\'equation du mouvement.

\begin{figure}[htb!]
	\begin{center}
		\includegraphics[width=10cm]{chapter_03_paragraph_14_fig_09}
		\caption{Trajectoire dans un champ central attracteur}\label{FIG:3_9}
	\end{center}
\end{figure}

L'\'equation (\ref{EQ:14_4}) :
\benn
	E = T + U = \dfrac{m}{2}\dot{r}^{2} + \dfrac{M^{2}}{2mr^{2}} + U(r)
\eenn
peut se r\'e\'ecrire en identifiant $\frac{m}{2}\dot{r}^{2}$ comme l'\'energie cin\'etique radiale de la particule comme :
\benn
	E = \dfrac{m}{2}\dot{r}^{2} + U_{\mathrm{eff}}(r)
\eenn
avec :
\be
	U_{\mathrm{eff}}(r) = \dfrac{M^{2}}{2mr^{2}} + U(r) \label{EQ:14_8}
\ee
se d\'efinissant comme une \'energie potentielle efficace o\`u le terme $\frac{M^{2}}{2mr^{2}}$ est l'\emph{\'energie centrifuge}. Ainsi, la partie radiale du mouvement de la particule peut \^etre consid\'er\'ee comme un mouvement lin\'eaire dans un champ dot\'e d'une \'energie potentielle efficace.

Les valeurs de la distance telle que l'\'energie cin\'etique radiale s'annule ou encore $E = U_{\mathrm{eff}}(r)$ d\'elimitent le domaine du mouvement de la particule car cela n\'ecessite de facto $\dot{r} = 0$. Cela d\'efinit les \emph{points de rebroussement} o\`u la particule peut encore avoir une vitesse angulaire $\dot{\varphi}\neq 0$. La fonction $r(\mathrm{t})$ devient ensuite croissante ou d\'ecroissante. Il existe alors deux possibilit\'es :
\begin{itemize}
	\item $r \gg r_{min}$ alors le mouvement est infini
	\item $r_{min} \leqslant r \leqslant r_{max}$ alors la trajectoire est contenue dans l'anneau d\'efini par les cercles de rayon $r_{min}$ et $r_{max}$
\end{itemize}
Dans le second cas, la variation angulaire $\Delta\varphi$ de la trajectoire de la particule pendant qu'elle fait le trajet $r_{max} \rightarrow r_{min} \rightarrow r_{max}$\footnote{Ou l'\'equivalent sym\'etrique $r_{min} \rightarrow r_{max} \rightarrow r_{min}$.} vaut d'apr\`es l'\'equation (\ref{EQ:14_7}) :
\be
	\Delta\varphi = 2\bigintsss_{r_{min}}^{r_{max}}{\dfrac{\frac{M}{r^{2}}\mathrm{d}r}{\sqrt{2m(E - U(r)) - \frac{M^{2}}{r^{2}}}}} \label{EQ:14_10}
\ee
La trajectoire de la particule est finie, i.e. elle se referme, si et seulement si $\Delta\varphi = \frac{m}{n}\times 2\pi$ avec ${m;n}\in\mathbb{N}$. Ainsi, quand le rayon vecteur parcourt $n$ p\'eriodes, il fait \'egalement $m$ tours complets et la trajectoire se referme. Toutefois, cela reste un cas particulier. En g\'en\'eral $U(r)$ ne permet par d'obtenir $\Delta\varphi$ comme une fraction rationnelle de $2\pi$ et la trajectoire ne se referme pas et finit par remplir tout l'espace de l'anneau compris entre $r_{min}$ et $r_{max}$. Finalement, la trajectoire est ferm\'ee pour les \'energies potentielles en $\frac{1}{r}$, voir le paragraphe \ref{PAR:15} et en $r^{2}$, voir l'exercice \ref{PAR:23_EX3}.

Aux points de rebroussement d\'efinis tels que $\dot{r} = 0$, il y a un changement de signe de la vitesse radiale et donc de la direction radiale prise par la particule. Que ce soit en $r_{min}$ ou $r_{max}$, la quantit\'e $\dfrac{\dot{r}(\mathrm{t} - \delta\mathrm{t})}{\dot{r}(\mathrm{t} + \delta\mathrm{t})}$ est n\'egative. La trajectoire est ainsi sym\'etrique par rapport \`a la direction indiqu\'ee.

Pour le mouvement d'une particule ayant une moment cin\'etique non nulle, la position $r=0$ est inatteignable car :
\benn
	\lim_{r\rightarrow 0}\dfrac{M^{2}}{2mr^{2}} \rightarrow +\infty
\eenn

L'\'equation (\ref{EQ:14_4}) implique de facto\footnote{Dans l'expression finale de cette s\'equence, le livre affiche le signe $<$ au lieu de $>$.} :
\bea
	E - U(r) - \dfrac{M^{2}}{2mr^{2}} & = & \dfrac{m\dot{r}^{2}}{2} > 0 \nonumber \\
	r^{2}E - r^{2}U(r) & > & \dfrac{M^{2}}{2m} \nonumber \\
	r^{2}E & > & r^{2}U(r) + \dfrac{M^{2}}{2m} \nonumber
\eea

Par conservation de l'\'energie, nous avons si $\lim_{r\rightarrow 0} r^{2}E = 0$, aussi $r$ ne peut tendre vers 0 que si :
\bea
	0 & > & \lim_{r\rightarrow 0}(r^{2}U(r)) + \dfrac{M^{2}}{2mr^{2}} \nonumber \\
	\lim_{r\rightarrow 0}(r^{2}U(r)) & < & -\dfrac{M^{2}}{2mr^{2}} \label{EQ:14_11}
\eea
Et cette derni\`ere in\'egalit\'e n'est vraie que dans deux situations et implique que $r$ peut ainsi tendre vers le centre du champ d'\'energie potentielle :
\begin{itemize}
	\item $U(r)=\dfrac{-\alpha}{r^{2}}$ avec $\alpha > \dfrac{M^{2}}{2mr^{2}}$
	\item $U(r)=\dfrac{-\alpha}{r^{n}}$ avec $n > 2$ car $\lim_{r\rightarrow 0}(r^{2}U(r)) = \lim_{r\rightarrow 0}\left(\dfrac{-\alpha}{r^{n-2}}\right) = -\infty < -\dfrac{M^{2}}{2mr^{2}}$
\end{itemize}

\section{Le probl\`eme de Kepler}\label{PAR:15}

\subsection{G\'en\'eralit\'es}

L'\'etude du champ central inversement proportionnel \`a $r$ est tr\`es important\footnote{Cela correspond \`a une force inversement proportionnel au carr\'e de la distance, voir l'\'equation (\ref{EQ:7_4}) donnant $\vec{F} = \dfrac{\partial L}{\partial U}$} car il s'agit du champ gravitationnel, attractif, et du champ \'electrostatique de Coulomb, attractif ou r\'epulsif. Ainsi, d\'efinissons un champ d'attraction tel que :
\be
	U(r) = -\dfrac{\alpha}{r} \label{EQ:15_1}
\ee
avec la constante $\alpha > 0$. Selon la relation (\ref{EQ:14_8}), l'\'energie potentielle efficace s'\'ecrit alors :
\be
	U_{eff}(r) = -\dfrac{\alpha}{r} + \dfrac{M^{2}}{2mr^{2}} \label{EQ:15_2}
\ee
qui est repr\'esent\'ee sur la figure (\ref{FIG:3_10}).

\begin{figure}[htb!]
	\begin{center}
		\includegraphics[width=10cm]{chapter_03_paragraph_15_fig_10}
		\caption{\'Energie potentielle attractrice pour $\alpha\in \{1;2;3;4;5\}$ et $M^{2}/m = 1$}\label{FIG:3_10}
	\end{center}
\end{figure}
Le minimum de la fonction pr\'ec\'edente s'obtient avec :
\bea
	\dfrac{\mathrm{d}U_{eff}(r)}{\mathrm{d}r} & = & 0 \Leftrightarrow \dfrac{\alpha}{r^{2}} - \dfrac{2M^{2}}{2mr^{3}} = 0 \nonumber \\
	\Leftrightarrow \dfrac{\alpha}{r^{2}} & = & \dfrac{M^{2}}{mr^{3}} \Leftrightarrow r = \dfrac{M^{2}}{\alpha m} \nonumber
\eea
et par cons\'equence, nous avons :
\be
	(U_{eff})_{min} = -\dfrac{\alpha^{2}m}{2M^{2}} \label{EQ:15_3}
\ee

D'apr\`es la courbe de la figure (\ref{FIG:3_10}), il est \'evident que pour $E > 0$, la trajectoire est infinie alors que pour $E < 0$, elle est finie. La forme g\'en\'erale de la trajectoire se d\'eduit de l'\'equation (\ref{EQ:14_7}) en y posant $U(r) = -\alpha / r$. Nous \'ecrivons alors :
\benn
	\varphi = \bigintsss{\dfrac{\dfrac{M}{r^{2}}\mathrm{d}r}{\sqrt{2m\left(E + \dfrac{\alpha}{r}\right) - \dfrac{M^{2}}{r^{2}}}}}
\eenn
L'objectif est ici de faire un changement de variable permettant de faire ressortir la fonction $\arccos$\footnote{En effet, $\arccos'(u(x)) = \dfrac{\mathrm{d}\arccos(u(x))}{\mathrm{d}u(x)}\dfrac{\mathrm{d}u(x)}{\mathrm{d}x}$. Soit avec $u(x) = 1/x$, $\arccos'(u(x)) = \dfrac{-1}{\sqrt{1 - \frac{1}{x^{2}}}}$.}. Pour cela, nous posons le changement de variable $u = \frac{M}{r} - \frac{m\alpha}{M}$. Ce qui implique :
\be
	\begin{cases}
		\dfrac{1}{r} = \dfrac{u}{M} + \dfrac{m\alpha}{M^{2}} \\
		\mathrm{d}u = -\dfrac{M}{r^{2}}\mathrm{d}r \nonumber
	\end{cases}
\ee

La variation de l'angle $\varphi$ dans le plan de la trajectoire devient donc :
\bea
	\varphi & = & \bigintsss{\dfrac{-\mathrm{d}u}{\sqrt{2m\left(E + \dfrac{\alpha u}{M} + \dfrac{m\alpha^{2}}{M^{2}}\right) - \left(u + \dfrac{m\alpha}{M}\right)^{2}}}} \nonumber \\
	& = & \bigintsss{\dfrac{-\mathrm{d}u}{\sqrt{2mE + 2m\dfrac{\alpha u}{M} + 2\dfrac{m^{2}\alpha^{2}}{M^{2}} - u^{2} - \dfrac{m^{2}\alpha^{2}}{M^{2}} - 2u\dfrac{\alpha m}{M}}}} \nonumber \\
	& = & \bigintsss{\dfrac{-\mathrm{d}u}{\sqrt{2mE + \dfrac{m^{2}\alpha^{2}}{M^{2}} - u^{2}}}} \nonumber \\
	& = & \dfrac{1}{\sqrt{2mE + \dfrac{m^{2}\alpha^{2}}{M^{2}}}}\bigintsss{\dfrac{-\mathrm{d}u}{\sqrt{1 - \dfrac{u^{2}}{2mE + \dfrac{m^{2}\alpha^{2}}{M^{2}}}}}} \nonumber
\eea
En posant enfin :
\be
	\begin{cases}
		v = \dfrac{u}{\sqrt{2mE + \dfrac{m^{2}\alpha^{2}}{M^{2}}}} \\
		\mathrm{d}u = \sqrt{2mE + \dfrac{m^{2}\alpha^{2}}{M^{2}}}\mathrm{d}v \nonumber
	\end{cases}
\ee
Nous obtenons :
\benn
	\varphi = \bigintsss{\dfrac{-\mathrm{d}v}{\sqrt{1 - v^{2}}}} = \arccos v + cste
\eenn
Soit finalement :
\benn
	\varphi = \arccos\left(\dfrac{\dfrac{M}{r} - \dfrac{m\alpha}{M}}{\sqrt{2mE + \dfrac{m^{2}\alpha^{2}}{M^{2}}}}\right) + cste
\eenn
En choisissant l'origine de $\varphi$ telle que la constante s'annule, nous posons :
\be
	p = \dfrac{M^{2}}{m\alpha}\text{ et } e = \sqrt{1 + \dfrac{2EM^{2}}{m\alpha^{2}}} \label{EQ:15_4}
\ee
pour permettre d'\'ecrire :
\bea
	\cos\varphi & = & \dfrac{\dfrac{M}{r} - \dfrac{m\alpha}{M}}{\sqrt{2mE + \dfrac{m^{2}\alpha^{2}}{M^{2}}}} = \dfrac{\dfrac{M}{r} - \dfrac{m\alpha}{M}}{\dfrac{m\alpha}{M}\sqrt{\dfrac{2mEM^{2}}{m^{2}\alpha^{2}}} + 1} \nonumber \\
	& = & \dfrac{\dfrac{M^{2}}{m\alpha r} - 1}{\sqrt{1 + \dfrac{2mEM^{2}}{m^{2}\alpha^{2}}}} \nonumber
\eea
Avec les deux d\'efinitions (\ref{EQ:15_4}), l'\'egalit\'e ci-dessus se r\'eduit \`a :
\be
	\dfrac{p}{r} = 1 + e\cos\varphi \label{EQ:15_5}
\ee
qui est l'\'equation d'une section conique ayant son foyer \`a l'origine des coordonn\'ees et telle que $p$ est le \emph{param\`etre} et $e$ l'\emph{excentricit\'e} de la trajectoire.

\begin{figure}[htb!]
	\begin{center}
		\begin{picture}(300,200)(0,0)
			%axis
			\linethickness{0.05mm}
			\multiput(50,100)(10,0){29}{\line(1,0){8}}\put(342,98){$x$}
			\multiput(250,10)(0,10){18}{\line(0,1){8}}\put(247,192){$y$}
			%ellipse
			\linethickness{0.5mm}
			\qbezier(75,100),(75,25),(200,25)
			\qbezier(200,25),(325,25),(325,100)
			\qbezier(325,100),(325,175),(200,175)
			\qbezier(200,175),(75,175),(75,100)
			%lines
			\linethickness{0.05mm}
			\put(75,100){\line(0,-1){100}}\put(325,100){\line(0,-1){100}}
			\put(200,25){\line(-1,0){175}}\put(200,175){\line(-1,0){175}}
			\put(200,100){\line(0,-1){50}}
			\put(220,170){\line(1,0){30}}
			%arrows
			\put(25,50){\vector(0,-1){25}}
			\put(25,65){\vector(0,1){110}}
			\put(20,55){$2b$}
			\put(125,10){\vector(-1,0){50}}
			\put(135,10){\vector(1,0){190}}
			\put(126,8){$2a$}
			\put(215,60){\vector(-1,0){15}}
			\put(225,60){\vector(1,0){25}}
			\put(215,58){$ae$}
			\put(225,155){\vector(0,1){15}}
			\put(225,145){\vector(0,-1){45}}
			\put(223,148){$p$}
		\end{picture}
		\caption{Trajectoire elliptique et ses param\`etres}\label{FIG:3_11}
	\end{center}
\end{figure}

Le choix pr\'ec\'edemment fait pour avoir la constante d'int\'egration nulle correspond au point le plus proche du centre du champ et il correspond \`a $\varphi = 0$ d'apr\`es l'\'equation de la trajectoire (\ref{EQ:15_5}). Ce point est appel\'e les \emph{p\'erih\'elie} de l'orbite.

Dans le cas o\`u deux corps interagissent selon le champ d'attraction (\ref{EQ:15_1}), alors l'orbite de chacun des deux corps est une section conique dont le foyer se trouve \`a leur centre d'inertie commun, voir les \'equations (\ref{EQ:8_3}) et (\ref{EQ:13_2}).

\subsection{Cas de la trajectoire elliptique}

D'apr\`es la relation (\ref{EQ:15_4}), il est \'evident que si $-\frac{m\alpha^{2}}{2M^{2}} \le E < 0$ alors $e < 1$. L'orbite est alors une ellipse telle que montr\'ee sur la figure (\ref{FIG:3_11}) et le mouvement est fini et la trajectoire ferm\'ee. Dans ce cas pr\'ecis et en utilisant $\lvert E \rvert$, le grand demi-axe $a$ s'\'ecrit :
\bea
	a & = & \dfrac{p}{1 - e^{2}} = \dfrac{M^{2}}{m\alpha\left(1 - 1 - \dfrac{2EM^{2}}{m\alpha^{2}}\right)} = \dfrac{M^{2}}{\dfrac{2\lvert E \rvert M^{2}}{m\alpha}} \nonumber \\
	& = & \dfrac{\alpha}{2\lvert E \rvert} \label{EQ:15_6a}
\eea
et le petit demi-axe $b$ :
\bea
	b & = & \dfrac{p}{\sqrt{1 - e^{2}}} = \dfrac{M^{2}}{m\alpha\sqrt{1 - 1 - \dfrac{2EM^{2}}{m\alpha^{2}}}} = \dfrac{M^{2}}{m\alpha\sqrt{\dfrac{2\lvert E \rvert M^{2}}{m\alpha^{2}}}} \nonumber \\
	& = & \dfrac{M}{\sqrt{2m\lvert E \rvert}} \label{EQ:15_6b}
\eea

De plus, la plus petite valeur de l'\'energie co\"incide avec l'\'energie cin\'etique nulle et par consé\'equent avec $(U_{eff})_{min}$, voir l'\'equation (\ref{EQ:15_3}). Dans ce cas, l'excentricit\'e $e$ de la section conique est \'egale \`a 1, correspond \`a un cercle dont le rayon vaut $p$. Pour d'autres valeurs de l'\'energie telle que $E < 0$, les distances minimum et maximum au centre du champ central s'obtiennent en r\'esolvant l'\'equation $E = U_{eff}(r)$ qui s'\'ecrit :
\benn
	Er^{2} = -\alpha r + \dfrac{M^{2}}{2m} \Leftrightarrow Er^{2} + \alpha r - \dfrac{M^{2}}{2m} = 0
\eenn
qui est une \'equation du second degr\'e\footnote{Les deux solutions de l'\'equation $ax^{2} + bx + c = 0$ sont $x = (b \pm \sqrt{\Delta})/2a$ avec $\Delta = b^{2} - 4ac$} en $r$ pour laquelle, nous avons :
\be
	\begin{cases}
		r_{min} = \dfrac{\alpha - \sqrt{\alpha^{2}\left(1 + \dfrac{2EM^{2}}{m\alpha^{2}}\right)}}{2\lvert E \rvert} = a(1 - e) \\
		r_{max} = \dfrac{\alpha + \sqrt{\alpha^{2}\left(1 + \dfrac{2EM^{2}}{m\alpha^{2}}\right)}}{2\lvert E \rvert} = a(1 + e) \label{EQ:15_7}
	\end{cases}
\ee
avec $r_{min}$ le pr\'erih\'elie de la trajectoire.

Dans notre cas qui nous occupe, il est aussi int\'eressant de d\'etermine la p\'eriode de r\'evolution $T$ sur l'orbite de la particule. La loi de conservation du mouvement associ\'ee \`a la loi des aires (\ref{EQ:14_3}) permet d'\'ecrire :
\bea
	M & = & 2m\dot{f} \Leftrightarrow \int_{0}^{T} M\mathrm{dt} = 2m\int_{0}^{T} \dot{f}\mathrm{dt} \Leftrightarrow MT = 2m\int_{0}^{T} \mathrm{d}f \nonumber \\
	\Leftrightarrow T & = & \dfrac{2m}{M}\pi ab \nonumber
\eea
car $\int_{0}^{T} \mathrm{d}f$ vaut par d\'efinition la surface totale de l'ellipse parcourue, soit $\pi ab$. \`A l'aide des formules (\ref{EQ:15_6a}) et (\ref{EQ:15_6b}) d\'efinissant $a$ et $b$, la p\'eriode de r\'evolution est :
\be
	T = \dfrac{2m}{M}\pi \dfrac{\alpha}{2\lvert E \rvert} \dfrac{M}{\sqrt{2m\lvert E \rvert}} = \alpha\pi\sqrt{\dfrac{m}{2{\lvert E \rvert}^{3}}} \label{EQ:15_8}
\ee
Au-del\`a du fait que la p\'eriode ne d\'epend que de l'\'energie de la particule, nous retrouvons la troisi\`eme loi de Kepler \'enonc\'ee dans l'\'equation (\ref{EQ:10_KEPLER}).

\subsection{Cas de la trajectoire hyperbolique}

\begin{figure}[htb!]
	\begin{center}
		\begin{picture}(300,200)(0,0)
			%axis
			\linethickness{0.05mm}
			\multiput(0,100)(10,0){29}{\line(1,0){8}}\put(292,98){$x$}
			\multiput(100,0)(0,10){19}{\line(0,1){8}}\put(98,193){$y$}
			%hyperbole
			\linethickness{0.5mm}
			\qbezier(0,0),(200,25),(200,100)
			\qbezier(200,100),(200,175),(0,200)
			%lines
			\linethickness{0.05mm}
			\put(100,174){\line(-1,0){30}}
			\put(200,100){\line(0,-1){30}}
			%arrows
			\put(80,130){\vector(0,-1){30}}
			\put(80,140){\vector(0,1){35}}
			\put(78,133){$p$}
			\put(130,80){\vector(-1,0){30}}
			\put(175,80){\vector(1,0){25}}
			\put(132,78){$a(e-1)$}
		\end{picture}
		\caption{Trajectoire hyperbolique}\label{FIG:3_12}
	\end{center}
\end{figure}

Pour $E \le 0$, le mouvement est infini et plus pr\'ecis\'ement pour $E > 0$, l'excentricit\'e $e > 1$ et la trajectoire est alors une hyperbole, voir la figure (\ref{FIG:3_12}). Dans ce cas, le p\'erih\'lie est :
\be
	r_{min} = a(e - 1) = \frac{p}{1 + e} \label{EQ:15_9}
\ee
avec le demi-axe de l'hyperbole $a = \frac{\alpha}{2\lvert E \rvert} = \frac{\alpha}{2E}$, car $E > 0$.
Dans le cas particulier o\`u $E = 0$, l'excentricit\'e $e$ vaut 1 et la particule a une trajectoire parabolique telle que la distance du p\'erih\'elie $r_{min}$ vaut $\frac{p}{2}$. $E = 0$ n'est possible que si la particule part de l'infini \`a $\mathrm{t} = 0$.

Cherchons d\'esormais la relation entre les coordonn\'ees et le temps \`a partir de l'\'equation (\ref{EQ:14_6}) sous une forme param\'etrique dans le cadre d'une orbite elliptique, i.e. $E < 0\Leftrightarrow \lvert E \rvert = -E$. La relation (\ref{EQ:14_6}) est :
\bea
	\mathrm{t} & = & \bigintsss{\dfrac{\mathrm{d}r}{\sqrt{\dfrac{2}{m}\left(E - U(r)\right) - \dfrac{M^{2}}{m^{2}r^{2}}}}} = \sqrt{\dfrac{m}{2\lvert E \rvert}}\bigintsss{\dfrac{\mathrm{d}r}{\sqrt{-1 + \dfrac{\alpha}{r\lvert E \rvert} - \dfrac{M^{2}}{2m\lvert E \rvert r^{2}}}}} \nonumber \\
	& = & \sqrt{\dfrac{m}{2\lvert E \rvert}}\bigintsss{\dfrac{r\mathrm{d}r}{\sqrt{-r^{2} + \dfrac{\alpha r}{\lvert E \rvert} - \dfrac{M^{2}}{2m\lvert E \rvert}}}} \nonumber
\eea
Or, nous remarquons que :
\bea
	a^{2}e^{2} - (r - a)^{2} & = & \dfrac{\alpha^{2}}{4E^{2}}\left(1+\dfrac{2EM^{2}}{m\alpha^{2}}\right) - r^{2} - \dfrac{\alpha^{2}}{4E^{2}} + \dfrac{2r\alpha}{2\lvert E \rvert} \nonumber \\
	& = & -r^{2} + \dfrac{\alpha r}{\lvert E \rvert} - \dfrac{M^{2}}{2m\lvert E \rvert} \nonumber
\eea
car $-E = \lvert E \rvert$. Et comme $\sqrt{\dfrac{m}{2\lvert E \rvert}} = \sqrt{\dfrac{m\alpha}{2\alpha\lvert E \rvert}} = \sqrt{\dfrac{ma}{\alpha}}$ alors :
\benn
	\mathrm{t} = \sqrt{\dfrac{ma}{\alpha}}\bigintsss{\dfrac{r\mathrm{d}r}{\sqrt{a^{2}e^{2} - (r - a)^{2}}}}
\eenn
En posant :
\benn
	\begin{cases}
		r - a = -ae\cos\xi \\
		\mathrm{d}(r - a) = \mathrm{d}r = ae\sin\xi\mathrm{d}\xi
	\end{cases}
\eenn
nous pouvons \'ecrire :
\bea
	\mathrm{t} & = & \sqrt{\dfrac{ma}{\alpha}}\bigintsss{\dfrac{a^{2}(1 - e\cos\xi)e\sin\xi\mathrm{d}\xi}{a^{2}e^{2} - a^{2}e^{2}\cos^{2}\xi}} = \sqrt{\dfrac{ma}{\alpha}}\bigintsss{\dfrac{a^{2}(1 - e\cos\xi)e\sin\xi\mathrm{d}\xi}{a^{2}e^{2}(1 - \cos^{2}\xi)}} \nonumber \\
	& = & \sqrt{\dfrac{ma}{\alpha}}\dfrac{a^{2}e}{ae}\int{(1 - e\cos\xi)\mathrm{d}\xi} = \sqrt{\dfrac{ma^{3}}{\alpha}}\int{(1 - e\cos\xi)\mathrm{d}\xi} \nonumber \\
	& = & \sqrt{\dfrac{ma^{3}}{\alpha}}(\xi - e\sin\xi) + cste \nonumber
\eea
En choisissant l'origine du temps telle que la constante d'int\'egration soit nulle, nous obtenons bien la repr\'esentation param\'etrique de la trajectoire entre $r$ et $t$ telle que :
\be
	r = a(1 - e\cos\xi)\text{ et }\mathrm{t} = \sqrt{\dfrac{ma^{3}}{\alpha}}(\xi - e\sin\xi) \label{EQ:15_10}
\ee
L'instant $t = 0$ est \'equivalent \`a $(\xi - e\sin\xi) = 0$ qui n'est possible que pour $\xi = 0$. Cela correspond exactement \`a la position du p\'erih\'elie car $r = a(1 - e) = r_{min}$.

Dans le plan de la trajectoire, en coordonn\'ees cart\'esiennes telles que les axes $x$ et $y$ suivent respectivement le grand et le petit demi-axe de l'ellipse, alors :
\benn
	\begin{cases}
		x = r\cos\varphi \\
		y = r\sin\varphi
	\end{cases}
\eenn
En utilisant les relations (\ref{EQ:15_5}) et (\ref{EQ:15_10}), nous pouvons \'ecrire :
\bea
	p & = & r + er\cos\varphi = r + ex \nonumber \\
	\Leftrightarrow x & = & \dfrac{p - r}{e} \nonumber
\eea
Or la relation (\ref{EQ:15_6a}) se prolonge en :
\bea
	p & = & a(1 - e^{2}) \nonumber \\
	ex & = & a(1 - e^{2}) - r = a(1 - e^{2}) - a(1 - e\cos\xi) \nonumber \\
	ex & = & ae(\cos\xi - e) \nonumber
\eea
et comme $y^{2} = r^{2} - x^{2}$, nous avons :
\bea
	y^{2} & = & a^{2}(1 - e\cos\xi)^{2}) - a^{2}(\cos\xi -e)^{2} \nonumber \\
	& = & a^{2}(1 + e^{2}\cos^{2}\xi - 2e\cos\xi - \cos^{2}\xi - e^{2} + 2e\cos\xi) \nonumber \\
	& = & a^{2}(1 + e^{2}\cos^{2}\xi - e^{2} - \cos^{2}\xi) \nonumber \\
	& = & a^{2}(1 + e^{2} - e^{2}\sin^{2}\xi - e^{2} - 1 + \sin^{2}\xi) \nonumber \\
	& = & a^{2}(- e^{2}\sin^{2}\xi + \sin^{2}\xi) \nonumber
\eea
Nous pouvons en conclure :
\be
	x = a(\cos\xi - 1)\text{ et }y = a\sqrt{1 - e^{2}}\sin\xi \label{EQ:15_11}
\ee
et nous pouvons observer que le parcours de $\xi$ de $0$ \`a $2\pi$ permet de d\'ecrire une r\'evolution compl\`ete sur l'ellipse.

Cherchons de la m\^eme mani\`ere \`a d\'eterminer la repr\'esentation param\'etrique de la trajectoire dans le cas de la solution hyperbolique, \`a savoir $E > 0$. La relation (\ref{EQ:14_6}) reste :
\bea
	\mathrm{t} & = & \bigintsss{\dfrac{\mathrm{d}r}{\sqrt{\dfrac{2}{m}\left(E + \dfrac{\alpha}{r}\right) - \dfrac{M^{2}}{m^{2}r^{2}}}}} \nonumber \\
	& = & \sqrt{\dfrac{m}{2E}}\bigintsss{\dfrac{r\mathrm{d}r}{\sqrt{r^{2} + \dfrac{\alpha}{E}r - \dfrac{M^{2}}{2mE}}}} \nonumber
\eea
Maintenant, en posant :
\benn
	\begin{cases}
		r = a(e\cosh\xi -1) \\
		\mathrm{d}r = ae\sinh\xi\mathrm{d}\xi
	\end{cases}
\eenn
nous pouvons remarquer que :
\bea
	a^{2}e^{2} - (r + a)^{2} & = & \dfrac{\alpha^{2}}{4E^{2}}\left(1 + \dfrac{2EM^{2}}{m\alpha^{2}}\right) - r^{2} - \dfrac{\alpha^{2}}{4E^{2}} - \dfrac{2\alpha r}{2E} \nonumber \\
	& = & -\left(r^{2} + \dfrac{\alpha}{E}r - \dfrac{M^{2}}{2mE}\right) \nonumber
\eea
Aussi, nous pouvons poursuivre avec $\mathrm{t}$ avec \'egalement\footnote{En ajoutant aussi que $\cosh^{2}\xi - \sinh^{2}\xi = 1$} $\sqrt{\dfrac{m}{2E}} = \sqrt{\dfrac{m\alpha}{2\alpha E}} = \sqrt{\dfrac{ma}{\alpha}}$  :
\bea
	\mathrm{t} & = & \sqrt{\dfrac{ma}{\alpha}}\bigintsss{\dfrac{r\mathrm{d}r}{\sqrt{-\left(r^{2} + \dfrac{\alpha}{E}r - \dfrac{M^{2}}{2mE}\right)}}} = \sqrt{\dfrac{ma}{\alpha}}\bigintsss{\dfrac{r\mathrm{d}r}{\sqrt{(r + a)^{2} - a^{2}e^{2}}}} \nonumber \\
	& = & \sqrt{\dfrac{ma}{\alpha}}\bigintsss{\dfrac{a(e\cosh\xi - 1)ae\sin\xi\mathrm{d}\xi}{\sqrt{\left(a(e\cosh\xi - 1) + a\right)^{2} - a^{2}e^{2}}}} = \sqrt{\dfrac{ma}{\alpha}}\bigintsss{\dfrac{a(e\cosh\xi - 1)ae\sin\xi\mathrm{d}\xi}{\sqrt{a^{2}e^{2}\cosh^{2}\xi - a^{2}e^{2}}}} \nonumber \\
	& = & \sqrt{\dfrac{ma}{\alpha}}\bigintsss{\dfrac{a(e\cosh\xi - 1)ae\sin\xi\mathrm{d}\xi}{ae\sqrt{\cosh^{2}\xi - 1}}} = \sqrt{\dfrac{ma}{\alpha}}\dfrac{a^{2}e}{ae}\int{(e\cosh\xi - 1)\mathrm{d}\xi} \nonumber \\
	& = & \sqrt{\dfrac{ma^{3}}{\alpha}}\int{(e\cosh\xi - 1)\mathrm{d}\xi} \nonumber \\
	\mathrm{t} & = & \sqrt{\dfrac{ma^{3}}{\alpha}}(e\sinh\xi - \xi) \nonumber
\eea
Avec la m\^eme hypoth\`ese que dans le cas elliptique qui permet d'\'ecrire $x = r\cos\varphi$ et $y = r\sin\varphi$ et sachant (\ref{EQ:15_9}), nous avons :
\benn
	p = a(e^{2} - 1)\text{ et }p - r = ex
\eenn
donc :
\bea
	ex & = & a(e^{2} - 1) - a(e\cosh\xi - 1) = ae^{2} - a - ae\cosh\xi + a \nonumber \\
	\Leftrightarrow x & = & a(e - \cosh\xi) \nonumber
\eea
et avec $y^{2} = r^{2} - x^{2}$ :
\bea
	y^{2} & = & a^{2}e^{2}\cosh^{2}\xi + a^{2} - 2ae\cosh\xi - a^{2}e^{2} - a^{2}\cosh^{2}\xi + 2ae\cosh\xi \nonumber \\
	& = & a^{2}e^{2}(1 + \sinh^{2}\xi) + a^{2} - a^{2}e^{2} - a^{2}(1 + \sinh^{2}\xi) \nonumber \\
	\Leftrightarrow y & = & a\sqrt{e^{2} - 1}\sinh\xi \nonumber
\eea
avec $\xi \in ]-\infty ; +\infty[$. Nous avons ainsi la trajectoire param\'etrique telle que :
\be
	\begin{cases}
		r = a(e\cosh\xi - 1)\text{ et }\mathrm{t} = \sqrt{\dfrac{ma^{3}}{\alpha}}(e\sinh\xi - \xi) \\
		x = a(e - \cosh\xi)\text{ et }y = a\sqrt{e^{2} - 1}\sinh\xi \label{EQ:15_12}
	\end{cases}
\ee

\subsection{Cas de la trajectoire hyperbolique dans un champ r\'epulsif}

\begin{figure}[htb!]
	\begin{center}
		\begin{picture}(300,200)(0,0)
			%axis
			\linethickness{0.05mm}
			\multiput(0,100)(10,0){29}{\line(1,0){8}}\put(292,98){$x$}
			\multiput(10,0)(0,10){19}{\line(0,1){8}}\put(8,193){$y$}
			%hyperbole
			\linethickness{0.5mm}
			\qbezier(250,0),(100,25),(100,100)
			\qbezier(100,100),(100,175),(250,200)
			%lines
			\linethickness{0.05mm}
			\put(100,100){\line(0,-1){30}}
			%arrows
			\put(30,80){\vector(-1,0){20}}
			\put(75,80){\vector(1,0){25}}
			\put(33,78){$a(1+e)$}
		\end{picture}
		\caption{Trajectoire hyperbolique dans le cas r\'epulsif}\label{FIG:3_13}
	\end{center}
\end{figure}

Dans le cas d'un champ d'\'energie potentielle r\'epulsif, champ de Coulomb entre deux charges de m\^eme signe par exemple, alors :
\be
	U(r) = \dfrac{\alpha}{r} \label{EQ:15_13}
\ee
avec $\alpha > 0$. Alors par analogie, l'\'equation (\ref{EQ:14_8}) peut s'\'ecrire et une illustration est partag\'ee sur la figure (\ref{FIG:3_13a}) :
\benn
	U_{eff}(r) = \dfrac{\alpha}{r} + \dfrac{M^{2}}{2mr^{2}}
\eenn

\begin{figure}[htb!]
	\begin{center}
		\includegraphics[width=10cm]{chapter_03_paragraph_15_fig_13a}
		\caption{\'Energie potentielle r\'epulsive pour $\alpha\in \{1;2;3;4;5\}$ et $M^{2}/m = 1$}\label{FIG:3_13a}
	\end{center}
\end{figure}

L'\'energie de la particule en jeu est de facto strictement positive \`a la vue de la courbe obtenue pour l'\'energie potentielle, sachant que l'\'energie cin\'etique est toujours positive. En reprenant l'\'equation (\ref{EQ:14_7}), soit :
\benn
	\varphi = \bigintsss{\dfrac{\frac{M}{r^{2}}\mathrm{d}r}{\sqrt{2m\left(E + \dfrac{\alpha}{r}\right) - \frac{M^{2}}{r^{2}}}}}
\eenn
et en substituant la quantit\'e $\alpha$ par $-\alpha$, nous obtenons :
\benn
	\cos\varphi = \dfrac{\dfrac{M}{r} + \dfrac{m\alpha}{M}}{\sqrt{2mE + \dfrac{m^{2}\alpha^{2}}{M^{2}}}}
\eenn
Avec les relations (\ref{EQ:15_4}), nous obtenons ici :
\bea
	\cos\varphi & = & \dfrac{\dfrac{p}{r} + 1}{e} \nonumber \\
	\dfrac{p}{r} & = & -1 + e\cos\varphi \label{EQ:15_14}
\eea
dont une repr\'esentation est sur la figure (\ref{FIG:3_13}). Nous pouvons aussi observer ici que $r$ est minimale quand la quantit\'e $\cos\varphi$ est maximale, donc le p\'erih\'elie s'\'ecrit ici :
\be
	r_{min} = \dfrac{p}{e - 1} = a(1 + e) \label{EQ:15_15}
\ee
car $a = \frac{\alpha}{2E}$.
L'\'equation (\ref{EQ:14_6}) se d\'eveloppe :
\benn
	\mathrm{t} = \bigintsss{\dfrac{\mathrm{d}r}{\sqrt{\dfrac{2}{m}\left(E - \dfrac{\alpha}{r}\right) - \dfrac{M^{2}}{m^{2}r^{2}}}}} = \sqrt{\dfrac{ma}{\alpha}}\bigintsss{\dfrac{r\mathrm{d}r}{\sqrt{r^{2} - \dfrac{\alpha}{E}r - \dfrac{M^{2}}{2mE}}}}
\eenn
En remarquant que :
\bea
	a^{2}e^{2} - (r - a)^{2} & = & a^{2}e^{2} - r^{2} - a^{2} + 2ar \nonumber \\
	& = & \dfrac{\alpha^{2}}{4E^{2}}\left(1 + \dfrac{2EM^{2}}{m\alpha^{2}}\right) - r^{2} - \dfrac{\alpha^{2}}{4E^{2}} + \dfrac{2\alpha r}{2E} \nonumber \\
	& = & -\left(r^{2} - \dfrac{\alpha}{E}r - \dfrac{M^{2}}{2mE}\right) \nonumber
\eea
nous pouvons reprendre tel que :
\benn
	\mathrm{t} = \sqrt{\dfrac{ma}{\alpha}}\bigintsss{\dfrac{r\mathrm{d}r}{\sqrt{(r - a)^{2} - a^{2}e^{e}}}}
\eenn
En posant :
\benn
	\begin{cases}
		r = a(e\cosh\xi + 1) \\
		\mathrm{d}r = ae\sinh\xi\mathrm{d}\xi
	\end{cases}
\eenn
nous prolongeons en :
\bea
	\mathrm{t} & = & \sqrt{\dfrac{ma}{\alpha}}\bigintsss{\dfrac{a(e\cosh\xi + 1)ae\sin\xi\mathrm{d}\xi}{\sqrt{a^{2}e^{2}\cosh^{2}\xi - a^{2}e^{2}}}} = \sqrt{\dfrac{ma}{\alpha}}\bigintsss{\dfrac{a(e\cosh\xi + 1)ae\sin\xi\mathrm{d}\xi}{ae\sqrt{\cosh^{2}\xi - 1}}} \nonumber \\
	& = & \sqrt{\dfrac{ma^{3}}{\alpha}}\int{(e\cosh\xi + 1)\mathrm{d}\xi} \nonumber \\
	\mathrm{t} & = & \sqrt{\dfrac{ma^{3}}{\alpha}}(e\sinh\xi + \xi) \nonumber
\eea
Avec la m\^eme technique calculatoire que pr\'ec\'edemment, nous posons $x = r\cos\varphi$ et $y = r\sin\varphi$ et sachant (\ref{EQ:15_14}), nous avons :
\benn
	p = a(e^{2} - 1)\text{ et }p + r = ex
\eenn
donc :
\bea
	ex & = & a(e^{2} - 1) + a(e\cosh\xi + 1) = ae^{2} + ae\cosh\xi \nonumber \\
	\Leftrightarrow x & = & a(\cosh\xi + e) \nonumber
\eea
et avec $y^{2} = r^{2} - x^{2}$ :
\bea
	y^{2} & = & a^{2}e^{2}\cosh^{2}\xi + a^{2} + 2ae\cosh\xi - a^{2}e^{2} - a^{2}\cosh^{2}\xi - 2ae\cosh\xi \nonumber \\
	& = & a^{2}e^{2}(1 + \sinh^{2}\xi) + a^{2} - a^{2}e^{2} - a^{2}(1 + \sinh^{2}\xi) \nonumber \\
	\Leftrightarrow y & = & a\sqrt{e^{2} - 1}\sinh\xi \nonumber
\eea
avec $\xi \in ]-\infty ; +\infty[$. Nous avons ainsi la trajectoire param\'etrique telle que :
\be
	\begin{cases}
		r = a(e\cosh\xi + 1)\text{ et }\mathrm{t} = \sqrt{\dfrac{ma^{3}}{\alpha}}(e\sinh\xi + \xi) \\
		x = a(\cosh\xi + e)\text{ et }y = a\sqrt{e^{2} - 1}\sinh\xi \label{EQ:15_16}
	\end{cases}
\ee

\subsection{Int\'egrale sp\'ecifique du champ en $\alpha/r$}

Cherchons une int\'egrale premi\`ere, i.e. constante dans le temps, sp\'ecifique \`a ce type de champ en $\frac{\alpha}{r}$ avec $\alpha\in\mathbb{R}$ en remarquant que :
\bea
	\dfrac{\mathrm{d}}{\mathrm{dt}}\left(\vec{v}\wedge\vec{M} + \dfrac{\alpha}{r}\vec{r}\right) & = & \vec{\dot{v}}\wedge\vec{M} + \vec{v}\wedge\vec{\dot{M}} + \dfrac{\alpha}{r}\vec{\dot{r}} - \dfrac{\alpha}{r^{3}}\vec{r}(\vec{v}\cdot\vec{r}) \nonumber \\
	& = & \vec{\dot{v}}\wedge\vec{M} + \dfrac{\alpha}{r}\vec{\dot{r}} - \dfrac{\alpha}{r^{3}}\vec{r}(\vec{v}\cdot\vec{r}) \nonumber
\eea
car il y a toujours conservation du moment cin\'etique et donc $\vec{\dot{M}} = \vec{0}$. De plus $\vec{M} = m\vec{r}\wedge\vec{v}$ donc :
\benn
	\dfrac{\mathrm{d}}{\mathrm{dt}}\left(\vec{v}\wedge\vec{M} + \dfrac{\alpha}{r}\vec{r}\right)  = m(\vec{\dot{v}}\cdot\vec{v})\vec{r} - m(\vec{\dot{v}}\cdot\vec{r})\vec{v} + \dfrac{\alpha}{r}\vec{\dot{r}} - \dfrac{\alpha}{r^{3}}\vec{r}(\vec{v}\cdot\vec{r})
\eenn
En revenant à l'\'equation du mouvement (\ref{EQ:5_3}) :
\benn
	m\vec{\dot{v}} = -\dfrac{\partial U}{\partial\vec{r}} = \dfrac{\alpha}{r^{3}}\vec{r}
\eenn
La quantit\'e :
\bea
	m(\vec{\dot{v}}\cdot\vec{v})\vec{r} - m(\vec{\dot{v}}\cdot\vec{r})\vec{v} + \dfrac{\alpha}{r}\vec{\dot{r}} - \dfrac{\alpha}{r^{3}}\vec{r}(\vec{v}\cdot\vec{r}) & = & m(\vec{\dot{v}}\cdot\vec{v})\vec{r} - m(\vec{\dot{v}}\cdot\vec{r})\vec{v} + \dfrac{\alpha}{r}\vec{\dot{r}} - m\vec{\dot{v}}(\vec{v}\cdot\vec{r}) \nonumber \\
	& = & m(\vec{\dot{v}}\cdot\vec{v})\vec{r} - m\vec{\dot{v}}(\vec{v}\cdot\vec{r}) - (m\vec{\dot{v}}\cdot\vec{r})\vec{v} + \dfrac{\alpha}{r}\vec{\dot{r}} \nonumber \\
	& = & m(\vec{\dot{v}}\cdot\vec{v})\vec{r} - m(\vec{\dot{v}}\cdot\vec{v})\vec{r} - \dfrac{\alpha}{r}\vec{v} + \dfrac{\alpha}{r}\vec{\dot{r}} \nonumber \\
	& = & \vec{0} \nonumber
\eea
Ainsi le vecteur :
\be
	\vec{v}\wedge\vec{M} + \dfrac{\alpha}{r}\vec{r} = \vec{cste} \label{EQ:15_17}
\ee
se conserve dans le temps, en direction et en norme. Par d\'efinition, le vecteur $\vec{v}\wedge\vec{M}$ a la m\^eme direction que $\vec{r}$ et \`a la position du p\'erih\'elie, la vitesse radiale est nulle, donc le vecteur conservatif d\'efinit par (\ref{EQ:15_17}) est dirig\'e vers le grand axe du foyer. De plus, au p\'erih\'elie, nous avons $r = r_{min}$ par d\'efinition et la composante radiale de la vitesse de la particule est nulle. Aussi, la quantité (\ref{EQ:15_17}) se d\'eveloppe au p\'erih\'elie ainsi :
\benn
	\vec{v}\wedge\vec{M} + \dfrac{\alpha}{r}\vec{r} = \begin{pmatrix} 0 \\ v_{\varphi} \\ 0 \end{pmatrix} \wedge \begin{pmatrix} 0 \\ 0 \\ M_{z} \end{pmatrix} + \begin{pmatrix} \alpha \\ 0 \\ 0 \end{pmatrix} = \begin{pmatrix} v_{\varphi}M_{z} + \alpha \\ 0 \\ 0 \end{pmatrix}
\eenn
avec $v_{\varphi}$ la vitesse selon $\varphi$ au p\'erih\'elie. Or l'\'equation (\ref{EQ:14_2}) donne $M = mr^{2}\dot{\varphi}$ ou encore dans notre cas \`a la position du p\'erih\'elie, $M_{z} = M = mr_{min}^{2}\dot{\varphi}$. Cette relation peut aussi s'\'ecrire :
\bea
	\dfrac{\mathrm{d}\varphi}{\mathrm{dt}} & = & \dfrac{M}{mr_{min}^{2}} \nonumber \\
	v_{\varphi} & = & \dfrac{M}{mr_{min}} \nonumber
\eea
car $v_{\varphi} = r_{min}\dot{\varphi}$. La quantité :
\bea
	v_{\varphi}M_{z} + \alpha & = & \dfrac{M^{2}}{mr_{min}} + \alpha = \dfrac{M^{2}(e - 1)}{mp} + \alpha \nonumber \\
	& = & \dfrac{M^{2}(e - 1)m\alpha}{mM^{2}} + \alpha = (e - 1)\alpha + \alpha \nonumber \\
	& = & \alpha e \nonumber
\eea
en prenant pour r\'ef\'erence pour la valeur du param\`etre $p$ la relation (\ref{EQ:15_15}). Le vecteur d\'efini en (\ref{EQ:15_17}) est donc constant en norme qui est \'egale \`a $\alpha e$. L'int\'egrale premi\`ere (\ref{EQ:15_17}) comme les lois de conservation de l'\'energie et du moment cin\'etique est une fonction uniforme de l'\'etat de la particule, position et vitesse. L'apparition de l'int\'egrale (\ref{EQ:15_17}) suppl\'ementaire est due à la d\'eg\'en\'erescence du mouvement, \'etudi\'e au paragraphe (\ref{PAR:50}).
\chapter{Chocs de particules}

\section{D\'esint\'egration des particules}

Les lois de conservations ne d\'ependent absolument pas de l'esp\`ece d'interaction entre les particules et permettent ainsi d'arriver \`a des conclusions pourtant importantes pour les processus m\'ecaniques en jeu.

Pour commencer, choisissons la d\'esint\'egration spontan\'ee, i.e. non provoqu\'ee par des forces ext\'erieures, d'une particule en deux autres particules se d\'epla\c{c}ant apr\`es la d\'esint\'egration ind\'ependamment l'une de l'autre. Sa forme la plus simple est quand il est consid\'er\'e dans un syst\`eme de r\'ef\'erence o\`u la particule initiale est au repos. La conservation d' l'impulsions implique :
\bea
	\vec{p}_{1} + \vec{p}_{2} & = & \vec{0} \nonumber \\
	\vec{p}_{1} & = & -\vec{p}_{2}
\eea
ainsi les particules r\'esultantes de la d\'esint\'egration s'\'eloignent l'une de l'autre avec des impulsions \'egales, ,$\lVert \vec{p}_{1} \rVert = \lVert \vec{p}_{2} \rVert = p_{0}$ et dirig\'ees en sens inverse.

En utilisant la loi de conservation de l'\'energie et en particulier l'\'equation (\ref{EQ:8_4}) et parce que l'\'energie cin\'etique de la particule initiale est nulle, nous pouvons \'ecrire :
\be
	E_{int} = E_{1int} + \dfrac{p_{0}^{2}}{2m_{1}} + E_{2int} + \dfrac{p_{0}^{2}}{2m_{2}}
\ee
avec $E_{int}$ l'\'energie interne de la particule initiale, $E_{1int}$ et $E_{2int}$ l'\'energie interne des deux particules cr\'e\'ees. En d\'efinissant l'\'energie de d\'esint\'egration $\epsilon$ telle que :
\be
	\epsilon = E_{int} - (E_{1int} + E_{2int}) \label{EQ:16_1}
\ee
sa valeur est de facto positive. Or :
\be
	E_{int} - (E_{1int} + E_{2int}) = p_{0}^{2}\left(\dfrac{1}{2m_{1}} + \dfrac{1}{2m_{2}}\right)
\ee
Donc :
\be
	\epsilon = p_{0}^{2}\left(\dfrac{1}{2m_{1}} + \dfrac{1}{2m_{2}}\right) = \dfrac{p_{0}^{2}}{2m} \label{EQ:16_2}
\ee
avec $m = \dfrac{m_{1} + m_{2}}{m_{1}m_{2}}$ qui est la masse r\'eduite du syst\`eme apr\`es d\'esint\'egration.

\begin{figure}[htb!]
	\begin{center}
		\begin{picture}(500,300)(0,0)
			%circles
			\linethickness{0.05mm}
			\put(100,150){\circle{200}}\put(90,25){$V < v_{0}$}
			\put(400,150){\circle{200}}\put(390,25){$V > v_{0}$}
			%vectors circle #1
			\linethickness{0.5mm}
			\put(40,150){\vector(1,0){60}}\put(65,135){$\vec{V}$}
			\put(100,150){\vector(1,1){72}}\put(140,180){$\vec{v}_{0}$}
			\put(40,150){\vector(9,5){130}}\put(90,185){$\vec{v}$}
			%angles circle #1
			\linethickness{0.05mm}
			\qbezier(55,150)(55,155)(50,155)\put(60,152){$\theta$}
			\qbezier(110,150)(110,155)(105,155)\put(112,153){$\theta_{0}$}
			\multiput(100,150)(10,0){10}{\line(1,0){8}}
			%points circle #1
			\put(30,147){$A$}
			%vectors circle #2
			\linethickness{0.5mm}
			\put(250,150){\vector(1,0){150}}\put(320,135){$\vec{V}$}
			\put(400,150){\vector(1,1){72}}\put(440,180){$\vec{v}_{0}$}
			\put(250,150){\vector(10,3){220}}\put(390,210){$\vec{v}$}
			%\theta_{max} case
			\linethickness{0.05mm}
			\put(250,150){\line(10,9){90}}
			\put(400,150){\line(-11,15){60}}
			%angles circle #2
			\linethickness{0.05mm}
			\qbezier(280,150)(280,158)(275,158)\put(287,152){$\theta$}
			\qbezier(410,150)(410,155)(405,155)\put(412,153){$\theta_{0}$}
			\qbezier(320,150)(320,190)(294,190)\put(315,182){$\theta_{max}$}
			\multiput(400,150)(10,0){10}{\line(1,0){8}}
			%points circle #2
			\put(240,147){$A$}
			\put(300,155){$B$}
			\put(475,220){$C$}
		\end{picture}
		\caption{Repr\'esentation g\'eom\'etrique de d\'esint\'egrations}\label{FIG:4_14}
	\end{center}
\end{figure}

\'Etudions maintenant le cas o\`u la particule initiale poss\`ede une vitesse $\vec{V}$ non nulle avant la d\'esint\'egration dans le syst\`eme de r\'ef\'erence. Il en existe deux :
\begin{itemize}
	\item le syst\`eme du laboratoire ou <<~l~>>,
	\item le syst\`eme du centre d'inertie ou <<~c~>> dans lequel, par d\'efinition, la somme des impulsions est nulle, voir le paragraphe (\ref{PAR:8}).
\end{itemize}
Pour une des particules r\'esultantes de la d\'esint\'egration, d\'efinissions $\vec{v}$ sa vitesse dans le syst\`eme <<~l~>> et $\vec{v}_{0}$, sa vitesse dans le syst\`eme <<~c~>>. Par construction, nous avons $\vec{v} = \vec{V} + \vec{v}_{0} \Leftrightarrow \vec{v}_{0} = \vec{v} + \vec{V}$. De plus, la formule d'Al-Kashi donne directement :
\bea
	v_{0}{2} & = & v^{2} + V^{2} + 2vV\cos(\langle \vec{v},\vec{V}\rangle) \nonumber \\
	& = & v^{2} + V^{2} + 2vV\cos(\theta)
\eea
en se basant sur la figure (\ref{FIG:4_14})et o\`u l'angle $\theta$ est l'angle de d\'eviation dans <<~l~>>. La repr\'esentation g\'eom\'etrique (\ref{FIG:4_14}) montre deux cas distincts :
\begin{itemize}
	\item $V < v_{0}$ o\`u l'angle $\theta$ peut alors prendre une valeur quelconque
	\item $V > v_{0}$ o\`u la particule r\'esultante est limit\'ee dans sa direction par la tangente au cercle de rayon $\lVert \vec{v}_{0} \rVert$ depuis le point A. Cela permet d'\'ecrire :
	\be
		\sin\theta_{max} = \dfrac{v_{0}}{V} \label{EQ:16_4}
	\ee
\end{itemize}

$\theta_{0}$ est l'angle de d\'eviation dans le r\'ef\'erentiel <<~c~>>, la repr\'esentation g\'eom\'etrique (\ref{FIG:4_14}) permet d'\'etablir la relation entre $\theta$ et $\theta_{0}$ telle que :
\be
	\tan\theta = \dfrac{v_{0}\sin\theta_{0}}{V + v_{0}\cos\theta_{0}} \label{EQ:16_5}
\ee
En d\'eveloppant et en recherchant une \'equation en $\cos\theta_{0}$, nous \'ecrivons :
\bea
	\dfrac{\sin^{2}\theta}{\cos^{2}\theta} & = & \dfrac{v_{0}^{2}(1 - \cos^{2}\theta_{0}}{(V + v_{0}\cos\theta_{0})^{2}} \nonumber \\
	\Leftrightarrow \sin^{2}\theta(V^{2} + v_{0}^{2}\cos^{2}\theta_{0} + 2Vv_{0}\cos\theta_{0}) & = & v_{0}^{2}\cos^{2}\theta - v_{0}^{2}\cos^{2}\theta\cos^{2}\theta_{0} \nonumber \\
\eea
ou encore en recherchant l'\'equation du second degr\'e :
\bea
	v_{0}^{2}(\sin^{2}\theta + cos^{2}\theta)\cos^{2}\theta_{0} + 2Vv_{0}\sin^{2}\theta\cos\theta_{0} + 2V^{2}\sin^{2}\theta - v_{0}^{2}\cos^{2}\theta & = & 0 \nonumber \\
	\Leftrightarrow \cos^{2}\theta_{0} + \dfrac{2V}{v_{0}}\sin^{2}\theta\cos\theta_{0} + \dfrac{V^{2}}{v_{0}^{2}}\sin^{2}\theta - \cos^{2}\theta & = & 0 \nonumber \\
\eea
qui a pour solutions :
\bea
	\cos\theta_{0} & = & \dfrac{-2\dfrac{V}{v_{0}}\sin^{2}\theta \pm \sqrt{4\dfrac{V^{2}}{v_{0}^{2}}\sin^{4}\theta - 4\dfrac{V^{2}}{v_{0}^{2}}\sin^{2}\theta + 4\cos^{2}\theta}}{2} \nonumber \\
	& = & -\dfrac{V}{v_{0}}\sin^{2}\theta \pm \sqrt{\dfrac{V^{2}}{v_{0}^{2}}(\sin^{2}\theta - 1)\sin^{2}\theta + \cos^{2}\theta} \nonumber \\
	\Leftrightarrow \cos\theta_{0} & = & -\dfrac{V}{v_{0}}\sin^{2}\theta \pm \cos\theta\sqrt{1 - \dfrac{V^{2}}{v_{0}^{2}}\sin^{2}\theta} \label{EQ:16_6}
\eea

\begin{figure}[htb!]
	\begin{center}
		\includegraphics[width=10cm]{chapter_04_paragraph_16_fig_14a}
		\caption{\'Equation (\ref{EQ:16_6}) dans le cas $v_{0} > V$ pour $\dfrac{V}{v_{0}}$ compris entre 0.1 et 0.9 avec un pas de 0.1 et pour $0 < \theta < \pi$}\label{FIG:4_14A}
	\end{center}
\end{figure}

Dans le cas o\`u $v_{0} > V$, le vecteur $\vec{v}$ ne croise le cercle de rayon $v_{0}$ qu'en un unique point et comme il est n\'ecessaire de choisir $\theta_{0} = 0$ quand $\theta = 0$, la solution (\ref{EQ:16_6}) se r\'eduit \`a $\cos\theta_{0} = -\dfrac{V}{v_{0}}\sin^{2}\theta + \cos\theta\sqrt{1 - \dfrac{V^{2}}{v_{0}^{2}}\sin^{2}\theta}$. Ce cas est repr\'esent\'e sur la figure (\ref{FIG:4_14A}) Dans le cas o\`u $v_{0} < V$, le vecteur $\vec{v}$ croise le m\^eme cercle en deux points, B et C sur la figure correspondante (\ref{FIG:4_14}) tel qu'il existe alors deux valeurs de $\theta_{0}$ pour chaque valeur de $\theta$. Ce cas est illustr\'e sur la figure (\ref{FIG:4_14B}).

\begin{figure}[htb!]
	\begin{center}
		\includegraphics[width=10cm]{chapter_04_paragraph_16_fig_14b}
		\caption{Deux solutions de l'\'equation (\ref{EQ:16_6}) dans le cas $v_{0} < V$ pour $\dfrac{V}{v_{0}}$ compris entre 1.1 et 1.9 avec un pas de 0.1 et pour $0 < \theta < \pi$}\label{FIG:4_14B}
	\end{center}
\end{figure}

Dans la plupart des applications physiques, ce sont de nombreuses particules qui se d\'esint\`egrent et il faut alors raisonner en termes de distribution, en \'energie, en impulsion, en directions, etc. Prenons d\'esormais l'hypoth\`ese de particules initiales orient\'ees de mani\`ere chaotique, i.e. en moyenne de fa\c{c}on isotrope. Dans le r\'ef\'erentiel <<~c~>>, l'isotropie est conserv\'ee apr\`es les d\'esint\'egrations et les particules r\'esultantes de m\^eme esp\`ece ont alors la m\^eme \'energie et la r\'epartition des trajectoires est isotrope. L'orientation chaotique peut se traduire comme la quantit\'e de particules traversant un angle solide\footnote{Par d\'efinition, un \'el\'ement d'angle solide est d\'efini par $\mathrm{d}^{2}\Omega = \dfrac{\vec{r}\cdot\vec{n}}{r^{3}}\mathrm{d}^{2}S$ avec $\vec{r}$ le vecteur rayon et $\vec{n}$ le vecteur normale de l'\'el\'ement de surface $\mathrm{d}^{2}S$. Dans le cadre d'une sph\`ere, nous avons $\mathrm{d}^{2}S = r\mathrm{d}\theta r\sin\theta\mathrm{d}\varphi$ et par cons\'equent $\mathrm{d}^{2}\Omega = \sin\theta\mathrm{d}\theta\mathrm{d}\varphi$ ou encore $\mathrm{d}\Omega = 2\pi\sin\theta\mathrm{d}\theta$.} $\mathrm{d}\omega_{0}$ qui est proportionnelle \`a la grandeur de cet \'el\'ement soit $\frac{\mathrm{d}\omega_{0}}{4\pi}$. Dans le cadre d'une sph\`ere, $\mathrm{d}\omega_{0} = 2\pi\sin\theta_{0}\mathrm{d}\theta_{0}$, donc :
\be
	\dfrac{\mathrm{d}\omega_{0}}{4\pi} = \dfrac{1}{2}\sin\theta_{0}\mathrm{d}\theta_{0} \label{EQ:16_7}
\ee
Pour obtenir la r\'epartition dans le r\'ef\'erentiel <<~l~>>, partons du calcul de l'\'energie cin\'etique et de sa distribution. Nous savons que :
\bea
	\vec{v} & = & \vec{V} + \vec{v}_{0} \nonumber \\
	\Leftrightarrow v^{2} & = & v_{0}^{2} + V^{2} + 2v_{0}V\cos\theta_{0} \nonumber \\
	\Leftrightarrow \cos\theta_{0} & = & \dfrac{v^{2} - v_{0}^{2} - V^{2}}{2v_{0}V}
\eea
Or par rapport \`a $\theta_{0}$, les quantit\'es $\lVert\vec{v}_{0}\rVert$ et $\lVert\vec{V}\rVert$ sont constantes au contraire de $\lVert\vec{v}\rVert$ aussi, nous pouvons en conclure que :
\be
	\mathrm{d}\cos\theta_{0} = -\sin\theta_{0}\mathrm{d}\theta_{0} = \dfrac{\mathrm{d}(v^{2})}{2v_{0}V}
\ee
Or l'\'energie cin\'etique d'une particule r\'esultante de masse $m$ s\'ecrit $T = \frac{1}{2}mv^{2} \Leftrightarrow \mathrm{d}T = \frac{1}{2}m\mathrm{d}(v^{2})$. Donc en reprenant l'\'equation (\ref{EQ:16_7}) :
\bea
	\dfrac{\mathrm{d}\omega_{0}}{4\pi} & = & \dfrac{1}{2}\pi\sin\theta_{0}\mathrm{d}\theta_{0} = -\dfrac{\mathrm{d}(v^{2})}{4v_{0}V} \nonumber \\
	\Leftrightarrow \dfrac{\mathrm{d}\omega_{0}}{4\pi} & = & -\dfrac{\mathrm{d}(T)}{4mv_{0}V} \label{EQ:16_8}
\eea
En reprenant $v^{2} = v_{0}^{2} + V^{2} + 2v_{0}V\cos\theta_{0}$, nous en d\'eduisons que :
\begin{itemize}
	\item l'\'energie cin\'etique maximale est obtenue pour $\theta_{0} = 0$ et $T_{max} = \frac{m}{2}(v_{0} + V)^{2}$
	\item l'\'energie cin\'etique minimale est obtenue pour $\theta_{0} = \pi$ et $T_{max} = \frac{m}{2}(v_{0} - V)^{2}$
\end{itemize}
et dans cet intervalle, l'\'energie cin\'etique se distribue suivant la relation (\ref{EQ:16_8}).

Si la d\'esint\'egration donne plus de deux composantes, cela complexifie l'\'etude et en particulier, l'\'energie des composantes est loin d'\^etre unique dans le r\'ef\'erentiel <<~c~>>. Toutefois il existe une valeur maximale de l'\'energie cin\'etique pour chaque particule r\'esultante. Parmi l'ensemble des particules r\'esultants, consid\'erons-en une de masse $m_{1}$ et en posant $E_{int}'$, l'\'energie <<~interne~>> de l'ensemble des particules restantes moins $m_{1}$. Puisque cette situation permet de revenir \`a un probl\`eme \`a deux corps, la formule (\ref{EQ:16_1}) permet d'\'ecrire :
\be
	E_{int} = E_{int}' + T' + T_{10} + E_{1int}
\ee
avec $T'$ l'\'energie cin\'etique de l'ensemble des particules restantes moins $m_{1}$, $T_{10}$ l'\'energie cin\'etique de $m_{1}$ et $E_{1int}$ son \'energie interne. Les \'energies cin\'etiques peuvent s'\'ecrire avec $M$ la masse de la particule initiale :
\be
	T' = \dfrac{p_{0}^{2}}{2(M - m_{1})}\text{ et }T_{10} = \dfrac{p_{0}^{2}}{2m_{1}}
\ee
donc :
\bea
	E_{int} - E_{int}' - E_{1int} & = & \dfrac{M}{M - m_{1}}\dfrac{p_{0}^{2}}{2m_{1}} \nonumber \\
	\Leftrightarrow T_{10} & = & \dfrac{M - m_{1}}{M}(E_{int} - E_{int}' - E_{1int})
\eea
Aussi $T_{10}$ est maximale si et seulement si $E_{int}'$ est minimale. Ceci intervient lorsque toutes les particules r\'esultantes, sauf $m_{1}$, ont la m\^eme vitesse, i.e. une agitation du syst\`eme minimale et $E_{int}'$ devient simplement la somme des \'energies internes de cet ensemble de particules. Dans ce cas pr\'ecis, la quantit\'e $E_{int} - E_{int}' - E_{1int}$ repr\'esente $\epsilon$, l'\'energie de d\'esint\'egration d\'efinie dans l'\'equation (\ref{EQ:16_2}). Nous en concluons :
\be
	T_{10max} = \dfrac{M - m_{1}}{M}\epsilon \label{EQ:16_9}
\ee

\section{Chocs \'elastiques des particules}

Le choc entre deux particules est dit \'elastique lorsqu'il n'y a pas de modifications de leur \'etat interne. Lors de l'application de la loi de conservation de l'\'energie, il n'est donc pas n\'ecessaire de prendre en compte les \'energies internes respectives. Dans le r\'ef\'erentiel du centre d'inertie <<~c~>>, ce dernier est de facto au repos. Avant le choc, par application de la conservation de l'impulsion, nous avons dans <<~c~>> : $m_{1}\vec{v}_{10} + m_{2}\vec{v}_{20} = \vec{0}$. Dans <<~l~>>, la position du centre d'inertie s'\'ecrit :
\be
	\vec{R} = \dfrac{m_{1}\vec{r}_{1} + m_{2}\vec{r}_{2}}{m_{1} + m_{2}}
\ee
et la loi de composition des vitesses am\`ene \`a \'ecrire :
\be
	\begin{cases}
		\vec{v}_{1} = \vec{v}_{10} + \frac{\mathrm{d}\vec{R}}{\mathrm{dt}} \\
		\vec{v}_{2} = \vec{v}_{20} + \frac{\mathrm{d}\vec{R}}{\mathrm{dt}}
	\end{cases}
\ee
donc en soustrayant les deux relations et en d\'efinissant\footnote{Voir une \'equivalence avec les relations (\ref{EQ:13_2})} $\vec{v} = \vec{v}_{1} - \vec{v}_{2}$ :
\be
	\vec{v} = \vec{v}_{10} - \vec{v}_{20}
\ee
En reprenant la conservation de l'impulsion dans <<~c~>>, nous avons alors :
\be
	\begin{cases}
		\vec{v} = \vec{v}_{10} + \frac{m_{1}}{m_{2}}\vec{v}_{10} \Leftrightarrow \vec{v}_{10} = \frac{m_{2}}{m_{1} + m_{2}}\vec{v} \\
		\vec{v} = -\vec{v}_{20} - \frac{m_{2}}{m_{1}}\vec{v}_{20} \Leftrightarrow \vec{v}_{20} = -\frac{m_{1}}{m_{1} + m_{2}}\vec{v}
	\end{cases}
\ee

Comme le choc est \'elastique, dans <<~c~>>, la conservation de l'impulsion après le choc donne $m_{1}\vec{v'}_{10} + m_{2}\vec{v'}_{20} = \vec{0}$ et comme avant le choc, nous avons $m_{1}\vec{v}_{10} + m_{2}\vec{v}_{20} = \vec{0}$, nous pouvons en conclure que c'est vrai si $m_{1}(\vec{v'}_{10} + \vec{v}_{10}) + m_{2}(\vec{v'}_{20} + \vec{v}_{20}) = \vec{0}$, soit $\vec{v'}_{10} = -\vec{v}_{10}$ et $\vec{v'}_{20} = -\vec{v}_{20}$. De m\^eme, la conservation de l'\'energie avant et apr\`es le choc, et qui ne concerne que les \'energies cin\'etiques, implique que l'\'energie de chacune des deux particules est conserv\'ees car $v'_{10} = v_{10}$ et $v'_{20} = v_{20}$.
Par cons\'equent, dans <<~c~>>, l'unique diff\'erence entre avant et apr\`es le choc se situe dans l'inversion de la direction de la vitesse de chacune des deux particules.

D\'efinissons le vecteur unitaire $\vec{n}_{0}$ dans la direction de la vitesse apr\`es le choc de la particule de masse $m_{1}$. Alors $\vec{v'}_{10} = v'_{10}\vec{n}_{0} = v_{10}\vec{n}_{0}$. Le vecteur $\vec{n}_{0}$ absorbe l'inversion de direction apr\`es le choc et permet de reprendre les relations ci-dessous pour en conclure :
\be
	\begin{cases}
		\vec{v'}_{10} = \frac{m_{2}}{m_{1} + m_{2}}v\vec{n}_{0} \\
		\vec{v'}_{20} = -\frac{m_{1}}{m_{1} + m_{2}}v\vec{n}_{0} \label{EQ:17_1}
	\end{cases}
\ee
Le passage de <<~c~>> \`a <<~l~>> via la loi de composition des vitesses et l'expression de la vitesse du centre d'inertie dans <<~l~>> permet d'\'ecrire :
\be
	\begin{cases}
		\vec{v'}_{1} = \vec{v'}_{10} + \frac{\mathrm{d}\vec{R'}}{\mathrm{dt}} \\
		\vec{v'}_{2} = \vec{v'}_{20} + \frac{\mathrm{d}\vec{R'}}{\mathrm{dt}}
	\end{cases}
\ee
avec :
\be
	\dfrac{\mathrm{d}\vec{R'}}{\mathrm{dt}} = \dfrac{m_{1}\vec{v'}_{1} + m_{2}\vec{v'}_{2}}{m_{1} + m_{2}} = \dfrac{m_{1}\vec{v}_{1} + m_{2}\vec{v}_{2}}{m_{1} + m_{2}}
\ee
par l'additivit\'e des int\'egrales du mouvement, voir le paragraphe (\ref{PAR:6}) qui implique que les lois de conservation ne d\'ependent pas des interactions en jeu, en particulier pour celle de l'impulsion ici. Par cons\'equent, en reportant les \'equations (\ref{EQ:17_1}) :
\be
	\begin{cases}
		\vec{v'}_{1} = \frac{m_{2}}{m_{1} + m_{2}}v\vec{n}_{0} + \frac{m_{1}\vec{v}_{1} + m_{2}\vec{v}_{2}}{m_{1} + m_{2}} \\
		\vec{v'}_{2} = -\frac{m_{1}}{m_{1} + m_{2}}v\vec{n}_{0} + \frac{m_{1}\vec{v}_{1} + m_{2}\vec{v}_{2}}{m_{1} + m_{2}} \label{EQ:17_2}
	\end{cases}
\ee
o\`u seul le vecteur $\vec{n}_{0}$ est une cons\'equence de la loi d'interaction entre les particules. En multipliant les relations (\ref{EQ:17_2}) par la masse respective des particules, nous arrivons \`a exprimer leur impulsion apr\`es le choc en fonction de celle avant le choc, le tout dans le r\'ef\'erentiel <<~l~>> :
\be
	\begin{cases}
		\vec{p'}_{1} = mv\vec{n}_{0} + \frac{m_{1}}{m_{1} + m_{2}}(\vec{p}_{1} + \vec{p}_{2}) \\
		\vec{p'}_{2} = -mv\vec{n}_{0} + \frac{m_{2}}{m_{1} + m_{2}}(\vec{p}_{1} + \vec{p}_{2}) \label{EQ:17_3}
	\end{cases}
\ee
avec $m$ la masse r\'eduite du syt\`eme d\'efinie telle que $m = \frac{m_{1}m_{2}}{m_{1} + m_{2}}$.

\begin{figure}[htb!]
	\begin{center}
		\begin{picture}(300,300)(0,0)
			%circle
			\linethickness{0.05mm}
			\put(150,150){\circle{200}}
			%vectors circle
			\linethickness{0.5mm}
			\put(60,150){\vector(1,0){90}}
			\put(150,150){\vector(1,0){50}}
			\put(60,150){\vector(7,6){115}}\put(90,190){$\vec{p'}_{1}$}
			\put(150,150){\vector(1,5){20}}\put(160,185){$\vec{n}_{0}$}
			\put(170,246){\vector(3,-10){29}}\put(195,190){$\vec{p'}_{2}$}
			%points circle
			\put(146,138){$O$}
			\put(51,148){$A$}
			\put(202,146){$B$}
			\put(175,250){$C$}
		\end{picture}
		\caption{Repr\'esentation g\'eom\'etrique d'un choc entre deux particules}\label{FIG:4_15}
	\end{center}
\end{figure}

Sur la figure (\ref{FIG:4_15}) est trac\'e le cercle de rayon $mv$. Les vecteurs $\vec{AC}$ et $\vec{CB}$ donnent respectivement $\vec{p'}_{1}$ et $\vec{p'}_{2}$ en accord avec les relations (\ref{EQ:17_3}). Les impulsions initiales $\vec{p}_{1}$ et $\vec{p}_{2}$ impliquent que les points $A$ et $B$ ne changent pas de position, aussi seul le point $C$ peut avoir une position quelconque sur le cercle de rayon $mv$. Par construction g\'eom\'etrique, les relations (\ref{EQ:17_3}) donnent :
\be
	\begin{cases}
		\vec{AO} = \frac{m_{1}}{m_{1} + m_{2}}(\vec{p}_{1} + \vec{p}_{2}) \\
		\vec{OB} = \frac{m_{2}}{m_{1} + m_{2}}(\vec{p}_{1} + \vec{p}_{2})
	\end{cases}
\ee

\subsection{Cas de la particule $m_{2}$ au repos avant le choc}

\begin{figure}[htb!]
	\begin{center}
		\begin{picture}(500,300)(0,0)
			%circles
			\linethickness{0.05mm}
			\put(100,150){\circle{200}}\put(90,25){$m_{1} < m_{2}$}
			\put(400,150){\circle{200}}\put(390,25){$m_{1} > m_{2}$}
			%vectors circle #1
			\linethickness{0.5mm}
			\put(40,150){\vector(1,0){160}}
			\put(40,150){\vector(9,5){130}}\put(90,190){$\vec{p'}_{1}$}
			\put(169,221){\vector(3,-7){32}}\put(165,180){$\vec{p'}_{2}$}
			%angles circle #1
			\linethickness{0.05mm}
			\multiput(100,150)(10,10){7}{\line(1,1){8}}
			\qbezier(55,150)(55,155)(50,155)\put(60,153){$\theta_{1}$}
			\qbezier(110,150)(110,155)(105,155)\put(114,154){$\xi$}
			\qbezier(180,150)(188,165)(194,165)\put(170,155){$\theta_{2}$}
			%points circle #1
			\put(95,135){$O$}
			\put(30,147){$A$}
			\put(202,147){$B$}
			\put(175,220){$C$}
			%vectors circle #2
			\linethickness{0.5mm}
			\put(250,150){\vector(1,0){250}}
			\put(250,150){\vector(10,3){220}}\put(390,200){$\vec{p'}_{1}$}
			\put(469,221){\vector(3,-7){32}}\put(465,180){$\vec{p'}_{2}$}
			%\theta_{max} case
			\linethickness{0.05mm}
			\put(250,150){\line(10,9){90}}
			\put(400,150){\line(-11,15){60}}
			%angles circle #2
			\linethickness{0.05mm}
			\multiput(400,150)(10,10){7}{\line(1,1){8}}
			\qbezier(280,150)(280,158)(275,158)\put(287,152){$\theta_{1}$}
			\qbezier(410,150)(410,155)(405,155)\put(412,153){$\xi$}
			\qbezier(480,150)(488,165)(494,165)\put(470,155){$\theta_{2}$}
			\qbezier(320,150)(320,190)(294,190)\put(315,182){$\theta_{max}$}
			%points circle #2
			\put(395,135){$O$}
			\put(240,147){$A$}
			\put(502,147){$B$}
			\put(475,220){$C$}
		\end{picture}
		\caption{Repr\'esentation g\'eom\'etrique de d\'esint\'egrations dans le cas $\vec{v}_{2} = \vec{0}$}\label{FIG:4_16}
	\end{center}
\end{figure}

Si la particule de masse $m_{2}$ est au repos avant le choc, cela se traduit par $v_{2} = \vec{0} = p_{2}$. Nous avons aussi :
\be
	\vec{OB} = \dfrac{m_{2}}{m_{1} + m_{2}}(\vec{p}_{1} + \vec{p}_{2}) = \frac{m_{1}m_{2}}{m_{1} + m_{2}}\vec{v}_{1} = m\vec{v}_{1} = m\vec{v}
\ee
car par d\'efinition, $\vec{v} = \vec{v}_{1} - \vec{v}_{2}$ et donc \'egal \`a $\vec{v}_{1}$ dans le cas qui nous occupe. Donc le point $B$ se trouve positionner sur le cercle de rayon $mv$. De plus :
\be
	\vec{AB} = \vec{AO} + \vec{OB} = \frac{m_{1}}{m_{1} + m_{2}}\vec{p}_{1} + \dfrac{m_{2}}{m_{1} + m_{2}}\vec{p}_{1} = \vec{p}_{1}
\ee
Le vecteur $\vec{AB}$ repr\'esente donc l'impulsion de d\'epart de la particule de masse $m_{1}$.

Les points $A$, $O$ et $B$ se situant sur la m\^eme droite, nous pouvons en d\'eduire :
\be
	AO = AB - OB = m_{1}v_{1} - \dfrac{m_{1}m_{2}}{m_{1} + m_{2}}v_{1} = \dfrac{m_{1}^{2}}{m_{1} + m_{2}}v_{1} = \dfrac{m_{1}}{m_{2}}OB
\ee
Par cons\'equent :
\begin{itemize}
	\item si $m_{1} < m_{2}$ alors le point $A$ se situe \`a l'int\'erieur du cercle
	\item si $m_{1} = m_{2}$ alors le point $A$ se situe sur le cercle
	\item si $m_{1} > m_{2}$ alors le point $A$ se situe \`a l'ext\'erieur du cercle
\end{itemize}
Ces cas sont repr\'esent\'es sur la figure (\ref{FIG:4_16}). Sur celle-ci, les angles $\theta_{1}$ et $\theta_{2}$ sont les angles de d\'eviation dans <<~l~>> par rapport \`a la direction du choc, d\'efinie par $\vec{p}_{1}$ et $\xi$ est l'angle de d\'eviation de la particule de masse $m_{1}$ dans <<~c~>>. Les angles $\theta_{1}$ et $\theta_{2}$ peuvent \^etre calcul\'es ainsi :
\be
	\begin{cases}
		\tan\theta_{1} = \dfrac{mv\sin\xi}{AO + mv\cos\xi} = \dfrac{\dfrac{m_{1}m_{2}}{m_{1} + m_{2}}v\sin\xi}{\dfrac{m_{1}^{2}}{m_{1} + m_{2}}v + \dfrac{m_{1}m_{2}}{m_{1} + m_{2}}v\cos\xi} = \dfrac{m_{2}\sin\xi}{m_{1} + m_{2}\cos\xi} \\
		\\
		\pi = \dfrac{\pi}{2} + \dfrac{\xi}{2} + \theta_{2} \Leftrightarrow \theta_{2} = \dfrac{\pi - \xi}{2} \label{EQ:17_4}
	\end{cases}
\ee

\subsection{Cas d'un choc entre deux particules de même masse dont l'une au repos avant le choc}

\begin{figure}[htb!]
	\begin{center}
		\begin{picture}(300,300)(0,0)
			%circles
			\linethickness{0.05mm}
			\put(150,150){\circle{200}}
			%vectors circle #1
			\linethickness{0.5mm}
			\put(50,150){\vector(1,0){200}}
			\put(50,150){\vector(9,4){170}}\put(140,200){$\vec{p'}_{1}$}
			\put(219,221){\vector(3,-7){32}}\put(215,180){$\vec{p'}_{2}$}
			%angles circle #1
			\linethickness{0.05mm}
			\multiput(150,150)(10,10){7}{\line(1,1){8}}
			\qbezier(65,150)(65,155)(60,155)\put(75,153){$\theta_{1}$}
			\qbezier(160,150)(160,155)(155,155)\put(164,154){$\xi$}
			\qbezier(230,150)(238,165)(244,165)\put(220,155){$\theta_{2}$}
			%points circle #1
			\put(145,135){$O$}
			\put(40,147){$A$}
			\put(252,147){$B$}
			\put(225,220){$C$}
		\end{picture}
		\caption{Repr\'esentation g\'eom\'etrique de d\'esint\'egrations dans le cas $\vec{v}_{2} = \vec{0}$ et $m_{1} = m_{2}$}\label{FIG:4_17}
	\end{center}
\end{figure}
\chapter{Petites oscillations}

\section{Oscillations lin\'eaires libres}\label{PAR:21}

Les \emph{petites oscillations} sont celles faites par un syst\`eme au voisinage de sa position d'\'equilibre stable et nous étudions ici le cas le plus simple, celui où il n'y a qu'un seul degr\'e de libert\'e. Un \'equilibre stable intervient quand l'\'energie potentielle $U(q)$ est minimale. D\'efinissons que $U$ soit minimale en $q = q_{0}$. L'\'equation (\ref{EQ:5_4}) montre qu'un \'ecart par rapport \`a la position $q_{0}$ engendre une force \'egale \`a $-\frac{\mathrm{d}U}{\mathrm{d}q}$ qui ram\`ene le syst\`eme \`a sa position d'\'equilibre stable. R\'ealisons un d\'eveloppement de Taylor de $U(q)$ jusqu'au second ordre :
\benn
	U(q) = U(q_{0}) + U'(q_{0})(q - q_{0}) + \dfrac{U''(q_{0})}{2}(q - q_{0})^{2} \Rightarrow U(q) - U(q_{0}) = \dfrac{k}{2}(q - q_{0})^{2}
\eenn
car $U'(q_{0}) = 0$ puisque $q_{0}$ est la position d'\'equilibre et nous avons pos\'e $k = U''(q_{0})$. En d\'efinissant :
\be
	x = q - q_{0} \label{EQ:21_1}
\ee
qui repr\'esente l'\'ecart par rapport \`a la position d'\'equilibre, nous avons, en posant $U(q_{0}) = 0$ :
\be
	U(x) = \dfrac{kx^{2}}{2} \label{EQ:21_2}
\ee
Au regard des relations (\ref{EQ:4_1}) et (\ref{EQ:5_5}), l'\'energie cin\'etique pour une particule et dans le cas d'un unique degr\'e de libert\'e, s'\'ecrit $T = \frac{1}{2}a(q)\dot{q}^{2}$. Or dans le cadre des petites oscillations, $x \approx q$. De la m\^eme mani\`ere, $a(q) \approx q(a_{0})$, quantit\'e qui peut s'identifier \`a la masse de la particule si $x$ est une coordonn\'ee cart\'esienne. L'\'energie cin\'etique peut donc se formuler comme $T = \frac{1}{2}m\dot{x}^{2}$. Par cons\'equent, la fonction de Lagrange pour un syst\`eme r\'ealisant de petites oscillations, d\'enomm\'e aussi \emph{oscillateur lin\'eaire} est :
\be
	L = \dfrac{m\dot{x}^{2}}{2} - \dfrac{kx^{2}}{2} \label{EQ:21_3}
\ee
Dans notre cas \'epur\'e, l'\'equation du mouvement s'obtient \`a partir de la relation (\ref{EQ:5_2}) :
\be
	\dfrac{\mathrm{d}}{\mathrm{dt}}\left(\dfrac{\partial L}{\partial \dot{x}}\right) = \dfrac{\partial L}{\partial x} \Leftrightarrow \dfrac{\mathrm{d}m\dot{x}}{\mathrm{dt}} = -kx \Leftrightarrow m\ddot{x} + kx = 0 \label{EQ:21_4}
\ee
\'Equation qui peut \'egalement s'\'ecrire :
\be
	\ddot{x} + \omega^{2}x = 0 \label{EQ:21_5}
\ee
en posant la quantit\'e :
\be
	\omega = \sqrt{\dfrac{k}{m}} \label{EQ:21_6}
\ee
qui est appel\'ee \emph{fr\'equence angulaire}. La quantit\'e $\omega$ ne d\'epend pas des conditions initiales du mouvement mais uniquement des propri\'et\'es m\'ecaniques du syst\`eme, c'est une constante fondamentale des oscillations. Ceci n'est valable que dans le cas qui nous concerne, celui des petites oscillations, ou encore quand $U \propto x^{2}$. Pour les cas g\'en\'eraux, nous pouvons nous reporter \`a l'\'equation (\ref{EQ:11_EX2A_1}) qui d\'efinit la p\'eriode d'oscillations $T$ pour $U = A\lvert x^{n} \rvert$, sachant que $\omega = 2\pi/T$.

L'\'equation (\ref{EQ:21_5} admet deux solutions ind\'ependantes en $\cos(\omega t)$ et en $\sin(\omega t)$. Donc la solution g\'en\'erale s'\'ecrit :
\be
	x = c_{1}\cos(\omega t) + c_{2}\sin(\omega t) \label{EQ:21_7}
\ee
ou encore :
\be
	x = a\cos(\omega t + \alpha) \label{EQ:21_8}
\ee
o\`u $a$ est l'\emph{amplitude} et $\alpha$ la valeur initiale de la \emph{phase}, qui d\'epend de l'origine prise pour $t$.Comme $\cos(\omega t + \alpha) = \cos(\omega t)\cos\alpha - \sin(\omega t)\sin\alpha$, alors nous pouvons en d\'eduire, \`a partir de la relation (\ref{EQ:21_7}), $c_{1} = a\cos\alpha$ et $c_{2} = -a\sin\alpha$. De plus :
\be
	\begin{cases}
		c_{1}^{2} + c_{2}^{2} = a^{2}(\cos^{2}\alpha + \sin^{2}\alpha) \Rightarrow a = \sqrt{c_{1}^{2} + c_{2}^{2}} \\
		\dfrac{\sin\alpha}{\cos\alpha} = \tan\alpha = -\dfrac{c_{2}}{c_{1}} \label{EQ:21_9}
	\end{cases}
\ee
L'\'energie totale du syst\`eme vaut, en y ajoutant la formule (\ref{EQ:21_8}) :
\bea
	E & = & T + U = \dfrac{m\dot{x}^{2}}{2} + \dfrac{kx^{2}}{2} = \dfrac{m}{2}(\dot{x}^{2} + \omega^{2}x^{2}) \nonumber \\
	& = & \dfrac{m}{2}(a^{2}\omega^{2}\sin^{2}(\omega t + \alpha) + a^{2}\omega^{2}\cos^{2}(\omega t + \alpha)) = \dfrac{m\omega^{2}a^{2}}{2} \label{EQ:21_10}
\eea
Il est souvent ais\'e de passer dans le domaine complexe pour simplifier les op\'erations math\'ematiques et de consid\'erer la solution (\ref{EQ:21_8}) comme :
\be
	x = \Re{\{Ae^{i\omega t}\}} \label{EQ:21_11}
\ee
avec l'\emph{amplitude complexe} s'exprimant comme :
\be
	A = ae^{i\alpha} \label{EQ:21_12}
\ee
dont le module est l'amplitude ordinaire et l'argument la phase initiale.

\section{Oscillations forc\'ees}\label{PAR:22}

Les \emph{oscillations forc\'ees} sont les oscillations d'un syst\`eme soumis \`a un champ ext\'erieur variable. En restant dans l'hypoth\`ese des petites oscillations, cela implique que le champ ext\'erieur est suffisamment faible pour provoquer des d\'eplacements faibles \'egalement. De plus, ici, nous restons dans le cadre d'un unique degr\'e de libert\'e.

\subsection{Cas g\'en\'eral}

L'\'energie potentielle totale du syst\`eme s'\'ecrit avec deux termes, \`a savoir l'\'energie potentielle propre en $\frac{1}{2}kx^{2}$ et celle due \`a l'action ext\'erieur $U_{e}(x,t)$. Au premire ordre, cette derni\`ere s'\'ecrit :
\benn
	U_{e}(x,t) = U_{e}(0,t) + x\left(\dfrac{\partial U_{e}}{\partial x}\right)(0,t)
\eenn
o\`u $U_{e}(0,t)$ est une fonction ne d\'ependant que du temps et de facto, elle est le r\'esultat d'une d\'eriv\'ee totale par rapport au temps. Dans cette partie est n\'eglig\'ee dans l'\'equation de Lagrange, voir (\ref{EQ:2_8}). De plus, la formule (\ref{EQ:5_8}) permet de d\'eduire que $-\frac{\partial U_{e}}{\partial x}$ est la force ext\'erieure qui s'exerce sur le syst\`eme. Elle est fonction du temps et nous la d\'esignons $F(t)$. Par cons\'equent, $U_{e}(x,t) = -xF(t)$ et la fonction de Lagrange peut se formuler ainsi :
\be
	L = \dfrac{m}{2}\dot{x}^{2} - \left(\frac{k}{2}x^{2} - xF(t)\right) \label{EQ:22_1}
\ee
L'\'equation du mouvement provenant de la formule (\ref{EQ:5_2}) permet d'en d\'eduite :
\bea
	\dfrac{\mathrm{d}}{\mathrm{dt}}\left(\dfrac{\partial L}{\partial \dot{x}}\right) & = & \dfrac{\partial L}{\partial x} \Leftrightarrow \dfrac{\mathrm{d}m\dot{x}}{\mathrm{dt}} = -kx + F(t) + x\dfrac{\partial F(t)}{\partial x} \nonumber \\
	& \Leftrightarrow & m\ddot{x} + kx = F(t) \Leftrightarrow \ddot{x} + \omega^{2}x = \dfrac{F(t)}{m} \label{EQ:22_2}
\eea
avec $\omega = \sqrt{\frac{k}{m}}$ la fr\'equence des oscillations. L'\'equation (\ref{EQ:22_2}) est une \'equation diff\'erentielle du second ordre avec des c{\oe}fficients constants et un second membre. La solution g\'en\'erale est de la forme $x = x_{0} + x_{1}$ telle que :
\begin{itemize}
	\item $x_{0}$ est la solution g\'en\'erale de l'\'equation sans second membre qui repr\'esente les oscillations libres d\'etermin\'ee aux \'equations (\ref{EQ:21_7}) et (\ref{EQ:21_8})
	\item $x_{1}$ est une int\'egrale particuli\`ere de l'\'equation avec le second membre
\end{itemize}

\subsection{Cas particulier d'une force ext\'erieure p\'eriodique}

Consid\'erons que la force ext\'erieure s'exprime telle que :
\be
	F(t) = f\cos(\gamma t + \beta) \label{EQ:22_3}
\ee
Pour cette hypoth\`ese, l'int\'egrale particuli\`ere de l'\'equation (\ref{EQ:22_2}) peut \^etre $x_{1} = b\cos(\gamma t + \beta)$, ce qui nous donne :
\bea
	& & -b\gamma^{2}\cos(\gamma t + \beta) + \omega^{2}b\cos(\gamma t + \beta) = \dfrac{f}{m}\cos(\gamma t + \beta) \Leftrightarrow b(\omega^{2} - \gamma^{2}) = \dfrac{f}{m} \nonumber \\
	& \Leftrightarrow & b = \dfrac{f}{m(\omega^{2} - \gamma^{2})} \nonumber
\eea
Le mouvement se compose donc en ajoutant la solution g\'en\'erale (\ref{EQ:21_8}) :
\be
	x = a\cos(\omega t + \alpha) + \frac{f}{m(\omega^{2} - \gamma^{2})}\cos(\gamma t + \beta) \label{EQ:22_4}
\ee
o\`u les quantit\'es $a$ et $\alpha$ sont d\'eduites des conditions initiales. Le mouvement est ainsi compos\'e de la somme de deux oscillations, la premi\`ere avec la fr\'equence propre du syst\`eme et la seconde avec la fr\'equence de la force ext\'erieure. La formule (\ref{EQ:22_4}) est ind\'etermin\'ee dans le cas o\`u $\omega = \gamma$, il y a alors \emph{r\'esonnance}.

\subsection{Solution r\'eelle lors de la r\'esonnance avec une force ext\'erieure}

Partons de la formule (\ref{EQ:22_4}) pour la reformuler ainsi :
\benn
	x = a\cos(\omega t + \alpha) + \frac{f}{m(\omega^{2} - \gamma^{2})}(\cos(\gamma t + \beta) - \cos(\omega t + \beta))
\eenn
Ainsi lorsque $\gamma \rightarrow \omega$, alors nous avons une ind\'etermination de type $0/0$ pour le second terme. Appliquons alors la r\`egle de L'Hospital\footnote{Si $f$ et $g$ sont deux fonctions d\'efinies sur $[a;b[$, d\'erivables en $a$ et telles que $f(a) = g(a) = 0$ et $g'(a) \neq 0$ alors $\lim_{x \rightarrow a^{+}}\dfrac{f(x)}{g(x)} = \dfrac{f'(a)}{g'(a)}$} avec :
\benn
	\begin{cases}
		f : \omega \mapsto \cos(\gamma t + \beta) - \cos(\omega t + \beta) \\
		g : \omega \mapsto w^{2} - \gamma^{2}
	\end{cases}
\eenn
Cela nous donne :
\benn
	\lim_{\gamma \rightarrow \omega}\dfrac{f(\gamma)}{g(\gamma)} = \dfrac{f'(\gamma = \omega)}{g'(\gamma = \omega)} = \dfrac{-t\sin(\omega t + \beta)}{-2\omega}
\eenn
et nous permet d'\'ecrire :
\be
	x = a\cos(\omega t + \alpha) + \frac{f}{2m\omega}t\sin(\omega t + \beta) \label{EQ:22_5}
\ee
Dans le cas de la r\'esonnance, l'amplitude augmente lin\'eairement avec le temps tant que nous restons dans le cadre des petites oscillations.

\subsection{Solution complexe lors de la r\'esonnance avec une force ext\'erieure}

Au voisinage de la r\'esonnance, d\'efinissons la fr\'equence $\gamma = \omega + \epsilon$ avec $\epsilon$ petit devant $\omega$. Dans le domaine complexe, il est possible de g\'en\'eraliser l'expression (\ref{EQ:22_4}) telle que :
\be
	x = Ae^{i\omega t} + Be^{i(\omega + \epsilon)t} = \left(A + Be^{i\epsilon t}\right)e^{i\omega t} \label{EQ:22_6}
\ee
Comme $\epsilon \ll \omega$, l'expression $A + Be^{i\epsilon t}$ varie peu lors d'une p\'eriode $2pi/\omega$. Au voisinage de la r\'esonnance, le mouvement se constitue de petites oscillations \`a amplitude variable. L'amplitude reste une grandeure r\'eelle aussi nous d\'efinissons $C = \lvert A + Be^{i\epsilon t} \lvert$. La relation (\ref{EQ:21_12}) permet d'\'ecrire $A = ae^{i\alpha}$ et $B = be^{i\beta}$, et ainsi l'amplitude $C$ devient :
\bea
	C & = & \lvert ae^{i\alpha} + be^{i(\epsilon t + \beta)} \rvert \nonumber \\
	\Leftrightarrow C^{2} & = & \lvert ae^{i\alpha} \rvert^{2} + \lvert be^{i(\epsilon t + \beta)} \rvert^{2} + 2\lvert ae^{i\alpha} \rvert \cdot \lvert be^{i(\epsilon t + \beta)} \rvert \cos(\epsilon t + \beta - \alpha) \nonumber \\
	C^{2} & = & a^{2} + b^{2} + 2ab\cos(\epsilon t + \beta - \alpha) \label{EQ:22_7}
\eea
Les valeurs extremum de $C$ sont :
\begin{itemize}
	\item $C_{max}^{2} = a^{2} + b^{2} + 2ab \Rightarrow C_{max} = a + b$
	\item $C_{min}^{2} = a^{2} + b^{2} - 2ab \Rightarrow C_{min} = \lvert a - b \rvert$
\end{itemize}
Par cons\'equent, l'amplitude $C$ oscille p\'eriodiquement entre les deux valeurs pr\'ec\'edentes avec une fr\'equence $\epsilon$, ph\'enom\`ene appel\'e \emph{battements}.

Ensuite, nous pouvons remarquer que l'\'equation (\ref{EQ:22_2}) peut s\'ecrire :
\bea
	\ddot{x} + i\omega\dot{x} - i\omega\dot{x} - i\omega\cdot i\omega x & = & \dfrac{F(t)}{m} \Leftrightarrow \dfrac{\mathrm{d}}{\mathrm{dt}}(\dot{x} + i\omega x) - i\omega(\dot{x} + i\omega x) = \dfrac{F(t)}{m} \nonumber \\
	\Rightarrow \dfrac{\mathrm{d}\xi}{\mathrm{dt}} - i\omega\xi = \dfrac{F(t)}{m} \label{EQ:22_8}
\eea
en d\'efinissant la quantit\'e $\xi$ telle que :
\be
	\xi = \dot{x} + i\omega x \label{EQ:22_9}
\ee
L'\'equation (\ref{EQ:22_8}) n'est plus du second ordre. Sa solution g\'en\'erale sans second membre peut d'\'ecrire comme $\xi_{0}e^{i\omega t}$ avec $\xi_{0}$ une constante d\'ependant des conditions initialies et \'egale \`a $\xi(t=0)$. Cherchons d\'esormais un int\'egrale particuli\`ere de la forme $A(t)e^{i\omega t}$ pour l'\'equation avec second membre. En ins\'erant cette solution particuli\`ere dans l'\'equation (\ref{EQ:22_8}) :
\bea
	\dot{A}(t)e^{i\omega t} + i\omega A(t)e^{i\omega t} - i\omega e^{i\omega t} & = & \dfrac{F(t)}{m} \Leftrightarrow \dot{A}(t) = \dfrac{1}{m}F(t)e^{-i\omega t} \nonumber \\
	\Rightarrow \xi(t) & = & e^{i\omega t}\left(\xi_{0} + \dfrac{1}{m}\int{F(t)e^{-i\omega t}\mathrm{dt}}\right) \label{EQ:22_10}
\eea
Ainsi la quantit\'e $x(t)$ est donn\'ee par la partie imaginaire de $\xi(t)$ donn\'ee par la relation (\ref{EQ:22_10}), divis\'ee par $\omega$.

Dans le cas d'oscillations forc\'ees, l'\'energie totale du syst\`eme ne se conservent \'evidemmment pas. L'\'energie totale transmise au syst\`eme par la force ext\'erieure sachant que cette \'energie est nulle initialement, peut se calculer \`a partir de la relation (\ref{EQ:22_10}) :
\be
	E = T + U = \dfrac{m}{2}\dot{x}^{2} + kx^{2} = \dfrac{m}{2}(\dot{x}^{2} + \omega^{2}x^{2}) = \dfrac{m}{2}(\dot{x} + i\omega x)(\dot{x} - i\omega x) = \dfrac{m}{2}\lvert \xi \rvert^{2} \label{EQ:22_11}
\ee
sachant que les quantit\'es $x$ et $\xi$ incluent d\'ej\`a les cons\'equences de la force ext\'erieure $F$ dans leur expression respective. En supposant $\xi(-\infty) = 0$ et parce que $\lvert e^{i\omega t} \rvert = 1$ par d\'efinition, l'\'energie transmise totale vaut en reprenant les relations (\ref{EQ:22_10}) et (\ref{EQ:22_11}) :
\be
	E_{transmise}(\infty) = \dfrac{m}{2}\dfrac{1}{m^{2}} \Big\lvert \int_{-\infty}^{\infty}{F(t)e^{-i\omega t}\mathrm{dt}} \Big\rvert^{2} = \dfrac{1}{2m} \Big\lvert \int_{-\infty}^{\infty}{F(t)e^{-i\omega t}\mathrm{dt}} \Big\rvert^{2} \label{EQ:22_12}
\ee
Nous pouvons remarquer que l'expression de l'\'energie transmise correspond \`a l'amplitude variable du mouvement de la relation (\ref{EQ:22_10}). De plus, le c{\oe}fficient de Fourier s'\'ecrit pour une fonction quelconque $f$, pour une p\'eriode $T$, comme :
\benn
	\forall n \le 0\text{, }c_{n}(f) = \dfrac{1}{T}\int_{T}f(t)e^{-i\frac{2n\pi}{T}t}\mathrm{dt}
\eenn
aussi l'\'energie transmise est le carr\'e du module du c{\oe}fficient de Fourier de la force $F(t)$ avec un fr\'equence \'egale \`a la fonction propre du syst\`eme physique.

Sur un intervalle de temps petit par rapport \`a $\frac{1}{\omega}$, nous avons $e^{-i\omega t} \approx 1$ et l'\'energie transmise :
\benn
	E = \dfrac{1}{2m} \left(\int_{-\infty}^{\infty}{F(t)\mathrm{dt}}\right)^{2}
\eenn
Par cons\'equent, une force qui ne s'exprime que sur une faible dur\'ee communique au syst\`eme une impulsion $\int F\mathrm{dt}$ qui ne permet pas de provoquer de d\'eplacement notable, ce qui revient \`a notre hypoth\`ese de base des petites oscillations.

\section{Oscillations des syst\`emes \`a plusieurs degr\'es de libert\'e}\label{PAR:23}

\subsection{Cas g\'en\'eral des oscillations libres}

Il s'agit d'une g\'en\'eralisation de l'\'etude men\'ee au paragraphe (\ref{PAR:21}), une m\'ethode similaire est applicable dans le cas qui nous occupe ici avec plusieurs degr\'es de libert\'es. L'\'energie potentielle est une fonction des coordonn\'ees g\'en\'eralis\'ees $\begin{Bmatrix}q_{i}\end{Bmatrix}^{s}_{1}$ et il est d\'efini qu'elle soit minimale pour $\forall i \in (1, 2, \ldots, s)\text{, }q_{i} = q_{i0}$. Cela permet de d\'efinir les petits d\'eplacements tels que :
\be
	\forall i \in (1, 2, \ldots, s)\text{, }x_{i} = q_{i} - q_{i0} \label{EQ:23_1}
\ee
Le d\'eveloppement de Taylor au second degr\'e de l'\'energie potentielle donne :
\benn
	U(\begin{Bmatrix}q_{i}\end{Bmatrix}) = U(\begin{Bmatrix}q_{i_{0}}\end{Bmatrix}) + \sum_{i}\left(\dfrac{\partial U}{\partial q_{i}}\right)_{q_{i}=q_{i0}}(q_{i} - q_{i0}) + \dfrac{1}{2}\sum_{i,j}\left(\dfrac{\partial^{2}U}{\partial q_{i}\partial q_{j}}\right)_{q_{i}=q_{i0},q_{j}=q_{j0}}(q_{i} - q_{i0})(q_{j} - q_{j0})
\eenn
Le syst\`eme est d\'efini comme \'etant \`a l'\'equilibre initialement, aussi :
\benn
	\left(\dfrac{\partial U}{\partial q_{i}}\right)_{q_{i}=q_{i0}} = 0
\eenn
donc :
\benn
	U(\begin{Bmatrix}q_{i}\end{Bmatrix}) - U(\begin{Bmatrix}q_{i_{0}}\end{Bmatrix}) = \dfrac{1}{2}\sum_{i,j}\left(\dfrac{\partial^{2}U}{\partial q_{i}\partial q_{j}}\right)_{q_{i}=q_{i0},q_{j}=q_{j0}}(q_{i} - q_{i0})(q_{j} - q_{j0})
\eenn
En comptant l'\'energie potentielle par rapport \`a sa valeur minimale et parce que $\partial x_{i} = \partial (q_{i} - q_{i0}) = \partial q_{i}$, alors l'\'energie potentielle est une forme d\'efinie positive :
\be
	U(\begin{Bmatrix}q_{i}\end{Bmatrix}) = \dfrac{1}{2}\sum_{i,j}k_{ij}x_{i}x_{j}\text{ avec }k_{ij} = \left(\dfrac{\partial^{2}U}{\partial x_{i}\partial x_{j}}\right)_{x_{i}=0,x_{j}=0} \label{EQ:23_2}
\ee
Sachant que $\partial x_{i}x_{j} = \partial x_{j}x_{i}$ alors $k_{ij} = k_{ji}$, les c{\oe}fficients sont sym\'etriques.

L'\'equation (\ref{EQ:5_5}) montre que l'\'energie cin\'etique s'\'ecrit sous la forme $\frac{1}{2}\sum_{i,j}a_{ij}{q}\dot{q}_{i}\dot{q}_{j}$. En faisant l'approximation suivante : $a_{ij}(q) = a_{ij}(q_{0}) = m_{ij} = m_{ji}$ avec $q_{0} = \begin{Bmatrix}q_{i_{0}}\end{Bmatrix}$ et $\forall i\text{, }\dot{q}_{i} = \dot{x}_{i}$, alors l'\'energie cin\'etique est la forme quadratique d\'efinie positive\footnote{Soit, $\forall x\text{, }T(x) \ge 0$ et $T(x) = 0 \Rightarrow x = 0$.} :
\be
	T = \dfrac{1}{2}\sum_{i,j}m_{ij}\dot{x}_{i}\dot{x}_{j} \label{EQ:23_3}
\ee
La fonction de Lagrange peut donc \^etre d\'eduite des relations (\ref{EQ:23_2}) et (\ref{EQ:23_3}) :
\be
	L = T - U = \dfrac{1}{2}\sum_{i,j}(m_{ij}\dot{x}_{i}\dot{x}_{j} - k_{ij}x_{i}x_{j}) \label{EQ:23_4}
\ee
Les \'equations du mouvement sont donn\'ees par (\ref{EQ:2_6}) :
\bea
	\forall i \in {1,\ldots,s}\text{, } \dfrac{\mathrm{d}}{\mathrm{dt}}\left(\dfrac{\partial L}{\partial\dot{q}_{i}}\right) - \dfrac{\partial L}{\partial q_{i}} & = & 0 \Leftrightarrow \dfrac{\mathrm{d}}{\mathrm{dt}}\left(\dfrac{\partial L}{\partial\dot{x}_{i}}\right) - \dfrac{\partial L}{\partial x_{i}} = 0 \nonumber \\
	\dfrac{\mathrm{d}}{\mathrm{dt}}\left(\dfrac{1}{2}\sum_{j}m_{ij}\dot{x}_{j}\right) + \dfrac{1}{2}\sum_{j}k_{ij}x_{j} & = & 0 \Leftrightarrow  \sum_{j}m_{ij}\ddot{x}_{j} + \sum_{j}k_{ij}x_{j} = 0 \nonumber \\
	\sum_{j}(m_{ij}\ddot{x}_{j} + k_{ij}x_{j}) & = & 0 \label{EQ:23_5}
\eea
qui forment, pour chaque particule $i$, le syst\`eme de $s$ \'equations diff\'erentielles lin\'eaires sans second membre et \`a c{\oe}fficients constants. En partant de la solution complexe construite par les relations (\ref{EQ:21_11}) et (\ref{EQ:21_12}), nous devons chercher les $s$ fonctions inconnues $x_{j}(t)$ telles que :
\be
	\forall j \in {1,\ldots,s}\text{, } x_{j}(t) = A_{j}e^{i\omega t} \label{EQ:23_6}
\ee
En utilisant la relation (\ref{EQ:23_5}), nous obtenons un syst\`eme de $s$ \'equations lin\'eaires et homog\`enes :
\be
	\sum_{j}(-\omega^{2}m_{ij} + k_{ij})A_{j}e^{i\omega t} = 0 \Leftrightarrow \sum_{j}(k_{ij} - \omega^{2}m_{ij})A_{j} = 0 \label{EQ:23_7}
\ee
qui s'\'ecrit sous forme matricielle ainsi :
\benn
	\begin{pmatrix}
		k_{11} - \omega^{2}m_{11} & \ldots & k_{1j} - \omega^{2}m_{1j} & \ldots & k_{1s} - \omega^{2}m_{1s} \\
		\vdots & \ddots & \vdots & \ddots & \vdots \\
		k_{i1} - \omega^{2}m_{i1} & \ldots & k_{ij} - \omega^{2}m_{ij} & \ldots & k_{is} - \omega^{2}m_{is} \\
		\vdots & \ddots & \vdots & \ddots & \vdots \\
		k_{i1} - \omega^{2}m_{s1} & \ldots & k_{sj} - \omega^{2}m_{sj} & \ldots & k_{ss} - \omega^{2}m_{ss} \\
	\end{pmatrix}
	\cdot
	\begin{pmatrix}
		A_{1} \\
		\vdots \\
		A_{j} \\
		\vdots \\
		A_{s}
	\end{pmatrix}
	= \vec{0}
\eenn
\`A l'exception de la solution \'evidente, i.e. $\forall j\text{, }A_{j} = 0$, le syst\`eme admet des solutions si et seulement si :
\benn
	\det{(k_{ij} - \omega^{2}m_{ij})} = 0 \label{EQ:23_8}
\eenn
qui est une \emph{\'equation caract\'eristique} de degr\'e $s$ par rapport \`a $\omega^{2}$. En g\'en\'eral, il y a $s$ racines r\'eelles et positives distinctes $\omega_{\alpha}$ avec $\alpha \in \{1, \cdots , s\}$. Les valeurs de $\omega_{\alpha}$ sont les \emph{fr\'equences propres} du syst\`eme.

Dans un syst\`eme physique, si $\omega_{\alpha}$ poss\`ede une partie imaginaire, $\omega_{\alpha}^{2} < 0$ ou $\omega_{\alpha} \in \mathbb{C}$ et peut donc s\'ecrire ainsi $\omega_{\alpha} = a + ib$ alors dans la relation (\ref{EQ:23_6}), $x_{j}(t) = A_{j}e^{-bt}e^{iat}$. Le terme $e^{-bt}$ est une terme exponentiel croissant ou d\'ecroissant suivant la valeur de $b$. L'\'energie totale du sys\`eme \'evolue alors avec le temps, ce qui est en contradiction avec la loi de conservation de l'\'energie.

D'un point de vue mathématique, l'\'equation (\ref{EQ:23_7}) permet d'\'ecrire :
\benn
	\sum_{j}(k_{ij} - \omega^{2}m_{ij})A_{j} = 0 \Leftrightarrow \sum_{i,j}(k_{ij} - \omega^{2}m_{ij})A_{i}^{*}A_{j} = 0
\eenn
avec $A_{i}^{*}$ la quantité conjugu\'ee de $A_{i}$. Nous obtenons alors :
\benn
	\omega^{2} = \dfrac{\sum_{i,j}k_{ij}A_{i}^{*}A_{j}}{\sum_{i,j}m_{ij}A_{i}^{*}A_{j}}
\eenn
o\`u les quantit\'es $k_{ij}$ et $m_{ij}$ sont r\'elles et sym\'etriques. Donc :
\benn
	\left(\sum_{ij}k_{ij}A_{i}^{*}A_{j}\right)^{*} = \sum_{ij}k_{ij}A_{i}A_{j}^{*} = \sum_{ij}k_{ij}A_{j}A_{i}^{*}
\eenn
Ceci est aussi vrai pour le d\'enominateur de l'expression donnant $\omega^{2}$. En accord avec les d\'finitions (\ref{EQ:23_2}) et (\ref{EQ:23_3}), $\omega^{2}$ a alors des formes d\'efinies positives au num\'erateur et au d\'enominateur. C'est donc pouquoi les valeurs de $\omega_{\alpha}^{2}$ ne peuvent \^etre que réelles et positives.

Les fr\'equences $\omega_{\alpha}$ \'etant connues, les \'equations (\ref{EQ:23_7}) permettent d'obtenir les valeurs $A_{j}$. Si les valeurs $\omega_{\alpha}$ sont toutes distinctes alors les c{\oe}fficients $A_{j}$ sont proportionneles aux mineurs\footnote{Le mineur est le d\'eterminant qu'une sous-matrice carr\'ee.} de $k_{ij} - \omega^{2}m_{ij}$ en posant $\omega = \begin{Bmatrix}\omega_{\alpha}\end{Bmatrix}^{s}_{1}$. Soit $\Delta_{j\alpha}$ le mineur permettant le calcul de la quantit\'e $A_{j}$ avec pour fr\'equence $\omega_{\alpha}$, l'expression (\ref{EQ:23_5}) peut alors s'\'ecrire : $x_{j} = \Delta_{j\alpha}C_{\alpha}e^{i\omega_{\alpha}t}$ avec $C_{\alpha} \in \mathbb{C}$ et $\Delta_{j\alpha} \in \mathbb{R}$. Il s'agit alors d'une solution particuli\`ere puisque calcul\'ee pour une fr\'equence $\omega_{\alpha}$ donn\'ee. La solutino g\'en\'erale est la sommes des solutions particuli\`eres qui vaut dans $\mathbb{R}$ :
\be
	x_{j} = \Re{\left(\sum_{\alpha=1}^{s}\Delta_{j\alpha}C_{\alpha}e^{i\omega_{\alpha}t}\right)} = \sum_{\alpha=1}^{s}\Delta_{j\alpha}\Re{\left(C_{\alpha}e^{i\omega_{\alpha}t}\right)} = \sum_{\alpha=1}^{s}\Delta_{j\alpha}\Theta_{\alpha} \label{EQ:23_9}
\ee
en d\'efinissant :
\be
	\Theta_{\alpha} = \Re{\left(C_{\alpha}e^{i\omega_{\alpha}t}\right)} \label{EQ:23_10}
\ee
qui repr\'esente une oscillation p\'eriodique simple avec une amplitude et une phase arbitraires mais avec une fr\'equence $\omega_{\alpha}$ bien d\'etermin\'ee, voir (\ref{EQ:21_11}).

Les relations (\ref{EQ:23_9}) peuvent \^etre interpr\^et\'ees comme un syst\`eme de $s$ \'equations \`a $s$ inconnues $\Theta_{\alpha}$, solutions qui peuvent \^etre exprim\'ees en fonction des coordonn\'ees $x_{j}$. Les solutions $\Theta_{\alpha}$ peuvent ainsi \^etre consid\'er\'ees comme de nouvelles coordonn\'ees g\'en\'eralis\'ees. Elles sont dites \emph{normales} et les oscillations p\'eriodiques simples associ\'ees sont dites \emph{oscillations simples} du syst\`eme.

En supposant que le param\`etre $C_{\alpha}$ s'écrive $a_{\alpha} + ib_{\alpha}$ alors :
\bea
	\Theta_{\alpha} & = & \Re{\left((a_{\alpha} + ib_{\alpha})e^{i\omega_{\alpha}t}\right)} = \Re{\left((a_{\alpha} + ib_{\alpha})(\cos(\omega{\alpha}t) + i\sin(\omega_{\alpha}t))\right)} \nonumber \\
	& = & \Re{\left(a_{\alpha}\cos(\omega{\alpha}t) - b_{\alpha}\sin(\omega{\alpha}t) + i(a_{\alpha}\sin(\omega{\alpha}t) + b_{\alpha}\cos(\omega{\alpha}t))\right)} = a_{\alpha}\cos(\omega{\alpha}t) - b_{\alpha}\sin(\omega{\alpha}t) \nonumber \\
	\Rightarrow \ddot{\Theta}_{\alpha} & = & -a_{\alpha}\omega_{\alpha}^{2}\cos(\omega{\alpha}t) + b_{\alpha}\omega_{\alpha}^{2}\sin(\omega{\alpha}t) \nonumber
\eea
Ceci permet d'\'ecrire en d\'efinitive que :
\be
	\ddot{\Theta}_{\alpha} + \omega_{\alpha}^{2}\Theta_{\alpha} = 0 \label{EQ:23_11}
\ee
Ainsi, pour les coordonn\'ees normales, nous avons $s$ \'equations ind\'ependantes et de facto, ces coordonn\'ees normales sont ind\'ependantes les unes des autres. Et la fonction de Lagrange du syst\`eme est :
\be
	L = \sum_{\alpha}\dfrac{m_{\alpha}}{2}\left(\dot{\Theta_{\alpha}}^{2} - \omega_{\alpha}^{2}\Theta_{\alpha}^{2}\right) \label{EQ:23_12}
\ee
avec $\forall \alpha \in \{1, \cdots , s\}\text{, }m_{\alpha} \ge 0$.

Math\'ematiquement, la transformation (\ref{EQ:23_9}) permet aux \'energies cin\'etique et potentielle d'avoir une forme diagonale. Habituellement, les coordonn\'ees normales sont choisies telles que les c{\oe}fficients des carr\'es des vitesses dans la fonction de Lagrange soient \'egaux \`a $\frac{1}{2}$. En posant les coordonn\'ees normales :
\be
	Q_{\alpha} = \sqrt{m_{\alpha}}\Theta_{\alpha} \label{EQ:23_13}
\ee
la fonction de Lagrange s'\'ecrit alors :
\benn
	L = \sum_{\alpha}\dfrac{m_{\alpha}}{2}\left(\dot{\Theta_{\alpha}}^{2} - \omega_{\alpha}^{2}\Theta_{\alpha}^{2}\right) = \sum_{\alpha}\dfrac{m_{\alpha}}{2}\left(\dfrac{\dot{Q_{\alpha}}^{2}}{m_{\alpha}} - \omega_{\alpha}^{2}\dfrac{Q_{\alpha}^{2}}{m_{\alpha}}\right) = \dfrac{1}{2}\sum_{\alpha}\left(\dot{Q_{\alpha}}^{2} - \omega_{\alpha}^{2}Q_{\alpha}^{2}\right)
\eenn

\subsection{Cas tridimensionnel}

Dans le cas d'oscillations tridimensionnelles d'un point mat\'eriel de masse $m$ dans un champ ext\'erieur constant dans le temps, l'\'energie cin\'etique ne d\'epend pas du choix de la direction des axes des coordonn\'ees et :
\benn
	T = \dfrac{m}{2}(\dot{x}^{2} + \dot{y}^{2} + \dot{z}^{2})
\eenn
En pla\c{c}ant l'origine des coordonn\'ees au minimum de l'\'energie potentielle $U(x,y,z)$, une rotation des axes permet de rendre diagonal l'\'energie potentielle, i.e. $\frac{\partial U}{\partial x} = f(x)$, $\frac{\partial U}{\partial y} = g(y)$ et $\frac{\partial U}{\partial z} = h(z)$. En reprenant la formulation de l'\'energie potentielle d\'ecrite dans la relation (\ref{EQ:21_2}), la fonction de Lagrange devient :
\be
	L = \dfrac{m}{2}(\dot{x}^{2} + \dot{y}^{2} + \dot{z}^{2}) - \dfrac{1}{2}(k_{x}x^{2} + k_{y}y^{2} + k_{z}z^{2}) \label{EQ:23_14}
\ee
avec le long des axes, des oscillations de fr\'equences : $\omega_{x} = \sqrt{k_{x}/m}$, $\omega_{y} = \sqrt{k_{y}/m}$ et $\omega_{z} = \sqrt{k_{z}/m}$. En g\'en\'eralisant l'\'ecriture, nous avons :
\benn
	L = \dfrac{1}{2}\sum_{\alpha = x,y,z}\left((\sqrt{m}\dot{x}_{\alpha})^{2} - (\omega_{\alpha}x_{\alpha})^{2}\right)
\eenn
Les coordonn\'ees normales sont alors dans ce cas pr\'ecis $Q_{\alpha} = \sqrt{m}x_{\alpha}$. Dans le cas sp\'ecifique d'un champ central, alors $k_{x} = k_{y} = k_{z} = k$ et $Q = \sqrt{m}r$.

\subsection{Oscillations forc\'ees}

La g\'en\'eralisation de la fonction de Lagrange formul\'ee dans l'\'equation (\ref{EQ:22_1}) \`a plusieurs degr\'e de libert\'e est :
\be
	L = L_{0} + \sum_{i}F_{i}(t)x_{i} \label{EQ:23_15}
\ee
avec $L_{0}$ la fonction de Lagrange des oscillations libres, voir la relation (\ref{EQ:23_14}), et $F_{k}$, la force ext\'erieure appliqu\'ee sur la coordonn\'ees $x_{k}$. La formule (\ref{EQ:23_9}) nous donne :
\benn
	x_{j} = \sum_{\alpha=1}^{s}\Delta_{j\alpha}\Theta_{\alpha} = \sum_{\alpha=1}^{s}\dfrac{\Delta_{j\alpha}}{\sqrt{m_{\alpha}}}Q_{\alpha}
\eenn
aussi la fonction de Lagrange se d\'eveloppe ainsi :
\bea
	L & = & \dfrac{1}{2}\sum_{\alpha}\left(\dot{Q_{\alpha}}^{2} - \omega_{\alpha}^{2}Q_{\alpha}^{2}\right) + \sum_{i}F_{i}(t)\sum_{\alpha}\dfrac{\Delta_{i\alpha}}{\sqrt{m_{\alpha}}}Q_{\alpha} \nonumber \\
	& = & \dfrac{1}{2}\sum_{\alpha}\left(\dot{Q_{\alpha}}^{2} - \omega_{\alpha}^{2}Q_{\alpha}^{2}\right) + \sum_{\alpha}\left(\sum_{i}\dfrac{F_{i}(t)\Delta_{i\alpha}}{\sqrt{m_{\alpha}}}\right)Q_{\alpha} \nonumber \\
	& = & \dfrac{1}{2}\sum_{\alpha}\left(\dot{Q_{\alpha}}^{2} - \omega_{\alpha}^{2}Q_{\alpha}^{2}\right) + \sum_{\alpha}f_{\alpha}(t)Q_{\alpha} \label{EQ:23_16}
\eea
avec $f_{\alpha}(t) = \sum_{i}\frac{F_{i}(t)\Delta_{i\alpha}}{\sqrt{m_{\alpha}}}$. Les \'equations du mouvement obtenues par l'\'equation (\ref{EQ:5_2}) sont alors :
\be
	\forall \alpha \in {1,\ldots,s}\text{, }\dfrac{\mathrm{d}}{\mathrm{dt}}\left(\dfrac{\partial L}{\partial\dot{Q}_{\alpha}}\right) - \dfrac{\partial L}{\partial Q_{\alpha}} = 0 \Leftrightarrow \ddot{Q}_{\alpha} + \omega_{\alpha}^{2}Q_{\alpha} = f_{\alpha}(t) \label{EQ:23_17}
\ee
dont la seule inconnue est la fonction $Q_{\alpha}(t)$.

\section{Oscillations des mol\'ecules}\label{PAR:24}

Dans le cas d'un syst\`eme de $n$ particules interagissant les unes sur les autres sans champ ext\'erieur, tous les degr\'es de libert\'e ne sont pas issus de ph\'enom\`enes vibratoires. Comme une mol\'ecule peut avoir des mouvements de translation et de rotation, le mouvement vibratoire occupe $3n - 6$ degr\'es de libert\'e. Si la mol\'ecule est form\'ee d'atomes align\'es, la rotation autour de l'axe n'a pas de sens et alors la mol\'ecule poss\`ede $3n - 5$ degr\'es de libert\'es vibratoires.

Pour r\'esoudre le probl\`eme m\'ecanique des oscillations d'une mol\'ecule, il est utile d'exclure les degr\'es de libert\'e issus des mouvements de translation et de rotation.

\subsection{\'Elimination de la translation}

Pour se faire, il est n\'ecessaire de se placer dans le r\'ef\'erentiel du centre d'intertie dans lequel l'impulsion totale du syst\`eme est nulle, voir le paragraphe (\ref{PAR:8}). Cela s'\'ecrit en reprenant la relation (\ref{EQ:8_3}) et en posant $\vec{r}_{a} = \vec{r}_{a0} + \vec{u}_{a}$ o\`u $\vec{r}_{a0}$ est la position d'\'equilibre immobile et $\vec{u}_{a}$ l'\'ecart par rapport \`a la position d'\'equilibre :
\benn
	\dfrac{\mathrm{d}\vec{R}}{\mathrm{dt}} = \dfrac{\mathrm{d}}{\mathrm{dt}}\left(\dfrac{\sum_{a}m_{a}\vec{r}_{a}}{\sum_{a}m_{a}}\right) = \vec{0} \Leftrightarrow \sum_{a}m_{a}\vec{\dot{r}}_{a} = \vec{0}
\eenn
En posant $\vec{r}_{a} = \vec{r}_{a0} + \vec{u}_{a}$ avec $\vec{r}_{a0}$ la position immobile, donc constant dans le temps, de l'atome $a$ \`a l'\'equilibre et $\vec{u}_{a}$, l'\'ecart par rapport \`a l'\'equilibre, nous obtenons :
\be
	\sum_{a}m_{a}\vec{\dot{u}}_{a} = \vec{0} \Leftrightarrow \sum_{a}m_{a}\vec{u}_{a} = \vec{cste} = \vec{0} \label{EQ:24_1}
\ee
par choix de la condition initiale.

\subsection{\'Elimination de la rotation}

De la m\^eme mani\`ere, il est n\'ecessaire de trouver un r\'ef\'erentiel ou une transformation pour laquelle le moment cin\'etique s'annule. En gardant la d\'efinition $\vec{r}_{a} = \vec{r}_{a0} + \vec{u}_{a}$ alors le moment cin\'etique du syst\`eme :
\benn
	\vec{M} = \sum_{a}m_{a}\vec{r}_{a}\wedge\vec{v}_{a} = \sum_{a}m_{a}(\vec{r}_{a0} + \vec{u}_{a}\wedge\vec{\dot{r}}_{a} \approx \sum_{a}m_{a}\vec{r}_{a0}\wedge\dfrac{\mathrm{d}\vec{\dot{u}_{a}}}{\mathrm{dt}}
\eenn
pour rester dans le cadre des petits d\'eplacements ou oscillations. Cela nous donne alors :
\be
	\vec{M} \approx \dfrac{\mathrm{d}}{\mathrm{dt}}\left(\sum_{a}m_{a}\vec{r}_{a0}\wedge\vec{u}_{a}\right) \Rightarrow \sum_{a}m_{a}\vec{r}_{a0}\wedge\vec{u}_{a} = \vec{0} \label{EQ:24_2}
\ee

\subsection{Mol\'ecule dans un plan}

Si les $n$ atomes d'une mol\'ecule sont dans un plan, nous pouvons distinguer les oscillations normales qui laissent les atomes dans le plan de celles qui les fait sortir. Dans un mouvement plan, il y a $2n$ degr\'es de libert\'e dont 2 de translations, le long des deux axes d\'efinissant le plan, et 1 de rotation, par rapport \`a un axe perpendiculaire au plan en question. Le nombre des oscillations normales laissant les atomes dans le plan est donc \'egal \`a $2n - 3$. Le nombre de degr\'es vibratoires faisant sortir les atomes du plan est donc de $(3n - 6) - (2n - 3) = n - 3$.

\subsection{Mol\'ecule lin\'eaire}

Il existe deux types d'oscillations : celles qui sont longitudinales et celles qui \'ecartent les atomes de la forme rectiligne de la mol\'ecule. Sur une droite, il y a $n$ degr\'es de libert\'e dont 1 de translation. Le nombre d'oscillations longitudinales est de $n - 1$. Nous avons vu pr\'ec\'edemment que le nombre de degr\'es vibratoires total pour une mol\'ecule lin\'eaire est de $3n - 5$. Il reste donc $(3n - 5) - (n - 1) = 2n - 4$ degr\'es de libert\'e vibratoires faisant sortir les atomes de la forme rectiligne de la mol\'ecule. Ces $2n - 4 = 2(n - 2)$ oscillations correspondant \`a $n - 2$ fr\'equences distinctes dans deux plans orthogonaux entre eux, correspondant \`a des oscillations normales.

\section{Oscillations amorties}\label{PAR:25}

En r\'{e}alit\'{e}, tout corps se meut dans un milieu qui lui offre un r\'{e}sistance. L'\'{e}nergie dissip\'{e}e se transforme alors en chaleur et se dissipe. Le mouvement n'est plus dans ces conditions un processus purement m\'{e}canique et son \'{e}tude exige aussi celle du mouvement du milieu, celle de son \'{e}tat interne et de celle du corps. Il existe cependant une cat\'{e}gorie de cas o\`{u} le mouvement dans un milieu peut \^{e}tre d\'{e}crit approximativement \`{a} l'aide des \'{e}quations de la M\'{e}canique en introduisant des termes suppl\'{e}mentaires, comme une force de frottement d\'{e}pendant uniquement de la vitesse dans un milieu homog\`{e}ne. Avec $x$, une coordonn\'{e}e g\'{e}n\'{e}ralis\'{e}e, nous pouvons supposer la force de frottement dans le cas de petites oscillations lin\'{e}aires telle que :
\benn
	f(\dot{x}) = f(0) + \dfrac{\mathrm{d}f}{\mathrm{d}\dot{x}}\bigg|_{\dot{x}=0}(\dot{x} - 0) = \dfrac{\mathrm{d}f}{\mathrm{d}\dot{x}}\bigg|_{\dot{x}=0}\dot{x} = -\alpha\dot{x}
\eenn
avec $alpha > 0$ car la force de frottement agit contre le mouvement libre.

Par extension de l'\'{e}quation (\ref{EQ:21_4}), nous obtenons :
\be
	m\ddot{x} = -kx  - \alpha\dot{x} \label{EQ:25_1}
\ee
En d\'{e}finissant $\omega_{0}$ comme la fr\'{e}quence des oscillations libres et $\lambda$ comme le \emph{c{\oe}fficient d'ammortissement}, telles que :
\be
	\begin{cases}
		\omega_{0}^{2} = \dfrac{k}{m} \\
		2\lambda = \dfrac{\alpha}{m}
	\end{cases} \label{EQ:25_2}
\ee
alors l'\'{e}quation (\ref{EQ:25_1}) s'\'{e}crit :
\be
	\ddot{x} + 2\lambda\dot{x} + \omega_{0}^{2}x = 0 \label{EQ:25_3}
\ee
En suivant les r\`{e}gles g\'{e}n\'{e}rales de r\'{e}solution des \'{e}quations lin\'{e}aires \`{a} c{\oe}fficients constants, nous posons $x = e^{r\mathrm{t}}$ qui, introduit dans l'{e}quation (\ref{EQ:25_3}) permet d'\'{e}crire l'\'{e}quation caract\'{e}ristique :
\benn
	r^{2} + 2\lambda r + \omega_{0}^{2} = 0
\eenn
qui a pour solutions :
\benn
	r_{1,2} = -\lambda \pm \sqrt{\lambda^{2} - \omega_{0}^{2}}
\eenn
La solution g\'{e}n\'{e}rale \`{a} l'\'{e}quation (\ref{EQ:25_3}) est donc :
\benn
	x = c_{1}e^{r_{1}\mathrm{t}} + c_{2}e^{r_{2}\mathrm{t}}
\eenn
Trois cas sont alors possibles et sont discut\'{e}s ici.

\subsection{$\lambda < \omega_{0}$}

Dans ce cas, les deux solutions $r_{1,2} \in \mathbb{C}$ et sont conjugu\'{e}es l'une de l'autre. Cela a pour cons\'{e}quence que $\Re{\{r_{1}\}} = \Re{\{r_{2}\}}$. La solution g\'{e}n\'{e}rale peut donc s'\'{e}crire :
\benn
	x = \Re{\left\{Ae^{\left(-\lambda + \sqrt{-(\omega_{0}^{2} - \lambda^{2})}\right)\mathrm{t}}\right\}} = \Re{\left\{Ae^{-\lambda\mathrm{t}}e^{i\sqrt{\omega_{0}^{2} - \lambda^{2}}\mathrm{t}}\right\}}
\eenn
avec $A \in \mathbb{C}$. En \'{e}crivant $A = a+ib$, la solution $x$ devient :
\bea
	x & = & e^{-\lambda\mathrm{t}}\Re{\left\{(a + ib)\left(\cos\left(\sqrt{\omega_{0}^{2} - \lambda^{2}}\mathrm{t}\right) + i\sin\left(\sqrt{\omega_{0}^{2} - \lambda^{2}}\mathrm{t}\right)\right)\right\}} \nonumber \\
	& = & e^{-\lambda\mathrm{t}}\left[a\cos\left(\sqrt{\omega_{0}^{2} - \lambda^{2}}\mathrm{t}\right) - b\sin\left(\sqrt{\omega_{0}^{2} - \lambda^{2}}\mathrm{t}\right)\right] \nonumber
\eea
La solution $x$ peut donc s'\'{e}crire comme :
\be
	x = ae^{-\lambda\mathrm{t}}\cos(\omega\mathrm{t} + \varphi) \label{EQ:25_4}
\ee
avec $\omega = \sqrt{\omega_{0}^{2} - \lambda^{2}}$ et $(a;\varphi) \in \mathbb{R}^{2}$. Il s'agit d'\emph{oscillations amorties} qui correspondent \`{a} une oscillation harmonique dont l'amplitude diminue exponentiellement et avec une fr\'{e}quence $\omega$ inf\'{e}rieure \`{a} $\omega_{0}$, la fr\'{e}quence des oscillations libres.

Dans le cadre o\`{u} le c{\oe}fficient d'ammortissement n\'{e}gligeable devant la fr\'{e}quence propre du syst\`{e}me, i.e. $\lambda \ll \omega_{0}$ alors $\omega = \omega_{0}\sqrt{1 - \lambda^{2}/\omega_{0}^{2}}$ diff\`{e}re de $\omega_{0}$ d'un infinimement petit du second ordre. Avec $\lambda \ll \omega_{0}$, sur une p\'{e}riode $2\pi/\omega$, le facteur d'ammortissement $e^{-2\pi\lambda/\omega}$ est proche de 1. Alors sur cette p\'{e}riode, les carr\'{e}s de la coordonn\'{e}e et de la vitesse, soit l'\'{e}nergie totale du syst\`{e}me, sont proportionnels \`{a} $(e^{-\lambda\mathrm{t}})^{2} = e^{-2\lambda\mathrm{t}}$. Ceci peut s'\'{e}crire :
\be
	\overline{E} = E_{0}e^{-2\lambda\mathrm{t}} \label{EQ:25_5}
\ee
L'\'{e}nergie du moyenne du syst\`{e}me diminue depuis sa valeur initiale $E_{0}$, dissip\'{e}e par les frottements.

\subsection{$\lambda > \omega_{0}$}

Alors, les deux valeurs pour $r_{1,2} \in \mathbb{R}$ et toutes les deux sont n\'{e}gatives car, de fait $\lambda > \sqrt{\lambda^{2} - \omega_{0}^{2}}$. La forme g\'{e}n\'{e}rale de la solution est :
\be
	x = c_{1}e^{\left(-\lambda \pm \sqrt{\lambda^{2} - \omega_{0}^{2}}\right)\mathrm{t}} + c_{2}e^{\left(-\lambda \pm \sqrt{\lambda^{2} - \omega_{0}^{2}}\right)\mathrm{t}} \label{EQ:25_6}
\ee
qui, parce que $r_{1,2} < 0$, repr\'{e}sente un mouvement d\'{e}croissant sans oscillation qui tend vers une position d'\'{e}quillibre quand $\mathrm{t} \rightarrow +\infty$. Il s'agit d'un \emph{amortissement ap\'{e}riodique}.

\subsection{$\lambda = \omega_{0}$}

Dans ce cas particulier, l'\'{e}quation caract\'{e}ristique n'a qu'une unique solution $r = -\lambda$ et la solution g\'{e}n\'{e}rale de l'\'{e}quation du mouvement devient :
\be
	x = (c_{1} + c_{2}\mathrm{t})e^{-\lambda\mathrm{t}} \label{EQ:25_7}
\ee
En effet :
\benn
	\begin{cases}
		\dot{x} = c_{2}e^{-\lambda\mathrm{t}} - \lambda(c_{1} + c_{2}\mathrm{t})e^{-\lambda\mathrm{t}} \\
		\ddot{x} = -\lambda c_{2}e^{-\lambda\mathrm{t}} - \lambda c_{2}e^{-\lambda\mathrm{t}} + \lambda^{2}(c_{1} + c_{2}\mathrm{t})e^{-\lambda\mathrm{t}}
	\end{cases}
\eenn
et nous avons alors \`{a} partir du premier membre de l'\'{e}quation (\ref{EQ:25_3}) :
\bea
	\ddot{x} + 2\lambda\dot{x} + \omega_{0}^{2}x & = & \ddot{x} +2\lambda\dot{x} + \lambda^{2}x \nonumber \\
	& = & -\lambda c_{2}e^{-\lambda\mathrm{t}} - \lambda c_{2}e^{-\lambda\mathrm{t}} + \lambda^{2}(c_{1} + c_{2}\mathrm{t})e^{-\lambda\mathrm{t}} + 2\lambda c_{2}e^{-\lambda\mathrm{t}} - 2\lambda^{2}(c_{1} + c_{2}\mathrm{t})e^{-\lambda\mathrm{t}} \nonumber \\
	& & + \lambda^{2}(c_{1} + c_{2}\mathrm{t})e^{-\lambda\mathrm{t}} = 0 \nonumber
\eea
La solution repr\'{e}sente \'{e}galement un amortissement ap\'{e}riodique, i.e. sans caract\`{e}re oscillatoire.

\subsection{G\'{e}n\'{e}ralisation pour un syst\`{e}me \`{a} plusieurs degr\'{e}s de libert\'{e}}

L'expression de la force de frottement vue pr\'{e}c\'{e}demment, \`{a} savoir $f = -\alpha\dot{x}$, peut \^{e}tre g\'{e}n\'{e}ralis\'{e}e telle que :
\be
	\forall i\text{, }f_{i} = -\sum_{j}\alpha_{ij}\dot{x}_{j} \label{EQ:25_8}
\ee
qui est la force de frottement agissant sur la coordonn\'{e}e g\'{e}n\'{e}ralis\'{e}e $x_{i}$. Les m\'{e}thodes de physique statistiques abord\'{e}es au paragraphe (\ref{PAR:123}) permettent de montrer que :
\benn
	\forall i,k\text{, }\alpha_{ik} = \alpha_{ki} \label{EQ:25_9}
\eenn
Les expressions (\ref{EQ:25_8}) peuvent \^{e}tre repr\'{e}sent\'{e}es comme les d\'{e}riv\'{e}es d\'{e}finies comme :
\be
	f_{i} = -\dfrac{\partial F}{\partial \dot{x}_{i}} \label{EQ:25_10}
\ee
de la forme quadratique suivante :
\be
	F = \frac{1}{2}\sum_{i,k}\alpha_{ik}\dot{x}_{i}\dot{x}_{j}  \label{EQ:25_11}
\ee
qui est appel\'{e}e la \emph{fonction de dissipation}. Le c{\oe}fficient $\frac{1}{2}$ permet de revenir \`{a} l'expression unidimensionnelle de la force de frottement en :
\benn
	-\dfrac{\partial}{\partial \dot{x}_{i}}\left(\frac{1}{2}\dot{x}^{\,2}\right) = -\alpha\dot{x}
\eenn
L'\'{e}quation de Lagrange pour la coordonn\'{e}e $x_{i}$ est alors en g\'{e}n\'{e}ralisant l'\'{e}quation (\ref{EQ:25_1}) est alors :
\be
	\dfrac{\mathrm{d}}{\mathrm{dt}}\left(\dfrac{\partial L}{\partial \dot{x}_{i}}\right) = \dfrac{\partial L}{\partial x_{i}} - \dfrac{\partial F}{\partial\dot{x}_{i}} \label{EQ:25_12}
\ee
L'\'{e}nergie du syt\`{e}me d\'{e}finie par \'{e}quation (\ref{EQ:6_1}) est $E = \sum_{i}\dot{x}_{i}\frac{\partial L}{\partial\dot{x}_{i}} - L$ et sa variation avec le temps s'\'{e}crit :
\benn
	\dfrac{\mathrm{d}E}{\mathrm{dt}} = \dfrac{\mathrm{d}}{\mathrm{dt}}\left(\sum_{i}\dot{x}_{i}\frac{\partial L}{\partial\dot{x}_{i}}\right) - \dfrac{\mathrm{d}L}{\mathrm{dt}} = \sum_{i}\ddot{x}_{i}\dfrac{\partial L}{\partial\dot{x}_{i}} + \sum_{i}\dot{x}_{i}\dfrac{\mathrm{d}}{\mathrm{dt}}\left(\sum_{i}\frac{\partial L}{\partial\dot{x}_{i}}\right) - \dfrac{\mathrm{d}L}{\mathrm{dt}}
\eenn
et en appliquant la relation (\ref{EQ:25_12}), nous pouvons continuer \`{a} d\'{e}velopper :
\bea
	\dfrac{\mathrm{d}E}{\mathrm{dt}} & = & \sum_{i}\ddot{x}_{i}\dfrac{\partial L}{\partial\dot{x}_{i}} + \sum_{i}\dot{x}_{i}\left(\dfrac{\partial L}{\partial x_{i}} - \dfrac{\partial F}{\partial\dot{x}_{i}}\right) - \dfrac{\mathrm{d}L}{\mathrm{dt}} \nonumber \\
	& = & \sum_{i}\ddot{x}_{i}\dfrac{\partial L}{\partial\dot{x}_{i}} + \sum_{i}\dot{x}_{i}\dfrac{\partial L}{\partial x_{i}} - \sum_{i}\dot{x}_{i}\dfrac{\partial F}{\partial\dot{x}_{i}} - \dfrac{\mathrm{d}L}{\mathrm{dt}} \nonumber
\eea
or nous avons \'{e}videmment :
\benn
	\mathrm{d}L = \sum_{i}\dfrac{\partial L}{\partial x_{i}}\mathrm{d}x_{i} + \sum_{i}\dfrac{\partial L}{\partial\dot{x}_{i}}\mathrm{d}\dot{x}_{i}
\eenn
donc :
\benn
	\dfrac{\mathrm{d}L}{\mathrm{dt}} = \sum_{i}\dfrac{\partial L}{\partial x_{i}}\dfrac{\mathrm{d}x_{i}}{\mathrm{dt}} + \sum_{i}\dfrac{\partial L}{\partial\dot{x}_{i}}\dfrac{\mathrm{d}\dot{x}_{i}}{\mathrm{dt}}
\eenn
ou encore :
\benn
	\dfrac{\mathrm{d}L}{\mathrm{dt}} = \sum_{i}\dfrac{\partial L}{\partial x_{i}}\dot{x}_{i} + \sum_{i}\dfrac{\partial L}{\partial\dot{x}_{i}}\ddot{x}_{i}
\eenn
La variation de l'\'{e}nergie s'\'{e}crit donc :
\benn
	\dfrac{\mathrm{d}E}{\mathrm{dt}} = -\sum_{i}\dot{x}_{i}\dfrac{\partial F}{\partial\dot{x}_{i}}
\eenn
Or la d\'{e}finition de $F$ en (\ref{EQ:25_11}) permet d'affirmer qu'elle est une fonction homog\`{e}ne du second degr\'{e} par rapport aux vitesses. En effet :
\benn
	F(\beta\dot{x}_{i},\beta\dot{x}_{j}) = \frac{1}{2}\alpha_{ik}\beta\dot{x}_{i}\beta\dot{x}_{j} = \frac{1}{2}\beta^{2}\alpha_{ik}\dot{x}_{i}\dot{x}_{j} = \beta^{2}F(\dot{x}_{i},\dot{x}_{j})
\eenn
Le th\'{e}or\`{e}me d'Euler sur les fonctions homog\`{e}nes, ici $F$, permet d'\'{e}crire :
\benn
	\sum_{i}\dot{x}_{i}\dfrac{\partial F}{\partial\dot{x}_{i}} = 2F
\eenn
et en conclusion :
\be
	\dfrac{\mathrm{d}E}{\mathrm{dt}} = -2F \label{EQ:25_13}
\ee
Ainsi la fonction de dissipation a une importance physique significative puisqu'elle d\'{e}termine l'intensit\'{e} de la dissipation d'\'{e}nergie dans le syst\`{e}me qui est deux fois la valeur de la fonction de dissipation. Le ph\'{e}nom\`{e}ne de dissipation entra\^{i}ne toujours une diminution de l'\'{e}nergie avec le temps. Par voie de cons\'{e}quence, nous avons toujours $F > 0$ et la forme quadratique (\ref{EQ:25_11} est donc essentiellement positive, i.e. pas chacun des termes mais au moins sa somme totale.

En reprenant les \'{e}quations (\ref{EQ:23_5}) et en y ajoutant au second membre les forces ext\`{e}rieures d\'{e}finies dans la relation (\ref{EQ:25_8}), cela permet d'obtenir les \'{e}quations des petites oscillations en pr\'{e}sence de frottements :
\be
	\forall i\text{ }\sum_{j}m_{ij}\ddot{x}_{j} + \sum_{j}k_{ij}x_{j} = -\sum_{j}\alpha_{ij}\dot{x}_{j} \label{EQ:25_14}
\ee
En posant $x_{j} = A_{j}e^{r\mathrm{t}}$, l'\'{e}quation (\ref{EQ:25_14}) devient :
\bea
	& & \forall i\text{ }\sum_{j}m_{ij}A_{kj}r^{2}e^{r\mathrm{t}} + \sum_{j}k_{ij}A_{j}e^{r\mathrm{t}} = -\sum_{j}\alpha_{ij}A_{j}re^{r\mathrm{t}} \nonumber \\
	& \Leftrightarrow & \forall i\text{ }\sum_{j}(m_{ij}r^{2} + \alpha_{ij}r + k_{ij})A_{j} = 0 \label{EQ:25_15}
\eea
De la m\^{e}me mani\`{e}re que lors du passage de l'\'{e}quation (\ref{EQ:23_7}) \`{a} (\ref{EQ:23_8}), nous arrivons \`{a} l'\'{e}quation caract\'{e}ristique :
\be
	\det{(m_{ij}r^{2} + \alpha_{ij}r + k_{ij})} = 0 \label{EQ:25_16}
\ee
qui est une \'{e}quation de degr\'{e} $2s$ par rapport \`{a} $r$. L'\'{e}quation (\ref{EQ:25_13}) indique que la force de dissipation entra\^{i}ne une diminuation de l'\'{e}nergie avec le temps. Cela implique que les solutions $r$ de l'\'{e}quation (\ref{EQ:25_16}) doivent \^{e}tre soit r\'{e}elles et n\'{e}gatives, soit complexes conjugu\'{e}es deux \`{a} deux et dont la partie r\'{e}elle est n\'{e}gative.

\section{Oscillations forc\'{e}es avec frottement}\label{PAR:26}

Nous allons nous focaliser sur le cas cas la force ext\'{e}rieure qui provoque les oscillations est p\'{e}riodique.

\subsection{Cas g\'{e}n\'{e}ral}

En combinant les \'{e}quations (\ref{EQ:22_2}) et (\ref{EQ:25_3}) et en supposant que la force ext\`{e}rieure s'\'{e}crit telle que $f\cos(\gamma\mathrm{t})$, alors l'\'{e}quation du mouvement est :
\be
	\ddot{x} + 2\lambda\dot{x} +\omega_{0}^{2}x = \dfrac{f}{m}\cos(\gamma\mathrm{t}) \label{EQ:26_1}
\ee
En \'{e}crivant la force sous forme complexe, \`{a} savoir $\frac{f}{m}e^{i\gamma\mathrm{t}}$ et en supposant que la solution dans le domaine complexe est $x = Be^{i\gamma\mathrm{t}}$, alors la relation (\ref{EQ:26_1}) devient :
\bea
	& & -B\gamma^{2}e^{i\gamma\mathrm{t}} + 2B\lambda\gamma ie^{i\gamma\mathrm{t}} + B\omega_{0}^{2}e^{i\gamma\mathrm{t}} = \dfrac{f}{m}e^{i\gamma\mathrm{t}} \nonumber \\
	& \Leftrightarrow & B = \dfrac{f}{m(\omega_{0}^{2} - \gamma^{2} + 2\lambda\gamma i)} \label{EQ:26_2}
\eea
qui peut se d\'{e}velopper sous la forme $a + ib$ :
\benn
	B = \dfrac{f}{m\left((\omega_{0}^{2} - \gamma^{2})^{2} + 4\lambda^{2}\gamma^{2} \right)}(\omega_{0}^{2} - \gamma^{2} - 2\lambda\gamma i) = \dfrac{f}{m\sqrt{\left((\omega_{0}^{2} - \gamma^{2})^{2} + 4\lambda^{2}\gamma^{2} \right)}}\left(\dfrac{\omega_{0}^{2} - \gamma^{2} - 2\lambda\gamma i}{\sqrt{\left((\omega_{0}^{2} - \gamma^{2})^{2} + 4\lambda^{2}\gamma^{2} \right)}}\right)
\eenn
qui permet en \'{e}crivant $B = be^{i\delta} = b(\cos\delta + i\sin\delta)$ d'avoir :
\be
	\begin{cases}
		b = \dfrac{f}{m\sqrt{\left((\omega_{0}^{2} - \gamma^{2})^{2} + 4\lambda^{2}\gamma^{2} \right)}} \\
		\tan\delta = \dfrac{-2\lambda\gamma}{\omega_{0}^{2} - \gamma^{2}}
	\end{cases}\label{EQ:26_3}
\ee

\begin{figure}[htb!]
	\begin{center}
		\includegraphics[width=10cm]{chapter_05_paragraph_26}
		\caption{Amplitude de la solution (\ref{EQ:26_3}) avec des valeurs du c{\oe}fficient de frottement compris entre 0.01 et 1.0 et $\omega_{0} = 1$, $f = 2$ et $m = 5$}\label{FIG:5_26}
	\end{center}
\end{figure}

De fait, la partie r\'{e}elle de la solution $x = Be^{i\gamma\mathrm{t}} = be^{i\gamma\mathrm{t} + \delta}$ est l'int\'{e}grale particuli\`{e}re de l'\'{e}quation (\ref{EQ:26_1}), i.e. $\cos(\gamma\mathrm{t} + \delta)$ alors que la solution de l'\'{e}quation sans second membre est celle obtenue en (\ref{EQ:25_4}), i.e. $ae^{-\lambda\mathrm{t}}\cos(\omega\mathrm{t} + \alpha)$ avec $\omega = \sqrt{\omega_{0}^{2} - \lambda^{2}}$dans le cas o\`{u} $\lambda < \omega_{0}$. Ainsi la solution est :
\be
	x = ae^{-\lambda\mathrm{t}}\cos(\omega\mathrm{t} + \alpha) + b\cos(\gamma\mathrm{t} + \delta) \label{EQ:26_4}
\ee
avec $\omega = \sqrt{\omega_{0}^{2} - \lambda^{2}}$. Elle devient après un temps suffisamment long pour que le permier terme tende vers 0 :
\be
	x = b\cos(\gamma\mathrm{t} + \delta) \label{EQ:26_5}
\ee
Lorsque les valeurs de $\gamma$ et $\omega_{0}$ se rapprochent, l'amplitude des oscillations forc\'{e}es obtenues dans la relation (\ref{EQ:26_3}) augmente mais ne tend pas vers l'infini comme cela est le cas en l'absence de frottement, voir la relation (\ref{EQ:22_5}). Quelque soit la valeur de l'amplitude $f$ de la force ext\'{e}rieure, l'amplitude de l'oscillation $b$ est maximale si et seulement si la quantit\'{e} $(\omega_{0}^{2} - \gamma^{2})^{2} + 4\lambda^{2}\gamma^{2}$ est minimale, i.e. :
\bea
	\dfrac{\partial\left((\omega_{0}^{2} - \gamma^{2})^{2} + 4\lambda^{2}\gamma^{2}\right)}{\partial\gamma^{2}} & = & 0 \nonumber \\
	\Leftrightarrow 2\gamma^{2} + 4\lambda^{2} - 2\omega_{0}^{2} & = & 0 \nonumber \\
	\Leftrightarrow \gamma = \sqrt{\omega_{0}^{2} - 2\lambda^{2}} \nonumber
\eea
ce qui permet de remarquer que dans la cas o\`{u} $\lambda \ll \omega_{0}$ alors la valeur de $\gamma$ ne diff\`{e}re de celle de $\omega_{0}$ que par un infiniment petit du second ordre.

\subsection{Au voisinage de la r\'{e}sonance}

Posons dans ce cas $\gamma = \omega_{0} + \epsilon$ avec $\epsilon$ petit devant $\omega_{0}$ et en consid\'{e}rant $\lambda \ll \omega_{0}$ alors 
\benn
	\begin{cases}
		\gamma^{2} - \omega_{0}^{2} = (\gamma - \omega_{0})(\gamma + \omega_{0}) = \epsilon(2\omega_{0} + \epsilon) = 2\omega_{0}\epsilon + \epsilon^{2} \approx 2\omega_{0}\epsilon \\
		2\lambda\gamma i = 2\lambda(\omega_{0} + \epsilon)i \approx 2\lambda\omega_{0}i
	\end{cases}
\eenn
de sorte que l'\'{e}quation (\ref{EQ:26_2}) devient :
\be
	B = - \dfrac{f}{2m\omega_{0}(\epsilon - 2\lambda i} \label{EQ:26_6}
\ee
et l'\'{e}quation (\ref{EQ:26_3}) :
\be
	\begin{cases}
		b = \dfrac{f}{m\sqrt{4\omega_{0}^{2}\epsilon^{2} + 4\lambda^{2}\omega_{0}^{2}}} = \dfrac{f}{2m\omega_{0}\sqrt{\epsilon^{2} + \lambda^{2}}} \\
		\tan\delta = \dfrac{2\lambda\gamma}{\gamma^{2} - \omega_{0}^{2}} = \dfrac{2\lambda\omega_{0}}{2\omega_{0}\epsilon} = \dfrac{\lambda}{\epsilon} \label{EQ:26_7}
	\end{cases}
\ee
Par d\'{e}finition, $\epsilon = \gamma - \omega_{0}$ soit :
\benn
	\tan\delta = \dfrac{\lambda}{\gamma - \omega_{0}} \Rightarrow \dfrac{\partial\tan\delta}{\partial\gamma} = -\dfrac{\lambda}{2(\gamma - \omega_{0})^{2}}
\eenn
ainsi, quelque soit la valeur de $\gamma$, la d\'{e}riv\'{e}e pr\'{e}c\'{e}dente, qui repr\'{e}sente la diff\'{e}rence de phase par rapport \`{a} la fr\'{e}quence de la force ext\`{e}rieure, est toujours n\'{e}gative. Ainsi, l'oscillation du syst\`{e}me retarde sur la force ext\`{e}rieure.

Loin de la r\'{e}sonance, i.e. en reprenant le r\'{e}sultat de (\ref{EQ:26_3}) et en continuant de supposer $\lambda \ll \omega_{0}$, nous avons :
\benn
	\begin{cases}
		\gamma < \omega_{0}\text{, }\tan\delta < 0\text{ et tend vers 0 quand }\gamma\rightarrow\infty \Rightarrow \delta \rightarrow 0\text{, i.e. cos > 0 et sin < 0} \\
		\gamma > \omega_{0}\text{, }\tan\delta > 0\text{ et tend vers 0 quand }\gamma\rightarrow\infty \Rightarrow \delta \rightarrow -\pi\text{, i.e. cos < 0 et sin < 0} \nonumber \\
	\end{cases}
\eenn
Alors qu'\`{a} la r\'{e}sonance, soit $\gamma = \omega_{0}$, nous avons $\delta = \frac{\pi}{2}$. De plus, la variation de la phase $\delta$ de l'oscillation forc\'{e}e entre $-\pi$ et $0$ s'op\`{e}re sur une largeur de fr\'{e}quence $\sim \lambda$ \'{e}troite puique nous avons toujours $\lambda \ll \omega_{0}$. En l'absence de frottement, $\lambda = 0$, le passage de $-\pi$ \`{a} $0$ pour la phase $\delta$ est un saut de valeur $\pi$ et indique que le frottement, finalement, \'{e}tale ce saut.

\subsection{\'{E}nergie}

Une fois le mouvement stabilis\'{e}, l'oscillation est r\'{e}gie par l'\'{e}quation (\ref{EQ:26_5}) et, de facto, l'\'{e}nergie du syst\`{e}me ne varie pas. Ainsi, l'\'{e}nergie absorb\'{e}e aux d\'{e}pens de la force ext\'{e}rieure est dissip\'{e}e par les frottements.

Si $I(\gamma)$ est la quanti\'{e} d'\'{e}nergie dissip\'{e}e en moyenne par unit\'{e} de temps et en fonction de la fr\'{e}quence de la force ext\'{e}rieure, alors l'\'{e}quation (\ref{EQ:25_13}) s'applique et donne :
\benn
	\dfrac{\mathrm{d}E}{\mathrm{t}} = -2F \Rightarrow \dfrac{\Delta E}{\Delta\mathrm{r}} = 2\overline{F} = I(\Gamma)
\eenn
avec $\overline{F}$ la valeur moyenne sur la p\'{e}riode d'une oscillation de la fonction de dissipation, voir (\ref{EQ:25_11}). Pour un unique degr\'{e} de libert\'{e}, cetter derni\`{e}re relation donne $F = \frac{\alpha}{2}\dot{x}^{\,2}$. En comparant les \'{e}quations (\ref{EQ:26_1}) et (\ref{EQ:25_14}) avec $(i,j) = (1,1)$, nous posons la relation $m\ddot{x} + kx + \alpha\dot{x} = m\ddot{x} + 2\lambda\dot{x} + \omega_{0}^{2}x$ qui est vraie si et seulement si  $\alpha = 2\lambda m$, impliquant que $F = \lambda m\dot{x}^{\,2}$.

Puisque le mouvement est stabilis\'{e}, l'\'{e}quation (\ref{EQ:26_5}) est exploitable et permet d'\'{e}crire :
\benn
	F = \lambda mb^{2}\gamma^{2}\sin^{2}(\gamma\mathrm{t} + \delta)
\eenn
Or sur une p\'{e}riode d'oscillation, $\langle\sin^{2}\rangle = 1/2$ donc :
\be
	I(\gamma) = \lambda mb^{2}\gamma^{2} \label{EQ:26_8}
\ee
Au voisinage de la r\'{e}sonance, $b$ est donn\'{e} par la relation (\ref{EQ:26_7}) et alors, $I(\gamma)$ devient plut\^{o}t $I(\epsilon)$ tel que :
\bea
	I(\epsilon) & = & \dfrac{\lambda mf^{2}}{4m^{2}\omega_{0}^{2}(\epsilon^{2} + \lambda^{2})}(\omega_{0} + \epsilon)^{2} = \dfrac{\lambda f^{2}}{4m(\epsilon^{2} + \lambda^{2})}\left(1 + \dfrac{\epsilon}{\omega_{0}}\right)^{2} \nonumber \\
	\Leftrightarrow I(\epsilon) & = & \dfrac{f^{2}\lambda}{4m(\epsilon^{2} + \lambda^{2})}\text{ car }\epsilon \ll \omega_{0} \label{EQ:26_9}
\eea
Cette forme de relation entre \'{e}nergie dissip\'{e}e, absorb\'{e}e, et la fr\'{e}quence $\epsilon$ est appel\'{e}e \emph{dispersive}. La valeur maximale de $I$ est atteinte quand $\epsilon$ vaut mieux et elle est de :
\benn
	I(0) = \dfrac{f^{2}}{4m\lambda}
\eenn
De m\^{e}me, la demi-largueur de la courbe, i.e. la largeur à demi-hauteur, se d\'{e}duit de l'\'{e}quation suivante :
\benn
	I(\epsilon) = \frac{I(0)}{2} \Leftrightarrow \dfrac{f^{2}\lambda}{4m(\epsilon^{2} + \lambda^{2})} = \dfrac{f^{2}}{8m\lambda} \Leftrightarrow 2\lambda^{2} = \epsilon^{2} + \lambda^{2} \Leftrightarrow \epsilon = \pm \lambda
\eenn
Ainsi, si le c{\oe}fficient d'ammortissement $\lambda$ diminue, alors la hauteur de la courbe de r\'{e}sonance devient plus importante mais sa demi-largeur diminue. D\'{e}montrons que la surface sous $I(\gamma)$ est invariable par rapport au c{\oe}fficient d'ammortissement $\lambda$.

Comme $\gamma = \omega_{0} + \epsilon$, alors $\mathrm{d}\gamma = \mathrm{d}\epsilon$. Cela permet d'\'{e}crire :
\benn
	\int_{0}^{+\infty}I(\gamma)\mathrm{d}\gamma = \int_{-\omega_{0}}^{+\infty}I(\epsilon)\mathrm{d}\epsilon
\eenn
Or $I(\epsilon) \propto \epsilon^{-2}$ donc $I(\epsilon)$ d\'{e}croit tr\`{e}s rapidement quant $\lvert\epsilon\rvert$ augmente. Par cons\'{e}quent :
\bea
	\int_{-\omega_{0}}^{+\infty}I(\epsilon)\mathrm{d}\epsilon & \approx & \int_{-\infty}^{+\infty}I(\epsilon)\mathrm{d}\epsilon = \dfrac{f^{2}\lambda}{4m}\int_{-\infty}^{+\infty}\dfrac{\mathrm{d}\epsilon}{\epsilon^{2} + \lambda^{2}} \nonumber \\
	& = & \dfrac{f^{2}\lambda}{4m}\int_{-\infty}^{+\infty}\dfrac{\lambda\mathrm{d}\epsilon/\lambda}{\lambda^{2}((\epsilon/\lambda)^{2} + 1} = \dfrac{f^{2}}{4m}\int_{-\infty}^{+\infty}\dfrac{\mathrm{d}X}{(1 + X^{2})} = \dfrac{f^{2}}{4m}\left[\arctan\right]_{-\infty}^{+\infty} \nonumber
\eea
Et enfin, nous pouvons \'{e}crire :
\be
	\int_{-\infty}^{+\infty}I(\epsilon)\mathrm{d}\epsilon = \dfrac{f^{2}\pi}{4m} \label{EQ:26_10}
\ee
permettant de conclure que la surface sous $I(\epsilon)$ ne d\'{e}pend du c{\oe}fficient d'ammortissement du syst\`{e}me.

\section{R\'{e}sonance param\'{e}trique}

\subsection{Cas g\'{e}n\'{e}ral}

Dans le cas d'un syst\`{e}me \`{a} une dimension et sans force ext\'{e}rieure, l'\'{e}quation de Lagrange (\ref{EQ:21_3}) donne, dans le cas o\`{u} $m$ et $k$ d\'{e}pendent du temps, pour \'{e}quation du mouvement :
\be
	\dfrac{\mathrm{d}}{\mathrm{dt}}\left(\dfrac{\partial L}{\partial\dot{x}}\right) = \dfrac{\partial L}{\partial x} \Leftrightarrow \dfrac{\mathrm{d}(m\dot{x})}{\mathrm{dt}} + kx = 0 \label{EQ:27_1}
\ee

Maintenant, introduisons la variable $\tau$ d\'{e}finie telle que $m\mathrm{d}\tau = \mathrm{t}$ avec $m$ une fonction du temps. Cela permet d'\'{e}crire :
\bea
	\dfrac{\mathrm{d}}{\mathrm{d}\mathrm{t}}(m\dot{x}) & = & \dfrac{\mathrm{d}m}{\mathrm{d}\mathrm{t}}\dot{x} + m\dfrac{\mathrm{d}\dot{x}}{\mathrm{d}\mathrm{t}} = \dfrac{\mathrm{d}m}{\mathrm{d}\mathrm{t}}\dfrac{\mathrm{d}x}{\mathrm{d}\mathrm{t}} + m\dfrac{\mathrm{d}}{\mathrm{d}\mathrm{t}}\left(\dfrac{\mathrm{d}x}{\mathrm{d}\mathrm{t}}\right) \nonumber \\
	& = & \dfrac{\mathrm{d}m}{m\mathrm{d}\mathrm{\tau}}\dfrac{\mathrm{d}x}{m\mathrm{d}\mathrm{\tau}} + m\dfrac{\mathrm{d}}{m\mathrm{d}\mathrm{\tau}}\left(\dfrac{\mathrm{d}x}{m\mathrm{d}\mathrm{\tau}}\right) = \dfrac{1}{m^{2}}\dfrac{\mathrm{d}m}{\mathrm{d}\mathrm{\tau}}\dfrac{\mathrm{d}x}{\mathrm{d}\mathrm{\tau}} + \dfrac{\mathrm{d}}{\mathrm{d}\mathrm{\tau}}\left(\dfrac{1}{m}\right)\dfrac{\mathrm{d}x}{\mathrm{d}\mathrm{\tau}} + \dfrac{1}{m}\dfrac{\mathrm{d}^{2}x}{\mathrm{d}\mathrm{\tau}^{2}} \nonumber \\
	& = & \dfrac{1}{m^{2}}\dfrac{\mathrm{d}m}{\mathrm{d}\mathrm{\tau}}\dfrac{\mathrm{d}x}{\mathrm{d}\mathrm{\tau}} - \dfrac{1}{m^{2}}\dfrac{\mathrm{d}m}{\mathrm{d}\mathrm{\tau}}\dfrac{\mathrm{d}x}{\mathrm{d}\mathrm{\tau}} + \dfrac{1}{m}\dfrac{\mathrm{d}^{2}x}{\mathrm{d}\mathrm{\tau}^{2}} = \dfrac{1}{m}\dfrac{\mathrm{d}^{2}x}{\mathrm{d}\mathrm{\tau}^{2}} \nonumber
\eea
Ainsi l'\'{e}quation pr\'{e}c\'{e}dente (\ref{EQ:27_1}) devient :
\benn
	\dfrac{\mathrm{d}^{2}x}{\mathrm{d}\tau^{2}} + mkx = 0
\eenn
o\`{u} la masse variable avec le temps $m$ n'est plus pr\'{e}sente dans la d\'{e}riv\'{e}e avec le temps. Par analogie avec (\ref{EQ:27_1}) et sans affaiblir le raisonnement, nous pouvons continuer en consid\'{e}rant l'\'{e}quation du mouvement telle que :
\be
	\dfrac{\mathrm{d}^{2}x}{\mathrm{dt}^{2}} + w^{2}(\mathrm{t})x = 0 \label{EQ:27_2}
\ee
mais cette fois-ci avec la masse $m$ constante avec le temps.

Faisons l'hypoth\`{e}se que $\omega(\mathrm{t})$ soit une fonction de fr\'{e}quence $\omega$ et de p\'{e}riode $T = 2\pi/\omega$. Par d\'{e}finion, $\omega(\mathrm{t} + T) = \omega(t)$ et en cons\'{e}quence, l'\'{e}quation (\ref{EQ:27_2}) est invariante lors de la transformation $\mathrm{t} \rightarrow \mathrm{t} + T$. Donc si $x(\mathrm{t})$ est une solution de cette m\^{e}me \'{e}quation alors $x(\mathrm{t} + T)$ l'est tout autant. En d'autres termes, si $x_{1}(\mathrm{t})$ et $x_{2}(\mathrm{t})$ sont deux int\'{e}grales particuli\`{e}res de l'\'{e}quation (\ref{EQ:27_2}) alors elles se transforment lin\'{e}airement l'une dans l'autre en rempla\c{c}ant $\mathrm{t}$ par $\mathrm{t} + T$.

Choisissons $x_{1}(\mathrm{t})$ et $x_{2}(\mathrm{t})$ de mani\`{e}re \`{a} ce qu'en rempla\c{c}ant $\mathrm{t}$ par $\mathrm{t} + T$ cela revient \`{a} la multiplication par un facteur constant, i.e. :
\benn
	x_{1}(\mathrm{t} + T) = \mu_{1}x_{1}(\mathrm{t})\text{, }x_{2}(\mathrm{t} + T) = \mu_{2}x_{2}(\mathrm{t})
\eenn
La forme la plus g\'{e}n\'{e}rale des fonctions poss\'{e}dant cette propri\'{e}t\'{e} est :
\be
	x_{1}(\mathrm{t}) = \mu_{1}^{\mathrm{t}/T}\Pi_{1}(\mathrm{t})\text{, }x_{2}(\mathrm{t}) = \mu_{2}^{\mathrm{t}/T}\Pi_{2}(\mathrm{t})\label{EQ:27_3}
\ee
o\`{u} les fonctions $\Pi_{i}(\mathrm{t})$ sont purement p\'{e}riodiques du temps et de de p\'{e}riode $T$. Les constantes $\mu_{i}$ doivent \^{e}tre li\'{e}es par une relation d\'{e}termin\'{e}e. Ainsi, pour $i = 2$, l'\'{e}quation (\ref{EQ:27_2}) donne :
\benn
	\begin{cases}
		\ddot{x_{1}} + w^{2}(\mathrm{t})x_{1} = 0 \\
		\ddot{x_{2}} + w^{2}(\mathrm{t})x_{2} = 0
	\end{cases}
	\Leftrightarrow
	\begin{cases}
		\ddot{x_{1}}x_{2} + w^{2}(\mathrm{t})x_{1}x_{2} = 0 \\
		\ddot{x_{2}}x_{1} + w^{2}(\mathrm{t})x_{2}x_{1} = 0
	\end{cases}
\eenn
implique en soustrayant la seconde\`{a} la premi\`{e}re :
\bea
	\ddot{x_{1}}x_{2} - \ddot{x_{2}}x_{1} = 0 & \Leftrightarrow & \ddot{x_{1}}x_{2} + \dot{x_{1}}\dot{x_{2}} - \ddot{x_{2}}x_{1} - \dot{x_{1}}\dot{x_{2}} = 0 \nonumber \\
	& \Leftrightarrow & \dfrac{\mathrm{d}(\dot{x_{1}}x_{2} - x_{1}\dot{x_{2}})}{\mathrm{dt}} = 0 \nonumber
\eea
ce qui permet de conclure \`{a} ce que :
\be
	\dot{x_{1}}x_{2} - x_{1}\dot{x_{2}} = cste/\mathrm{t}\label{EQ:27_4}
\ee
En g\'{e}n\'{e}ralisant les relations (\ref{EQ:27_3}), nous avons :
\benn
	\forall i\text{, }\begin{cases}
		x_{i}(\mathrm{t} + T) = \mu_{i}x_{i}(\mathrm{t}) \\
		\dot{x}_{i}(\mathrm{t}) = \frac{\ln(\mu_{i})}{T}\mu_{i}^{\mathrm{t}/T}\Pi_{i}(\mathrm{t}) + \mu_{i}^{\mathrm{t}/T}\dot{\Pi}(\mathrm{t}) \Rightarrow \dot{x}_{i}(\mathrm{t} + T) = \frac{\ln(\mu_{i})}{T}\mu_{i}\mu_{i}^{\mathrm{t}/T}\Pi_{i}(\mathrm{t}) + \mu_{i}\mu_{i}^{\mathrm{t}/T}\dot{\Pi}(\mathrm{t}) = \mu_{i}\dot{x}_{i}(\mathrm{t})
	\end{cases}
\eenn
car par d\'{e}finition $\Pi_{i}(\mathrm{t} + T) = \Pi_{i}(\mathrm{t})$. Reprenons maintenant l'\'{e}quation (\ref{EQ:27_4}) qui permet d'\'{e}crire :
\bea
	(\dot{x_{1}}x_{2} - x_{1}\dot{x_{2}})\big|_{\mathrm{t}} & = & (\dot{x_{1}}x_{2} - x_{1}\dot{x_{2}})\big|_{\mathrm{t} + T} \nonumber \\
	\Leftrightarrow \dot{x_{1}}x_{2} - x_{1}\dot{x_{2}} & = & \mu_{1}\mu_{2}\dot{x_{1}}x_{2} - \mu_{1}\mu_{2}x_{1}\dot{x_{2}} = \mu_{1}\mu_{2}(\dot{x_{1}}x_{2} - x_{1}\dot{x_{2}}) \nonumber \\
	\Leftrightarrow \mu_{1}\mu_{2} = 1 \label{EQ:27_5}
\eea

\subsection{Conclusions sur la r\'{e}sonance param\'{e}trique}

Dans l'\'{e}quation (\ref{EQ:27_2}), supposons que $x(\mathrm{t})$ en est une intégrale dans $\mathbb{C}$. Elle s'écrit $a(\mathrm{t}) + ib(\mathrm{t})$ et nous pouvons d\'{e}duire de l'\'{e}quation (\ref{EQ:27_2}) :
\benn
	\ddot{a}(\mathrm{t}) + i\ddot{b}(\mathrm{t}) + \omega^{2}(\mathrm{t})a(\mathrm{t}) + i\omega^{2}(\mathrm{t})b(\mathrm{t}) = 0
\eenn
qui n'est possible qu'aux conditions suivantes :
\benn
	\begin{cases}
		\ddot{a}(\mathrm{t}) + \omega^{2}(\mathrm{t})a(\mathrm{t}) = 0 \\
		\ddot{b}(\mathrm{t}) + \omega^{2}(\mathrm{t})b(\mathrm{t}) = 0
	\end{cases}
\eenn
Le conjugu\'{e} $\overline{x} = a(\mathrm{t}) + ib(\mathrm{t})$ permet d'\'{e}crire :
\benn
	\dfrac{\mathrm{d}^{2}\overline{x}}{\mathrm{dt}^{2}} + w(\mathrm{t})^{2}(\mathrm{t})\overline{x} = \ddot{a}(\mathrm{t}) - i\ddot{b}(\mathrm{t}) + \omega^{2}(\mathrm{t})a(\mathrm{t}) - i\omega^{2}(\mathrm{t})b(\mathrm{t})
\eenn
Cette expression s'annule si et seulement si $\omega(\mathrm{t})\in\mathbb{R}$.

Reprenons d\'{e}sormais la relation (\ref{EQ:27_5}) avec $\forall i\in \{1,2\}\text{, } \mu_{i} = a_{i} + ib_{i}$. Cela permet d'\'{e}crire :
\bea
	& & \mu_{1}\mu_{2} = 1 \Leftrightarrow a_{1}a_{2} - b_{1}b_{2} + i(a_{1}b_{1} + a_{2}b_{2}) = 1 \Rightarrow a_{1}b_{1} + a_{2}b_{2} = 0 \nonumber \\
	& \Leftrightarrow & (a_{1}b_{1})^{2} = (a_{2}b_{2})^{2} \Leftrightarrow \lVert\mu_{1}\rVert^{2} = \lVert\mu_{2}\rVert^{2} \nonumber
\eea
Le module de $\mu_{1}$ et $\mu_{2}$ doit donc \^{e}tre \'{e}gal \`{a} l'unit\'{e} et de mani\`{e}re triviale, la relation (\ref{EQ:27_5}) ne se r\'{e}alise que si $\mu_{1}$ et $\mu_{2}$ sont dans $\mathbb{R}$.

En allant plus loin et en supposons $(\mu_{1},\mu_{2})\in\mathbb{R}^{2}$ alors, comme $\mu_{1}\mu_{2} = 1$, nous pouvons \'{e}crire $\mu_{1} = 1/\mu_{2}$ et les relations (\ref{EQ:27_3}) peuvent se simplifier telles que :
\benn
	x_{1}(\mathrm{t}) = \mu^{\mathrm{t}/T}\Pi_{1}(\mathrm{t})\text{, }x_{2}(\mathrm{t}) = \Pi_{2}(\mathrm{t})/\mu^{\mathrm{t}/T} = \mu^{-\mathrm{t}/T}\Pi_{2}(\mathrm{t})\label{EQ:27_6}
\eenn
avec $\mu > 0$ ou $\mu < 0$ mais diff\'{e}rent de l'unit\'{e} pour \'{e}viter le cas trivial. Dans tous les cas, nous avons des fonctions exponentielles du temps et ou $x_{1}$, ou $x_{2}$, va cro\^{i}tre exponentiellement avec le temps. L'\'{e}tat de repos du syst\`{e}me sera donc instable car le moindre petit \'{e}cart de la position d'\'{e}quilibre va engendrer un d\'{e}placement exponentiel de $x$. Il s'agit de la \emph{r\'{e}sonance param\'{e}trique}.

Si, \`{a} $\mathrm{t} = 0$, nous mesurons $\dot{x}(0) = 0$ et $x(0) = 0$, alors comme $\forall\mathrm{t}$, $x(\mathrm{t}) = x(\mathrm{t} + T)$ et $\dot{x}(\mathrm{t}) = \dot{x}(\mathrm{t} + T)$, le syst\`{e}me ne bougera pas de sa position initiale, au contraire de la r\'{e}sonnance ordinaire \'{e}tudi\'{e}e au paragraphe (\ref{PAR:22}) et en particulier dans l'\'{e}quation (\ref{EQ:22_5}) o\`{u} l'amplitude des oscillations cro\^{i}t avec le temps.

\subsection{Cas particulier de $\omega(\mathrm{t})$ p\'{e}riodique et \`{a} faible amplitude}

\'{E}tudions les conditions pour lesquelles la r\'{e}sonnance param\'{e}trique appara\^{i}t dans le cas o\`{u} $\omega(\mathrm{t})$ diff\`{e}re peu d'une valeur constante $\omega_{0}$ par application d'une fonction p\'{e}riodique simple telle que :
\be
	\omega^{2}(\mathrm{t}) = \omega_{0}^{2}(1 + h\cos(\gamma\mathrm{t})) \label{EQ:27_7}
\ee
avec $0 < h \ll 1$. Supposons $\gamma$ proche du double de $\omega_{0}$ telle que : $\gamma = 2\omega_{0} + \epsilon$ avec \'{e}videmment $\epsilon \ll \omega_{0}$. L'\'{e}quation (\ref{EQ:27_2}) devient alors :
\be
	\ddot{x} + \omega_{0}^{2}\left(1 + h\cos\left[(2\omega_{0} + \epsilon)\mathrm{t}\right]\right)x = 0\label{EQ:27_8}
\ee
pour laquelle nous chercherons la solution sous la forme :
\be
	x(\mathrm{t}) = a(\mathrm{t})\cos\left[\left(\omega_{0} + \frac{\epsilon}{2}\right)\mathrm{t}\right] + b(\mathrm{t})\sin\left[\left(\omega_{0} + \frac{\epsilon}{2}\right)\mathrm{t}\right]\label{EQ:27_9}
\ee
o\`{u} $a(\mathrm{t})$ et $b(\mathrm{t})$ sont des fonctions du temps variant avec des pulsations bien plus faibles que $\omega_{0}$. La solution exacte \`{a} l'\'{e}quation (\ref{EQ:27_8}) contient \'{e}rgalement des termes dont la fr\'{e}quence est multiple de la quantit\'{e} $2\omega_{0} + \epsilon$, ce qui en fait des infiniments petits d'ordre sup\'{e}rieur et n\'{e}gligeable en premi\`{e}re approximation (voir exercice (\ref{PAR:27_EX1})).

Connaissant la relation (\ref{EQ:27_9}), et en d\'{e}veloppant d'abord :
\bea
	\dfrac{\mathrm{d}^{2}\left(a(\mathrm{t})\cos(\alpha\mathrm{t})\right)}{\mathrm{dt^{2}}} & = & \dfrac{\mathrm{d}\left(\dot{a}(\mathrm{t})\cos(\alpha\mathrm{t}) - \alpha a(\mathrm{t})\sin(\alpha\mathrm{t})\right)}{\mathrm{dt}} \nonumber \\
	& = & \ddot{a}(\mathrm{t})\cos(\alpha\mathrm{t}) - 2\alpha\dot{a}(\mathrm{t})\sin(\alpha\mathrm{t}) - \alpha^{2}a(\mathrm{t})\cos(\alpha\mathrm{t}) \nonumber \\
	\dfrac{\mathrm{d}^{2}\left(b(\mathrm{t})\sin(\alpha\mathrm{t})\right)}{\mathrm{dt^{2}}} & = & \dfrac{\mathrm{d}\left(\dot{b}(\mathrm{t})\sin(\alpha\mathrm{t}) + \alpha b(\mathrm{t})\cos(\alpha\mathrm{t})\right)}{\mathrm{dt}} \nonumber \\
	& = & \ddot{b}(\mathrm{t})\sin(\alpha\mathrm{t}) + 2\alpha\dot{b}(\mathrm{t})\cos(\alpha\mathrm{t}) - \alpha^{2}b(\mathrm{t})\sin(\alpha\mathrm{t}) \nonumber
\eea
l'\'{e}quation (\ref{EQ:27_8}) devient en posant $\alpha = \omega_{0} + \frac{\epsilon}{2}$ :
\bea
	& & \ddot{a}(\mathrm{t})\cos(\alpha\mathrm{t}) - 2\alpha\dot{a}(\mathrm{t})\sin(\alpha\mathrm{t}) - \alpha^{2}a(\mathrm{t})\cos(\alpha\mathrm{t}) \nonumber \\
	& + & \ddot{b}(\mathrm{t})\sin(\alpha\mathrm{t}) + 2\alpha\dot{b}(\mathrm{t})\cos(\alpha\mathrm{t}) - \alpha^{2}b(\mathrm{t})\sin(\alpha\mathrm{t}) \nonumber \\
	& + & \omega^{2}a(\mathrm{t})\cos(\alpha\mathrm{t}) + \omega^{2}b(\mathrm{t})\sin(\alpha\mathrm{t}) \nonumber \\
	& + & a(\mathrm{t})\omega^{2}h\cos(2\alpha\mathrm{t})\cos(\alpha\mathrm{t}) + b(\mathrm{t})\omega^{2}h\cos(2\alpha\mathrm{t})\sin(\alpha\mathrm{t}) = 0 \nonumber
\eea
En n\'{e}gligeant les termes en $\epsilon^{2}$, nous avons $\alpha^{2} \approx \omega_{0}^{2} + \omega_{0}\epsilon$ et en utilisant les formules de Simpson telles que :
\bea
	\cos(2\alpha\mathrm{t})\cos(\alpha\mathrm{t}) & = & \frac{1}{2}\left(\cos(3\alpha\mathrm{t}) + \cos(\alpha\mathrm{t})\right) \nonumber \\
	\cos(2\alpha\mathrm{t})\sin(\alpha\mathrm{t}) & = & \frac{1}{2}\left(\sin(3\alpha\mathrm{t}) - \sin(\alpha\mathrm{t})\right) \nonumber
\eea
alors nous pouvons poursuivre le d\'{e}veloppement tel que :
\bea
	& & \ddot{a}(\mathrm{t})\cos(\alpha\mathrm{t}) - 2\alpha\dot{a}(\mathrm{t})\sin(\alpha\mathrm{t}) - (\omega_{0}^{2} + \omega_{0}\epsilon)a(\mathrm{t})\cos(\alpha\mathrm{t}) \nonumber \\
	& + & \ddot{b}(\mathrm{t})\sin(\alpha\mathrm{t}) + 2\alpha\dot{b}(\mathrm{t})\cos(\alpha\mathrm{t}) - (\omega_{0}^{2} + \omega_{0}\epsilon)b(\mathrm{t})\sin(\alpha\mathrm{t}) \nonumber \\
	& + & \omega_{0}^{2}a(\mathrm{t})\cos(\alpha\mathrm{t}) + \omega_{0}^{2}b(\mathrm{t})\sin(\alpha\mathrm{t}) \nonumber \\
	& + & a(\mathrm{t})\frac{\omega_{0}^{2}h}{2}\left(\cos(3\alpha\mathrm{t}) + \cos(\alpha\mathrm{t})\right) + b(\mathrm{t})\frac{\omega_{0}^{2}h}{2}\left(\sin(3\alpha\mathrm{t}) - \sin(\alpha\mathrm{t})\right) = 0 \nonumber \\
	\Leftrightarrow & & \ddot{a}(\mathrm{t})\cos(\alpha\mathrm{t}) - 2\alpha\dot{a}(\mathrm{t})\sin(\alpha\mathrm{t}) - \omega_{0}\epsilon a(\mathrm{t})\cos(\alpha\mathrm{t}) \nonumber \\
	& + & \ddot{b}(\mathrm{t})\sin(\alpha\mathrm{t}) + 2\alpha\dot{b}(\mathrm{t})\cos(\alpha\mathrm{t}) - \omega_{0}\epsilon b(\mathrm{t})\sin(\alpha\mathrm{t}) \nonumber \\
	& + & a(\mathrm{t})\frac{\omega_{0}^{2}h}{2}\cos(\alpha\mathrm{t}) - b(\mathrm{t})\frac{\omega_{0}^{2}h}{2}\sin(\alpha\mathrm{t}) = 0 \nonumber
\eea
qui permet d'arriver \`{a} :
\benn
	\left[\ddot{a}(\mathrm{t}) - \omega_{0}\epsilon a(\mathrm{t}) + 2\alpha\dot{b}(\mathrm{t}) + a(\mathrm{t})\frac{\omega_{0}^{2}h}{2}\right]\cos(\alpha\mathrm{t}) + \left[-2\alpha\dot{a}(\mathrm{t}) + \ddot{b}(\mathrm{t}) - \omega_{0}\epsilon b(\mathrm{t}) - b(\mathrm{t})\frac{\omega_{0}^{2}h}{2}\right]\sin(\alpha\mathrm{t}) = 0
\eenn
En prenant les hypoth\`{e}ses suivantes : $\dot{a}\approxeq\epsilon a$ et $\dot{b}\approxeq\epsilon b$ alors nous pouvons n\'{e}gliger les termes en $\ddot{a}$, $\ddot{b}$, $\epsilon\dot{a}$ et $\epsilon\dot{b}$ et nous obtenons :
\benn
	\left[2\dot{b}(\mathrm{t}) - \epsilon a(\mathrm{t}) + a(\mathrm{t})\frac{\omega_{0}h}{2}\right]\cos(\alpha\mathrm{t}) - \left[2\dot{a}(\mathrm{t}) + \epsilon b(\mathrm{t}) + b(\mathrm{t})\frac{\omega_{0}h}{2}\right]\sin(\alpha\mathrm{t}) = 0
\eenn
\'{E}videmment, l'\'{e}galit\'{e} pr\'{e}c\'{e}dente n'est valable que si les c{\oe}fficients de facteur $\cos$ et $\sin$ s'annulent simultan\'{e}ment, i.e. :
\benn
	\begin{cases}
		2\dot{b}(\mathrm{t}) - \epsilon a(\mathrm{t}) + a(\mathrm{t})\frac{\omega_{0}h}{2} = 0 \\
		2\dot{a}(\mathrm{t}) + \epsilon b(\mathrm{t}) + b(\mathrm{t})\frac{\omega_{0}h}{2} = 0
	\end{cases}
\eenn
soient deux \'{e}quations diff\'{e}rentielles lin\'{e}aires. Recherchons une paire de solution telle que $a(\mathrm{t}) = ae^{s\mathrm{t}}$ et $b(\mathrm{t}) = be^{s\mathrm{t}}$. Les deux \'{e}quations pr\'{e}c\'{e}dentes deviennent alors, quelque soit $\mathrm{t}$ :
\benn
	\begin{cases}
		2bs - \left(\epsilon - \frac{\omega_{0}h}{2}\right)a = 0 \\
		2as + \left(\epsilon + \frac{\omega_{0}h}{2}\right)b = 0
	\end{cases}
	\Leftrightarrow
	\begin{cases}
		bs - \left(\epsilon - \frac{\omega_{0}h}{2}\right)\frac{a}{2} = 0 \\
		\left(\epsilon + \frac{\omega_{0}h}{2}\right)\frac{b}{2} + as = 0
	\end{cases}
\eenn
qui a une solution non triviale si et seulement si le d\'{e}terminant est nul, soit :
\bea
	s^{2} + \frac{1}{4}\left(\epsilon - \frac{\omega_{0}h}{2}\right)\left(\epsilon + \frac{\omega_{0}h}{2}\right) & = & 0 \nonumber \\
	\Leftrightarrow s^{2} = \frac{1}{4}\left(\frac{\omega_{0}h}{2} - \epsilon\right)\left(\frac{\omega_{0}h}{2} + \epsilon\right) & \Leftrightarrow & s^{2} = \frac{1}{4}\left(\left(\frac{\omega_{0}h}{2}\right)^{2} - \epsilon^{2}\right) \label{EQ:27_10}
\eea

Le raisonnement d\'{e}rivant des relations (\ref{EQ:27_6}) am\`{e}ne \`{a} la r\'{e}sonance param\'{e}trique si le c{\oe}fficient $\mu \in \mathbb{R}$, soit dans le cas pr\'{e}sent, si $s^{2} \in \mathbb{R}$ et plus pr\'{e}cis\`{e}ment $s^{2} > 0$. Donc la r\'{e}sonance param\'{e}trique a lieu pour :
\be
	\left(\frac{\omega_{0}h}{2}\right)^{2} - \epsilon^{2} > 0 \Leftrightarrow -\frac{h\omega_{0}}{2} < \epsilon < +\frac{h\omega_{0}}{2} \label{EQ:27_11}
\ee
qui repr\'{e}sente l'intervalle de fr\'{e}quence autour de la pulsation $2\omega_{0}$ d\'{e}finie dans la relation (\ref{EQ:27_8}).

Dans les \'{e}quations (\ref{EQ:27_7}) et (\ref{EQ:27_7}), $\gamma$ peut \^{e}tre choisi comme $2\omega_{0}/n$ avec $n\in\mathbb{N}$. Dans ce cas, la largeur des domaines de r\'{e}sonance param\'{e}trique diminue rapidement.

\subsection{Cas \`{a} faible c{\oe}fficient de frottement}

Le paragraphe (\ref{PAR:25}) et en particulier la solution (\ref{EQ:25_4}) montrent que le frottement induit un ammortissement exponentiel de l'amplitude des oscillations, en $e^{-\lambda\mathrm{t}}$. En reprenant le cheminement pour arriver à l'\'{e}quation (\ref{EQ:27_10}), l'amplitude des oscillations dans la r\'{e}sonnance param\'{e}trique suit une loi en $e^{(s-\lambda)\mathrm{t}}$ avec $s > 0$. La limite du domaine d'instabilit\'{e} est donn\'{e} par $s - \lambda = 0$. L'\'{e}quation (\ref{EQ:27_10}) devient dans ce cas pr\'{e}cis :
\benn
	(s - \lambda)^{2} = \frac{1}{4}\left(\left(\frac{\omega_{0}h}{2}\right)^{2} - \epsilon^{2}\right) \Leftrightarrow s^{2} - 2\lambda s + \lambda^{2} - \frac{1}{4}\left(\left(\frac{\omega_{0}h}{2}\right)^{2} - \epsilon^{2}\right) = 0
\eenn
qui a pour solutions :
\benn
	s = \frac{2\lambda \pm \sqrt{4\lambda^{2} - 4\left(\lambda^{2} - \frac{1}{4}\left(\left(\frac{\omega_{0}h}{2}\right)^{2} - \epsilon^{2}\right)\right)}}{2} = \lambda \pm \frac{1}{2}\sqrt{\left(\frac{\omega_{0}h}{2}\right)^{2} - \epsilon^{2}}
\eenn
avec \'{e}videmment $\epsilon < \frac{\omega_{0}h}{2}$ qui est donn\'{e} par les in\'{e}galit\'{e}s (\ref{EQ:27_11}). Nous avons donc le domaine de r\'{e}sonance param\'{e}trique suivant, i.e. $s > 0$ :
\be
	-\sqrt{\left(\frac{\omega_{0}h}{2}\right)^{2} - 4\lambda^{2}} < \epsilon < \sqrt{\left(\frac{\omega_{0}h}{2}\right)^{2} - 4\lambda^{2}} \label{EQ:27_12}
\ee
qui n'est possible que pour $\left(\frac{\omega_{0}h}{2}\right)^{2} - 4\lambda^{2} > 0$. Il y a done un seuil d'amplitude au-dessus duquel la r\'{e}sonance n'est pas possible. Cette valeur $h_{k}$ est :
\benn
	\left(\frac{\omega_{0}h_{k}}{2}\right)^{2} = 4\lambda^{2} \Leftrightarrow h_{k} = \frac{4\lambda}{\omega_{0}}
\eenn

\section {Oscillations anharmoniques}\label{PAR:28}

\subsection{Cas g\'{e}n\'{e}ral par la m\'{e}thode des approximations successives}

La th\'{e}orie des petites oscillations pr\'{e}sent\'{e}e jusque maintenant est bas\'{e}e sur un d\'{e}veloppement des \'{e}nergies cin\'{e}tique et potentielle par rapport aux positions et aux vitesses du syst\`{e}me o\`{u} seuls les termes du deuxi\`{e}me ordre ont \'{e}t\'{e} conserv\'{e}s. Cela a permis d'obtenir des \'{e}quations du mouvement lin\'{e}aires. Dans ce cadre th\'{e}orique, nous avons ainsi discuter des \emph{oscillations lin\'{e}aires}, cadre tout \`{a} fait l'\'{e}gitime pour des oscillations suffisamment petites. Prendre en compte les approximations suivantes, i.e. \emph{l'anharmonicit\'{e}} ou \emph{non-lin\'{e}arit\'{e}} des oscillations conduit \`{a} l'apparition de particularit\'{e}s qualitativement nouvelles.

Effectuons le d\'{e}veloppement de la fonction de Lagrange jusqu'au troisi\`{e}me ordre, extension du paragraphe (\ref{PAR:23}), en utilisant le d\'{e}veloppement de Taylor en coordonn\'{e}es g\'{e}n\'{e}ralis\'{e}es $q_{i}$ par rapport \`{a} la position d'\'{e}quilibre $q_{i0}$ :
\bea
	U(\left\{q_{i}\right\}) = U(\left\{q_{i0}\right\}) + \sum_{i}\left(\frac{\partial U}{\partial\dot{q}_{i}}\right)_{q_{i}=q_{i0}}(q_{i} - q_{i0}) & + & \frac{1}{2}\sum_{i,j}\left(\frac{\partial^{2} U}{\partial\dot{q}_{i}\partial\dot{q}_{j}}\right)_{\begin{subarray}{l}q_{i}=q_{i0}\\q_{j}=q_{j0}\end{subarray}}(q_{i} - q_{i0})(q_{j} - q_{j0}) \nonumber \\
	& + & \frac{1}{6}\sum_{i,j,k}\left(\frac{\partial^{3} U}{\partial\dot{q}_{i}\partial\dot{q}_{j}\partial\dot{q}_{k}}\right)_{\begin{subarray}{l}q_{i}=q_{i0}\\q_{j}=q_{j0}\\q_{k}=q_{k0}\end{subarray}}(q_{i} - q_{i0})(q_{j} - q_{j0})(q_{k} - q_{k0}) \nonumber
\eea
et comme nous avons choisi le syst\`{e}me \`{a} l'\'{e}quilibre en $\forall i\text{, }q_{i} = q_{i0}$ et $\forall i\text{, }U(q_{i0}) = 0$ alors l'\'{e}nergie potentielle se r\'{e}duit \`{a} :
\benn
	U(\left\{q_{i}\right\}) = \frac{1}{2}\sum_{i,j}\left(\frac{\partial^{2} U}{\partial\dot{q}_{i}\partial\dot{q}_{j}}\right)_{\begin{subarray}{l}q_{i}=q_{i0}\\q_{j}=q_{j0}\end{subarray}}(q_{i} - q_{i0})(q_{j} - q_{j0}) + \frac{1}{6}\sum_{i,j,k}\left(\frac{\partial^{3} U}{\partial\dot{q}_{i}\partial\dot{q}_{j}\partial\dot{q}_{k}}\right)_{\begin{subarray}{l}q_{i}=q_{i0}\\q_{j}=q_{j0}\\q_{k}=q_{k0}\end{subarray}}(q_{i} - q_{i0})(q_{j} - q_{j0})(q_{k} - q_{k0})
\eenn
que nous pouvons \'{e}crire ainsi :
\benn
	U = \frac{1}{2}\sum_{i,j}k_{ij}x_{i}x_{j} + \frac{1}{6}\sum_{i,j,k}l_{ijk}x_{i}x_{j}x_{k}
\eenn
De la m\^{e}me mani\`{e}re, l'\'{e}nergie cin\'{e}tique s'\'{e}crit avec les m\^{e}mes hypoth\`{e}ses au troisi\`{e}me ordre :
\benn
	T = \frac{1}{2}\sum_{i,j}m_{ij}\dot{x}_{i}\dot{x}_{j} + \frac{1}{6}\sum_{i,j,k}n_{ijk}\dot{x}_{i}\dot{x}_{j}\dot{x}_{k}
\eenn
Cela nous permet alors d'assembler pour \'{e}crire la fonction de Lagrange :
\be
	L = \frac{1}{2}\sum_{i,j}(m_{ij}\dot{x}_{i}\dot{x}_{j} - k_{ij}x_{i}x_{j}) + \frac{1}{6}\sum_{i,j,k}n_{ijk}\dot{x}_{i}\dot{x}_{j}\dot{x}_{k} - \frac{1}{6}\sum_{i,j,k}l_{ijk}x_{i}x_{j}x_{k}\label{EQ:28_1}
\ee
\emph{Pour cette formulation, je n'arrive pas \`{a} aboutir aux c{\oe}fficients obtenus dans le livre pour les termes du troisi\`{e}me ordre, \`{a} savoir $1/2$ pour $n_{ijk}$ et $-1/3$ pour $l_{ijk}$.}
En utilisant les coordonn\'{e}es normales $Q_{\alpha}$ et en supposant que les transformations des coordonn\'{e}es quelconques \`{a} normales $x_{i} \rightarrow Q_{\alpha}$ et $\dot{x}_{i} \rightarrow \dot{Q}_{\alpha}$ soient lin\'{e}aires, alors la fonction de Lagrange devient :
\be
	L = \frac{1}{2}\sum_{\alpha}(\dot{Q}^{2}_{\alpha} - \omega_{\alpha}^{2}Q_{\alpha}^{2}) + \frac{1}{6}\sum_{\alpha,\beta,\gamma}\lambda_{\alpha\beta\gamma}\dot{Q}_{\alpha}\dot{Q}_{\beta}\dot{Q}_{\gamma} - \frac{1}{6}\sum_{\alpha,\beta,\gamma}\mu_{\alpha\beta\gamma}Q_{\alpha}Q_{\beta}Q_{\gamma}\label{EQ:28_2}
\ee
Quelque soit la coordonn\'{e}e normale, nous pouvons \'{e}crire son \'{e}quation du mouvement :
\bea
	& \forall\alpha\text{, } & \frac{\mathrm{d}}{\mathrm{dt}}\left(\frac{\partial L}{\partial\dot{Q}_{\alpha}}\right) = \frac{\partial L}{\partial Q_{\alpha}} \nonumber \\
	& \Leftrightarrow & \ddot{Q}_{\alpha} + \frac{1}{6}\sum_{\beta,\gamma}\lambda_{\alpha\beta\gamma}(\ddot{Q}_{\beta}\dot{Q}_{\gamma} + \dot{Q}_{\beta}\ddot{Q}_{\gamma}) = -\omega_{\alpha}^{2}Q_{\alpha}  - \frac{1}{6}\sum_{\beta,\gamma}\mu_{\alpha\beta\gamma}Q_{\beta}Q_{\gamma}) \nonumber
\eea
qui peut s'\'{e}crire sous la forme :
\be
	\forall\alpha\text{, }\ddot{Q}_{\alpha} + \omega_{\alpha}^{2}Q_{\alpha} = f_{\alpha}(Q,\dot{Q},\ddot{Q})\label{EQ:28_3}
\ee
avec $f_{\alpha}$ une fonction homog\`{e}ne du second ordre des coordonn\'{e}es normales $Q_{\alpha}$ et de ses d\'{e}riv\'{e}es premi\`{e}re et seconde par rapport au temps.

En appliquant la m\'{e}thode des approximations successives, nous allon chercher une solution du type :
\be
	Q_{\alpha} = Q_{\alpha}^{(1)} + Q_{\alpha}^{(2)}\label{EQ:28_4}
\ee
avec $Q_{\alpha}^{(2)} \ll Q_{\alpha}^{(1)}$ et o\`{u} les $Q_{\alpha}^{(1)}$ satisfont aux \'{e}quations non perturb\'{e}es, i.e. :
\benn
	\forall\alpha\text{, }\ddot{Q}_{\alpha}^{(1)} + \omega_{\alpha}^{2}Q_{\alpha}^{(1)} = 0
\eenn
qui ont pour solutions des oscillations harmoniques ordinaires telles que :
\be
	Q_{\alpha}^{(1)} = a_{\alpha}\cos(\omega_{\alpha}\mathrm{t} + \alpha_{\alpha})\label{EQ:28_5}
\ee
L'\'{e}quation (\ref{EQ:28_3}) devient pour son membre de gauche $\ddot{Q}_{\alpha} + \omega_{\alpha}^{2}Q_{\alpha} = \ddot{Q}_{\alpha}^{(1)} + \ddot{Q}_{\alpha}^{(2)} + \omega_{\alpha}^{2}Q_{\alpha}^{(1)} + \omega_{\alpha}^{2}Q_{\alpha}^{(2)} = \ddot{Q}_{\alpha}^{(2)} + \omega_{\alpha}^{2}Q_{\alpha}^{(2)}$. Pour son membre de droite qui correspond d\'{e}j\`{a} au troisi\`{e}me ordre des \'{e}nergies potentielle et cin\'{e}tique, il est suffisant de ne l'exprimer qu'\`{a} partir de la solution non perturb\'{e}e. Ainsi, nous avons :
\be
	\forall\alpha\text{, }\ddot{Q}_{\alpha}^{(2)} + \omega_{\alpha}^{2}Q_{\alpha}^{(2)} = f_{\alpha}(Q^{(1)},\dot{Q}^{(1)},\ddot{Q}^{(1)})\label{EQ:28_6}
\ee
qui sont des \'{e}quations diff\'{e}rentielles lin\'{e}aires non homog\`{e}nes pour lesquelles le second membre est compos\'{e} de sommes de fonctions p\'{e}riodiques simples. En effet, dans le cadre des solutions (\ref{EQ:28_5}), nous pouvons d\'{e}velopper par utilisation d'une des formules de Simpson :
\bea
	Q_{\alpha}^{(1)}Q_{\beta}^{(1)} & = & a_{\alpha}a_{\beta}\cos(\omega_{\alpha}\mathrm{t} + \alpha_{\alpha})\cos(\omega_{\beta}\mathrm{t} + \alpha_{\beta}) \nonumber \\
	& = & \frac{1}{2}\left(\cos((\omega_{\alpha} + \omega_{\beta})\mathrm{t} + \alpha_{\alpha} + \alpha_{\beta}) + \cos((\omega_{\alpha} - \omega_{\beta})\mathrm{t} + \alpha_{\alpha} - \alpha_{\beta})\right) \nonumber
\eea
qui comprend donc des oscillations \'{e}gales \`{a} la somme et \`{a} la diff\'{e}rence des fr\'{e}quences propres du syst\`{e}me. Dans ce cas, nous pouvnos nous reporter aux oscillations forc\'{e}es du paragraphe (\ref{PAR:26}) o\`{u} la solution (\ref{EQ:26_4}), en plus de la fr\'{e}quence propre du syst\`{e}me $\omega_{0}$, comporte les fr\'{e}quences de la force ext\'{e}rieure, soient ici :
\be
	\omega_{\alpha} \pm \omega_{\beta}\label{EQ:28_7}
\ee
En particulier, si $\omega_{\alpha} = \omega_{\beta}$, les solutions sont soit les fr\'{e}quences double $2\omega_{\alpha}$ soit la fr\'{e}quence nulle, qui, elle, correspond \`{a} un d\'{e}placement constant. Il s'agit l\`{a} de \emph{fr\'{e}quences combinatoires}. De la m\^{e}me mani\`{e}re, par identification \`{a} (\ref{EQ:26_3}), les amplitudes des oscillations combinatoires sont proportionnelles \`{a} $a_{\alpha}a_{\beta}$.

Pour les approximations d'ordre sup\'{e}rieur de la fonction de Lagrange, et donc des \'{e}nergies cin\'{e}tique et potentielle, le second membre de l'\'{e}quation du mouvement voit appara\^{i}tre des termes dont les oscillations sont des combinaisons des fr\'{e}quences des solutions p\'{e}riodiques simples en utilisant les formules de Simpson. Ainsi les fr\'{e}quences combinatoires sont $\omega_{\alpha} \pm \omega_{\beta} \pm \omega_{\gamma} \pm \ldots$. Ainsi, parmi les fr\'{e}quences solution, nous trouvons des fr\'{e}quences co\"{i}ncidant avec celle des solutions non perturb\'{e}es $\omega_{\alpha}$. Dans ces cas, nous nous retrouvons avec des oscillations forc\'{e}es, voir paragraphe (\ref{PAR:22}), pour lesquelles la solution (\ref{EQ:22_5}) a une amplitude qui cro\^{i}t avec le temps. Ceci n'est \'{e}videmment pas possible dans notre contexte de syst\`{e}me isol\'{e}.

Pour les approximations suivantes, la m\'{e}thode des approximations successives doit \^{e}tre modifi\'{e}e afin que les facteurs p\'{e}riodiques de la soltion contiennent d\`{e}s le d\'{e}but des valeurs exactes et non pas approch\'{e}es.

\subsection{Cas particulier}

Nous allons ici nous int\'{e}resser \`{a} l'application de la m\'{e}thode des oscillations anharmoniques avec la fonction de Lagrange \`{a} un degr\'{e} de libert\'{e} suivante :
\be
	L = \frac{m\dot{x}^{2}}{2} - \frac{m\omega_{0}^{2}}{2}x^{2} - \frac{m\alpha}{3}x^{3} - \frac{m\beta}{4}x^{4}\label{EQ:28_8}
\ee
L'\'{e}quation du mouvement d'\'{e}crire :
\bea
	& & \frac{\mathrm{d}}{\mathrm{dt}}\left(\frac{\partial L}{\partial\dot{x}}\right) = \frac{\partial L}{\partial x} \nonumber \\
	& \Leftrightarrow & m\ddot{x} = m\omega_{0}^{2}x - m\alpha x^{2} - m\beta x^{3} \nonumber \\
	& \Leftrightarrow & \ddot{x} + \omega_{0}^{2}x = - \alpha x^{2} - \beta x^{3}\label{EQ:28_9}
\eea
\`{A} cette \'{e}quation, nous cherchons une solution sous la forme d'une s\'{e}rie d'approximations successives :
\benn
	x = x_{0} + x_{1} + x_{2}
\eenn
telle que :
\be
	x_{0}(\mathrm{t}) = a\cos(\omega\mathrm{t})\label{EQ:28_10}
\ee
et en choisissant l'origine du temps telle que la phase initiale de $x_{0}$ soit nulle et avec $\omega$ une valeur exacte que nous chercherons par la suite sous la forme d'une s'\'{e}rie $\omega_{0} + \omega_{1} + \ldots$.

L'\'{e}quation (\ref{EQ:28_9}) s'\'{e}crit de mani\`{e}re \'{e}quivalente :
\bea
	\ddot{x} + \omega_{0}^{2}x & = & - \alpha x^{2} - \beta x^{3} \nonumber \\
	\left(1 + \frac{\omega_{0}^{2}}{\omega^{2}} - \frac{\omega_{0}^{2}}{\omega^{2}}\right)\ddot{x} + \omega_{0}^{2}x & = & - \alpha x^{2} - \beta x^{3} \nonumber \\
	\frac{\omega_{0}^{2}}{\omega^{2}}\ddot{x} + \omega_{0}^{2}x & = & - \alpha x^{2} - \beta x^{3} - \left(1 - \frac{\omega_{0}^{2}}{\omega^{2}}\right)\ddot{x} \label{EQ:28_11}
\eea
qui permet d'avoir un premier membre nul pour $x = x_{0}$. Posons d\'{e}sormais $\omega = \omega_{0} + \omega_{1}$ et $x = x_{0} + x_{1}$ en n\'{e}gligeant les infiniments petits d'ordre sup\'{e}rieur. Dans ce cas, l'\'{e}quation (\ref{EQ:28_11}) peut se d\'{e}velopper ainsi :
\benn
	\frac{\omega_{0}^{2}}{\omega^{2}}(\ddot{x_{0}} + \ddot{x_{1}}) + \omega_{0}^{2}(x_{0} + x_{1}) = - \alpha (x_{0} + x_{1})^{2} - \beta (x_{0} + x_{1})^{3} - \left(1 - \frac{\omega_{0}^{2}}{\omega^{2}}\right)(\ddot{x}_{0} + \ddot{x}_{1})
\eenn
qui peut se r\'{e}quire \`{a} car $(x_{0} + x_{1})^{3}$ est d'ordre sup\'{e}rieur \`{a} 2 :
\benn
	\frac{\omega_{0}^{2}}{\omega^{2}}\ddot{x_{0}} + \omega_{0}^{2}x_{0} + \frac{\omega_{0}^{2}}{\omega^{2}}\ddot{x_{1}} + \omega_{0}^{2}x_{1} = -\alpha x_{0}^{2} - \alpha x_{1}^{2} - 2\alpha x_{0}x_{1} - \left(\frac{\omega^{2} - \omega_{0}^{2}}{\omega^{2}}\right)\ddot{x}_{0} - \ddot{x}_{1} + \frac{\omega_{0}^{2}}{\omega^{2}}\ddot{x}_{1}
\eenn
Or nous avons d'apr\`{e}s la relation (\ref{EQ:28_10}) $\ddot{x}_{0} + \omega_{0}^{2}x_{0} = 0$ et les termes en $x_{1}^{2}$ et $x_{0}x_{1}$ sont d'ordre sup\'{e}rieur \`{a} 2. Nous pouvons donc, en utilisant \'{e}galement le fait que $\omega = \omega_{0} + \omega_{1}$, encore r\'{e}duire l'\'{e}quation pr\'{e}c\'{e}dente \`{a} :
\benn
	\ddot{x_{1}} + \omega_{0}^{2}x_{1} = -\alpha a^{2}\cos^{2}(\omega\mathrm{t}) + a\cos(\omega\mathrm{t})\left(\frac{\omega_{0}^{2} + \omega_{1}^{2} + 2\omega_{0}\omega_{1} - \omega_{0}^{2}}{\omega^{2}}\right)
\eenn
Cela donne finalement car $\omega_{1}^{2}$ et $\cos^{2}a = (cos(2a) + 1)/2$ est insignifiant :
\benn
	\ddot{x_{1}} + \omega_{0}^{2}x_{1} = -\frac{\alpha a^{2}}{2} - -\frac{\alpha a^{2}}{2}\cos(2\omega\mathrm{t}) + 2\omega_{0}\omega_{1}\cos(\omega\mathrm{t})
\eenn
Pour interdire la r\'{e}sonance dans cette derni\`{e}re \'{e}quation, il est n\'{e}cessaire que dans le second membre, il n'y ait pas d'oscillations telles que $\omega = \omega_{0}$, ce qui donne la condition $\omega_{1} = 0$ pour annuler le terme en $\cos(\omega\mathrm{t})$. Cela permet d'en arriver \`{a} l'\'{e}quation lin\'{e}aire non homog\`{e}ne :
\benn
	\ddot{x_{1}} + \omega_{0}^{2}x_{1} = -\frac{\alpha a^{2}}{2} - \frac{\alpha a^{2}}{2}\cos(2\omega\mathrm{t})
\eenn
dont la solution est la somme de deux fonctions diff\'{e}rentes permettant de r\'{e}soudre par lin\'{e}arité :
\benn
	\begin{cases}
		\ddot{x_{1}} + \omega_{0}^{2}x_{1} = -\frac{\alpha a^{2}}{2} \\
		\ddot{x_{1}} + \omega_{0}^{2}x_{1} = -\frac{\alpha a^{2}}{2}\cos(2\omega\mathrm{t})
	\end{cases}
\eenn
Pour la premi\`{e}re \'{e}quation, la solution est \'{e}videmment $-\alpha a^{2}/(2\omega_{0}^{2})$. Pour la seconde, cherchons une solution en $x_{10}\cos(2\omega\mathrm{t})$ qui une fois inject\'{e}e donne :
\bea
	-4x_{10}\omega^{2}\cos(2\omega\mathrm{t}) + \omega_{0}^{2}x_{10}\cos(2\omega\mathrm{t}) & = & -\frac{\alpha a^{2}}{2}\cos(2\omega\mathrm{t}) \nonumber \\
	\Leftrightarrow x_{10} = \frac{\alpha a^{2}}{6\omega^{2}} & \Leftrightarrow & x_{10} = \frac{\alpha a^{2}}{6\omega_{0}^{2}}
\eea
car $\omega^{2} = (\omega_{0} + \omega_{1})^{2} = \omega_{0}^{2}$ car $\omega_{1}$ reste nulle. Nous obtenons finalement :
\be
	x_{1}(\mathrm{t}) = -\frac{\alpha a^{2}}{2\omega_{0}^{2}} + \frac{\alpha a^{2}}{6\omega_{0}^{2}}\cos(2\omega\mathrm{t})\label{EQ:28_12}
\ee
Passons d\'{e}sormais au troisi\`{e}me ordre tel que $x = x_{0} + x_{1} + x_{2}$ et $\omega = \omega_{0} + 0 + \omega_{2}$. Alors l'\'{e}quation (\ref{EQ:28_11}) devient :
\bea
	\frac{\omega_{0}^{2}}{\omega^{2}}\ddot{x_{0}} + \omega_{0}^{2}x_{0} + \frac{\omega_{0}^{2}}{\omega^{2}}\ddot{x_{1}} + \omega_{0}^{2}x_{1} + \frac{\omega_{0}^{2}}{\omega^{2}}\ddot{x_{2}} + \omega_{0}^{2}x_{2} & = & -\alpha x_{0}^{2} - \alpha x_{1}^{2} - \alpha x_{2}^{2} - 2\alpha x_{0}x_{1} - 2\alpha x_{0}x_{2} - 2\alpha x_{1}x_{2} \nonumber \\
	& & -\beta x_{0}^{3} - \beta x_{1}^{3} - \beta x_{2}^{3} - 3\beta x_{0}^{2}x_{1} - 3\beta x_{0}^{2}x_{2} - 3\beta x_{1}^{2}x_{2} \nonumber \\
	& & - 3\beta x_{1}x_{2}^{2} - 3\beta x_{0}x_{1}^{2} - 3\beta x_{0}x_{2}^{2} - 9\beta x_{0}x_{1}x_{2} \nonumber \\
	& & - \frac{(\omega^{2} - \omega_{0}^{2})}{\omega^{2}}\ddot{x}_{0} - \frac{(\omega^{2} - \omega_{0}^{2})}{\omega^{2}}\ddot{x}_{1} - \ddot{x}_{2} + \frac{\omega_{0}^{2}}{\omega^{2}}x_{2} \nonumber
\eea
En supprimant les infiniments petits d'ordre sup\'{e}rieur \`{a} 3, l'\'{e}quation du mouvement devient :
\bea
	\frac{\omega_{0}^{2}}{\omega^{2}}\ddot{x_{0}} + \omega_{0}^{2}x_{0} + \frac{\omega_{0}^{2}}{\omega^{2}}\ddot{x_{1}} + \omega_{0}^{2}x_{1} + \frac{\omega_{0}^{2}}{\omega^{2}}\ddot{x_{2}} + \omega_{0}^{2}x_{2} & = & -\alpha x_{0}^{2} - 2\alpha x_{0}x_{1} \nonumber \\
	& & -\beta x_{0}^{3} \nonumber \\
	& & - \frac{(\omega^{2} - \omega_{0}^{2})}{\omega^{2}}\ddot{x}_{0} - \frac{(\omega^{2} - \omega_{0}^{2})}{\omega^{2}}\ddot{x}_{1} - \ddot{x}_{2} + \frac{\omega_{0}^{2}}{\omega^{2}}x_{2} \nonumber
\eea
Mais nous savons \'{e}galement que, d'apr\`{e}s les relations (\ref{EQ:28_10}) et (\ref{EQ:28_12}) :
\benn
	\begin{cases}
		\frac{\omega_{0}^{2}}{\omega^{2}}\ddot{x_{0}} + \omega_{0}^{2}x_{0} = 0 \\
		\frac{\omega_{0}^{2}}{\omega^{2}}\ddot{x_{1}} + \omega_{0}^{2}x_{1} = -\alpha x_{0}^{2} - \frac{(\omega^{2} - \omega_{0}^{2})}{\omega^{2}}\ddot{x}_{1} + 2\omega_{0}\omega_{1}x_{0}
	\end{cases}
\eenn
Avec toujours $\omega_{1} = 0$ puis avec $\omega^{2} - \omega_{0}^{2} = \omega_{0}^{2} + \omega_{2}^{2} + 2\omega_{0}\omega_{2} - \omega_{0}^{2} \approx 2\omega_{0}\omega_{2}$, nous pouvons en d\'{e}duire l'\'{e}quation du mouvement :
\bea
	-\alpha x_{0}^{2} + \ddot{x}_{2} + \omega_{0}^{2}x_{2} & = & -\alpha x_{0}^{2} - 2\alpha x_{0}x_{1} - \beta x_{0}^{3} + 2\omega_{0}\omega_{2}x_{0} \nonumber \\
	\Leftrightarrow  \ddot{x}_{2} + \omega_{0}^{2}x_{2} & = & -2\alpha x_{0}x_{1} - \beta x_{0}^{3} + 2\omega_{0}\omega_{2}x_{0} \nonumber
\eea
En y ins\'{e}rant les approximations du premier et du deuxi\`{e}me ordres, (\ref{EQ:28_10}) et (\ref{EQ:28_12}), l'\'{e}quation pr\'{e}c\'{e}dente devient :
\bea
	\ddot{x}_{2} + \omega_{0}^{2}x_{2} & = & \frac{2\alpha^{2}a^{3}}{2\omega_{0}^{2}}\cos(\omega\mathrm{t}) - \frac{2\alpha^{2}a^{3}}{6\omega_{0}^{2}}\cos(\omega\mathrm{t})\cos(2\omega\mathrm{t}) - \beta a^{3}\cos^{3}(\omega\mathrm{t}) + 2\omega_{0}\omega_{2}a\cos(\omega\mathrm{t}) \nonumber \\
	& = & \frac{\alpha^{2}a^{3}}{\omega_{0}^{2}}\cos(\omega\mathrm{t}) - \frac{\alpha^{2}a^{3}}{6\omega_{0}^{2}}\left(\cos(3\omega\mathrm{t}) + \cos(-\omega\mathrm{t})\right) - \frac{\beta a^{3}}{2}\cos(\omega\mathrm{t})\left(\cos(2\omega\mathrm{t}) + 1\right) \nonumber \\
	& & + 2\omega_{0}\omega_{2}a\cos(\omega\mathrm{t}) \nonumber \\
	& = & \frac{\alpha^{2}a^{3}}{\omega_{0}^{2}}\cos(\omega\mathrm{t}) - \frac{\alpha^{2}a^{3}}{6\omega_{0}^{2}}\cos(3\omega\mathrm{t}) - \frac{\alpha^{2}a^{3}}{6\omega_{0}^{2}}\cos(\omega\mathrm{t}) - \frac{\beta a^{3}}{2}\cos(\omega\mathrm{t}) \nonumber \\
	& & - \frac{\beta a^{3}}{4}\left(\cos(3\omega\mathrm{t}) + \cos(-\omega\mathrm{t})\right) + 2\omega_{0}\omega_{2}a\cos(\omega\mathrm{t}) \nonumber \\
	& = & \left(\frac{\alpha^{2}a^{2}}{\omega_{0}^{2}} - \frac{\alpha^{2}a^{2}}{6\omega_{0}^{2}} - \frac{\beta a^{2}}{2} - \frac{\beta a^{2}}{4} + 2\omega_{0}\omega_{2}\right)a\cos(\omega\mathrm{t}) - \left(\frac{\alpha^{2}}{6\omega_{0}^{2}} + \frac{\beta}{4}\right)a^{3}\cos(3\omega\mathrm{t}) \nonumber \\
	& = & \left(\frac{5\alpha^{2}a^{2}}{6\omega_{0}^{2}} - \frac{3\beta a^{2}}{4\omega_{0}^{2}} + 2\omega_{0}\omega_{2}\right)a\cos(\omega\mathrm{t}) - \left(\frac{\alpha^{2}}{6\omega_{0}^{2}} + \frac{\beta}{4}\right)a^{3}\cos(3\omega\mathrm{t}) \nonumber
\eea
Le terme en $\cos(\omega\mathrm{t})$ dans le second membre est \`{a} annuler car la premi\`{e}re approximation $x_{0}$ contient d\'{e}j\`{a} cette pulsation $\omega$. Cela permet d'\'{e}crire :
\bea
	\frac{5\alpha^{2}a^{2}}{6\omega_{0}^{2}} - \frac{3\beta a^{2}}{4} + 2\omega_{0}\omega_{2} & = & 0 \nonumber \\
	\Leftrightarrow \omega_{2} & = & \left(\frac{3\beta}{8} - \frac{5\alpha^{2}}{12\omega_{0}^{2}}\right)\frac{a^{2}}{\omega_{0}} \label{EQ:28_13}
\eea
Cela donne enfin pour l'approximation du troisi\`{e}me ordre, la r\'{e}solution de l'\'{e}quation suivante :
\benn
	\ddot{x}_{2} + \omega_{0}^{2}x_{2} = -\left(\frac{\alpha^{2}}{6\omega_{0}^{2}} + \frac{\beta}{4}\right)a^{3}\cos(3\omega\mathrm{t})
\eenn
En supposant $x_{2} = x_{20}\cos(3\omega\mathrm{t})$, nous obtenons :
\bea
	-9x_{20}\omega^{2}\cos(3\omega\mathrm{t}) + \omega_{0}^{2}x_{20}\cos(3\omega\mathrm{t}) & = &  -\left(\frac{\alpha^{2}}{6\omega_{0}^{2}} + \frac{\beta}{4}\right)a^{3}\cos(3\omega\mathrm{t}) \nonumber \\
	\Leftrightarrow (\omega_{0}^{2} - 9\omega^{2})x_{20} & = & -\left(\frac{\alpha^{2}}{6\omega_{0}^{2}} + \frac{\beta}{4}\right)a^{3} \nonumber
\eea
or nous avons toujours $\omega^{2} \approx \omega_{0}^{2}$, donc :
\benn
	x_{20} = \left(\frac{\alpha^{2}}{3\omega_{0}^{2}} + \frac{\beta}{2}\right)\frac{a^{3}}{16\omega_{0}^{2}}
\eenn
Finalement, nous pouvons donc \'{e}crire la solution approch\'{e}e \`{a} l'ordre 3 :
\bea
	\omega & = & \omega_{0} + \left(\frac{3\beta}{8} - \frac{5\alpha^{2}}{12\omega_{0}^{2}}\right)\frac{a^{2}}{\omega_{0}} \nonumber \\
	x & = & a\cos(\omega\mathrm{t}) - \frac{\alpha a^{2}}{2\omega_{0}^{2}} + -\frac{\alpha a^{2}}{6\omega_{0}^{2}}\cos(2\omega\mathrm{t}) + \left(\frac{\alpha^{2}}{3\omega_{0}^{2}} + \frac{\beta}{2}\right)\frac{a^{3}}{16\omega_{0}^{2}}\cos(3\omega\mathrm{t}) \label{EQ:28_14}
\eea

\begin{figure}[htb!]
	\begin{center}
		\includegraphics[width=10cm]{chapter_05_paragraph_28}
		\caption{Amplitude de la solution (\ref{EQ:28_14}) avec des valeurs $a = 1$, $\alpha = 1$, $\beta = 0.5$ et $\omega_{0} = \left\{0.3\,0.4\,0.5\,0.75\,1.0\,2.0\,3.0\right\}$}\label{FIG:5_28}
	\end{center}
\end{figure}

\section{R\'{e}sonance dans les oscillations non lin\'{e}aires}\label{PAR:29}

\subsection{\'{E}quation de l'amplitude des oscillations forc\'{e}es}

En ajoutant une force ext\'{e}rieure oscillante de fr\'{e}quence $\gamma$ comme celle de l'{e}quation (\ref{EQ:26_1}) et une force de frottement avec un c{\oe}fficient d'ammortissement $\lambda$ \`{a} l'{e}quation (\ref{EQ:28_9}), cette derni\`{e}re devient :
\be
	\ddot{x} + 2\lambda\dot{x} + \omega_{0}^{2}x = \frac{f}{m}\cos(\gamma\mathrm{t}) - \alpha x^{2} - \beta x^{3} \label{EQ:29_1}
\ee
D\'{e}sormais au voisinage de la r\'{e}sonance ordinaire, i.e. $\gamma = \omega_{0} + \epsilon$. En premi\`{e}re approximation, i.e lin\'{e}aire, la relation entre l'amplitude de la solution oscillatoire forc\'{e}e $b$, l'amplitude $f$ et la fr\'{e}quence de la force ext\'{e}rieure $\gamma$ est donn\'{e}e, au voisinage de la r\'{e}sonance, par la formule de (\ref{EQ:26_7}) :
\be
	b = \frac{f}{2m\omega_{0}\sqrt{\epsilon^{2} + \lambda^{2}}} \Leftrightarrow b^{2}(\epsilon^{2} + \lambda^{2}) = \frac{f}{4m^{2}\omega_{0}^{2}} \label{EQ:29_2}
\ee

\appendix

\chapter{Probl\`emes d'\'equations du mouvement}

L'objectif est de trouver la fonction de Lagrange dans les cas suivants plac\'es dans un champ de pesanteur constant $g$.

\section{Probl\`eme 1}

\begin{figure}[htb!]
	\begin{center}
		\begin{picture}(150,200)(0,0)
			%axis
			\linethickness{0.05mm}
			\multiput(0,200)(10,0){15}{\line(1,0){8}}\put(150,195){$x$}
			\multiput(0,0)(0,10){20}{\line(0,1){8}}\put(-5,-10){$y$}
			\multiput(75,90)(0,10){7}{\line(0,-1){8}}
			%socle
			\put(0,200){\color{black}\circle*{5}}
			%arms
			\linethickness{0.5mm}
			\put(0,200){\line(10,-15){75}}\put(75,90){\color{black}\circle*{10}}\put(80,95){$m_{1}$}
			\put(75,90){\line(15,-10){50}}\put(125,55){\color{black}\circle*{10}}\put(130,60){$m_{2}$}
		\end{picture}
		\caption{Pendule double oscillant}\label{FIG:1_1}
	\end{center}
\end{figure}

\chapter{Probl\`emes sur les lois de conservation}

\section{Changement de direction d'une particule}

\begin{figure}[htb!]
	\begin{center}
		\begin{picture}(150,100)(0,0)
			%axis
			\linethickness{0.05mm}
			\multiput(75,0)(0,10){10}{\line(0,1){8}}
			\multiput(20,50)(7,0){6}{\line(1,0){5}}
			\multiput(100,50)(7,0){6}{\line(1,0){5}}
			%particles
			\put(20,50){\color{black}\circle*{5}}\put(6,47){$m$}
			\put(100,50){\color{black}\circle*{5}}
			%velocities
			\put(20,50){\color{black}\vector(3,1){40}}\put(35,65){$\vec{v}_{1}$}
			\put(100,50){\color{black}\vector(2,1){40}}\put(110,65){$\vec{v}_{2}$}
			%angles
			\linethickness{0.05mm}
			\qbezier(40,50),(40,52),(38,55)
			\put(45,52){$\theta_{1}$}
			\qbezier(115,50),(115,52),(113,55)
			\put(120,52){$\theta_{2}$}
			%potential energies
			\put(25,10){$U_{1}(t_{1})$}
			\put(110,10){$U_{2}(t_{2})$}
		\end{picture}
		\caption{Trajectoire de la particule}\label{FIG:2_1}
	\end{center}
\end{figure}
\chapter{Probl\`emes d'int\'egration des \'equations du mouvement}

\section{Mouvement lin\'eaire}

\subsection{Oscillations d'un pendule math\'ematique plan}

\begin{figure}[htb!]
	\begin{center}
		\begin{picture}(100,100)(0,0)
			%axis
			\linethickness{0.05mm}
			\multiput(0,100)(10,0){10}{\line(1,0){8}}\put(102,95){$x$}
			\multiput(0,0)(0,10){10}{\line(0,1){8}}\put(-5,-10){$y$}
			%socle
			\put(0,100){\color{black}\circle*{5}}
			%arms
			\linethickness{0.5mm}
			\put(0,100){\line(10,-15){70}}
			\put(40,50){$l$}
			\put(68,-5){\color{black}\circle*{10}}\put(75,-7){$m$}
			%angles
			\linethickness{0.05mm}
			\qbezier(0,80),(5,82),(10,85)
			\put(3,70){$\varphi$}
		\end{picture}
		\caption{Pendule double oscillant}\label{FIG:3_1_1}
	\end{center}
\end{figure}

Il s'agit de trouver la p\'eriode d'oscillation du pendule en fonction de l'amplitude du mouvement. En supposant $\varphi_{0}$ l'angle maximal par rapport \`a la verticale, o\`u la vitesse du pendule est nulle, l'\'energie totale du pendule est :
\bea
	E & = & \frac{m}{2}\left[\dfrac{\mathrm{d}(l\varphi)}{\mathrm{dt}}\right]^{2} - mgl\cos\varphi \nonumber \\
	& = & \frac{m}{2}l^{2}\dot{\varphi}^{2} - mgl\cos\varphi
\eea
et par conservation de l'\'energie totale :
\be
	E = E(\varphi_{0}) = -mgl\cos\varphi_{0}
\ee
Par sym\'etrie du mouvement, la p\'eriode est \'egale au temps de parcours entre $\varphi=0$ et $\varphi=\varphi_{0}$. Sachant que la quantit\'e, $E-U$ vaut $-mgl\cos\varphi_{0} + mgl\cos\varphi$, l'\'equation (\ref{EQ:11_3}) peut s'\'ecrire :
\bea
	\mathrm{T} & = & 4\sqrt{\frac{m}{2}}\int_{0}^{\varphi_{0}}{\dfrac{\mathrm{d}(l\varphi)}{\sqrt{mgl(\cos\varphi - \cos\varphi_{0})}}} \nonumber \\
	& = & 4\sqrt{\frac{l}{2g}}\int_{0}^{\varphi_{0}}{\dfrac{\mathrm{d}\varphi}{\sqrt{\cos\varphi - \cos\varphi_{0}}}}
\eea
Notons que :
\bea
	\sin^{2}\frac{\alpha}{2} & = & \left(\dfrac{e^{i\frac{\alpha}{2}} - e^{-i\frac{\alpha}{2}}}{2i}\right)^{2} \nonumber \\
	& = & -\frac{1}{4}\left(e^{i\alpha} + e^{-i\alpha} - 2e^{i\frac{\alpha}{2}-i\frac{\alpha}{2}}\right) \nonumber \\
	& = & -\frac{1}{4}\left(e^{i\alpha} + e^{-i\alpha} - 2\right) \nonumber \\
	& = & \frac{1}{2}\left(1-\cos\alpha\right) \nonumber \\
	\Leftrightarrow \cos\alpha & = & 1 - 2\sin^{2}\frac{\alpha}{2}
\eea
Cela permet de d\'evelopper la p\'eriode telle que :
\bea
	\mathrm{T} & = & 4\sqrt{\frac{l}{2g}}\int_{0}^{\varphi_{0}}{\dfrac{\mathrm{d}\varphi}{\sqrt{1-2\sin\frac{\varphi}{2} - 1 + \sin\frac{\varphi_{0}}{2}}}} \nonumber \\
	& = & 4\sqrt{\frac{l}{2g}}\int_{0}^{\varphi_{0}}{\dfrac{\mathrm{d}\varphi}{\sqrt{2\left(\sin\frac{\varphi_{0}}{2} - \sin\frac{\varphi}{2}\right)}}} \nonumber \\
	& = & 2\sqrt{\frac{l}{g}}\int_{0}^{\varphi_{0}}{\dfrac{\mathrm{d}\varphi}{\sqrt{\sin\frac{\varphi_{0}}{2} - \sin\frac{\varphi}{2}}}}
\eea
En posant $\sin\xi = \dfrac{\sin(\varphi/2)}{\sin(\varphi_{0}/2)}$, nous avons pour $\varphi = 0$, $\sin\xi = 0$ soit $\xi = 0$ et pour $\varphi = \varphi_{0}$, $\sin\xi = 1$ soit $\xi = \frac{\pi}{2}$. En d\'eveloppant la d\'efinition de $\sin\xi$ pour l'inverser, nous trouvons :
\bea
	\sin^{2}\xi\sin^{2}\left(\frac{\varphi_{0}}{2}\right) & = & \sin^{2}\left(\frac{\varphi}{2}\right) \nonumber \\
	& = & 1 - \cos^{2}\left(\frac{\varphi}{2}\right) \nonumber \\
	\Leftrightarrow \cos\left(\frac{\varphi}{2}\right) & = & \sqrt{1 - \sin^{2}\left(\frac{\varphi_{0}}{2}\right)\sin^{2}\xi}
\eea
et pour conna\^itre la correspondance entre $\mathrm{d}\varphi$ et $\mathrm{d}\xi$ :
\bea
	\sin\xi & = & \dfrac{\sin\left(\frac{\varphi}{2}\right)}{\sin\left(\frac{\varphi_{0}}{2}\right)} \nonumber \\
	\Leftrightarrow \cos\xi\mathrm{d}\xi & = & \dfrac{\cos\left(\frac{\varphi}{2}\right)}{2\sin\left(\frac{\varphi_{0}}{2}\right)}\mathrm{d}\varphi \nonumber \\
	\mathrm{d}\varphi & = & \dfrac{2\sin\left(\frac{\varphi_{0}}{2}\right)\cos\xi}{\cos\left(\frac{\varphi}{2}\right)}\mathrm{d}\xi \nonumber \\
	& = & \dfrac{2\sin\left(\frac{\varphi_{0}}{2}\right)\sqrt{1 - \sin^{2}\xi}}{\cos\left(\frac{\varphi}{2}\right)}\mathrm{d}\xi \nonumber \\
	& = & \dfrac{2\sin\left(\frac{\varphi_{0}}{2}\right)\sqrt{1 - \dfrac{\sin^{2}\left(\frac{\varphi}{2}\right)}{\sin^{2}\left(\frac{\varphi_{0}}{2}\right)}}}{\cos\left(\frac{\varphi}{2}\right)}\mathrm{d}\xi \nonumber \\
	& = & \dfrac{2\sqrt{\sin^{2}\left(\frac{\varphi_{0}}{2}\right) - \sin^{2}\left(\frac{\varphi}{2}\right)}}{\cos\left(\frac{\varphi}{2}\right)}\mathrm{d}\xi \nonumber \\
	\Leftrightarrow \dfrac{\mathrm{d}\varphi}{\sqrt{\sin^{2}\left(\frac{\varphi_{0}}{2}\right) - \sin^{2}\left(\frac{\varphi}{2}\right)}} & = & \dfrac{2\mathrm{d}\xi}{\sqrt{1 - \sin^{2}\left(\frac{\varphi_{0}}{2}\right)\sin^{2}\xi}}
\eea
En cons\'equence, en posant :
\be
	K(k) = \int_{0}^{\frac{\pi}{2}}\dfrac{\mathrm{d}\xi}{\sqrt{1 - k^{2}\sin^{2}\xi}}
\ee
la p\'eriode $\mathrm{T}$ se r\'e\'ecrit ainsi\footnote{Dans le livre, il est indiqu\'e que l'argument de la fonction $K$ est $\frac{\sin\left(\varphi_{0}\right)}{2}$, je pense qu'il ne s'agit que d'une coquille.} :
\be
	\mathrm{T} = 4\sqrt{\dfrac{l}{g}}K\left(\sin(\varphi_{0}/2)\right)
\ee
La fonction $K$ est l'int\'egrale elliptique compl\`ete de premi\`ere esp\`ece. Elle admet comme d\'eveloppement en s\'erie enti\`ere :
\bea
	K(k) & = & \dfrac{\pi}{2}\sum_{n=0}^{+\infty}\left[\dfrac{(2n)!}{2^{2n}(n!)^{2}}\right]^{2}k^{2n} \nonumber \\
	& =& \dfrac{\pi}{2}\left(1 + \left(\dfrac{2}{4 \times 1}\right)^{2}k^{2} + \cdots\right)
\eea
donc pour de petites oscillations telles que $\sin(\varphi_{0}/2)\simeq\varphi_{0}/2\ll 1$ :
\be
	K(\sin\varphi_{0}/2) = \dfrac{\pi}{2}\left(1 + \dfrac{1}{16}\varphi_{0}^{2} + \cdots\right)
\ee
soit :
\be
	\mathrm{T} = 2\pi\sqrt{\dfrac{l}{g}}\left(1 + \dfrac{1}{16}\varphi_{0}^{2} + \cdots\right)
\ee
et en premi\`ere approximation, la p\'eriode du mouvement n'est d\'ependante ni de la masse de la particule ni de l'amplitude initiale.

\subsection{Cas d'\'energies potentielles particuli\`eres}

Il s'agit de d\'eterminer la p\'eriode d'oscillations en fonction de l'\'energie pour le mouvement d'une particule de masse $m$ dans un champ o\`u l'\'energie potentielle $U$ est d\'efinie telle que :

\subsubsection{$U = A\lvert x \rvert^{n}$}

\begin{figure}[htb!]
	\begin{center}
		\includegraphics[width=10cm]{chapter_03_exercice_2a}
		\caption{$U = \lvert x \rvert^{n}$ pour $n \in \{1,2,3,4,5\}$}\label{FIG:3_2_a}
	\end{center}
\end{figure}

Dans ce cas pr\'ecis, commen\c{c}ons par d\'efinir les points d'arr\^et tels que $E=U$, soit $x = \pm\left(\dfrac{E}{A}\right)^{\frac{1}{n}}$. Cela permet d'\'ecrire l'\'equation (\ref{EQ:11_5}) telle que :
\bea
	\mathrm{T}(E) & = & \sqrt{2m}\int_{-\left(\frac{E}{A}\right)^{\frac{1}{n}}}^{+\left(\frac{E}{A}\right)^{\frac{1}{n}}}\dfrac{\mathrm{d}x}{\sqrt{E - A\lvert x \rvert^{n}}} \nonumber \\
	& = & 2\sqrt{2m}\int_{0}^{+\left(\frac{E}{A}\right)^{\frac{1}{n}}}\dfrac{\mathrm{d}x}{\sqrt{E - Ax^{n}}} \nonumber \\
	& = & 2\sqrt{\dfrac{2m}{E}}\int_{0}^{+\left(\frac{E}{A}\right)^{\frac{1}{n}}}\dfrac{\mathrm{d}x}{\sqrt{1 - \dfrac{Ax^{n}}{E}}}
\eea
car la fonction $U$ est paire, i.e. $U(x) = U(-x)$. En posant :
\be
	y = \left(\dfrac{A}{E}\right)^{\frac{1}{n}}x
\ee
qui permet d'\'ecrire :
\be
	\begin{cases}
		y^{n} = \dfrac{A}{E}x^{n} \\
		\mathrm{d}y = \left(\dfrac{A}{E}\right)^{\frac{1}{n}}\mathrm{d}x
	\end{cases}
\ee
et :
\be
	\begin{cases}
		x = 0 \Rightarrow y = 0 \\
		x = \left(\dfrac{E}{A}\right)^{\frac{1}{n}} \Rightarrow y = \left(\dfrac{A}{E}\right)^{\frac{1}{n}}\left(\dfrac{E}{A}\right)^{\frac{1}{n}} = 1
	\end{cases}
\ee
La p\'eriode $\mathrm{T}$ devient :
\bea
	\mathrm{T} & = & 2\sqrt{\dfrac{2m}{E}}\int_{0}^{1}\left(\dfrac{E}{A}\right)^{\frac{1}{n}}\dfrac{\mathrm{d}y}{\sqrt{1 - y^{n}}} \nonumber \\
	& = & 2\dfrac{\sqrt{2mE^{\frac{1}{n}-\frac{1}{2}}}}{A^{\frac{1}{n}}}\int_{0}^{1}\dfrac{\mathrm{d}y}{\sqrt{1 - y^{n}}}
\eea
De la m\^eme mani\`ere, en posant $u=y^{n}$, nous avons :
\be
	\begin{cases}
		y = 0 \Rightarrow u = 0 \\
		y = 1 \Rightarrow u = 1
	\end{cases}
\ee
ainsi que :
\be
	\begin{cases}
		y = u^{\frac{1}{n}} \Rightarrow y^{n-1} = u^{\frac{n-1}{n}} = u^{1 - \frac{1}{n}} \\
		ny^{n-1}\mathrm{d}y = \mathrm{d}u \Leftrightarrow \mathrm{d}y = \dfrac{u^{\frac{1}{n} - 1}}{n}\mathrm{d}u
	\end{cases}
\ee
Nous arrivons \`a :
\bea
	\mathrm{T} & = & 2\dfrac{\sqrt{2m}E^{\frac{1}{n}-\frac{1}{2}}}{A^{\frac{1}{n}}}\int_{0}^{1}\dfrac{u^{\frac{1}{n} - 1}\mathrm{d}u}{n(1-u)^{\frac{1}{2}}} \nonumber \\
	& = & 2\dfrac{\sqrt{2m}E^{\frac{1}{n}-\frac{1}{2}}}{nA^{\frac{1}{n}}}\int_{0}^{1}u^{\frac{1}{n} - 1}(1-u)^{\frac{-1}{2}}\mathrm{d}u \label{EQ:APP3_2_a}
\eea
Utilisons ici les int\'egrales d'Euler de premi\`ere esp\`ece, \emph{fonction Béta}, d\'efinies telles que :
\be
	\mathrm{B}(x,y) = \int_{0}^{1}\mathrm{t}^{x-1}(1-\mathrm{t})^{y-1}\mathrm{dt} = \dfrac{\Gamma(x)\Gamma(y)}{\Gamma(x+y)} \label{EQ:INT_EULER_BETA}
\ee
et celles de seconde esp\`ece, \emph{fonction Gamma} :
\be
	\Gamma(z) = \int_{0}^{+\infty}\mathrm{t}^{z-1}e^{-\mathrm{t}}\mathrm{dt} \label{EQ:INT_EULER_GAMMA}
\ee
L'\'equation (\ref{EQ:APP3_2_a}) s'\'ecrit alors :
\bea
	\mathrm{T} & = & 2\dfrac{\sqrt{2m}E^{\frac{1}{n}-\frac{1}{2}}}{nA^{\frac{1}{n}}}\mathrm{B}\left(\frac{1}{n},\frac{1}{2}\right) \nonumber \\
	& = & 2\dfrac{\sqrt{2m}E^{\frac{1}{n}-\frac{1}{2}}}{nA^{\frac{1}{n}}}\dfrac{\Gamma\left(\frac{1}{n}\right)\Gamma\left(\frac{1}{2}\right)}{\Gamma\left(\frac{1}{n}+\frac{1}{2}\right)} \nonumber
\eea
Or $\Gamma(1/2) = \sqrt{\pi}$, nous avons donc finalement :
\be
	\mathrm{T} = 2\dfrac{\sqrt{2\pi m}\Gamma\left(\frac{1}{n}\right)}{nA^{\frac{1}{n}}\Gamma\left(\frac{1}{n}+\frac{1}{2}\right)}E^{\frac{1}{n}-\frac{1}{2}}
\ee

\begin{figure}[htb!]
	\begin{center}
		\includegraphics[width=10cm]{chapter_03_exercice_2a_result}
		\caption{$\mathrm{T}(E)$ pour $m=1$, $A=1$ et $n \in \{1,2,3,4,5\}$}\label{FIG:3_2_a_result}
	\end{center}
\end{figure}

\subsubsection{$U = -U_{0}/\cosh^{2}(\alpha x)$ avec $-U_{0} < E < 0$}

\begin{figure}[htb!]
	\begin{center}
		\includegraphics[width=10cm]{chapter_03_exercice_2b}
		\caption{$U = -2 / \cosh^{2}(\alpha x)$ pour $\alpha \in \{0.25,0.35,0.45,0.55,0.65,0.75\}$}\label{FIG:3_2_b}
	\end{center}
\end{figure}

Les point d'arr\^et sont toujours d\'efinis tels que $U=E$, i.e. :
\bea
	\cosh^{2}(\alpha x) & = & -\dfrac{U_{0}}{E} \nonumber \\
	\Leftrightarrow \alpha x & = & \pm \arccosh\left(\sqrt{\dfrac{-U_{0}}{E}}\right)
\eea
Ceci est possible car $-U_{0} < E < 0$ et donc $-U_{0}/E > 0$. Les points d'arr\^et sont donc :
\be
	x = \pm \dfrac{1}{\alpha}\arccosh\left(\sqrt{\dfrac{-U_{0}}{E}}\right)
\ee

L'\'equation (\ref{EQ:11_5}) peut donc s'\'ecrire :
\bea
	\mathrm{T}(E) & = & \sqrt{2m}\int_{-\frac{1}{\alpha}\arccosh\left(\sqrt{\frac{-U_{0}}{E}}\right)}^{\frac{1}{\alpha}\arccosh\left(\sqrt{\frac{-U_{0}}{E}}\right)}\dfrac{\mathrm{d}x}{\sqrt{E + \dfrac{U_{0}}{\cosh^{2}(\alpha x)}}} \nonumber \\
	& = & \sqrt{\dfrac{2m}{\lvert E \rvert}}\int_{-\frac{1}{\alpha}\arccosh\left(\sqrt{\frac{-U_{0}}{E}}\right)}^{\frac{1}{\alpha}\arccosh\left(\sqrt{\frac{-U_{0}}{E}}\right)}\dfrac{\mathrm{d}x}{\sqrt{1 + \dfrac{U_{0}}{E\cosh^{2}(\alpha x)}}}
\eea
car $E < 0$. En choisissant :
\be
	y = \sqrt{\frac{-U_{0}}{E}}\dfrac{1}{\cosh(\alpha x)}
\ee
nous obtenons pour les bornes de l'int\'egrale :
\be
	\begin{cases}
		x = -\frac{1}{\alpha}\arccosh\left(\sqrt{\frac{-U_{0}}{E}}\right) \Rightarrow y = -1 \\
		x = \frac{1}{\alpha}\arccosh\left(\sqrt{\frac{-U_{0}}{E}}\right) \Rightarrow y = 1 \\
	\end{cases}
\ee
et :
\bea
	y^{2} & = & \frac{-U_{0}}{E}\dfrac{1}{\cosh^{2}(\alpha x)} \nonumber \\
	\mathrm{d}y & = & \sqrt{\frac{-U_{0}}{E}}\dfrac{-\alpha}{\cosh^{2}(\alpha x)}\mathrm{d}x \nonumber \\
	\Leftrightarrow & = & -\alpha\sqrt{\frac{-U_{0}}{E}}\dfrac{E}{-U_{0}}y^{2}\mathrm{d}x \nonumber \\
	\Leftrightarrow \mathrm{d}x & = & -\sqrt{\frac{-U_{0}}{E}}\dfrac{\mathrm{d}y}{\alpha y^{2}}
\eea
La p\'eriode s'\'ecrit alors :
\bea
	\mathrm{T}(E) & = & \sqrt{\dfrac{2m}{\lvert E \rvert}}\int_{-1}^{1}\dfrac{y^{-2}\mathrm{d}y}{\sqrt{1-y^{2}}}\times \dfrac{-1}{\alpha}\sqrt{\dfrac{-U_{0}}{E}} \nonumber \\
	& = & -\dfrac{2}{\alpha}\sqrt{\dfrac{-U_{0}}{E}}\sqrt{\dfrac{2m}{\lvert E \rvert}}\int_{0}^{1}\dfrac{y^{-2}\mathrm{d}y}{\sqrt{1-y^{2}}}
\eea

En posant $u = y^{2}$, soit $y = u^{1/2}$ et $\mathrm{d}u = 2y\mathrm{d}y = 2u^{1/2}\mathrm{d}y \Leftrightarrow \mathrm{d}y = \frac{1}{2}u^{-1/2}\mathrm{d}u$, alors la p\'eriode devient :
\bea
	\mathrm{T}(E) & = & -\dfrac{2}{\alpha}\sqrt{\dfrac{-U_{0}}{E}}\sqrt{\dfrac{2m}{\lvert E \rvert}}\int_{0}^{1}\frac{1}{2}\dfrac{u^{-3/2}\mathrm{d}u}{(1-u)^{1/2}} \nonumber \\
	& = & -\dfrac{1}{\alpha}\sqrt{\dfrac{-U_{0}}{E}}\sqrt{\dfrac{2m}{\lvert E \rvert}} \mathrm{B}\left(-\frac{1}{2},\frac{1}{2}\right) \nonumber \\
	& = & \dfrac{2\pi}{\alpha}\sqrt{\dfrac{-U_{0}}{E}}\sqrt{\dfrac{2m}{\lvert E \rvert}}
\eea
car $\Gamma(0) = 1$, $\Gamma(\frac{1}{2}) = \sqrt{\pi}$ et $\Gamma(-\frac{1}{2}) = -2\sqrt{\pi}$. Ici, je trouve une solution diff\'erente de celle propos\'ee par le livre qui est : $\mathrm{T}(E) = \dfrac{\pi}{\alpha}\sqrt{\dfrac{2m}{\lvert E \rvert}}$ et j'avoue ne pas savoir en quoi la p\'eriode ne pourrait pas d\'ependre de la profondeur du puits d'\'energie potentielle $U_{0}$.

\subsubsection{$U = U_{0}\tan^{2}(\alpha x)$}

Tout d'abord quelques rappels sur la fonction \emph{tangente} :
\bea
	\tan\alpha & = & \dfrac{\sin\alpha}{\cos\alpha} = \dfrac{e^{i\alpha} - e^{-i\alpha}}{i(e^{i\alpha} + e^{-i\alpha})} \nonumber \\
	\tan'\alpha & = & \dfrac{1}{\cos^{2}\alpha} = 1 + \tan^{2}\alpha
\eea

\begin{figure}[htb!]
	\begin{center}
		\includegraphics[width=10cm]{chapter_03_exercice_2c}
		\caption{$U = 4\tan^{2}(\alpha x)$ pour $\alpha \in \{0.25,0.35,0.45,0.55,0.65,0.75\}$}\label{FIG:3_2_c}
	\end{center}
\end{figure}

Les points d'arr\^et du mobile sont d\'efinis tels que $U = E = U_{0}\tan^{2}(\alpha x)$, soit $\tan(\alpha x) = \pm\sqrt(E/U_{0})$ :
\be
	\begin{cases}
		x_{1} = -\frac{1}{\alpha}\arctan\left(\sqrt{\dfrac{E}{U_{0}}}\right) \\
		x_{2} = +\frac{1}{\alpha}\arctan\left(\sqrt{\dfrac{E}{U_{0}}}\right)
	\end{cases}
\ee
car la fonction $\arctan$ est impaire. Par cons\'equence, la p\'eriode $\mathrm{T}$ s'\'ecrit \`a partir de l'\'equation (\ref{EQ:11_5}) :
\bea
	\mathrm{T}(E) & = & \sqrt{2m}\int_{-\frac{1}{\alpha}\arctan\left(\sqrt{\frac{E}{U_{0}}}\right)}^{\frac{1}{\alpha}\arctan\left(\sqrt{\frac{E}{U_{0}}}\right)}\dfrac{\mathrm{d}x}{\sqrt{E - U_{0}\tan^{2}(\alpha x)}} \nonumber \\
	& = & 2\sqrt{\dfrac{2m}{E}}\int_{0}^{\frac{1}{\alpha}\arctan\left(\sqrt{\frac{E}{U_{0}}}\right)}\dfrac{\mathrm{d}x}{\sqrt{1 - \frac{U_{0}}{E}\tan^{2}(\alpha x)}}
\eea
car la fonction $\tan^{2}$ est paire.

Toutefois, \`a partir de ce moment l\`a de la d\'emonstration, en posant :
\be
	y = \sqrt{\dfrac{U_{0}}{E}}\tan(\alpha x)
\ee
je n'arrive pas \`a aller jusqu'au bout et trouver le r\'esultat annonc\'e par le livre, \`a savoir :
\be
	\mathrm{T}(E) = \dfrac{\pi\sqrt{2m}}{\alpha\sqrt{(E + U_{0})}}
\ee

\section{Contraction de la fonction de Lagrange pour un syst\`eme de n + 1 particules}

Consid\'erons un syst\`eme ferm\'e compos\'e de $n$ particules de masse $m$ dont chancune a pour rayon vecteur $\vec{R}_{a}$ et d'une particule de masse $M$ qui a pour rayon vecteur $\vec{R}$. Le fonction de Lagrange du syst\`eme s'\'ecrit alors :
\be
	L = \dfrac{M}{2}\vec{\dot{R}}^{\,2} + \sum_{a=1}^{n}\dfrac{m}{2}\vec{\dot{R}}_{a}^{\,2} - U(\vec{R},\begin{Bmatrix}\vec{R}_{a}\end{Bmatrix}_{1}^{n})
\ee
En d\'efinissant $\vec{r}_{a} = \vec{R}_{a} - \vec{R}$ et en appliquant le centre de masse en tant qu'origine du syst\`eme de coordonn\'ees alors $M\vec{R} + \sum_{a}m\vec{R}_{a} = \vec{0}$, cela permet d'avancer :
\bea
	M\vec{R} + m\sum_{a}\vec{R}_{a} & = & \vec{0} \nonumber \\
	M\vec{R} + m\sum_{a}(\vec{r}_{a} - \vec{R}) & = & \vec{0} \nonumber \\
	m\sum_{a}\vec{r}_{a} & = & -(M + nm)\vec{R} \nonumber \\
	\vec{R} & = & \dfrac{-m}{M + nm}\sum_{a}\vec{r}_{a}
\eea
La fonction de Lagrange \'ecrite plus haut peut alors se d\'evelopper en :
\bea
	L + U& = & \dfrac{M}{2}\vec{\dot{R}}^{\,2} + \dfrac{m}{2}\sum_{a}(\vec{\dot{R}}^{\,2} + 2\vec{\dot{r_{a}}}\cdot\vec{\dot{R}} + \vec{\dot{r}}_{a}^{\,2}) \nonumber \\
	& = & \dfrac{Mm^{2}}{2(M + nm)^{2}}\left(\sum_{a}\vec{r}_{a}\right)^{2} + \dfrac{m}{2}\sum_{a}\left(\dfrac{m^{2}}{(M + nm)^{2}}\left(\sum_{a}\vec{r}_{a}\right)^{2} + \vec{\dot{r}}_{a}^{\,2} - 2\vec{\dot{r_{a}}}\dfrac{m}{M + nm}\sum_{a}\vec{r}_{a}\right) \nonumber \\
	& = & \dfrac{Mm^{2}}{2(M + nm)^{2}}\left(\sum_{a}\vec{r}_{a}\right)^{2} + \dfrac{m^{3}}{2(M + nm)^{2}}\left(\sum_{a}\vec{r}_{a}\right)^{2} + \dfrac{m}{2}\vec{\dot{r}}_{a}^{\,2} - \dfrac{m^{2}}{M + nm}\sum_{a}\vec{r}_{a}\cdot\sum_{a}\vec{r}_{a} \nonumber \\
	& = & \dfrac{m^{2}(M + nm)}{2(M + nm)^{2}}\left(\sum_{a}\vec{r}_{a}\right)^{2} + \dfrac{m}{2}\vec{\dot{r}}_{a}^{\,2} - \dfrac{m^{2}}{(M + nm)}\left(\sum_{a}\vec{r}_{a}\right)^{2} \nonumber
\eea
soit :
\be
	L = \dfrac{m}{2}\vec{\dot{r}}_{a}^{\,2} - \dfrac{m^{2}}{2(M + nm)}\left(\sum_{a}\vec{r}_{a}\right)^{2} - U
\ee
Comme $U$ ne d\'epend que de la distance entre toutes les particules, elle peut s'\'ecrire comme une fonction de $\vec{r}_{a}$, la fonction de Lagrange d\'eduite se r\'esoud \`a celle de n particules.

\section{Mouvement dans un champ central}

L'objectif des exercices ci-dessous est de trouver la solution g\'en\'erale par rapport aux situations pos\'ees.

\subsection{Cas du pendule sphérique}

\begin{figure}[htb!]
	\begin{center}
		\begin{picture}(400,200)(0,0)
			%axis
			\linethickness{0.05mm}
			\multiput(50,100)(10,0){9}{\line(1,0){8}}\put(142,98){$x$}
			\multiput(50,100)(0,-10){9}{\line(0,-1){8}}\put(48,0){$y$}
			\put(42,102){$O$}
			\multiput(250,100)(10,0){9}{\line(1,0){8}}\put(341,98){$x$}
			\multiput(250,100)(0,-10){9}{\line(0,-1){8}}\put(248,0){$z$}
			\put(242,102){$O$}
			%circles
			\put(50,100){\color{black}\circle{100}}
			\put(250,100){\color{black}\circle{100}}
			%arm
			\linethickness{0.5mm}
			\put(50,100){\line(10,-10){35}}
			\put(75,80){$l$}
			\put(85,65){\color{black}\circle*{10}}\put(92,62){$m$}
			\put(250,100){\line(10,-10){35}}
			\put(275,80){$l$}
			\put(285,65){\color{black}\circle*{10}}\put(292,62){$m$}
			%angles
			\linethickness{0.05mm}
			\qbezier(50,80),(60,80),(60,90)
			\put(53,70){$\theta$}
			\qbezier(260,90),(270,90),(270,100)
			\put(268,85){$\varphi$}
		\end{picture}
		\caption{Vues de face, \`gauche et du dessus, \`a droite du pendule sph\'erique}\label{FIG:3_EX1}
	\end{center}
\end{figure}

La fonction de Lagrange en coordonn\'ees sph\'eriques s'\'ecrit, voir l'\'equation (\ref{EQ:4_6}), dans notre cas pr\'esent :
\be
	L = \dfrac{m}{2}(\dot{l}^{2} + l^{2}\dot{\theta}^{2} + l^{2}\sin^{2}\theta\dot{\varphi}^{2}) + mgl\cos\theta
\ee
or la distance $l$ est constante dans le mouvement, donc nous pouvons smplifier :
\be
	L = \dfrac{ml^{2}}{2}(\dot{\theta}^{2} + \sin^{2}\theta\dot{\varphi}^{2}) + mgl\cos\theta
\ee
La coordonn\'ee $\varphi$ n'appara\^it pas dans l'espression de la fonction de Lagrange, elle est donc une coordonn\'ee cyclique. Et il y a conservation de l'impulsion g\'en\'eralis\'ee telle que d\'ej\`a vu avec l'\'equation (\ref{EQ:14_2}), soit :
\be
	p_{\varphi} = M_{z} = m(l\sin\theta)^{2}\dot{\varphi} = cste
\ee

L'\'energie de la particule peut donc s'\'ecrire :
\bea
	E &= & T + U \nonumber \\
	& = & \dfrac{ml^{2}}{2}(\dot{\theta}^{2} + \sin^{2}\theta\dot{\varphi}^{2}) - mgl\cos\theta \nonumber \\
	& = & \dfrac{ml^{2}}{2}\dot{\theta}^{2} + \dfrac{ml^{2}}{2}\dfrac{\sin^{2}\theta M_{z}^{2}}{m^{2}l^{4}\sin^{4}\theta} - mgl\cos\theta \nonumber \\
	& = & \dfrac{ml^{2}}{2}\dot{\theta}^{2} + \dfrac{M_{z}^{2}}{2ml^{2}\sin^{2}\theta} - mgl\cos\theta
\eea
et nous pouvons continuer tel que :
\bea
	\dot{\theta} = \dfrac{\mathrm{d}\theta}{\mathrm{dt}} & = & \sqrt{\dfrac{2}{ml^{2}}\left(E - \left(\dfrac{M_{z}^{2}}{2ml^{2}\sin^{2}\theta} - mgl\cos\theta\right)\right)} \nonumber \\
	\Leftrightarrow \mathrm{dt} & = & \dfrac{\mathrm{d}\theta}{\sqrt{\dfrac{2}{ml^{2}}\left(E - U_{eff}(\theta)\right)}} \nonumber \\
	\mathrm{t} & = & \int{\dfrac{\mathrm{d}\theta}{\sqrt{\dfrac{2}{ml^{2}}\left(E - U_{eff}(\theta)\right)}}}
\eea
Nous avons donc la relation entre $\mathrm{r}$ et $\theta$ qui se trouve \^etre une int\'egrale elliptique de premi\`ere esp\`ece d\'efinie ici avec :
\be
	U_{eff}(\theta) = \dfrac{M_{z}^{2}}{2ml^{2}\sin^{2}\theta} - mgl\cos\theta
\ee

Ensuite, nous avons $M_{z} = m(l\sin\theta)^{2}\dot{\varphi}$ qui permet d'arriver \`a :
\be
	\mathrm{dt} = \dfrac{ml^{2}}{M_{z}}\sin^{2}\theta\mathrm{d}\varphi
\ee
Connaissant $\mathrm{dt}$ en fonction de $\theta$, nous pouvons exprimer $\varphi$ telle que :
\bea
	\mathrm{\varphi} & = & \dfrac{M_{z}\mathrm{d}\theta}{ml^{2}\sin^{2}\theta\sqrt{\dfrac{2}{ml^{2}}(E-U_{eff}(\theta))}} \nonumber \\
	& = & \dfrac{M_{z}}{l\sqrt{2m}}\dfrac{\mathrm{d}\theta}{\sin^{2}\theta\sqrt{\dfrac{2}{ml^{2}}(E-U_{eff}(\theta))}} \nonumber \\
	\varphi & = & \dfrac{M_{z}}{l\sqrt{2m}}\int{\dfrac{\mathrm{d}\theta}{\sin^{2}\theta\sqrt{\dfrac{2}{ml^{2}}(E-U_{eff}(\theta))}}}
\eea
Nous avons ici la relation entre $\varphi$ et $\theta$ qui est une int\'egrale elliptique de troisi\`eme esp\`ece. $\theta$ varie telle que $E - U_{eff}(\theta) > 0$ et ses valeurs limites se d\'efinissent telles que $E = U_{eff}(\theta)$, soit :
\bea
	E & = & \dfrac{M_{z}^{2}}{2ml^{2}\sin^{2}\theta} - mgl\cos\theta \nonumber \\
	\Leftrightarrow lE(1-\cos^{2}\theta) & = & \dfrac{M_{z}^{2}}{2ml} - mgl^{2}(1-\cos^{2}\theta)\cos\theta \nonumber \\
	\Leftrightarrow lE - lE\cos^{2}\theta & = & \dfrac{M_{z}^{2}}{2ml} - mgl^{2}\cos\theta + mgl^{2}\cos^{3}\theta \nonumber \\
	\Leftrightarrow mgl^{2}\cos^{3}\theta + lE\cos^{2}\theta - mgl^{2}\cos\theta + \dfrac{M_{z}^{2}}{2ml} - lE & = & 0
\eea
est une \'equation du troisi\`eme degr\'e en $\cos\theta$ qui a implicitement deux racines dans l'intervalle $\left]-1;1\right[$. Il existe donc deux angles $\theta_{min}$ et $\theta_{max}$ qui d\'efinissent deux cercles respectifs entre lesqueles toute la trajectoire de la particule est comprise.

\subsection{Cas du pendule sur un c\^one}

\begin{figure}[htb!]
	\begin{center}
		\begin{picture}(150,300)(0,0)
			%axis
			\linethickness{0.05mm}
			\multiput(75,0)(0,10){25}{\line(0,1){8}}\put(73,253){$y$}
			\put(71,-10){$O$}
			%cone
			\put(75,0){\line(2,4){125}}
			\put(75,0){\line(-2,4){125}}
			\qbezier(75,225),(200,225),(200,250)
			\qbezier(75,275),(200,275),(200,250)
			\qbezier(75,225),(-50,225),(-50,250)
			\qbezier(75,275),(-50,275),(-50,250)
			%circle
			\put(175,198){\color{black}\circle*{10}}\put(182,194){$m$}
			%arm
			\linethickness{0.5mm}
			\put(75,0){\line(2,4){100}}\put(135,100){$r$}
			%angles
			\linethickness{0.05mm}
			\qbezier(75,25),(85,25),(85,20)
			\put(77,27){$\alpha$}
		\end{picture}
		\caption{Trajectoire sur la surface d'un c\^one}\label{FIG:3_EX2}
	\end{center}
\end{figure}

Dans le r\'ef\'erentiel dont l'origine est plac\'e \`a la pointe du c\^one et dont l'axe polaire est point\'e vers le haut et co\"incide avec l'axe du c\^one, la fonction de Lagrange s'\'ecrit naturellement :
\be
	L = \dfrac{m}{2}(\dot{r}^{2} + r^{2}\sin^{2}\alpha\dot{\varphi}^{2}) - mgr\cos\alpha
\ee
La coordonn\'ee $\varphi$ est cyclique et permet d'\'ecrire, comme dans l'exercice pr\'ec\'edent, que :
\be
	p_{\varphi} = M_{z} = mr^{2}\sin^{2}\alpha\dot{\varphi}
\ee
L'\'energie de la particule vaut :
\bea
	E & = & \dfrac{m}{2}(\dot{r}^{2} + r^{2}\sin^{2}\alpha\dot{\varphi}^{2}) + mgr\cos\alpha \nonumber \\
	& = & \dfrac{m}{2}\dot{r}^{2} + \dfrac{m}{2}r^{2}\sin^{2}\alpha\dfrac{M_{z}^{2}}{m^{2}r^{4}\sin^{4}\alpha} + mgr\cos\alpha \nonumber \\
	& = & \dfrac{m}{2}\dot{r}^{2} + \dfrac{M_{z}^{2}}{2mr^{2}\sin^{2}\alpha} + mgr\cos\alpha
\eea
Et par cons\'equent :
\bea
	\dot{r} = \dfrac{\mathrm{d}r}{\mathrm{dt}} & = & \sqrt{\dfrac{2}{m}\left(E - \left(\dfrac{M_{z}^{2}}{2mr^{2}\sin^{2}\alpha} + mgr\cos\alpha\right)\right)} \nonumber \\
	\Leftrightarrow \mathrm{dt} & = & \dfrac{\mathrm{d}r}{\sqrt{\dfrac{2}{m}\left(E - U_{eff}(r)\right)}} \nonumber \\
	\Leftrightarrow \mathrm{t} & = & \int{\dfrac{\mathrm{d}r}{\sqrt{\dfrac{2}{m}\left(E - U_{eff}(r)\right)}}}
\eea
avec :
\be
	U_{eff}(r) = \dfrac{M_{z}^{2}}{2mr^{2}\sin^{2}\alpha} + mgr\cos\alpha
\ee
De plus, comme $M_{z} = mr^{2}\sin^{2}\alpha\dot{\varphi}$, alors :
\be
	\mathrm{dt} = \dfrac{mr^{2}sin^{2}\alpha}{M_{z}}\mathrm{d}\varphi
\ee
Soit :
\bea
	\dfrac{mr^{2}sin^{2}\alpha}{M_{z}}\mathrm{d}\varphi & = & \dfrac{\mathrm{d}r}{\sqrt{\dfrac{2}{m}\left(E - U_{eff}(r)\right)}} \nonumber \\
	\Leftrightarrow \mathrm{d}\varphi & = & \dfrac{M_{z}}{mr^{2}sin^{2}\alpha}\dfrac{\mathrm{d}r}{\sqrt{\dfrac{2}{m}\left(E - U_{eff}(r)\right)}} \nonumber \\
	\mathrm{d}\varphi & = & \dfrac{M_{z}}{\sqrt{2m}\sin^{2}\alpha}\dfrac{\mathrm{d}r}{r^{2}\sqrt{E - U_{eff}(r)}} \nonumber \\
	\varphi & = & \dfrac{M_{z}}{\sqrt{2m}\sin^{2}\alpha}\int{\dfrac{\mathrm{d}r}{r^{2}\sqrt{E - U_{eff}(r)}}}
\eea

Les limites de $r$ sont d\'efinies telles que $E = U_{eff}(r)$, i.e. :
\bea
	E & = & \dfrac{M_{z}^{2}}{2mr^{2}\sin^{2}\alpha} + mgr\cos\alpha \nonumber \\
	\Leftrightarrow  Er^{2} & = & \dfrac{M_{z}^{2}}{2m\sin^{2}\alpha} + mgr^{3}\cos\alpha \nonumber \\
	\Leftrightarrow r^{2}(mgr\cos\alpha  - E) & = & -\dfrac{M_{z}^{2}}{2m\sin^{2}\alpha} < 0
\eea
car $\alpha \in ]0;\frac{\pi}{2}[$. Et comme $r^{2} > 0$, alors c'est la quantit\'e $mgr\cos\alpha  - E$ qui doit \^etre n\'egative, soit $r < \frac{E}{mg\cos\alpha}$. La derni\`ere \'equation est aussi une \'equation du troisi\`eme degr\'e en $r$ qui a au moins deux racines positives. Elles d\'eterminent les deux cercles horizontaux entre lesquels la particule se meut.

\subsection{Cas du pendule plan avec le point de suspension mobile}

Ici, nous reprenons le cas pr\'esent\'e sur la figure (\ref{FIG:1_2}) o\`u la fonction de Lagrange a d\'ej\`a \'et\'e calcul\'ee :
\be
	L = \dfrac{m_{1} + m_{2}}{2}\dot{x}^{2} + \dfrac{m_{2}}{2}\left(l^{2}\dot{\varphi}^{2} + 2l\cos\varphi\dot{x}\dot{\varphi}\right) + m_{2}gl\cos\varphi
\ee
La coordonn\'ee $x$ n'appara\^it pas dans la fonction de Lagrange et est donc une coordonn\'ee cyclique qui permet de conclure la constance dans le mouvement de l'impulsion g\'en\'eralis\'ee $p_{x}$ qui se d\'efinit ainsi\footnote{Dans ce cas, il n'y a pas de rotation donc l\'energie cin\'etique est \'evidemment nulle.} :
\bea
	p_{x} = \dfrac{\partial L}{\partial\dot{x}} & = & (m_{1} + m_{2})\dot{x} + 2l\dfrac{m_{2}}{2}\dot{\varphi}\cos\varphi \nonumber \\
	& = & (m_{1} + m_{2})\dot{x} + m_{2}l\dot{\varphi}\cos\varphi = cste
\eea

Pour le mouvement du centre d'inertie du syst\`eme suivant l'horizontale, il y a $p_{x} = 0$ puisqu'il est immobile, au repos et permet d'\'ecrire :
\be
	\dot{x} = \dfrac{-m_{2}l\dot{\varphi}\cos\varphi}{(m_{1} + m_{2})}
\ee
L'\'energie du syst\`eme est :
\bea
	E & = & \dfrac{(m_{1} + m_{2})}{2}\dfrac{m_{2}^{2}l^{2}\dot{\varphi}^{2}\cos^{2}\varphi}{(m_{1} + m_{2})^{2}} + \dfrac{m_{2}}{2}\left(l^{2}\dot{\varphi}^{2} - \dfrac{2m_{2}l^{2}\cos^{2}\varphi\dot{\varphi}^{2}}{(m_{1} + m_{2})}\right) - m_{2}gl\cos\varphi \nonumber \\
	& = & \dfrac{m_{2}^{2}l^{2}\dot{\varphi}^{2}\cos^{2}\varphi}{2(m_{1} + m_{2})} + \dfrac{m_{2}l^{2}\dot{\varphi}^{2}}{2} - \dfrac{m_{2}^{2}l^{2}\cos^{2}\varphi\dot{\varphi}^{2}}{(m_{1} + m_{2})} - m_{2}gl\cos\varphi \nonumber \\
	& = & \dfrac{m_{2}l^{2}\dot{\varphi}^{2}}{2}\left(1 - \dfrac{m_{2}\cos^{2}\varphi}{(m_{1} + m_{2})}\right) - m_{2}gl\cos\varphi
\eea
et permet de d\'eterminer une solution g\'en\'erale :
\bea
	\Leftrightarrow \dot{\varphi} = \dfrac{\mathrm{d}\varphi}{\mathrm{dt}} & = & \sqrt{\dfrac{2}{m_{2}l^{2}}\dfrac{E + m_{2}gl\cos\varphi}{\left(1 - \dfrac{m_{2}\cos^{2}\varphi}{(m_{1} + m_{2})}\right)}} \nonumber \\
	\Leftrightarrow \mathrm{dt} & = & l\sqrt{\dfrac{m_{2}}{2}}\dfrac{\mathrm{d}\varphi}{\sqrt{\dfrac{(m_{1} + m_{2})(E + m_{2}gl\cos\varphi)}{m_{1} + m_{2} - m_{2} + m_{2}\cos^{2}\varphi}}} \nonumber \\
	& = & l\sqrt{\dfrac{m_{2}}{m_{1} + m_{2}}}\dfrac{m_{1} + m_{2}\cos^{2}\varphi}{\sqrt{E + m_{2}gl\cos\varphi}}\mathrm{d}\varphi \nonumber \\
	\Leftrightarrow \mathrm{t} & = & l\sqrt{\dfrac{m_{2}}{m_{1} + m_{2}}}\int{\dfrac{m_{1} + m_{2}\cos^{2}\varphi}{\sqrt{E + m_{2}gl\cos\varphi}}\mathrm{d}\varphi}
\eea
Remarquons enfin, en int\'egrant $p_{x} = 0 \Rightarrow (m_{1} + m_{2})x + m_{2}l\sin\varphi = K$ avec K une constante par rapport au temps. Or :
\be
	\begin{cases}
		x_{2} = x + l\sin\varphi \\
		y_{2} = l\cos\varphi
	\end{cases}
\ee
soit :
\bea
	\Leftrightarrow	x_{2} & = & \dfrac{K-m_{2}l\sin\varphi}{(m_{1} + m_{2})} + l\sin\varphi \nonumber \\
	& = & \dfrac{K}{(m_{1} + m_{2})} + l\left(1 - \dfrac{m_{2}}{(m_{1} + m_{2})}\right)\sin\varphi \nonumber \\
	& = & \dfrac{K}{(m_{1} + m_{2})} + \dfrac{m_{1}l}{(m_{1} + m_{2})}\sin\varphi
\eea
L'ensemble des positions issues de $x_{2}$ et $y_{2}$ d\'efinissent une portion d'ellipse de demi-axe horizontal $\frac{m_{1}l}{(m_{1} + m_{2})}$ et de demi-axe verticale $l$.

\section{Probl\`eme de Kepler}

\subsection{Repr\'esentation param\'trique pour $U = -\alpha/r$ et $E = 0$}

Nous sommes dans le cas elliptique o\`u $e = 1$ et $p = \dfrac{M^{2}}{m\alpha}$. La relation (\ref{EQ:14_6}) donne dans notre cas pr\'ecis :
\be
	\mathrm{t} = \int{\dfrac{r\mathrm{d}r}{\sqrt{\dfrac{2\alpha}{m}r - \dfrac{M^{2}}{m}}}}
\ee

\subsection{Solution pour $U = -\alpha/r^{2}$ et $\alpha > 0$}

\subsection{Perturbation dans l'\'energie potentielle}
\chapter{Probl\`emes de chocs des particules}

\section{D\'esint\'egrations de deux particules}

\subsection{Relation entre $\theta_{1}$ et $\theta_{2}$ dans <<~l~>>}

L'objectif ici est d'obtenir la relation entre $\theta_{1}$ et $\theta_{2}$ dans le r\'ef\'erentiel <<~l~>> dans le cadre de la d\'esint\'egration d'une particule de vitesse initiale $\vec{V}$ en deux particules r\'esultantes de masse respective $m_{1}$ et $m_{2}$. Dans le r\'ef\'erentiel <<~c~>>, le centre d'inertie est immobile, aussi :
\be
	\vec{R} = \dfrac{m_{1}\vec{r}_{10} + m_{2}\vec{r}_{20}}{m_{1} + m_{2}}
\ee
donne :
\bea
	\dfrac{\mathrm{d}\vec{R}}{\mathrm{dt}} & = & \vec{0} \nonumber \\
	m_{1}\vec{v}_{10} + m_{2}\vec{v}_{20} & = & \vec{0}
\eea
qui permet d'\'ecrire :
\bea
	m_{1}\lVert \vec{v}_{10} \rVert & = & m_{2}\lVert \vec{v}_{20} \rVert \nonumber \\
	\dfrac{m_{1}}{m_{2}} & = & \dfrac{v_{20}}{v_{10}}
\eea
mais \'egalement en utilisant le r\'esultat pr\'ec\'edent :
\bea
	m_{1}^{2}v_{10}^{2} + m_{2}^{2}v_{20}^{2} + 2 m_{1}m_{2}v_{10}v_{20}\cos(\theta_{10} + \theta_{20}) & = & 0 \nonumber \\
	m_{1}^{2}v_{10}^{2} + m_{1}^{2}v_{10}^{2} + 2 m_{1}^{2}v_{10}^{2}\cos(\theta_{10} + \theta_{20}) & = & 0 \nonumber \\
	\cos(\theta_{10} + \theta_{20}) & = & -1 \nonumber \\
	\theta_{10} + \theta_{20} & = & \pi
\eea

Les particules r\'esultantes n'interagissant pas l'une sur l'autre, la relation (\ref{EQ:16_5}) est applicable \`a l'une et l'autre telle que :
\bea
	\tan\theta_{1} = \dfrac{v_{10}\sin\theta_{10}}{V + v_{10}\cos\theta_{10}} & \text{ et } & \tan\theta_{2} = \dfrac{v_{20}\sin\theta_{20}}{V + v_{20}\cos\theta_{20}} \nonumber \\
	v_{10}\cos\theta_{10} + V = \dfrac{v_{10}\sin\theta_{10}}{\tan\theta_{1}} & \text{ et } & v_{20}\cos\theta_{20} + V = \dfrac{v_{20}\sin\theta_{20}}{\tan\theta_{2}} \nonumber \\
	V + v_{10}\cos\theta_{10} = \dfrac{v_{10}\sin\theta_{10}}{\tan\theta_{1}} & \text{ et } & V - v_{20}\cos\theta_{10} = \dfrac{v_{20}\sin\theta_{10}}{\tan\theta_{2}}
\eea
La premi\`ere relation permet d'\'ecrire :
\be
	\cos\theta_{10} = \dfrac{\sin\theta_{10}}{\tan\theta_{1}} - \dfrac{V}{v_{10}}
\ee
qui, inject\'ee dans la seconde :
\bea
	V - \dfrac{v_{20}\sin\theta_{10}}{\tan\theta_{1}} + \dfrac{v_{20}V}{v_{10}} & = & \dfrac{v_{20}\sin\theta_{10}}{\tan\theta_{2}} \nonumber \\
	\left(1 + \dfrac{v_{20}}{v_{10}}\right)V & = & \left(\dfrac{1}{\tan\theta_{1}} + \dfrac{1}{\tan\theta_{2}}\right)v_{20}\sin\theta_{10} \nonumber \\
	\sin\theta_{10} & = & V\dfrac{\frac{1}{v_{10}} + \frac{1}{v_{20}}}{\left(\frac{1}{\tan\theta_{1}} + \frac{1}{\tan\theta_{2}}\right)}
\eea
et par voie de cons\'equence :
\be
	\cos\theta_{10} = V\dfrac{\frac{1}{v_{10}} + \frac{1}{v_{20}}}{\left(1 + \frac{\tan\theta_{1}}{\tan\theta_{2}}\right)} - \dfrac{V}{v_{10}}
\ee
En utilisant la relation bien connue : $\cos^{2} + \sin^{2} = 1$, nous pouvons continuer avec $\theta_{10}$ telle que :
\be
	V^{2}\dfrac{\left(\frac{1}{v_{10}} + \frac{1}{v_{20}}\right)^{2}}{\left(\frac{1}{\tan\theta_{1}} + \frac{1}{\tan\theta_{2}}\right)^{2}} + V^{2}\dfrac{\left(\frac{1}{v_{10}} + \frac{1}{v_{20}}\right)^{2}}{\left(1 + \frac{\tan\theta_{1}}{\tan\theta_{2}}\right)^{2}} + \dfrac{V^{2}}{v_{10}^{2}} - 2\dfrac{V^{2}}{v_{10}}\dfrac{\frac{1}{v_{10}} + \frac{1}{v_{20}}}{\left(\frac{1}{\tan\theta_{1}} + \frac{1}{\tan\theta_{2}}\right)} = 1
\ee
Sachant que :
\be
	\dfrac{1}{v_{10}} + \dfrac{1}{v_{20}} = \frac{1}{v_{10}}\left( 1 + \frac{v_{10}}{v_{20}}\right) =  \frac{1}{v_{10}}\left( 1 + \frac{m_{2}}{m_{1}}\right)
\ee
l'\'equation devient :
\bea
	\dfrac{V^{2}}{v_{10}^{2}}\dfrac{\left(1 + \frac{m_{2}}{m_{1}}\right)^{2}}{\left(\frac{1}{\tan\theta_{1}} + \frac{1}{\tan\theta_{2}}\right)^{2}} + \dfrac{V^{2}}{v_{10}^{2}}\dfrac{\left(1 + \frac{m_{2}}{m_{1}}\right)^{2}}{\left(1 + \frac{\tan\theta_{1}}{\tan\theta_{2}}\right)^{2}} + \dfrac{V^{2}}{v_{10}^{2}} - 2\dfrac{V^{2}}{v_{10}^{2}}\dfrac{1 + \frac{m_{2}}{m_{1}}}{\left(\frac{1}{\tan\theta_{1}} + \frac{1}{\tan\theta_{2}}\right)} & = & 1 \nonumber \\
	\dfrac{(m_{1} + m_{2})^{2}}{m_{1}^{2}\left(\frac{1}{\tan\theta_{1}} + \frac{1}{\tan\theta_{2}}\right)^{2}} + \dfrac{(m_{1} + m_{2})^{2}}{m_{1}^{2}\left(1 + \frac{\tan\theta_{1}}{\tan\theta_{2}}\right)^{2}} - 2\dfrac{(m_{1} + m_{2})}{m_{1}\left(\frac{1}{\tan\theta_{1}} + \frac{1}{\tan\theta_{2}}\right)} & = & \dfrac{v_{10}^{2}}{V^{2}} - 1 \nonumber \\
	\dfrac{m_{1} + m_{2}}{m_{1}\left(\frac{1}{\tan\theta_{1}} + \frac{1}{\tan\theta_{2}}\right)^{2}} + \dfrac{m_{1} + m_{2}}{m_{1}\left(1 + \frac{\tan\theta_{1}}{\tan\theta_{2}}\right)^{2}} - \dfrac{2}{\left(\frac{1}{\tan\theta_{1}} + \frac{1}{\tan\theta_{2}}\right)} & = & \dfrac{(v_{10}^{2} - V^{2})m_{1}}{(m_{1} + m_{2})V^{2}} \nonumber \\
\eea
Or il convient de d\'evelopper :
\be
	\dfrac{1}{\tan\theta_{1}} + \dfrac{1}{\tan\theta_{2}} = \dfrac{\cos\theta_{1}}{\sin\theta_{1}} + \dfrac{\cos\theta_{2}}{\sin\theta_{2}} = \dfrac{\cos\theta_{1}\sin\theta_{2} + \sin\theta_{1}\cos\theta_{2}}{\sin\theta_{1}\sin\theta_{2}} = \dfrac{\sin(\theta_{1} + \theta_{2})}{\sin\theta_{1}\sin\theta_{2}}
\ee
et
\bea
	1 + \dfrac{\tan\theta_{1}}{\tan\theta_{2}} & = & \dfrac{\tan\theta_{1} + \tan\theta_{2}}{\tan\theta_{2}} = \left(\dfrac{\cos\theta_{1}}{\sin\theta_{1}} + \dfrac{\cos\theta_{2}}{\sin\theta_{2}}\right)\dfrac{\cos\theta_{2}}{\sin\theta_{2}} \nonumber \\
	& = & \dfrac{(\cos\theta_{1}\sin\theta_{2} + \sin\theta_{1}\cos\theta_{2})\cos\theta_{2}}{\cos\theta_{1}\cos\theta_{2}\sin\theta_{2}} = \dfrac{\sin(\theta_{1} + \theta_{2})}{\cos\theta_{1}\sin\theta_{2}}
\eea
L'\'equation principale devient alors :
\be
	\dfrac{(v_{10}^{2} - V^{2})m_{1}}{(m_{1} + m_{2})V^{2}}\sin^{2}(\theta_{1} + \theta_{2}) = \dfrac{m_{1} + m_{2}}{m_{1}}\sin^{2}\theta_{2} - 2\cos\theta_{1}\sin\theta_{2}\sin(\theta_{1} + \theta_{2})
\ee
Toutefois :
\bea
	\cos\theta_{1}\sin\theta_{2}\sin(\theta_{1} + \theta_{2}) & = & \cos\theta_{1}\sin\theta_{2}(\cos\theta_{1}\sin\theta_{2} + \sin\theta_{1}\cos\theta_{2}) \nonumber \\
	& = & \cos^{2}\theta_{1}\sin^{2}\theta_{2} + \cos\theta_{1}\sin\theta_{1}\cos\theta_{2}\sin\theta_{2} \nonumber \\
	& = & \sin^{2}\theta_{2} - \sin^{2}\theta_{1}\sin^{2}\theta_{2} + \cos\theta_{1}\sin\theta_{1}\cos\theta_{2}\sin\theta_{2} \nonumber \\
	& = & \sin^{2}\theta_{2} - \sin\theta_{1}\sin\theta_{2}(\sin\theta_{1}\sin\theta_{2} - \cos\theta_{1}\cos\theta_{2}) \nonumber \\
	& = & \sin^{2}\theta_{2} + \sin\theta_{1}\sin\theta_{2}\cos(\theta_{1} + \theta_{2})
\eea
ce qui permet d'avancer ainsi :
\bea
	\dfrac{(v_{10}^{2} - V^{2})m_{1}}{(m_{1} + m_{2})V^{2}}\sin^{2}(\theta_{1} + \theta_{2}) & = & \dfrac{m_{1} + m_{2}}{m_{1}}\sin^{2}\theta_{2} - 2\sin^{2}\theta_{2} - 2\sin\theta_{1}\sin\theta_{2}\cos(\theta_{1} + \theta_{2})\nonumber \\
	& = & \dfrac{m_{2}}{m_{1}}\sin^{2}\theta_{2} + \sin^{2}\theta_{2} - 2\sin^{2}\theta_{2} - 2\sin\theta_{1}\sin\theta_{2}\cos(\theta_{1} + \theta_{2})\nonumber \\
	& = & \dfrac{m_{2}}{m_{1}}\sin^{2}\theta_{2} - \sin^{2}\theta_{2} - 2\sin\theta_{1}\sin\theta_{2}\cos(\theta_{1} + \theta_{2})
\eea

\subsection{Distribution des directions dans <<~l~>>}

Pour \'etablie la distribution des directions des particules r\'esultantes dans le r\'ef\'erentiel <<~l~>>, nous allons repartir de l'\'equation (\ref{EQ:16_6}) qui apporte la solution pour deux domaines diff\'erents.

\subsubsection{$v_{0} > V$}

Dans ce cas, $\theta \in [0;\pi]$ \`a la vue de la figure (\ref{FIG:4_14}) et l'\'equation (\ref{EQ:16_6}) s'\'ecrit :
\be
	\cos\theta_{0} = -\dfrac{V}{v_{0}}\sin^{2}\theta + \cos\theta\sqrt{1 - \dfrac{V^{2}}{v_{0}^{2}}\sin^{2}\theta}
\ee
La distribution $\dfrac{\mathrm{d}\omega_{0}}{4\pi}$ vaut $\dfrac{1}{2}\sin\theta_{0}\mathrm{d}\theta_{0} = -\dfrac{\mathrm{d}(\cos\theta_{0})}{2}$ selon l'\'equation (\ref{EQ:16_7}). Cela permet donc d'\'ecrire avec la relation pr\'ec\'edente :
\bea
	\begin{Bmatrix}\dfrac{\mathrm{d}\omega_{0}}{4\pi}\end{Bmatrix}_{1} & = & \dfrac{V}{v_{0}}\sin\theta\cos\theta\mathrm{d}\theta + \dfrac{\sin\theta}{2}\sqrt{1 - \dfrac{V^{2}}{v_{0}^{2}}\sin^{2}\theta}\mathrm{d}\theta - \dfrac{\cos\theta}{4}\dfrac{-2\dfrac{V^{2}}{v_{0}^{2}}\sin\theta\cos\theta\mathrm{d}\theta}{\sqrt{1 - \dfrac{V^{2}}{v_{0}^{2}}\sin^{2}\theta}} \nonumber \\
	& = & \dfrac{V}{v_{0}}\sin\theta\cos\theta\mathrm{d}\theta + \dfrac{\mathrm{d}\theta}{2\sqrt{1 - \dfrac{V^{2}}{v_{0}^{2}}\sin^{2}\theta}}\left(\sin\theta - \dfrac{V^{2}}{v_{0}^{2}}\sin^{3}\theta + \dfrac{V^{2}}{v_{0}^{2}}\sin\theta\cos^{2}\theta\right) \nonumber \\
	& = & \dfrac{1}{2}\sin\theta\mathrm{d}\theta\left[2\dfrac{V}{v_{0}}\cos\theta + \dfrac{V^{2}}{v_{0}^{2}\sqrt{1 - \dfrac{V^{2}}{v_{0}^{2}}\sin^{2}\theta}}(\dfrac{v_{0}^{2}}{V^{2}} + \cos^{2}\theta - \sin^{2}\theta)\right] \nonumber \\
	& = & \dfrac{1}{2}\sin\theta\left[2\dfrac{V}{v_{0}}\cos\theta + \dfrac{1+\dfrac{V^{2}}{v_{0}^{2}}\cos(2\theta)}{\sqrt{1 - \dfrac{V^{2}}{v_{0}^{2}}\sin^{2}\theta}}\right]\mathrm{d}\theta\label{EQ:16_EX2A}
\eea

\begin{figure}[htb!]
	\begin{center}
		\includegraphics[width=10cm]{chapter_04_paragraph_16_exercice_2a}
		\caption{Exemples de distribution pour diff\'erentes valeurs de $\frac{V}{v_{0}}$, de 0.1 à 0.99, par int\'egration de l'\'equation (\ref{EQ:16_EX2A})}\label{FIG:4_16_EX2A}
	\end{center}
\end{figure}

\subsubsection{$v_{0} < V$}

Dans ce cas, la m\^eme s\'equence calculatoire am\`ene \`a \'ecrire pour $\theta \in [0;\theta_{max}]$ :
\be
	\begin{Bmatrix}\dfrac{\mathrm{d}\omega_{0}}{4\pi}\end{Bmatrix}_{2} = \dfrac{1}{2}\sin\theta\left[2\dfrac{V}{v_{0}}\cos\theta - \dfrac{1 + \dfrac{V^{2}}{v_{0}^{2}}\cos(2\theta)}{\sqrt{1 - \dfrac{V^{2}}{v_{0}^{2}}\sin^{2}\theta}}\right]\mathrm{d}\theta
\ee
sachant qu'il faut prendre cette fois-ci dans l'\'equation (\ref{EQ:16_6}) les deux solutions, celles avec le signe - et celle avec le signe +, d\'eduite pr\'ec\`edemment. Ainsi la distribution totale pour $v_{0} < V$ s'obtient en faisant :
\be
	\begin{Bmatrix}\dfrac{\mathrm{d}\omega_{0}}{4\pi}\end{Bmatrix}_{1} - \begin{Bmatrix}\dfrac{\mathrm{d}\omega_{0}}{4\pi}\end{Bmatrix}_{2} = \dfrac{1 + \dfrac{V^{2}}{v_{0}^{2}}\cos(2\theta)}{\sqrt{1 - \dfrac{V^{2}}{v_{0}^{2}}\sin^{2}\theta}}\sin\theta\mathrm{d}\theta\label{EQ:16_EX2B}
\ee

\begin{figure}[htb!]
	\begin{center}
		\includegraphics[width=10cm]{chapter_04_paragraph_16_exercice_2b}
		\caption{Exemples de distribution pour diff\'erentes valeurs de $\frac{V}{v_{0}}$, de 1.5 à 10, par int\'egration de l'\'equation (\ref{EQ:16_EX2B})}\label{FIG:4_16_EX2B}
	\end{center}
\end{figure}

\subsection{Intervalles angulaires dans <<~l~>>}

Dans le r\'ef\'erentiel <<~l~>>, d\'eterminons l'angle $\theta$ d\'efini comme la somme $\theta_{1} + \theta_{2}$ :
\bea
	\tan\theta & = & \tan(\theta_{1} + \theta_{2}) = \dfrac{\tan\theta_{1} + \tan\theta_{2}}{1 - \tan\theta_{1}\tan\theta_{2}} \nonumber \\
	& = & \dfrac{\dfrac{v_{10}\sin\theta_{10}}{V + v_{10}\cos\theta_{10}} + \dfrac{v_{20}\sin\theta_{20}}{V + v_{20}\cos\theta_{20}}}{1 - \dfrac{v_{10}\sin\theta_{10}}{V + v_{10}\cos\theta_{10}}\dfrac{v_{20}\sin\theta_{20}}{V + v_{20}\cos\theta_{20}}} \nonumber \\
\eea
or $\theta_{20} = \pi - \theta_{10}$, donc :
\bea
	\tan\theta & = & \dfrac{\dfrac{v_{10}\sin\theta_{10}}{V + v_{10}\cos\theta_{10}} + \dfrac{v_{20}\sin\theta_{10}}{V - v_{20}\cos\theta_{10}}}{1 - \dfrac{v_{10}\sin\theta_{10}}{V + v_{10}\cos\theta_{10}}\dfrac{v_{20}\sin\theta_{10}}{V - v_{20}\cos\theta_{10}}} \nonumber \\
	& = & \dfrac{v_{10}\sin\theta_{10}(V - v_{20}\cos\theta_{10}) + v_{20}\sin\theta_{10}(v_{10}\cos\theta_{10} + V)}{(V + v_{10}\cos\theta_{10})(V - v_{20}\cos\theta_{10}) - v_{10}v_{20}\sin^{2}\theta_{10}} \nonumber \\
	& = & \dfrac{v_{10}V\sin\theta_{10} - v_{10}v_{20}\cos\theta_{10}\sin\theta_{10} + v_{10}v_{20}\cos\theta_{10}\sin\theta_{10} + v_{20}V\sin\theta_{10}}{v_{10}V\cos\theta_{10} - v_{10}v_{20}\cos^{2}\theta_{10} + V^{2} - v_{20}V\cos\theta_{10} - v_{10}v_{20}\sin^{2}\theta_{10}} \nonumber \\
	& = & \dfrac{(v_{10} + v_{20})V\sin\theta_{10}}{V^{2} - v_{10}v_{20} + (v_{10} - v_{20})V\cos\theta_{10}}
\eea

\section{Chocs \'elastiques de deux particules}

Dans le r\'ef\'erentiel <<~l~>>, exprimons les vitesses apr\`es le choc $v'_{1}$ et $v'_{2}$ en fonction de l'angle de d\'eviation sous lequel elles s'\'ecartent, sachant qu'avant le choc $v_{2} = 0$ et donc $v_{1} = v$.

Puisque la grandeur $2OB$ est un diam\`etre du cercle de rayon $mv$, nous avons $p'_{2} = 2OB\cos\theta_{2}$, donc :
\be
	v'_{2} = \dfrac{2OB}{m_{2}}\cos\theta_{2} = \dfrac{2mv}{m_{2}}\cos\theta_{2}
\ee
avec $m$ la masse r\'eduite du syst\`eme compos\'e des deux particules.

Ensuite, nous utilisons la formule d'Al-Kashi dans le triangle form\'e des points $A$, $O$ et $C$, voir la figure (\ref{FIG:4_16}) pour faire intervenir l'angle $\theta_{1}$ tel que :
\bea
	OC^{2} & = & AO^{2} + {p'}_{1}^{2} - 2 AOp'_{1}\cos\theta_{1} \nonumber \\
	m^{2}v^{2} & = & \dfrac{m_{1}^{2}}{m_{2}^{2}}OB^{2} + m_{1}^{2}{v'}_{1}^{2} - 2\dfrac{m_{1}^{2}}{m_{2}}OBv'_{1}\cos\theta_{1} \nonumber \\
	\left(1 - \dfrac{m_{1}^{2}}{m_{2}^{2}}\right)m^{2}v^{2} & = & m_{1}^{2}{v'}_{1}^{2} - 2\dfrac{m_{1}^{2}}{m_{2}}mvv'_{1}\cos\theta_{1} \nonumber \\
	\dfrac{(m_{2} + m_{1})(m_{2} - m_{1})}{m_{2}^{2}}m^{2} & = & m_{1}^{2}\dfrac{v'_{1}}{v}^{2} - 2\dfrac{m_{1}^{2}}{m_{2}}m\dfrac{v'_{1}}{v}\cos\theta_{1} \nonumber \\
	\dfrac{(m_{2} + m_{1})(m_{2} - m_{1})m_{1}^{2}m_{2}^{2}}{m_{2}^{2}(m_{1} + m_{2})} & = & m_{1}^{2}\dfrac{v'_{1}}{v}^{2} - 2\dfrac{m_{1}^{2}}{m_{2}}m\dfrac{v'_{1}}{v}\cos\theta_{1} \nonumber \\
	0 & = & \left(\dfrac{v'_{1}}{v}\right)^{2} - 2\dfrac{m}{m_{2}}\dfrac{v'_{1}}{v}\cos\theta_{1} + \dfrac{m_{1} - m_{2}}{m_{1} + m_{2}}
\eea

Il s'agit d'une \'equation du second degr\'e en $v'_{1}/v$ dont les solutions sont :
\bea
	\dfrac{v'_{1}}{v} & = & \dfrac{1}{2}\left(\dfrac{2m_{1}m_{2}}{m_{2}(m_{1} + m_{2})}\cos\theta_{1} \pm \sqrt{\dfrac{4m_{1}^{2}m_{2}^{2}}{m_{2}^{2}(m_{1} + m_{2})^{2}}\cos^{2}\theta_{1} - \dfrac{4(m_{1} - m_{2})}{m_{2}(m_{1} + m_{2})}}\right) \nonumber \\
	& = & \dfrac{m_{1}}{m_{1} + m_{2}}\cos\theta_{1} \pm \sqrt{\dfrac{m_{1}^{2}}{(m_{1} + m_{2})^{2}}\cos^{2}\theta_{1} - \dfrac{m_{1}^{2} - m_{2}^{2}}{(m_{1} + m_{2})^{2}}} \nonumber \\
	& = & \dfrac{m_{1}}{m_{1} + m_{2}}\cos\theta_{1} \pm \dfrac{1}{m_{1} + m_{2}}\sqrt{m_{2}^{2} + (\cos^{2}\theta_{1} - 1)m_{1}^{2}} \nonumber \\
	& = & \dfrac{m_{1}}{m_{1} + m_{2}}\cos\theta_{1} \pm \dfrac{m_{2}}{m_{1} + m_{2}}\sqrt{1 + \dfrac{m_{1}^{2}}{m_{2}^{2}}\sin^{2}\theta_{1}}
\eea
De mani\`ere \'equivalente \`a l'\'equation (\ref{EQ:16_6}), pour $m_{1} > m_{2}$, la solution est univoque avec le signe $+$ alors que pour $m_{1} < m_{2}$, les deux solutions sont admises. La repr\'esentation de ces solutions est \'egalement similaire \`a ce qui est repr\'esent\'e sur les figures (\ref{FIG:4_14A}) et (\ref{FIG:4_14B}).

\section{Diffusion des particules}

\subsection{Diffusion par une bille solide}\label{PAR:18EX}

\begin{figure}[htb!]
	\begin{center}
		\begin{picture}(300,200)(0,0)
			%circle
			\linethickness{0.05mm}
			\put(50,50){\circle{100}}
			%dashed lines
			\linethickness{0.05mm}
			\multiput(50,50)(10,0){25}{\line(1,0){8}}
			\put(65,75){$a$}
			\multiput(50,50)(10,10){10}{\line(1,1){8}}
			%angles
			\qbezier(60,50)(60,55)(55,55)
			\put(60,57){$\varphi_{0}$}
			%vectors
			\put(150,73){\vector(0,1){12}}
			\put(150,63){\vector(0,-1){13}}
			\put(147,66){$\rho$}
			%trajectories
			\linethickness{0.5mm}
			\put(85,85){\line(1,0){220}}
			\put(85,85){\line(2,5){40}}
		\end{picture}
		\caption{Diffusion par une bille solide de rayon $a$}\label{FIG:4_18_EX1}
	\end{center}
\end{figure}

L'objectif est de calculer la section efficace pour la diffusion de particules par une bille de rayon $a$ parfatiement solide, i.e. $U(r>a) = 0$ et $U(r<a) = \infty$. La figure (\ref{FIG:4_18_EX1}) illustre la trajectoire, \`a savoir que les particules incidentes se meuvent librement avant et apr\`es le choc, qui se compose de deux droites. D'apr\`es la m\^eme figure, nous pouvons \'etablir que :
\bea
	\rho & = & a\sin\varphi_{0} = a\sin\left(\dfrac{\pi - \xi}{2}\right) = a\left(\cos\dfrac{\pi}{2}\sin\dfrac{-\xi}{2} + \sin\dfrac{\pi}{2}\cos\dfrac{-\xi}{2}\right) \nonumber \\
	& = & a\cos\dfrac{\xi}{2}
\eea
En reportant dans l'\'equation (\ref{EQ:18_7}), cela donne :
\be
	\mathrm{d}\sigma = 2\pi a\cos\dfrac{\xi}{2}\lvert -\dfrac{a}{2}\sin\dfrac{\xi}{2} \rvert \mathrm{d}\xi = \pi a^{2}\cos\dfrac{\xi}{2}\sin\dfrac{\xi}{2}\mathrm{d}\xi
\ee
or $\sin\xi = \sin(\frac{\xi}{2} + \frac{\xi}{2}) = 2\cos\frac{\xi}{2}\sin\frac{\xi}{2}$, donc :
\be
	\mathrm{d}\sigma = \dfrac{\pi a^{2}}{2}\sin\xi\mathrm{d}\xi
\ee
et puisque l'angle solide se d\'efinit pour un anneau comme $\mathrm{d}\omega = 2\pi\sin\xi\mathrm{d}\xi$, donc :
\be
	\mathrm{d}\sigma = \dfrac{a^{2}}{4}\mathrm{d}\omega
\ee
Dans le r\'ef\'erentiel <<~c~>>, la distribution est donc isotrope, i.e. ne d\'epend pas de l'angle de d\'eviation, puisque :
\be
	\sigma = \int_{0}^{4\pi}\dfrac{a^{2}}{4}\mathrm{d}\omega = \pi a^{2}
\ee
et la section efficace est \'egale \`a la section de la bille et assez logiquement, une particule pour \^etre diffus\'ee doit avoir une distance de vis\'ee inf\'erieure au rayon\footnote{Les particules incdents sont consid\'er\'ees comme ponctuelles.} de la bille $a$. Pour d\'eterminer la section efficace $\mathrm{d}\sigma$ dans le r\'ef\'erentiel <<~l~>>, nous pouvons repartir des \'equations (\ref{EQ:17_4}) o\`u $m_{1}$ est la masse de la particule incidente et $m_{2}$ celle de la bille. Nous avons vu plus haut que :
\be
	\mathrm{d}\sigma = \dfrac{\pi a^{2}}{2}\sin\xi\mathrm{d}\xi = -\dfrac{\pi a^{2}}{2}\mathrm{d}(\cos\xi)
\ee
La quantit\'e $\cos\xi$ est reprise de l'\'equation (\ref{EQ:18_9}) et il y a trois cas diiff\'erents.

\subsubsection{$m_{1} < m_{2}$}

Dans ce cas, $\cos\xi$ s'\'ecrit :
\be
	\cos\xi = -\dfrac{m_{1}}{m_{2}}\sin^{2}\theta_{1} + \cos\theta_{1}\sqrt{1 - \dfrac{m_{1}^{2}}{m_{2}^{2}}\sin^{2}\theta_{1}}
\ee
aussi :
\bea
	\mathrm{d}(\cos\xi) & = & -2\dfrac{m_{1}}{m_{2}}\cos\theta_{1}\sin\theta_{1}\mathrm{d}\theta_{1} - \sin\theta_{1}\mathrm{d}\theta_{1}\sqrt{1 - \dfrac{m_{1}^{2}}{m_{2}^{2}}\sin^{2}\theta_{1}} + \dfrac{1}{2}\cos\theta_{1}\dfrac{-2\dfrac{m_{1}^{2}}{m_{2}^{2}}\cos\theta_{1}\sin\theta_{1}\mathrm{d}\theta_{1}}{\sqrt{1 - \dfrac{m_{1}^{2}}{m_{2}^{2}}\sin^{2}\theta_{1}}} \nonumber \\
	& = & -\sin\theta_{1}\mathrm{d}\theta_{1}\left(2\dfrac{m_{1}}{m_{2}}\cos\theta_{1} + \sqrt{1 - \dfrac{m_{1}^{2}}{m_{2}^{2}}\sin^{2}\theta_{1}} + \dfrac{\dfrac{m_{1}^{2}}{m_{2}^{2}}\cos^{2}\theta_{1}}{\sqrt{1 - \dfrac{m_{1}^{2}}{m_{2}^{2}}\sin^{2}\theta_{1}}}\right) \nonumber \\
	& = & -\sin\theta_{1}\mathrm{d}\theta_{1}\left(2\dfrac{m_{1}}{m_{2}}\cos\theta_{1} + \dfrac{1}{\sqrt{1 - \dfrac{m_{1}^{2}}{m_{2}^{2}}\sin^{2}\theta_{1}}}\left(1 - \dfrac{m_{1}^{2}}{m_{2}^{2}}\sin^{2}\theta_{1} + \dfrac{m_{1}^{2}}{m_{2}^{2}}\cos^{2}\theta_{1}\right)\right) \nonumber \\
	& = & -\sin\theta_{1}\mathrm{d}\theta_{1}\left(2\dfrac{m_{1}}{m_{2}}\cos\theta_{1} + \dfrac{1}{\sqrt{1 - \dfrac{m_{1}^{2}}{m_{2}^{2}}\sin^{2}\theta_{1}}}\left(1 + \dfrac{m_{1}^{2}}{m_{2}^{2}}(\cos^{2}\theta_{1} - \sin^{2}\theta_{1})\right)\right) \nonumber \\
	& = & -\sin\theta_{1}\mathrm{d}\theta_{1}\left(2\dfrac{m_{1}}{m_{2}}\cos\theta_{1} + \dfrac{1}{\sqrt{1 - \dfrac{m_{1}^{2}}{m_{2}^{2}}\sin^{2}\theta_{1}}}\left(1 + \dfrac{m_{1}^{2}}{m_{2}^{2}}\cos(2\theta_{1})\right)\right)
\eea
donc :
\be
	\mathrm{d}\sigma_{1} = \dfrac{\pi a^{2}}{2}\left(2\dfrac{m_{1}}{m_{2}}\cos\theta_{1} + \dfrac{1 + \dfrac{m_{1}^{2}}{m_{2}^{2}}\cos(2\theta_{1})}{\sqrt{1 - \dfrac{m_{1}^{2}}{m_{2}^{2}}\sin^{2}\theta_{1}}}\right)\sin\theta_{1}\mathrm{d}\theta_{1}
\ee
et puisque $\mathrm{d}\omega_{1} = 2\pi\sin\theta_{1}\mathrm{d}\theta_{1}$, nous avons finalement :
\be
	\mathrm{d}\sigma_{1+} = \dfrac{a^{2}}{4}\left(2\dfrac{m_{1}}{m_{2}}\cos\theta_{1} + \dfrac{1 + \dfrac{m_{1}^{2}}{m_{2}^{2}}\cos(2\theta_{1})}{\sqrt{1 - \dfrac{m_{1}^{2}}{m_{2}^{2}}\sin^{2}\theta_{1}}}\right)\mathrm{d}\omega_{1}
\ee

\subsubsection{$m_{1} > m_{2}$}

Dans ce cas, il nous faut prendre les deux solutions de l'\'equation (\ref{EQ:18_9}). Nous pouvons reprendre celle avec le signe $+$ de la pr\'ec\'edente section et calculons la section efficace pour la seconde solution de $\cos\xi$, celle qui s'\'ecrit ainsi :
\be
	\cos\xi = -\dfrac{m_{1}}{m_{2}}\sin^{2}\theta_{1} - \cos\theta_{1}\sqrt{1 - \dfrac{m_{1}^{2}}{m_{2}^{2}}\sin^{2}\theta_{1}}
\ee
En appliquant les m\^emes calculs, nous arrivons \`a :
\be
	\mathrm{d}\sigma_{1-} = \dfrac{a^{2}}{4}\left(2\dfrac{m_{1}}{m_{2}}\cos\theta_{1} - \dfrac{1 + \dfrac{m_{1}^{2}}{m_{2}^{2}}\cos(2\theta_{1})}{\sqrt{1 - \dfrac{m_{1}^{2}}{m_{2}^{2}}\sin^{2}\theta_{1}}}\right)\mathrm{d}\omega_{1}
\ee
En d\'efinitive, nous obtenons :
\be
	\mathrm{d}\sigma_{1} = \mathrm{d}\sigma_{1+} - \mathrm{d}\sigma_{1-} = \dfrac{a^{2}}{2}\dfrac{1 + \dfrac{m_{1}^{2}}{m_{2}^{2}}\cos(2\theta_{1})}{\sqrt{1 - \dfrac{m_{1}^{2}}{m_{2}^{2}}\sin^{2}\theta_{1}}}\mathrm{d}\omega_{1}
\ee

\subsubsection{$m_{1} = m_{2}$}

Dans ce cas pr\'ecis, nous en d\'eduisons :
\bea
	\mathrm{d}\sigma_{1} & = & \dfrac{a^{2}}{2}\dfrac{1 + \cos(2\theta_{1})}{\sqrt{1 - \sin^{2}\theta_{1}}}\mathrm{d}\omega_{1} = \dfrac{a^{2}}{2}\dfrac{1 + \cos^{2}\theta_{1} - \sin^{2}\theta_{1}}{\lvert \cos\theta_{1} \rvert}\mathrm{d}\omega_{1} = \dfrac{a^{2}}{2}\dfrac{1 + \cos^{2}\theta_{1} - 1 + \cos^{2}\theta_{1}}{\lvert \cos\theta_{1} \rvert}\mathrm{d}\omega_{1} \nonumber \\
	& = & a^{2}\lvert \cos\theta_{1} \rvert\mathrm{d}\omega_{1}
\eea
$\cos\theta_{1}$ n'est pas forc\`ement positif, aussi il nous faut utiliser sa valeur absolue car la section efficace reste une surface et donc a une valeur positive.

Enfin, dans le cas o\`u initialement, i.e. avant le choc, la bille de masse $m_{2}$ est immobile, comme $\xi = \pi - 2\theta_{2}$, alors :
\bea
	\mathrm{d}\sigma_{2} & = & \dfrac{\pi a^{2}}{2}\sin(\pi - 2\theta_{2})\mathrm{d}(\pi - 2\theta_{2}) = \dfrac{\pi a^{2}}{2}\sin(2\theta_{2})\cdot -2\mathrm{d}(\theta_{2}) = -\pi a^{2}\sin(2\theta_{2})\mathrm{d}(\theta_{2}) \nonumber \\
	& = & -2\pi a^{2}\cos\theta_{2}\sin\theta_{2}\mathrm{d}\theta_{2}
\eea
or comme $\mathrm{d}\omega_{2} = 2\pi\sin\theta_{2}\mathrm{d}\theta_{2}$ et $\mathrm{d}\sigma_{2} > 0$, nous avons finalement :
\be
	\mathrm{d}\sigma_{2} = a^{2}\lvert \cos\theta_{2} \rvert\mathrm{d}\omega_{2}
\ee

\subsection{\'Energie perdue}

Il s'agit de calculer la section efficace de diffusion en fonction de l'\'energie perdue $\epsilon$ par les particules diffus\'ees de masse $m_{1}$ dans le m\^eme cas que dans le paragraphe (\ref{PAR:18EX}). La loi de conservation de l'\'energie du syst\`eme ferm\'e implique que $\epsilon$ vaut exactement l'\'energie gagn\'ee par $m_{2}$. Cette derni\`ere est l'\'energie cin\'etique gagn\'ee par $m_{2}$ apr\`es le choc car dans le cas pr\'esent, avant le choc, $v_{2} = 0$. L'utilisation de l'\'equation (\ref{EQ:17_5B}) permet d'\'ecrire :
\be
	v'_{2} = \dfrac{2m_{1}v}{m_{1} + m_{2}}\sin\left(\dfrac{\xi}{2}\right) = \dfrac{2m_{1}v_{\infty}}{m_{1} + m_{2}}\sin\left(\dfrac{\xi}{2}\right)
\ee
car $v = v_{1} = v_{\infty}$ puisque $v_{1}$ est la vitesse avant le choc de la particule de masse $m_{1}$. Donc l'\'energie de la particule de masse $m_{2}$ apr\`es le choc vaut :
\bea
	E'_{2} & = & \dfrac{m_{2}}{2}{v'}_{2}^{2} = \dfrac{m_{2}}{2}\dfrac{m_{1}^{2}v_{\infty}^{2}}{(m_{1} + m_{2})^{2}}\sin^{2}\left(\dfrac{\xi}{2}\right) \nonumber \\
	\Leftrightarrow \epsilon & = & \dfrac{2m_{1}^{2}m_{2}}{(m_{1} + m_{2})^{2}}v_{\infty}^{2}\sin^{2}\left(\dfrac{\xi}{2}\right)
\eea
Par d\'efinition, $\sin^{2}(\frac{\xi}{2}) \leq 1$, nous pouvons donc en d\'eduire que l'\'energie perdue maximale $\epsilon_{max}$ vaut pr\'ecis\`ement :
\be
	\epsilon_{max} = \dfrac{2m_{1}^{2}m_{2}}{(m_{1} + m_{2})^{2}}v_{\infty}^{2}
\ee
telle que :
\be
	\epsilon = \epsilon_{max}\sin^{2}\left(\dfrac{\xi}{2}\right)
\ee
En d\'erivant, nous obtenons :
\be
	\mathrm{d}\epsilon = \epsilon_{max}\sin\left(\dfrac{\xi}{2}\right)\cos\left(\dfrac{\xi}{2}\right)\mathrm{d}\xi = \dfrac{1}{2}\epsilon_{max}\sin\xi\mathrm{d}\xi
\ee
car $\sin\xi = 2\sin(\frac{\xi}{2})\cos(\frac{\xi}{2})$. Nous savons que :
\be
	\mathrm{d}\sigma = \dfrac{\pi a^{2}}{2}\sin\xi\mathrm{d}\xi
\ee
donc, finalement :
\be
	\mathrm{d}\sigma = \dfrac{\pi a^{2}}{\epsilon_{max}}\mathrm{d}\epsilon
\ee
et la distribution des particules diffus\'ees est homog\`ene par rapport \`a $\epsilon$ qui est l'\'energie perdue par celles-ci.

\subsection{Cas d'un champ $\propto r^{-n}$}

\subsection{Cas d'un champ $U = -\alpha / r^{2}$}

\subsection{Cas d'un champ $U = -\alpha / r^{n}$}

\subsection{Cas d'un champ suivant la loi de Newton}

\subsection{M\'ethode d'inversion de Firsov}

Oleg Firsov, 1953 cit\'e dans un article de 1971\footnote{Uniqueness of the Firsov Inversion Method and Focusing Potentials, Yu.N. Demkov, V.N. Ostrovskii, et N.B. Berezina}
\chapter{Petites oscillations}

\section{Oscillations lin\'eaires libres}

\subsection{Amplitude et phase initiale}

\end{document}