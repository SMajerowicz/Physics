\chapter{\'Equations du mouvement}
\section{Coordonn\'ees g\'en\'eralis\'ees}

Pour un point mat\'eriel, nous avons en coordonn\'ees cart\'esiennes :
\begin{itemize}
\item le rayon vecteur tel que :
	\be
		\vec{r}=\begin{pmatrix} x \\ y \\ z \end{pmatrix}
	\ee
\item la vitesse telle que :
	\be
		\vec{v}=\frac{{\rm d}\vec{r}}{{\rm dt}}
	\ee
\item l'acc\'el\'eration telle que :
	\be
		\vec{a}=\frac{{\rm d}\vec{v}}{{\rm d}t}=\frac{{\rm d}^{2}\vec{r}}{{\rm dt^{2}}}
	\ee
\end{itemize}

Les coordonn\'ees cart\'esiennes ne sont pas toujours les plus adapt\'ees. Un autre syt\`eme de coordonn\'ees peut \^etre plus commode \`a utiliser. Il convient de choisir alors $s$ grandeurs quelconques $\begin{Bmatrix}q_{i}\end{Bmatrix}^{s}_{1}$ pour d\'efinir la position d'un syst\`eme ($s$ degr\'es de libert\'es), ce sont ses \textbf{coordonn\'ees g\'en\'eralis\'ees} et les d\'riv\'ees $\begin{Bmatrix}\dot{q}_{i}\end{Bmatrix}^{s}_{1}$, ses \textbf{vitesses g\'en\'eralis\'ees}.

Les relations qui lient les acc\'el\'erations aux coordonn\'ees et aux vitesses sont appel\'ees les \textbf{\'equations du mouvement}.

\section{Le principe de moindre action}