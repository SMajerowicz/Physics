\chapter{Chocs de particules}

\section{D\'esint\'egration des particules}

Les lois de conservations ne d\'ependent absolument pas de l'esp\`ece d'interaction entre les particules et permettent ainsi d'arriver \`a des conclusions pourtant importantes pour les processus m\'ecaniques en jeu.

Pour commencer, choisissons la d\'esint\'egration spontan\'ee, i.e. non provoqu\'ee par des forces ext\'erieures, d'une particule en deux autres particules se d\'epla\c{c}ant apr\`es la d\'esint\'egration ind\'ependamment l'une de l'autre. Sa forme la plus simple est quand il est consid\'er\'e dans un syst\`eme de r\'ef\'erence o\`u la particule initiale est au repos. La conservation d' l'impulsions implique :
\bea
	\vec{p}_{1} + \vec{p}_{2} & = & \vec{0} \nonumber \\
	\vec{p}_{1} & = & -\vec{p}_{2}
\eea
ainsi les particules r\'esultantes de la d\'esint\'egration s'\'eloignent l'une de l'autre avec des impulsions \'egales, ,$\lVert \vec{p}_{1} \rVert = \lVert \vec{p}_{2} \rVert = p_{0}$ et dirig\'ees en sens inverse.

En utilisant la loi de conservation de l'\'energie et en particulier l'\'equation (\ref{EQ:8_4}) et parce que l'\'energie cin\'etique de la particule initiale est nulle, nous pouvons \'ecrire :
\be
	E_{int} = E_{1int} + \dfrac{p_{0}^{2}}{2m_{1}} + E_{2int} + \dfrac{p_{0}^{2}}{2m_{2}}
\ee
avec $E_{int}$ l'\'energie interne de la particule initiale, $E_{1int}$ et $E_{2int}$ l'\'energie interne des deux particules cr\'e\'ees. En d\'efinissant l'\'energie de d\'esint\'egration $\epsilon$ telle que :
\be
	\epsilon = E_{int} - (E_{1int} + E_{2int}) \label{EQ:16_1}
\ee
sa valeur est de facto positive. Or :
\be
	E_{int} - (E_{1int} + E_{2int}) = p_{0}^{2}\left(\dfrac{1}{2m_{1}} + \dfrac{1}{2m_{2}}\right)
\ee
Donc :
\be
	\epsilon = p_{0}^{2}\left(\dfrac{1}{2m_{1}} + \dfrac{1}{2m_{2}}\right) = \dfrac{p_{0}^{2}}{2m} \label{EQ:16_2}
\ee
avec $m = \dfrac{m_{1} + m_{2}}{m_{1}m_{2}}$ qui est la masse r\'eduite du syst\`eme apr\`es d\'esint\'egration.

\begin{figure}[htb!]
	\begin{center}
		\begin{picture}(500,300)(0,0)
			%circles
			\linethickness{0.05mm}
			\put(100,150){\circle{200}}\put(90,25){$V < v_{0}$}
			\put(400,150){\circle{200}}\put(390,25){$V > v_{0}$}
			%vector circle #1
			\linethickness{0.5mm}
			\put(40,150){\vector(1,0){60}}\put(65,135){$\vec{V}$}
			\put(100,150){\vector(1,1){72}}\put(140,180){$\vec{v}_{0}$}
			\put(40,150){\vector(9,5){130}}\put(90,185){$\vec{v}$}
			%angles circle #1
			\linethickness{0.05mm}
			\qbezier(55,150)(55,155)(50,155)\put(60,152){$\theta$}
			\qbezier(110,150)(110,155)(105,155)\put(112,153){$\theta_{0}$}
			\multiput(100,150)(10,0){10}{\line(1,0){8}}
			%vector circle #2
			\linethickness{0.5mm}
			\put(250,150){\vector(1,0){150}}\put(320,135){$\vec{V}$}
			\put(400,150){\vector(1,1){72}}\put(440,180){$\vec{v}_{0}$}
			\put(250,150){\vector(10,3){220}}\put(370,190){$\vec{v}$}
			%angles circle #2
			\linethickness{0.05mm}
			\qbezier(280,150)(280,158)(275,158)\put(287,152){$\theta$}
			\qbezier(410,150)(410,155)(405,155)\put(412,153){$\theta_{0}$}
			\multiput(400,150)(10,0){10}{\line(1,0){8}}
		\end{picture}
		\caption{Repr\'esentation g\'eom\'etrique de d\'esint\'egrations}\label{FIG:4_14}
	\end{center}
\end{figure}

\'Etudions maintenant le cas o\`u la particule initiale poss\`ede une vitesse $\vec{V}$ non nulle avant la d\'esint\'egration dans le syst\`eme de r\'ef\'erence. Il en existe deux :
\begin{itemize}
	\item le syst\`eme du laboratoire ou <<~l~>>,
	\item le syst\`eme du centre d'inertie ou <<~c~>> dans lequel, par d\'efinition, la somme des impulsions est nulle, voir le paragraphe (\ref{PAR:8}).
\end{itemize}
Pour une des particules r\'esultantes de la d\'esint\'egration, d\'efinissions $\vec{v}$ sa vitesse dans le syst\`eme <<~l~>> et $\vec{v}_{0}$, sa vitesse dans le syst\`eme <<~c~>>. Par construction, nous avons $\vec{v} = \vec{V} + \vec{v}_{0} \Leftrightarrow \vec{v}_{0} = \vec{v} + \vec{V}$. De plus, la formule d'Al-Kashi donne directement :
\bea
	v_{0}{2} & = & v^{2} + V^{2} + 2vV\cos(\langle \vec{v},\vec{V}\rangle) \nonumber \\
	& = & v^{2} + V^{2} + 2vV\cos(\theta)
\eea
en se basant sur la figure (\ref{FIG:4_14}).