\chapter{Chocs de particules}

\section{D\'esint\'egration des particules}

Les lois de conservations ne d\'ependent absolument pas de l'esp\`ece d'interaction entre les particules et permettent ainsi d'arriver \`a des conclusions pourtant importantes pour les processus m\'ecaniques en jeu.

Pour commencer, choisissons la d\'esint\'egration spontan\'ee, i.e. non provoqu\'ee par des forces ext\'erieures, d'une particule en deux autres particules se d\'epla\c{c}ant apr\`es la d\'esint\'egration ind\'ependamment l'une de l'autre. Sa forme la plus simple est quand il est consid\'er\'e dans un syst\`eme de r\'ef\'erence o\`u la particule initiale est au repos. La conservation d' l'impulsions implique :
\bea
	\vec{p}_{1} + \vec{p}_{2} & = & \vec{0} \nonumber \\
	\vec{p}_{1} & = & -\vec{p}_{2}
\eea
ainsi les particules r\'esultantes de la d\'esint\'egration s'\'eloignent l'une de l'autre avec des impulsions \'egales, ,$\lVert \vec{p}_{1} \rVert = \lVert \vec{p}_{2} \rVert = p_{0}$ et dirig\'ees en sens inverse.

En utilisant la loi de conservation de l'\'energie et en particulier l'\'equation (\ref{EQ:8_4}) et parce que l'\'energie cin\'etique de la particule initiale est nulle, nous pouvons \'ecrire :
\be
	E_{int} = E_{1int} + \dfrac{p_{0}^{2}}{2m_{1}} + E_{2int} + \dfrac{p_{0}^{2}}{2m_{2}}
\ee
avec $E_{int}$ l'\'energie interne de la particule initiale, $E_{1int}$ et $E_{2int}$ l'\'energie interne des deux particules cr\'e\'ees. En d\'efinissant l'\'energie de d\'esint\'egration $\epsilon$ telle que :
\be
	\epsilon = E_{int} - (E_{1int} + E_{2int}) \label{EQ:16_1}
\ee
sa valeur est de facto positive. Or :
\be
	E_{int} - (E_{1int} + E_{2int}) = p_{0}^{2}\left(\dfrac{1}{2m_{1}} + \dfrac{1}{2m_{2}}\right)
\ee
Donc :
\be
	\epsilon = p_{0}^{2}\left(\dfrac{1}{2m_{1}} + \dfrac{1}{2m_{2}}\right) = \dfrac{p_{0}^{2}}{2m} \label{EQ:16_2}
\ee
avec $m = \dfrac{m_{1} + m_{2}}{m_{1}m_{2}}$ qui est la masse r\'eduite du syst\`eme apr\`es d\'esint\'egration.

\begin{figure}[htb!]
	\begin{center}
		\begin{picture}(500,300)(0,0)
			%circles
			\linethickness{0.05mm}
			\put(100,150){\circle{200}}\put(90,25){$V < v_{0}$}
			\put(400,150){\circle{200}}\put(390,25){$V > v_{0}$}
			%vectors circle #1
			\linethickness{0.5mm}
			\put(40,150){\vector(1,0){60}}\put(65,135){$\vec{V}$}
			\put(100,150){\vector(1,1){72}}\put(140,180){$\vec{v}_{0}$}
			\put(40,150){\vector(9,5){130}}\put(90,185){$\vec{v}$}
			%angles circle #1
			\linethickness{0.05mm}
			\qbezier(55,150)(55,155)(50,155)\put(60,152){$\theta$}
			\qbezier(110,150)(110,155)(105,155)\put(112,153){$\theta_{0}$}
			\multiput(100,150)(10,0){10}{\line(1,0){8}}
			%points circle #1
			\put(30,147){$A$}
			%vectors circle #2
			\linethickness{0.5mm}
			\put(250,150){\vector(1,0){150}}\put(320,135){$\vec{V}$}
			\put(400,150){\vector(1,1){72}}\put(440,180){$\vec{v}_{0}$}
			\put(250,150){\vector(10,3){220}}\put(390,210){$\vec{v}$}
			%\theta_{max} case
			\linethickness{0.05mm}
			\put(250,150){\line(10,9){90}}
			\put(400,150){\line(-11,15){60}}
			%angles circle #2
			\linethickness{0.05mm}
			\qbezier(280,150)(280,158)(275,158)\put(287,152){$\theta$}
			\qbezier(410,150)(410,155)(405,155)\put(412,153){$\theta_{0}$}
			\qbezier(320,150)(320,190)(294,190)\put(315,182){$\theta_{max}$}
			\multiput(400,150)(10,0){10}{\line(1,0){8}}
			%points circle #2
			\put(240,147){$A$}
			\put(300,155){$B$}
			\put(475,220){$C$}
		\end{picture}
		\caption{Repr\'esentation g\'eom\'etrique de d\'esint\'egrations}\label{FIG:4_14}
	\end{center}
\end{figure}

\'Etudions maintenant le cas o\`u la particule initiale poss\`ede une vitesse $\vec{V}$ non nulle avant la d\'esint\'egration dans le syst\`eme de r\'ef\'erence. Il en existe deux :
\begin{itemize}
	\item le syst\`eme du laboratoire ou <<~l~>>,
	\item le syst\`eme du centre d'inertie ou <<~c~>> dans lequel, par d\'efinition, la somme des impulsions est nulle, voir le paragraphe (\ref{PAR:8}).
\end{itemize}
Pour une des particules r\'esultantes de la d\'esint\'egration, d\'efinissions $\vec{v}$ sa vitesse dans le syst\`eme <<~l~>> et $\vec{v}_{0}$, sa vitesse dans le syst\`eme <<~c~>>. Par construction, nous avons $\vec{v} = \vec{V} + \vec{v}_{0} \Leftrightarrow \vec{v}_{0} = \vec{v} + \vec{V}$. De plus, la formule d'Al-Kashi donne directement :
\bea
	v_{0}{2} & = & v^{2} + V^{2} + 2vV\cos(\langle \vec{v},\vec{V}\rangle) \nonumber \\
	& = & v^{2} + V^{2} + 2vV\cos(\theta)
\eea
en se basant sur la figure (\ref{FIG:4_14})et o\`u l'angle $\theta$ est l'angle de d\'eviation dans <<~l~>>. La repr\'esentation g\'eom\'etrique (\ref{FIG:4_14}) montre deux cas distincts :
\begin{itemize}
	\item $V < v_{0}$ o\`u l'angle $\theta$ peut alors prendre une valeur quelconque
	\item $V > v_{0}$ o\`u la particule r\'esultante est limit\'ee dans sa direction par la tangente au cercle de rayon $\lVert \vec{v}_{0} \rVert$ depuis le point A. Cela permet d'\'ecrire :
	\be
		\sin\theta_{max} = \dfrac{v_{0}}{V} \label{EQ:16_4}
	\ee
\end{itemize}

$\theta_{0}$ est l'angle de d\'eviation dans le r\'ef\'erentiel <<~c~>>, la repr\'esentation g\'eom\'etrique (\ref{FIG:4_14}) permet d'\'etablir la relation entre $\theta$ et $\theta_{0}$ telle que :
\be
	\tan\theta = \dfrac{v_{0}\sin\theta_{0}}{V + v_{0}\cos\theta_{0}} \label{EQ:16_5}
\ee
En d\'eveloppant et en recherchant une \'equation en $\cos\theta_{0}$, nous \'ecrivons :
\bea
	\dfrac{\sin^{2}\theta}{\cos^{2}\theta} & = & \dfrac{v_{0}^{2}(1 - \cos^{2}\theta_{0}}{(V + v_{0}\cos\theta_{0})^{2}} \nonumber \\
	\Leftrightarrow \sin^{2}\theta(V^{2} + v_{0}^{2}\cos^{2}\theta_{0} + 2Vv_{0}\cos\theta_{0}) & = & v_{0}^{2}\cos^{2}\theta - v_{0}^{2}\cos^{2}\theta\cos^{2}\theta_{0} \nonumber \\
\eea
ou encore en recherchant l'\'equation du second degr\'e :
\bea
	v_{0}^{2}(\sin^{2}\theta + cos^{2}\theta)\cos^{2}\theta_{0} + 2Vv_{0}\sin^{2}\theta\cos\theta_{0} + 2V^{2}\sin^{2}\theta - v_{0}^{2}\cos^{2}\theta & = & 0 \nonumber \\
	\Leftrightarrow \cos^{2}\theta_{0} + \dfrac{2V}{v_{0}}\sin^{2}\theta\cos\theta_{0} + \dfrac{V^{2}}{v_{0}^{2}}\sin^{2}\theta - \cos^{2}\theta & = & 0 \nonumber \\
\eea
qui a pour solutions :
\bea
	\cos\theta_{0} & = & \dfrac{-2\dfrac{V}{v_{0}}\sin^{2}\theta \pm \sqrt{4\dfrac{V^{2}}{v_{0}^{2}}\sin^{4}\theta - 4\dfrac{V^{2}}{v_{0}^{2}}\sin^{2}\theta + 4\cos^{2}\theta}}{2} \nonumber \\
	& = & -\dfrac{V}{v_{0}}\sin^{2}\theta \pm \sqrt{\dfrac{V^{2}}{v_{0}^{2}}(\sin^{2}\theta - 1)\sin^{2}\theta + \cos^{2}\theta} \nonumber \\
	\Leftrightarrow \cos\theta_{0} & = & -\dfrac{V}{v_{0}}\sin^{2}\theta \pm \cos\theta\sqrt{1 - \dfrac{V^{2}}{v_{0}^{2}}\sin^{2}\theta} \label{EQ:16_6}
\eea

\begin{figure}[htb!]
	\begin{center}
		\includegraphics[width=10cm]{chapter_04_paragraph_16_fig_14a}
		\caption{\'Equation (\ref{EQ:16_6}) dans le cas $v_{0} > V$ pour $\dfrac{V}{v_{0}}$ compris entre 0.1 et 0.9 avec un pas de 0.1 et pour $0 < \theta < \pi$}\label{FIG:4_14A}
	\end{center}
\end{figure}

Dans le cas o\`u $v_{0} > V$, le vecteur $\vec{v}$ ne croise le cercle de rayon $v_{0}$ qu'en un unique point et comme il est n\'ecessaire de choisir $\theta_{0} = 0$ quand $\theta = 0$, la solution (\ref{EQ:16_6}) se r\'eduit \`a $\cos\theta_{0} = -\dfrac{V}{v_{0}}\sin^{2}\theta + \cos\theta\sqrt{1 - \dfrac{V^{2}}{v_{0}^{2}}\sin^{2}\theta}$. Ce cas est repr\'esent\'e sur la figure (\ref{FIG:4_14A}) Dans le cas o\`u $v_{0} < V$, le vecteur $\vec{v}$ croise le m\^eme cercle en deux points, B et C sur la figure correspondante (\ref{FIG:4_14}) tel qu'il existe alors deux valeurs de $\theta_{0}$ pour chaque valeur de $\theta$. Ce cas est illustr\'e sur la figure (\ref{FIG:4_14B}).

\begin{figure}[htb!]
	\begin{center}
		\includegraphics[width=10cm]{chapter_04_paragraph_16_fig_14b}
		\caption{Deux solutions de l'\'equation (\ref{EQ:16_6}) dans le cas $v_{0} < V$ pour $\dfrac{V}{v_{0}}$ compris entre 1.1 et 1.9 avec un pas de 0.1 et pour $0 < \theta < \pi$}\label{FIG:4_14B}
	\end{center}
\end{figure}

Dans la plupart des applications physiques, ce sont de nombreuses particules qui se d\'esint\`egrent et il faut alors raisonner en termes de distribution, en \'energie, en impulsion, en directions, etc. Prenons d\'esormais l'hypoth\`ese de particules initiales orient\'ees de mani\`ere chaotique, i.e. en moyenne de fa\c{c}on isotrope. Dans le r\'ef\'erentiel <<~c~>>, l'isotropie est conserv\'ee apr\`es les d\'esint\'egrations et les particules r\'esultantes de m\^eme esp\`ece ont alors la m\^eme \'energie et la r\'epartition des trajectoires est isotrope. L'orientation chaotique peut se traduire comme la quantit\'e de particules traversant un angle solide\footnote{Par d\'efinition, un \'el\'ement d'angle solide est d\'efini par $\mathrm{d}^{2}\Omega = \dfrac{\vec{r}\cdot\vec{n}}{r^{3}}\mathrm{d}^{2}S$ avec $\vec{r}$ le vecteur rayon et $\vec{n}$ le vecteur normale de l'\'el\'ement de surface $\mathrm{d}^{2}S$. Dans le cadre d'une sph\`ere, nous avons $\mathrm{d}^{2}S = r\mathrm{d}\theta r\sin\theta\mathrm{d}\varphi$ et par cons\'equent $\mathrm{d}^{2}\Omega = \sin\theta\mathrm{d}\theta\mathrm{d}\varphi$ ou encore $\mathrm{d}\Omega = 2\pi\sin\theta\mathrm{d}\theta$.} $\mathrm{d}\omega_{0}$ qui est proportionnelle \`a la grandeur de cet \'el\'ement soit $\frac{\mathrm{d}\omega_{0}}{4\pi}$. Dans le cadre d'une sph\`ere, $\mathrm{d}\omega_{0} = 2\pi\sin\theta_{0}\mathrm{d}\theta_{0}$, donc :
\be
	\dfrac{\mathrm{d}\omega_{0}}{4\pi} = \dfrac{1}{2}\sin\theta_{0}\mathrm{d}\theta_{0} \label{EQ:16_7}
\ee
Pour obtenir la r\'epartition dans le r\'ef\'erentiel <<~l~>>, partons du calcul de l'\'energie cin\'etique et de sa distribution. Nous savons que :
\bea
	\vec{v} & = & \vec{V} + \vec{v}_{0} \nonumber \\
	\Leftrightarrow v^{2} & = & v_{0}^{2} + V^{2} + 2v_{0}V\cos\theta_{0} \nonumber \\
	\Leftrightarrow \cos\theta_{0} & = & \dfrac{v^{2} - v_{0}^{2} - V^{2}}{2v_{0}V}
\eea
Or par rapport \`a $\theta_{0}$, les quantit\'es $\lVert\vec{v}_{0}\rVert$ et $\lVert\vec{V}\rVert$ sont constantes au contraire de $\lVert\vec{v}\rVert$ aussi, nous pouvons en conclure que :
\be
	\mathrm{d}\cos\theta_{0} = -\sin\theta_{0}\mathrm{d}\theta_{0} = \dfrac{\mathrm{d}(v^{2})}{2v_{0}V}
\ee
Or l'\'energie cin\'etique d'une particule r\'esultante de masse $m$ s\'ecrit $T = \frac{1}{2}mv^{2} \Leftrightarrow \mathrm{d}T = \frac{1}{2}m\mathrm{d}(v^{2})$. Donc en reprenant l'\'equation (\ref{EQ:16_7}) :
\bea
	\dfrac{\mathrm{d}\omega_{0}}{4\pi} & = & \dfrac{1}{2}\pi\sin\theta_{0}\mathrm{d}\theta_{0} = -\dfrac{\mathrm{d}(v^{2})}{4v_{0}V} \nonumber \\
	\Leftrightarrow \dfrac{\mathrm{d}\omega_{0}}{4\pi} & = & -\dfrac{\mathrm{d}(T)}{4mv_{0}V} \label{EQ:16_8}
\eea
En reprenant $v^{2} = v_{0}^{2} + V^{2} + 2v_{0}V\cos\theta_{0}$, nous en d\'eduisons que :
\begin{itemize}
	\item l'\'energie cin\'etique maximale est obtenue pour $\theta_{0} = 0$ et $T_{max} = \frac{m}{2}(v_{0} + V)^{2}$
	\item l'\'energie cin\'etique minimale est obtenue pour $\theta_{0} = \pi$ et $T_{max} = \frac{m}{2}(v_{0} - V)^{2}$
\end{itemize}
et dans cet intervalle, l'\'energie cin\'etique se distribue suivant la relation (\ref{EQ:16_8}).

Si la d\'esint\'egration donne plus de deux composantes, cela complexifie l'\'etude et en particulier, l'\'energie des composantes est loin d'\^etre unique dans le r\'ef\'erentiel <<~c~>>. Toutefois il existe une valeur maximale de l'\'energie cin\'etique pour chaque particule r\'esultante. Parmi l'ensemble des particules r\'esultants, consid\'erons-en une de masse $m_{1}$ et en posant $E_{int}'$, l'\'energie <<~interne~>> de l'ensemble des particules restantes moins $m_{1}$. Puisque cette situation permet de revenir \`a un probl\`eme \`a deux corps, la formule (\ref{EQ:16_1}) permet d'\'ecrire :
\be
	E_{int} = E_{int}' + T' + T_{10} + E_{1int}
\ee
avec $T'$ l'\'energie cin\'etique de l'ensemble des particules restantes moins $m_{1}$, $T_{10}$ l'\'energie cin\'etique de $m_{1}$ et $E_{1int}$ son \'energie interne. Les \'energies cin\'etiques peuvent s'\'ecrire avec $M$ la masse de la particule initiale :
\be
	T' = \dfrac{p_{0}^{2}}{2(M - m_{1})}\text{ et }T_{10} = \dfrac{p_{0}^{2}}{2m_{1}}
\ee
donc :
\bea
	E_{int} - E_{int}' - E_{1int} & = & \dfrac{M}{M - m_{1}}\dfrac{p_{0}^{2}}{2m_{1}} \nonumber \\
	\Leftrightarrow T_{10} & = & \dfrac{M - m_{1}}{M}(E_{int} - E_{int}' - E_{1int})
\eea
Aussi $T_{10}$ est maximale si et seulement si $E_{int}'$ est minimale. Ceci intervient lorsque toutes les particules r\'esultantes, sauf $m_{1}$, ont la m\^eme vitesse, i.e. une agitation du syst\`eme minimale et $E_{int}'$ devient simplement la somme des \'energies internes de cet ensemble de particules. Dans ce cas pr\'ecis, la quantit\'e $E_{int} - E_{int}' - E_{1int}$ repr\'esente $\epsilon$, l'\'energie de d\'esint\'egration d\'efinie dans l'\'equation (\ref{EQ:16_2}). Nous en concluons :
\be
	T_{10max} = \dfrac{M - m_{1}}{M}\epsilon \label{EQ:16_9}
\ee

\section{Chocs \'elastiques des particules}

Le choc entre deux particules est dit \'elastique lorsqu'il n'y a pas de modifications de leur \'etat interne. Lors de l'application de la loi de conservation de l'\'energie, il n'est donc pas n\'ecessaire de prendre en compte les \'energies internes respectives. Dans le r\'ef\'erentiel du centre d'inertie <<~c~>>, ce dernier est de facto au repos. Avant le choc, par application de la conservation de l'impulsion, nous avons dans <<~c~>> : $m_{1}\vec{v}_{10} + m_{2}\vec{v}_{20} = \vec{0}$. Dans <<~l~>>, la position du centre d'inertie s'\'ecrit :
\be
	\vec{R} = \dfrac{m_{1}\vec{r}_{1} + m_{2}\vec{r}_{2}}{m_{1} + m_{2}}
\ee
et la loi de composition des vitesses am\`ene \`a \'ecrire :
\be
	\begin{cases}
		\vec{v}_{1} = \vec{v}_{10} + \frac{\mathrm{d}\vec{R}}{\mathrm{dt}} \\
		\vec{v}_{2} = \vec{v}_{20} + \frac{\mathrm{d}\vec{R}}{\mathrm{dt}}
	\end{cases}
\ee
donc en soustrayant les deux relations et en d\'efinissant\footnote{Voir une \'equivalence avec les relations (\ref{EQ:13_2})} $\vec{v} = \vec{v}_{1} - \vec{v}_{2}$ :
\be
	\vec{v} = \vec{v}_{10} - \vec{v}_{20}
\ee
En reprenant la conservation de l'impulsion dans <<~c~>>, nous avons alors :
\be
	\begin{cases}
		\vec{v} = \vec{v}_{10} + \frac{m_{1}}{m_{2}}\vec{v}_{10} \Leftrightarrow \vec{v}_{10} = \frac{m_{2}}{m_{1} + m_{2}}\vec{v} \\
		\vec{v} = -\vec{v}_{20} - \frac{m_{2}}{m_{1}}\vec{v}_{20} \Leftrightarrow \vec{v}_{20} = -\frac{m_{1}}{m_{1} + m_{2}}\vec{v}
	\end{cases}
\ee

Comme le choc est \'elastique, dans <<~c~>>, la conservation de l'impulsion après le choc donne $m_{1}\vec{v'}_{10} + m_{2}\vec{v'}_{20} = \vec{0}$ et comme avant le choc, nous avons $m_{1}\vec{v}_{10} + m_{2}\vec{v}_{20} = \vec{0}$, nous pouvons en conclure que c'est vrai si $m_{1}(\vec{v'}_{10} + \vec{v}_{10}) + m_{2}(\vec{v'}_{20} + \vec{v}_{20}) = \vec{0}$, soit $\vec{v'}_{10} = -\vec{v}_{10}$ et $\vec{v'}_{20} = -\vec{v}_{20}$. De m\^eme, la conservation de l'\'energie avant et apr\`es le choc, et qui ne concerne que les \'energies cin\'etiques, implique que l'\'energie de chacune des deux particules est conserv\'ees car $v'_{10} = v_{10}$ et $v'_{20} = v_{20}$.
Par cons\'equent, dans <<~c~>>, l'unique diff\'erence entre avant et apr\`es le choc se situe dans l'inversion de la direction de la vitesse de chacune des deux particules.

D\'efinissons le vecteur unitaire $\vec{n}_{0}$ dans la direction de la vitesse apr\`es le choc de la particule de masse $m_{1}$. Alors $\vec{v'}_{10} = v'_{10}\vec{n}_{0} = v_{10}\vec{n}_{0}$. Le vecteur $\vec{n}_{0}$ absorbe l'inversion de direction apr\`es le choc et permet de reprendre les relations ci-dessous pour en conclure :
\be
	\begin{cases}
		\vec{v'}_{10} = \frac{m_{2}}{m_{1} + m_{2}}v\vec{n}_{0} \\
		\vec{v'}_{20} = -\frac{m_{1}}{m_{1} + m_{2}}v\vec{n}_{0} \label{EQ:17_1}
	\end{cases}
\ee
Le passage de <<~c~>> \`a <<~l~>> via la loi de composition des vitesses et l'expression de la vitesse du centre d'inertie dans <<~l~>> permet d'\'ecrire :
\be
	\begin{cases}
		\vec{v'}_{1} = \vec{v'}_{10} + \frac{\mathrm{d}\vec{R'}}{\mathrm{dt}} \\
		\vec{v'}_{2} = \vec{v'}_{20} + \frac{\mathrm{d}\vec{R'}}{\mathrm{dt}}
	\end{cases}
\ee
avec :
\be
	\dfrac{\mathrm{d}\vec{R'}}{\mathrm{dt}} = \dfrac{m_{1}\vec{v'}_{1} + m_{2}\vec{v'}_{2}}{m_{1} + m_{2}} = \dfrac{m_{1}\vec{v}_{1} + m_{2}\vec{v}_{2}}{m_{1} + m_{2}}
\ee
par l'additivit\'e des int\'egrales du mouvement, voir le paragraphe (\ref{PAR:6}) qui implique que les lois de conservation ne d\'ependent pas des interactions en jeu, en particulier pour celle de l'impulsion ici. Par cons\'equent, en reportant les \'equations (\ref{EQ:17_1}) :
\be
	\begin{cases}
		\vec{v'}_{1} = \frac{m_{2}}{m_{1} + m_{2}}v\vec{n}_{0} + \frac{m_{1}\vec{v}_{1} + m_{2}\vec{v}_{2}}{m_{1} + m_{2}} \\
		\vec{v'}_{2} = -\frac{m_{1}}{m_{1} + m_{2}}v\vec{n}_{0} + \frac{m_{1}\vec{v}_{1} + m_{2}\vec{v}_{2}}{m_{1} + m_{2}} \label{EQ:17_2}
	\end{cases}
\ee
o\`u seul le vecteur $\vec{n}_{0}$ est une cons\'equence de la loi d'interaction entre les particules. En multipliant les relations (\ref{EQ:17_2}) par la masse respective des particules, nous arrivons \`a exprimer leur impulsion apr\`es le choc en fonction de celle avant le choc, le tout dans le r\'ef\'erentiel <<~l~>> :
\be
	\begin{cases}
		\vec{p'}_{1} = mv\vec{n}_{0} + \frac{m_{1}}{m_{1} + m_{2}}(\vec{p}_{1} + \vec{p}_{2}) \\
		\vec{p'}_{2} = -mv\vec{n}_{0} + \frac{m_{2}}{m_{1} + m_{2}}(\vec{p}_{1} + \vec{p}_{2}) \label{EQ:17_3}
	\end{cases}
\ee
avec $m$ la masse r\'eduite du syt\`eme d\'efinie telle que $m = \frac{m_{1}m_{2}}{m_{1} + m_{2}}$.

\begin{figure}[htb!]
	\begin{center}
		\begin{picture}(300,300)(0,0)
			%circle
			\linethickness{0.05mm}
			\put(150,150){\circle{200}}
			%vectors circle
			\linethickness{0.5mm}
			\put(60,150){\vector(1,0){90}}
			\put(150,150){\vector(1,0){50}}
			\put(60,150){\vector(7,6){115}}\put(90,190){$\vec{p'}_{1}$}
			\put(150,150){\vector(1,5){20}}\put(160,185){$\vec{n}_{0}$}
			\put(170,246){\vector(3,-10){29}}\put(195,190){$\vec{p'}_{2}$}
			%points circle
			\put(146,138){$O$}
			\put(51,148){$A$}
			\put(202,146){$B$}
			\put(175,250){$C$}
		\end{picture}
		\caption{Repr\'esentation g\'eom\'etrique d'un choc entre deux particules}\label{FIG:4_15}
	\end{center}
\end{figure}

Sur la figure (\ref{FIG:4_15}) est trac\'e le cercle de rayon $mv$. Les vecteurs $\vec{AC}$ et $\vec{CB}$ donnent respectivement $\vec{p'}_{1}$ et $\vec{p'}_{2}$ en accord avec les relations (\ref{EQ:17_3}). Les impulsions initiales $\vec{p}_{1}$ et $\vec{p}_{2}$ impliquent que les points $A$ et $B$ ne changent pas de position, aussi seul le point $C$ peut avoir une position quelconque sur le cercle de rayon $mv$. Par construction g\'eom\'etrique, les relations (\ref{EQ:17_3}) donnent :
\be
	\begin{cases}
		\vec{AO} = \frac{m_{1}}{m_{1} + m_{2}}(\vec{p}_{1} + \vec{p}_{2}) \\
		\vec{OB} = \frac{m_{2}}{m_{1} + m_{2}}(\vec{p}_{1} + \vec{p}_{2})
	\end{cases}
\ee

\subsection{Cas de la particule $m_{2}$ au repos avant le choc}

\begin{figure}[htb!]
	\begin{center}
		\begin{picture}(500,300)(0,0)
			%circles
			\linethickness{0.05mm}
			\put(100,150){\circle{200}}\put(90,25){$m_{1} < m_{2}$}
			\put(400,150){\circle{200}}\put(390,25){$m_{1} > m_{2}$}
			%vectors circle #1
			\linethickness{0.5mm}
			\put(40,150){\vector(1,0){160}}
			\put(40,150){\vector(9,5){130}}\put(90,190){$\vec{p'}_{1}$}
			\put(169,221){\vector(3,-7){32}}\put(165,180){$\vec{p'}_{2}$}
			%angles circle #1
			\linethickness{0.05mm}
			\multiput(100,150)(10,10){7}{\line(1,1){8}}
			\qbezier(55,150)(55,155)(50,155)\put(60,153){$\theta_{1}$}
			\qbezier(110,150)(110,155)(105,155)\put(114,154){$\xi$}
			\qbezier(180,150)(188,165)(194,165)\put(170,155){$\theta_{2}$}
			%points circle #1
			\put(95,135){$O$}
			\put(30,147){$A$}
			\put(202,147){$B$}
			\put(175,220){$C$}
			%vectors circle #2
			\linethickness{0.5mm}
			\put(250,150){\vector(1,0){250}}
			\put(250,150){\vector(10,3){220}}\put(390,200){$\vec{p'}_{1}$}
			\put(469,221){\vector(3,-7){32}}\put(465,180){$\vec{p'}_{2}$}
			%\theta_{max} case
			\linethickness{0.05mm}
			\put(250,150){\line(10,9){90}}
			\put(400,150){\line(-11,15){60}}
			%angles circle #2
			\linethickness{0.05mm}
			\multiput(400,150)(10,10){7}{\line(1,1){8}}
			\qbezier(280,150)(280,158)(275,158)\put(287,152){$\theta_{1}$}
			\qbezier(410,150)(410,155)(405,155)\put(412,153){$\xi$}
			\qbezier(480,150)(488,165)(494,165)\put(470,155){$\theta_{2}$}
			\qbezier(320,150)(320,190)(294,190)\put(315,182){$\theta_{max}$}
			%points circle #2
			\put(395,135){$O$}
			\put(240,147){$A$}
			\put(502,147){$B$}
			\put(475,220){$C$}
		\end{picture}
		\caption{Repr\'esentation g\'eom\'etrique de d\'esint\'egrations dans le cas $\vec{v}_{2} = \vec{0}$}\label{FIG:4_16}
	\end{center}
\end{figure}

Si la particule de masse $m_{2}$ est au repos avant le choc, cela se traduit par $v_{2} = \vec{0} = p_{2}$. Nous avons aussi :
\be
	\vec{OB} = \dfrac{m_{2}}{m_{1} + m_{2}}(\vec{p}_{1} + \vec{p}_{2}) = \frac{m_{1}m_{2}}{m_{1} + m_{2}}\vec{v}_{1} = m\vec{v}_{1} = m\vec{v}
\ee
car par d\'efinition, $\vec{v} = \vec{v}_{1} - \vec{v}_{2}$ et donc \'egal \`a $\vec{v}_{1}$ dans le cas qui nous occupe. Donc le point $B$ se trouve positionner sur le cercle de rayon $mv$. De plus :
\be
	\vec{AB} = \vec{AO} + \vec{OB} = \frac{m_{1}}{m_{1} + m_{2}}\vec{p}_{1} + \dfrac{m_{2}}{m_{1} + m_{2}}\vec{p}_{1} = \vec{p}_{1}
\ee
Le vecteur $\vec{AB}$ repr\'esente donc l'impulsion de d\'epart de la particule de masse $m_{1}$.

Les points $A$, $O$ et $B$ se situant sur la m\^eme droite, nous pouvons en d\'eduire :
\be
	AO = AB - OB = m_{1}v_{1} - \dfrac{m_{1}m_{2}}{m_{1} + m_{2}}v_{1} = \dfrac{m_{1}^{2}}{m_{1} + m_{2}}v_{1} = \dfrac{m_{1}}{m_{2}}OB
\ee
Par cons\'equent :
\begin{itemize}
	\item si $m_{1} < m_{2}$ alors le point $A$ se situe \`a l'int\'erieur du cercle
	\item si $m_{1} = m_{2}$ alors le point $A$ se situe sur le cercle
	\item si $m_{1} > m_{2}$ alors le point $A$ se situe \`a l'ext\'erieur du cercle
\end{itemize}
Ces cas sont repr\'esent\'es sur la figure (\ref{FIG:4_16}). Sur celle-ci, les angles $\theta_{1}$ et $\theta_{2}$ sont les angles de d\'eviation dans <<~l~>> par rapport \`a la direction du choc, d\'efinie par $\vec{p}_{1}$ et $\xi$ est l'angle de d\'eviation de la particule de masse $m_{1}$ dans <<~c~>>. Les angles $\theta_{1}$ et $\theta_{2}$ peuvent \^etre calcul\'es ainsi :
\be
	\begin{cases}
		\tan\theta_{1} = \dfrac{mv\sin\xi}{AO + mv\cos\xi} = \dfrac{\dfrac{m_{1}m_{2}}{m_{1} + m_{2}}v\sin\xi}{\dfrac{m_{1}^{2}}{m_{1} + m_{2}}v + \dfrac{m_{1}m_{2}}{m_{1} + m_{2}}v\cos\xi} = \dfrac{m_{2}\sin\xi}{m_{1} + m_{2}\cos\xi} \\
		\\
		\pi = \dfrac{\pi}{2} + \dfrac{\xi}{2} + \theta_{2} \Leftrightarrow \theta_{2} = \dfrac{\pi - \xi}{2} \label{EQ:17_4}
	\end{cases}
\ee

\subsection{Cas d'un choc entre deux particules de même masse dont l'une au repos avant le choc}

\begin{figure}[htb!]
	\begin{center}
		\begin{picture}(300,300)(0,0)
			%circles
			\linethickness{0.05mm}
			\put(150,150){\circle{200}}
			%vectors circle #1
			\linethickness{0.5mm}
			\put(50,150){\vector(1,0){200}}
			\put(50,150){\vector(9,4){170}}\put(140,200){$\vec{p'}_{1}$}
			\put(219,221){\vector(3,-7){32}}\put(215,180){$\vec{p'}_{2}$}
			%angles circle #1
			\linethickness{0.05mm}
			\multiput(150,150)(10,10){7}{\line(1,1){8}}
			\qbezier(65,150)(65,155)(60,155)\put(75,153){$\theta_{1}$}
			\qbezier(160,150)(160,155)(155,155)\put(164,154){$\xi$}
			\qbezier(230,150)(238,165)(244,165)\put(220,155){$\theta_{2}$}
			%points circle #1
			\put(145,135){$O$}
			\put(40,147){$A$}
			\put(252,147){$B$}
			\put(225,220){$C$}
		\end{picture}
		\caption{Repr\'esentation g\'eom\'etrique de d\'esint\'egrations dans le cas $\vec{v}_{2} = \vec{0}$ et $m_{1} = m_{2}$}\label{FIG:4_17}
	\end{center}
\end{figure}