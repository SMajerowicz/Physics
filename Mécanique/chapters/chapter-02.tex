\chapter{Lois de conservation}

\section{\'Energie}

Les \emph{int\'egrales premi\`eres} ou \emph{int\'egrales du mouvement}, sont les fonctions de $q_{i}$ et $\dot{q}_{i}$ qui conservent une valeur constante au cours du mouvement. Elles sont additives, c'est-\`a-dire qu'avant et après une int\'eraction, leur somme garde une valeur identique.

Commen\c{c}ons par l'int\'egrale premi\`ere qui d\'ecoule de l'\emph{uniformi\'e du temps}.

Pour un syst\`eme ferm\'e \`a $s$ degr\'es de libert\'e, nous avons :
\be
	\dfrac{\mathrm{d}L}{\mathrm{dt}} = \sum_{i=1}^{s}\dfrac{\partial L}{\partial q_{i}}\dfrac{\mathrm{d} q_{i}}{\mathrm{dt}} + \sum_{i=1}^{s}\dfrac{\partial L}{\partial \dot{q}_{i}}\dfrac{\mathrm{d}\dot{q}_{i}}{\mathrm{dt}} + \dfrac{\partial L}{\partial \mathrm{t}}
\ee
L'uniformit\'e du temps donne $\dfrac{\partial L}{\partial \mathrm{t}} = 0$ et en utilisant l'\'equation (\ref{EQ:2_6}), nous avons :
\bea
	\dfrac{\mathrm{d}L}{\mathrm{dt}} & = & \sum_{i=1}^{s}\dfrac{\mathrm{d}}{\mathrm{dt}}\left(\dfrac{\partial L}{\partial \dot{q}_{i}}\dot{q}_{i}\right) + \sum_{i=1}^{s}\dfrac{\partial L}{\partial \dot{q}_{i}}\ddot{q}_{i} \nonumber \\
	& = & \sum_{i=1}^{s}\dfrac{\mathrm{d}}{\mathrm{dt}}\left(\dfrac{\partial L}{\partial \dot{q}_{i}}\dot{q}_{i}\right) \nonumber \\
	0 & = & \dfrac{\mathrm{d}}{\mathrm{dt}}\left(\sum_{i=1}^{s}\dfrac{\partial L}{\partial \dot{q}_{i}}\dot{q}_{i} - L\right)
\eea
Par cons\'equent, la quantit\'e :
\be
	E = \sum_{i=1}^{s}\dfrac{\partial L}{\partial \dot{q}_{i}}\dot{q}_{i} - L \label{EQ:6_1}
\ee
est constante dans le temps pour un syst\`eme ferm\'e. $E$ est l'\emph{\'energie} du syst\`eme. Et puisque la fonction de Lagrange est additive, l'\'energie l'est aussi. Les syst\`emes m\'ecaniques dont l'\'energie se conserve sont appel\'es \emph{syst\`emes conservatifs}. Enfin, si le syst\`eme est plac\'e dans un champ ext\'erieur constant par rapport au temps alors la loi de conservation de l'\'energie reste valable car la fonction de Lagrange est implicitement ind\'ependant du temps.

\`A partir de l'\'equation (\ref{EQ:5_1}, nous pouvons \'ecrire :
\be
	\sum_{i=1}^{s}\dfrac{\partial L}{\partial\dot{q}_{i}}\dot{q}_{i} = \sum_{i=1}^{s}\dfrac{\partial (T-U)}{\partial\dot{q}_{i}}\dot{q}_{i} = \sum_{i=1}^{s}\dfrac{\partial T}{\partial\dot{q}_{i}}\dot{q}_{i}
\ee
car l'\'energie potentielle $U$ ne d\'epend que des coordonn\'ees (voire du temps).
Nous savons que l'\'energie cin\'etique $T = f(\dot{q}^{2})$. Or pour une fonction homog\`ene de degr\'e $\alpha$, telle $f(t.x) = t^{\alpha}f(x)$, alors le th\'eor\`eme d'Euler donne pour cette fonction :
\be
	\sum_{i} x_{i}\dfrac{\partial f(x)}{\partial x_{i}} = \alpha f(x)
\ee
Appliqu\'ee \`a l'\'energie cin\'etique, cela donne :
\be
	\sum_{i=1}^{s}\dfrac{\partial T}{\partial\dot{q}_{i}}\dot{q}_{i} = 2T = \sum_{i=1}^{s}\dfrac{\partial L}{\partial\dot{q}_{i}}\dot{q}_{i}
\ee
ce qui permet de conclure \`a, en compl\'etant l'\'equation (\ref{EQ:6_1}) :
\be
	E = T(q,\dot{q}) + U(q) \label{EQ:6_2}
\ee
soit en coordonn\'ees cart\'esiennes :
\be
	E = \sum_{a}\dfrac{m_{a}\vec{v}_{a}^{\,2}}{2} + U(\begin{Bmatrix}\vec{r}_{i}\end{Bmatrix}_{1}^{s}) \label{EQ:6_3}
\ee

\section{Impulsion}

Apr\`es l'unformit\'e du temps, nous allons \'etudier les cons\'equences de l'homog\'en\'eit\'e de l'espace. Dans ce cadre, les propri\'et\'es m\'ecaniques d'un syst\`eme ferm\'e ne changent pas lors d'un d\'eplacement parall\`ele du syst\`eme dans son entier. Supposons ainsi que la fonction de Lagrange reste inchang\'ee pour un d\'eplacement infinit\'esimal tel que : $\vec{r}_{a} \rightarrow \vec{r}_{a} + \vec{\epsilon}$, i.e. $\delta L = 0$ :
\bea
	\delta L(\vec{r}_{a},\vec{v}_{a}) & = & \sum_{a}\dfrac{\partial L}{\partial\vec{r}_{a}}\delta\vec{r}_{a} + \sum_{a}\dfrac{\partial L}{\partial\vec{v}_{a}}\delta\vec{v}_{a} \nonumber \\
	& = & \sum_{a}\dfrac{\partial L}{\partial\vec{r}_{a}}\vec{\epsilon} \text{ puisque }\delta\vec{v}_{a} = \vec{0} \nonumber \\
\eea
donc $\delta L = 0$ est \'equivalent \`a :
\bea
	\forall \vec{\epsilon}\text{, }\sum_{a}\dfrac{\partial L}{\partial\vec{r}_{a}}\vec{\epsilon} & = & 0 \nonumber \\
	\forall \vec{\epsilon}\text{, }\vec{\epsilon}\cdot\sum_{a}\dfrac{\partial L}{\partial\vec{r}_{a}} & = & 0 \nonumber \\
	\Leftrightarrow \sum_{a}\dfrac{\partial L}{\partial\vec{r}_{a}} & = & 0 \label{EQ:7_1}
\eea
Or les \'equations de Lagrange (\ref{EQ:5_2}) donnent :
\bea
	\dfrac{\mathrm{d}}{\mathrm{dt}}\left(\dfrac{\partial L}{\partial \vec{v}_{a}}\right) & = & \dfrac{\partial L}{\partial \vec{r}_{a}} \nonumber \\
	\Leftrightarrow \sum_{a}\left(\dfrac{\mathrm{d}}{\mathrm{dt}}\left(\dfrac{\partial L}{\partial \vec{v}_{a}}\right)\right) & = & \sum_{a}\left(\dfrac{\partial L}{\partial \vec{r}_{a}}\right) \nonumber
\eea
En appliquant l'\'equation (\ref{EQ:7_1}), nous avons :
\be
	\sum_{a}\left(\dfrac{\mathrm{d}}{\mathrm{dt}}\left(\dfrac{\partial L}{\partial \vec{v}_{a}}\right)\right) = \dfrac{\mathrm{d}}{\mathrm{dt}}\sum_{a}\left(\dfrac{\partial L}{\partial \vec{v}_{a}}\right) = \vec{0}
\ee
Ainsi, dans un syst\`eme ferm\'e, la quantit\'e vectorielle :
\be
	\vec{P} = \sum_{a}\dfrac{\partial L}{\partial \vec{v}_{a}} \label{EQ:7_2}
\ee
est inchang\'ee pendant le mouvement. Le vecteur $\vec{P}$ est l'\emph{impulsion} du syt\`eme. En utilisant l'\'equation (\ref{EQ:5_1}), nous obtenons :
\bea
	L & = & \sum_{a}\dfrac{m_{a}\vec{v}_{a}^{\,2}}{2} - U(\begin{Bmatrix}\vec{r}_{a}\end{Bmatrix}_{1}^{s}) \nonumber \\
	\Rightarrow \dfrac{\partial L}{\partial \vec{v}_{a}} & = & \dfrac{m_{a}}{2}\dfrac{2\partial\vec{v}_{a}}{\partial\vec{v}_{a}} - \dfrac{\partial U}{\partial\vec{v}_{a}} \nonumber \\
	& = & m_{a}\vec{v}_{a}
\eea
Ainsi :
\be
	\vec{P} = \sum_{a}m_{a}\vec{v}_{a} \label{EQ:7_3}
\ee