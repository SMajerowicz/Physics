\chapter{Probl\`emes d'int\'egration des \'equations du mouvement}

\section{Mouvement lin\'eaire}

\subsection{Oscillations d'un pendule math\'ematique plan}

\begin{figure}[htb!]
	\begin{center}
		\begin{picture}(100,100)(0,0)
			%axis
			\linethickness{0.05mm}
			\multiput(0,100)(10,0){10}{\line(1,0){8}}\put(102,95){$x$}
			\multiput(0,0)(0,10){10}{\line(0,1){8}}\put(-5,-10){$y$}
			%socle
			\put(0,100){\color{black}\circle*{5}}
			%arms
			\linethickness{0.5mm}
			\put(0,100){\line(10,-15){70}}
			\put(40,50){$l$}
			\put(68,-5){\color{black}\circle*{10}}\put(75,-7){$m$}
			%angles
			\linethickness{0.05mm}
			\qbezier(0,80),(5,82),(10,85)
			\put(3,70){$\varphi$}
		\end{picture}
		\caption{Pendule double oscillant}\label{FIG:3_1_1}
	\end{center}
\end{figure}

Il s'agit de trouver la p\'eriode d'oscillation du pendule en fonction de l'amplitude du mouvement. En supposant $\varphi_{0}$ l'angle maximal par rapport \`a la verticale, o\`u la vitesse du pendule est nulle, l'\'energie totale du pendule est :
\bea
	E & = & \frac{m}{2}\left[\dfrac{\mathrm{d}(l\varphi)}{\mathrm{dt}}\right]^{2} - mgl\cos\varphi \nonumber \\
	& = & \frac{m}{2}l^{2}\dot{\varphi}^{2} - mgl\cos\varphi
\eea
et par conservation de l'\'energie totale :
\be
	E = E(\varphi_{0}) = -mgl\cos\varphi_{0}
\ee
Par sym\'etrie du mouvement, la p\'eriode est \'egale au temps de parcours entre $\varphi=0$ et $\varphi=\varphi_{0}$. Sachant que la quantit\'e, $E-U$ vaut $-mgl\cos\varphi_{0} + mgl\cos\varphi$, l'\'equation (\ref{EQ:11_3}) peut s'\'ecrire :
\bea
	\mathrm{T} & = & 4\sqrt{\frac{m}{2}}\int_{0}^{\varphi_{0}}{\dfrac{\mathrm{d}(l\varphi)}{\sqrt{mgl(\cos\varphi - \cos\varphi_{0})}}} \nonumber \\
	& = & 4\sqrt{\frac{l}{2g}}\int_{0}^{\varphi_{0}}{\dfrac{\mathrm{d}\varphi}{\sqrt{\cos\varphi - \cos\varphi_{0}}}}
\eea
Notons que :
\bea
	\sin^{2}\frac{\alpha}{2} & = & \left(\dfrac{e^{i\frac{\alpha}{2}} - e^{-i\frac{\alpha}{2}}}{2i}\right)^{2} \nonumber \\
	& = & -\frac{1}{4}\left(e^{i\alpha} + e^{-i\alpha} - 2e^{i\frac{\alpha}{2}-i\frac{\alpha}{2}}\right) \nonumber \\
	& = & -\frac{1}{4}\left(e^{i\alpha} + e^{-i\alpha} - 2\right) \nonumber \\
	& = & \frac{1}{2}\left(1-\cos\alpha\right) \nonumber \\
	\Leftrightarrow \cos\alpha & = & 1 - 2\sin^{2}\frac{\alpha}{2}
\eea
Cela permet de d\'evelopper la p\'eriode telle que :
\bea
	\mathrm{T} & = & 4\sqrt{\frac{l}{2g}}\int_{0}^{\varphi_{0}}{\dfrac{\mathrm{d}\varphi}{\sqrt{1-2\sin\frac{\varphi}{2} - 1 + \sin\frac{\varphi_{0}}{2}}}} \nonumber \\
	& = & 4\sqrt{\frac{l}{2g}}\int_{0}^{\varphi_{0}}{\dfrac{\mathrm{d}\varphi}{\sqrt{2\left(\sin\frac{\varphi_{0}}{2} - \sin\frac{\varphi}{2}\right)}}} \nonumber \\
	& = & 2\sqrt{\frac{l}{2g}}\int_{0}^{\varphi_{0}}{\dfrac{\mathrm{d}\varphi}{\sqrt{\sin\frac{\varphi_{0}}{2} - \sin\frac{\varphi}{2}}}}
\eea