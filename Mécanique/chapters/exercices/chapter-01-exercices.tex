\chapter{Probl\`emes d'\'equations du mouvement}

L'objectif est de trouver la fonction de Lagrange dans les cas suivants plac\'es dans un champ de pesanteur constant $g$.

\section{Probl\`eme 1}

\begin{figure}[htb!]
	\begin{center}
		\begin{picture}(150,200)(0,0)
			%axis
			\linethickness{0.05mm}
			\multiput(0,200)(10,0){15}{\line(1,0){8}}\put(150,195){$x$}
			\multiput(0,0)(0,10){20}{\line(0,1){8}}\put(-5,-10){$y$}
			\multiput(75,90)(0,-10){7}{\line(0,-1){8}}
			%socle
			\put(0,200){\color{black}\circle*{5}}
			%arms
			\linethickness{0.5mm}
			\put(0,200){\line(10,-15){75}}
			\put(40,150){$l_{1}$}
			\put(75,90){\color{black}\circle*{10}}\put(80,95){$m_{1}$}
			\put(75,90){\line(15,-10){50}}
			\put(105,75){$l_{2}$}
			\put(125,55){\color{black}\circle*{10}}\put(130,60){$m_{2}$}
			%angles
			\linethickness{0.05mm}
			\qbezier(0,180),(5,182),(10,185)
			\put(3,170){$\varphi_{1}$}
			\qbezier(75,70),(82,70),(93,80)
			\put(83,60){$\varphi_{2}$}
		\end{picture}
		\caption{Pendule double oscillant}\label{FIG:1_1}
	\end{center}
\end{figure}

\section{Probl\`eme 2}

\begin{figure}[htb!]
	\begin{center}
		\begin{picture}(150,150)(0,0)
			%axis
			\linethickness{0.05mm}
			\multiput(0,150)(10,0){15}{\line(1,0){8}}\put(150,145){$x$}
			\multiput(0,0)(0,10){15}{\line(0,1){8}}\put(-5,-10){$y$}
			\multiput(50,150)(0,-10){10}{\line(0,-1){8}}
			%arms
			\linethickness{0.5mm}
			\put(50,150){\color{black}\circle*{10}}\put(60,140){$m_{1}$}
			\put(50,150){\line(10,-15){75}}
			\put(90,100){$l$}
			\put(125,40){\color{black}\circle*{10}}\put(130,45){$m_{2}$}
			%angles
			\linethickness{0.05mm}
			\qbezier(50,130),(55,132),(60,135)
			\put(53,120){$\varphi$}
		\end{picture}
		\caption{Pendule plan}\label{FIG:1_2}
	\end{center}
\end{figure}

\section{Probl\`eme 3}

\subsection{Cas 1}

\subsection{Cas 2}

\subsection{Cas 3}

\section{Probl\`eme 4}